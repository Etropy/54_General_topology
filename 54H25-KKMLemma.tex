\documentclass[12pt]{article}
\usepackage{pmmeta}
\pmcanonicalname{KKMLemma}
\pmcreated{2013-03-22 16:26:31}
\pmmodified{2013-03-22 16:26:31}
\pmowner{uriw}{288}
\pmmodifier{uriw}{288}
\pmtitle{KKM lemma}
\pmrecord{14}{38597}
\pmprivacy{1}
\pmauthor{uriw}{288}
\pmtype{Theorem}
\pmcomment{trigger rebuild}
\pmclassification{msc}{54H25}
\pmclassification{msc}{47H10}
\pmsynonym{K-K-M lemma}{KKMLemma}
\pmrelated{BrouwerFixedPointTheorem}

\usepackage{amsmath, amsthm, amssymb}

\newtheorem{theorem}{Theorem}
\newtheorem{corollary}[theorem]{Corollary}
\newtheorem{lemma}[theorem]{Lemma}
\newtheorem{claim}[theorem]{Claim}
\newtheorem{proposition}[theorem]{Proposition}
\newtheorem{example}[theorem]{Example}
\newtheorem{conjecture}[theorem]{Conjecture}
\newtheorem{remark}[theorem]{Remark}
\newtheorem{definition}[theorem]{Definition}

\DeclareMathOperator{\diam}{diam}

\begin{document}
\section{Preliminaries}
We start by introducing some standard notation. $\mathbb{R}^{n+1}$
is the $(n+1)$-dimensional real space with Euclidean norm and
metric. For a subset $A\subset \mathbb{R}^{n+1}$ we denote by
$\diam(A)$ the diameter of $A$.

The $n$-dimensional simplex $\mathcal{S}_n$ is the following
subset of $\mathbb{R}^{n+1}$
\[
\left\{(\alpha_1,\alpha_2,\ldots,\alpha_{n+1}) \, \Big| \,
\sum_{i=1}^{n+1}\alpha_i=1, \quad \alpha_i\geq0 \quad \forall
i=1,\ldots,n+1\right\}
\]
More generally, if $V=\{v_1,v_2,\ldots,v_k\}$ is a set of vectors,
then $S(V)$ is the simplex spanned by $V$:
\[
S(V)=\left\{\sum_{i=1}^k\alpha_i v_i\ ,\Big|\,
\sum_{i=1}^k\alpha_i=1, \quad \alpha_i\geq0 \quad \forall
i=1,\ldots,k\right\}
\]
Let $\mathcal{E}=\{e_1,e_1,\ldots,e_{n+1}\}$ be the standard
orthonormal basis of $\mathbb{R}^{n+1}$. So, $\mathcal{S}_n$ is
the simplex spanned by $\mathcal{E}$. Any element $v$ of $S(V)$ is
represented by a vector of coordinates
$(\alpha_1,\alpha_2,\ldots,\alpha_k)$ such that $v=\sum_i\alpha_i
v_i$; these are called a barycentric coordinates of $v$. If the
set $V$ is in general position then the above representation is
unique and we say that $V$ is a basis for $S(V)$. If we write
$S(V)$ then $V$ is always a basis.

Let $v$ be in $S(V)$, $V=\{v_1,v_2,\ldots,v_k\}$ a basis and $v$
represented (uniquely) by barycentric coordinates
$(\alpha_1,\alpha_2,\ldots,\alpha_k)$. We denote by $F_V(v)$ the
subset $\left\{j \,|\, \alpha_j \neq 0 \right\}$ of
$\{1,2,\ldots,k\}$ (i.e., the set of non-null coordinates). Let
$I\subset \{1,2,\ldots,k\}$, the $I$-th face of $S(V)$ is the set
$\{v\in S(V)|F_V(v) \subseteq I\}$. A \emph{face} of $S(V)$ is an
$I$-face for some $I$ (note that this is independent of the choice
of basis).

\section{KKM Lemma}

The main result we prove is the following:

\begin{theorem}[Knaster-Kuratowski-Mazurkiewicz Lemma \cite{KKM}]
Let $\mathcal{S}_n$ be the standard simplex spanned by
$\mathcal{E}$ the standard orthonormal basis for
$\mathbb{R}^{n+1}$. Assume we have $n+1$ closed subsets
$C_1,\ldots,C_{n+1}$ of $\mathcal{S}_n$ with the property that for
every subset $I$ of $\{1,2,\ldots,n+1\}$ the following holds: the
$I$-th face of $\mathcal{S}_n$ is a subset of $\cup_{i\in I}C_i$.
Then, the intersection of the sets $C_1,C_2,\ldots,C_{n+1}$ is
non-empty.
\end{theorem}

We prove the KKM Lemma by using Sperner's Lemma; Sperner's Lemma
is based on the notion of simplicial subdivision and coloring.

\begin{definition}[Simplicial subdivision of $\mathcal{S}_n$]
A \emph{simplicial subdivision} of $\mathcal{S}_n$ is a couple
$D=(V,\mathcal{B})$; $V$ are the vertices, a finite subset of
$\mathcal{S}_n$; $\mathcal{B}$ is a set of simplexes $S(V_1),
S(V_2), \ldots, S(V_k)$ where each $V_i$ is a subset of $V$ of
size $n+1$. $D$ has the following properties:
\begin{enumerate}
    \item The union of the simplexes in $\mathcal{B}$ is
    $\mathcal{S}_n$.
    \item If $S(V_i)$ and $S(V_j)$ intersect then the
    intersection is a face of both $S(V_i)$ and $S(V_j)$.
\end{enumerate}
The \emph{norm} of $D$, denoted by $|D|$, is the diameter of the
largest simplex in $\mathcal{B}$.
\end{definition}

An $(n+1)$-\emph{coloring} of a subdivision $D=(V,\mathcal{B})$ of
$\mathcal{S}_n$ is a function
\[
C:V\to\{1,2,\ldots,n+1\}
\]
A \emph{Sperner Coloring} of $D$ is an $(n+1)$-coloring $C$ such
that $C(v)\in F_\mathcal{E}(v)$ for every $v\in V$, that is, if
$v$ is on the $I$-th face then its color is from $I$. For example,
if $D=(V,\mathcal{B})$ is a subdivision of the standard simplex
$\mathcal{S}_n$ then the standard basis $\mathcal{E}$ is a subset
of $V$ and $F_\mathcal{E}(e_i)=i$. Hence, If $C$ is a Sperner
Coloring of $D$ then $C(e_i)=i$ for all $i=1,2,\ldots,n+1$.

\begin{theorem}[Sperner's Lemma]
Let $D=(V,\mathcal{B})$ be a simplicial subdivision of
$\mathcal{S}_n$ and $C:V\to\{1,2,\ldots,n+1\}$ a Sperner Coloring
of $D$. Then, there is a simplex $S(V_i)\in\mathcal{B}$ such that
$C(V_i)=\{1,2,\ldots,n+1\}$.
\end{theorem}

It is a standard result, for example by barycentric subdivisions,
that $\mathcal{S}_n$ has a sequence of simplicial subdivisions
$D_1,D_2,\ldots$ such that $|D_i|\to0$. We use this fact to prove
the KKM Lemma:

\begin{proof}[Proof of KKM Lemma]
Let $C_1,C_2,\ldots,C_{n+1}$ be closed subsets of $\mathcal{S}_n$
as given in the lemma. We define the following function
$\gamma:\mathcal{S}_n\to\{1,2,\ldots,n+1\}$ as follows:
\[
\gamma(v) = \min\{i|i\in F_\mathcal{E}(v) \textrm{ and } v\in
C_i\}
\]
$\gamma$ is well defined by the hypothesis of the lemma and
$\gamma(v) \in F_\mathcal{E}(v)$. Also, if $\gamma(v)=i$ then
$v\in C_i$. Let $D_1,D_2,\ldots$ be a sequence of simplicial
subdivisions such that $|D_i|\to0$. We set the color of every
vertex $v$ in $D_i$ to be $\gamma(v)$. This is a Sperner Coloring
since if $v$ is in $I$-fact then $\gamma(v) \in F_\mathcal{E}(v)
\subseteq I$. Therefore, by Sperner's Lemma we have in each
subdivision $D_i$ a simplex $S(V_i)$ such that
$\gamma(V_i)=\{1,2,\ldots,n+1\}$. Moreover, $\diam(S(V_i))\to0$.
By the properties of $\gamma$ for every $i=1,2,\ldots$ and every
$j\in\{1,2,\ldots,n+1\}$ we have that $S(V_i)\cap C_j\neq\phi$.
Let $u_i$ be the arithmetic mean of the elements of $V_i$ (this is
an element of $S(V_i)$ and thus an element of $\mathcal{S}_n$).
Since $\mathcal{S}_n$ is bounded and closed we get that $u_i$ has
a converging subsequence with a limit $L\in\mathcal{S}_n$. Now,
every set $C_j$ is closed, and for every $\epsilon>0$ we have an
element of $C_j$ of $\epsilon$-distance from $L$; thus $L$ is in
$C_j$.

Therefore, $L$ is in the intersection of all the sets
$C_1,C_2,\ldots,C_{n+1}$, and that proves the assertion.
\end{proof}

\begin{thebibliography}{1}

\bibitem{KKM}
B. Knaster, C. Kuratowski, and S. Mazurkiewicz, Ein Beweis des
Fixpunktsatzes f\"ur n-dimensionale Simplexe,
\newblock {\em Fund. Math. 14 (1929) 132-137.}

\end{thebibliography}


%%%%%
%%%%%
\end{document}
