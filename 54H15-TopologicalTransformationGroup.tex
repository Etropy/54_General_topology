\documentclass[12pt]{article}
\usepackage{pmmeta}
\pmcanonicalname{TopologicalTransformationGroup}
\pmcreated{2013-03-22 16:43:58}
\pmmodified{2013-03-22 16:43:58}
\pmowner{CWoo}{3771}
\pmmodifier{CWoo}{3771}
\pmtitle{topological transformation group}
\pmrecord{5}{38955}
\pmprivacy{1}
\pmauthor{CWoo}{3771}
\pmtype{Definition}
\pmcomment{trigger rebuild}
\pmclassification{msc}{54H15}
\pmclassification{msc}{22F05}
\pmdefines{effective topological transformation group}

\endmetadata

\usepackage{amssymb,amscd}
\usepackage{amsmath}
\usepackage{amsfonts}

% used for TeXing text within eps files
%\usepackage{psfrag}
% need this for including graphics (\includegraphics)
%\usepackage{graphicx}
% for neatly defining theorems and propositions
\usepackage{amsthm}
% making logically defined graphics
%%\usepackage{xypic}
\usepackage{pst-plot}
\usepackage{psfrag}

% define commands here
\newtheorem{prop}{Proposition}
\newtheorem{thm}{Theorem}
\newtheorem{ex}{Example}
\newcommand{\real}{\mathbb{R}}
\begin{document}
Let $G$ be a topological group and $X$ any topological space.  We say that $G$ is a \emph{topological transformation group} of $X$ if $G$ \emph{acts} on $X$ continuously, in the following sense:
\begin{enumerate}
\item there is a continuous function $\alpha:G\times X\to X$, where $G\times X$ is given the product topology
\item $\alpha(1,x)=x$, and
\item $\alpha(g_1g_2,x)=\alpha(g_1,\alpha(g_2,x))$.
\end{enumerate}

The function $\alpha$ is called the \emph{(left) action} of $G$ on $X$.  When there is no confusion, $\alpha(g,x)$ is simply written $gx$, so that the two conditions above read $1x=x$ and $(g_1g_2)x=g_1(g_2x)$.

If a topological transformation group $G$ on $X$ is \emph{effective}, then $G$ can be viewed as a group of homeomorphisms on $X$: simply define $h_g:X\to X$ by $h_g(x)=gx$ for each $g\in G$ so that $h_g$ is the identity function precisely when $g=1$.

\textbf{Some Examples.}
\begin{enumerate}
\item
Let $X=\mathbb{R}^n$, and $G$ be the group of $n\times n$ matrices over $\mathbb{R}$.  Clearly $X$ and $G$ are both topological spaces with the usual topology.  Furthermore, $G$ is a topological group.  $G$ acts on $X$ continuous if we view elements of $X$ as column vectors and take the action to be the matrix multiplication on the left.
\item
If $G$ is a topological group, $G$ can be considered a topological transformation group on itself.  There are many continuous actions that can be defined on $G$.  For example, $\alpha:G\times G\to G$ given by $\alpha(g,x)=gx$ is one such action.  It is continuous, and satisfies the two action axioms.  $G$ is also effective with respect to $\alpha$.
\end{enumerate}
%%%%%
%%%%%
\end{document}
