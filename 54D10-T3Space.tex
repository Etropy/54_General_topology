\documentclass[12pt]{article}
\usepackage{pmmeta}
\pmcanonicalname{T3Space}
\pmcreated{2013-03-22 12:18:24}
\pmmodified{2013-03-22 12:18:24}
\pmowner{yark}{2760}
\pmmodifier{yark}{2760}
\pmtitle{T3 space}
\pmrecord{14}{31866}
\pmprivacy{1}
\pmauthor{yark}{2760}
\pmtype{Definition}
\pmcomment{trigger rebuild}
\pmclassification{msc}{54D10}
\pmrelated{Tychonoff}
\pmrelated{T2Space}
\pmrelated{T1Space}
\pmrelated{T0Space}
\pmdefines{T3}
\pmdefines{regular}
\pmdefines{regular space}

\endmetadata

\def\T{\mathrm{T}}
\begin{document}
\PMlinkescapeword{imply}
\PMlinkescapeword{separated}
\PMlinkescapeword{opposite}
\PMlinkescapeword{words}

A \emph{regular space} is a topological space
in which points and closed sets can be separated by open sets;
in other words, given a closed set $A$ and a point $x\notin A$,
there are disjoint open sets $U$ and $V$ such that $x\in U$ and $A\subseteq V$.

A \emph{$\T_3$ space} is a regular \PMlinkname{$\T_0$-space}{T0Space}.
A $\T_3$ space is necessarily also $\T_2$, that is, Hausdorff.

Note that some authors make the opposite distinction between
$\T_3$ spaces and regular spaces,
that is, they define $\T_3$ spaces to be topological spaces
in which points and closed sets can be separated by open sets,
and then define regular spaces to be topological spaces
that are both $\T_3$ and $\T_0$.
(With these definitions, $\T_3$ does not imply $\T_2$.)

%%%%%
%%%%%
\end{document}
