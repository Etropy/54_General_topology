\documentclass[12pt]{article}
\usepackage{pmmeta}
\pmcanonicalname{CharacterizationOfT2Spaces}
\pmcreated{2013-03-22 14:41:47}
\pmmodified{2013-03-22 14:41:47}
\pmowner{matte}{1858}
\pmmodifier{matte}{1858}
\pmtitle{characterization of $T2$ spaces}
\pmrecord{7}{36306}
\pmprivacy{1}
\pmauthor{matte}{1858}
\pmtype{Theorem}
\pmcomment{trigger rebuild}
\pmclassification{msc}{54D10}
\pmrelated{LocallyCompactHausdorffSpace}

% this is the default PlanetMath preamble.  as your knowledge
% of TeX increases, you will probably want to edit this, but
% it should be fine as is for beginners.

% almost certainly you want these
\usepackage{amssymb}
\usepackage{amsmath}
\usepackage{amsfonts}
\usepackage{amsthm}

\usepackage{mathrsfs}

% used for TeXing text within eps files
%\usepackage{psfrag}
% need this for including graphics (\includegraphics)
%\usepackage{graphicx}
% for neatly defining theorems and propositions
%
% making logically defined graphics
%%%\usepackage{xypic}

% there are many more packages, add them here as you need them

% define commands here

\newcommand{\sR}[0]{\mathbb{R}}
\newcommand{\sC}[0]{\mathbb{C}}
\newcommand{\sN}[0]{\mathbb{N}}
\newcommand{\sZ}[0]{\mathbb{Z}}

 \usepackage{bbm}
 \newcommand{\Z}{\mathbbmss{Z}}
 \newcommand{\C}{\mathbbmss{C}}
 \newcommand{\R}{\mathbbmss{R}}
 \newcommand{\Q}{\mathbbmss{Q}}



\newcommand*{\norm}[1]{\lVert #1 \rVert}
\newcommand*{\abs}[1]{| #1 |}



\newtheorem{thm}{Theorem}
\newtheorem{defn}{Definition}
\newtheorem{prop}{Proposition}
\newtheorem{lemma}{Lemma}
\newtheorem{cor}{Corollary}
\begin{document}
\begin{prop}\cite{steen, bourbaki}
Suppose $X$ is a topological space. Then $X$ is a \PMlinkname{$T_2$ space}{T2Space} if and only if
for all $x\in X$, we have
\begin{eqnarray}
\label{ceq}
  \{x\} &=& \bigcap \{A \mid A\subseteq X\ \mbox{closed}, \mbox{$\exists$ open set}\  U\ \mbox{such that}\ x\in U\subseteq A\}.
\end{eqnarray}
\end{prop}

\begin{proof}
By manipulating the definition using de Morgan's laws, the claim 
can be rewritten as 
$$
  \{x\}^\complement = \bigcup \{V \mid V\subseteq X\ \mbox{open}, \mbox{$\exists$ open set}\  U\ \mbox{such that}\ x\in U\subseteq V^\complement\}.
$$
Suppose $y\in \{x\}^\complement$. As $X$ is a $T_2$ space,
there are open sets $U,V$ such that $x\in U, y\in V$, and $U\cap V=\emptyset$. 
Thus, the inclusion from left to right holds. 
On the other hand, suppose $y\in V$ for some open $V$ such that 
$\{x\}\subseteq V^\complement$. Then 
$$
  y\in V\subseteq \{x\}^\complement
$$
and the claim follows.
\end{proof}

\subsubsection*{Notes}
If we adopt the notation that a neighborhood of $x$ 
is any set containing an open set containing $x$, then the equation \ref{ceq}
can be written as 
\begin{eqnarray*}
  \{x\} &=& \bigcap \{A \mid A\subseteq X\ \mbox{is a closed neighborhood of $x$} \}.
\end{eqnarray*}



\begin{thebibliography}{9}
\bibitem{steen} L.A. Steen, J.A.Seebach, Jr.,
\emph{Counterexamples in topology},
Holt, Rinehart and Winston, Inc., 1970.
\bibitem{bourbaki} N. Bourbaki, \emph{General Topology, Part 1},
Addison-Wesley Publishing Company, 1966.
\end{thebibliography}
%%%%%
%%%%%
\end{document}
