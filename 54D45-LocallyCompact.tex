\documentclass[12pt]{article}
\usepackage{pmmeta}
\pmcanonicalname{LocallyCompact}
\pmcreated{2013-03-22 12:38:24}
\pmmodified{2013-03-22 12:38:24}
\pmowner{djao}{24}
\pmmodifier{djao}{24}
\pmtitle{locally compact}
\pmrecord{6}{32904}
\pmprivacy{1}
\pmauthor{djao}{24}
\pmtype{Definition}
\pmcomment{trigger rebuild}
\pmclassification{msc}{54D45}
\pmrelated{Compact}
\pmdefines{local compactness}

\endmetadata

% this is the default PlanetMath preamble.  as your knowledge
% of TeX increases, you will probably want to edit this, but
% it should be fine as is for beginners.

% almost certainly you want these
\usepackage{amssymb}
\usepackage{amsmath}
\usepackage{amsfonts}

% used for TeXing text within eps files
%\usepackage{psfrag}
% need this for including graphics (\includegraphics)
%\usepackage{graphicx}
% for neatly defining theorems and propositions
%\usepackage{amsthm}
% making logically defined graphics
%%%\usepackage{xypic} 

% there are many more packages, add them here as you need them

% define commands here
\begin{document}
A topological space $X$ is {\em locally compact} at a point $x \in X$ if there exists a compact set $K$ which contains a nonempty neighborhood $U$ of $x$. The space $X$ is {\em locally compact} if it is locally compact at every point $x \in X$.

Note that local compactness at $x$ does not require that $x$ have a neighborhood which is actually compact, since compact open sets are fairly rare and the more relaxed condition turns out to be more useful in practice. However, it is true that a space is locally compact at $x$ if and only if $x$ has a precompact neighborhood.
%%%%%
%%%%%
\end{document}
