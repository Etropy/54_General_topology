\documentclass[12pt]{article}
\usepackage{pmmeta}
\pmcanonicalname{Path}
\pmcreated{2013-03-22 12:00:15}
\pmmodified{2013-03-22 12:00:15}
\pmowner{rspuzio}{6075}
\pmmodifier{rspuzio}{6075}
\pmtitle{path}
\pmrecord{15}{30942}
\pmprivacy{1}
\pmauthor{rspuzio}{6075}
\pmtype{Definition}
\pmcomment{trigger rebuild}
\pmclassification{msc}{54D05}
\pmsynonym{pathwise connected}{Path}
\pmsynonym{path-connected}{Path}
\pmsynonym{path connected}{Path}
\pmrelated{SimplePath}
\pmrelated{DistanceInAGraph}
\pmrelated{LocallyConnected}
\pmrelated{ExampleOfAConnectedSpaceWhichIsNotPathConnected}
\pmrelated{PathConnectnessAsAHomotopyInvariant}
\pmdefines{path}
\pmdefines{arc}
\pmdefines{arcwise connected}
\pmdefines{initial point}
\pmdefines{terminal point}

\endmetadata

\usepackage{amssymb}
\usepackage{amsmath}
\usepackage{amsfonts}
\usepackage{graphicx}
%%%\usepackage{xypic}
\begin{document}
Let $I=[0,1] \subset \mathbb{R}$ and let $X$ be a topological space.

A continuous map $f:I\rightarrow X$ such that $f(0)=x$ and $f(1)=y$ is called a \emph{path} in $X$.  The point $x$ is called the {\bf initial point} of the path and $y$ is called its {\bf terminal point}.  If, in addition, the map is one-to-one, then it is known as an {\bf arc}.

Sometimes, it is convenient to regard two paths or arcs as equivalent if they differ by a reparameterization.  That is to say, we regard $f \colon I \to X$ and $g \colon I \to X$ as equivalent if there exists a homeomorphism $h \colon I \to I$ such that $h(0) = 0$ and $h(1) = 1$ and $f = g \circ h$.

If the space $X$ has extra structure, one may choose to restrict the classes of paths and reparameterizations.  For example, if $X$ has a differentiable structure, one may consider the class of differentiable paths.  Likewise, one can speak of piecewise linear paths, rectifiable paths, and analytic paths in suitable contexts.

The space $X$ is said to be {\bf pathwise connected} if, for every two points $x, y \in X$, there exists a path having $x$ as initial point and $y$ as terminal point.  Likewise, the space $X$ is said to be {\bf arcwise connected} if, for every two distinct points $x, y \in X$, there exists an \emph{arc} having $x$ as initial point and $y$ as terminal point.  

A pathwise connected space is always a connected space, but a connected space need not be path connected.  An arcwise connected space is always a pathwise connected space, but a pathwise connected space need not be arcwise connected.
As it turns out, for Hausdorff spaces these two notions coincide --- a Hausdorff space is pathwise connected iff it is arcwise connected.
%%%%%
%%%%%
%%%%%
\end{document}
