\documentclass[12pt]{article}
\usepackage{pmmeta}
\pmcanonicalname{RegularlyOpen}
\pmcreated{2013-03-22 12:19:43}
\pmmodified{2013-03-22 12:19:43}
\pmowner{drini}{3}
\pmmodifier{drini}{3}
\pmtitle{regularly open}
\pmrecord{5}{31954}
\pmprivacy{1}
\pmauthor{drini}{3}
\pmtype{Definition}
\pmcomment{trigger rebuild}
\pmclassification{msc}{54-00}

\endmetadata

%\usepackage{graphicx}
%%%\usepackage{xypic} 
\usepackage{bbm}
\newcommand{\Z}{\mathbbmss{Z}}
\newcommand{\C}{\mathbbmss{C}}
\newcommand{\R}{\mathbbmss{R}}
\newcommand{\Q}{\mathbbmss{Q}}
\newcommand{\mathbb}[1]{\mathbbmss{#1}}
\begin{document}
Given a topological space $(X,\tau)$, a \emph{regularly open set} is an open set $A\in \tau$ such that 
$$\mathrm{int}\, \overline{A}=A$$
(the interior of the closure is the set itself).

An example of non regularly open set on the standard topology for $\R$ is
$A=(0,1)\cup(1,2)$ since $\mathrm{int}\overline{A}=(0,2)$.
%%%%%
%%%%%
\end{document}
