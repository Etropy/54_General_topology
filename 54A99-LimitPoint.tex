\documentclass[12pt]{article}
\usepackage{pmmeta}
\pmcanonicalname{LimitPoint}
\pmcreated{2013-03-22 12:06:51}
\pmmodified{2013-03-22 12:06:51}
\pmowner{mathcam}{2727}
\pmmodifier{mathcam}{2727}
\pmtitle{limit point}
\pmrecord{15}{31240}
\pmprivacy{1}
\pmauthor{mathcam}{2727}
\pmtype{Definition}
\pmcomment{trigger rebuild}
\pmclassification{msc}{54A99}
\pmsynonym{accumulation point}{LimitPoint}
\pmsynonym{cluster point}{LimitPoint}
%\pmkeywords{topology}
\pmrelated{AlternateStatementOfBolzanoWeierstrassTheorem}

\endmetadata

\usepackage{amssymb}
\usepackage{amsmath}
\usepackage{amsfonts}
\usepackage{graphicx}
%%%\usepackage{xypic}
\begin{document}
Let $X$ be a topological space, and let $A\subseteq X$.  An element $x\in X$ is said to be a \emph{limit point} of $A$ if every open set containing $x$ also contains at least one point of $A$ distinct from $x$.  Note that we can often take a nested sequence of open such sets, and can thereby construct a sequence of points which converge to $x$, partially motivating the terminology "limit'' in this case.

Equivalently:
\begin{itemize}
\item $x$ is a limit point of $A$ if and only if there is a net in $A$ converging to $x$ which is not residually constant.
\item $x$ is a limit point of $A$ if and only if there is a filter on $A$ \PMlinkname{converging}{filter} to $x$.
\item If $X$ is a metric (or first countable) space, $x$ is a limit point of $A$ if and only if there is a sequence of points in $A\setminus\{x\}$ converging to $x$.
\end{itemize}
%%%%%
%%%%%
%%%%%
\end{document}
