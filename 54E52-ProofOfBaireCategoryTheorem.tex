\documentclass[12pt]{article}
\usepackage{pmmeta}
\pmcanonicalname{ProofOfBaireCategoryTheorem}
\pmcreated{2013-03-22 13:06:55}
\pmmodified{2013-03-22 13:06:55}
\pmowner{rmilson}{146}
\pmmodifier{rmilson}{146}
\pmtitle{proof of Baire category theorem}
\pmrecord{10}{33543}
\pmprivacy{1}
\pmauthor{rmilson}{146}
\pmtype{Proof}
\pmcomment{trigger rebuild}
\pmclassification{msc}{54E52}

\usepackage{amsmath}
\usepackage{amsfonts}
\usepackage{amssymb}
\newcommand{\reals}{\mathbb{R}}
\newcommand{\natnums}{\mathbb{N}}
\newcommand{\cnums}{\mathbb{C}}
\newcommand{\znums}{\mathbb{Z}}
\newcommand{\lp}{\left(}
\newcommand{\rp}{\right)}
\newcommand{\lb}{\left[}
\newcommand{\rb}{\right]}
\newcommand{\supth}{^{\text{th}}}
\newtheorem{proposition}{Proposition}
\newtheorem{definition}[proposition]{Definition}

\newtheorem{theorem}[proposition]{Theorem}
\begin{document}
Let $(X,d)$ be a complete metric space, and $U_k$ a countable
collection of dense, open subsets.  Let $x_0\in X$ and $\epsilon_0>0$ be
given.  We must show that there exists a $x\in \bigcap_k U_k$ such that
$$d(x_0,x)<\epsilon_0.$$
Since $U_1$ is dense and open, we may choose an $\epsilon_1>0$ and an
$x_1\in U_1$ such that
$$d(x_0,x_1)<\frac{\epsilon_0}{2},\quad \epsilon_1<\frac{\epsilon_0}{2},$$
and  such that the open ball of
radius $\epsilon_1$ about $x_1$ lies entirely  
in $U_1$.  We then choose an $\epsilon_2>0$ and a $x_2\in U_2$ such that
$$d(x_1,x_2)<\frac{\epsilon_1}{2},\quad \epsilon_2<\frac{\epsilon_1}{2},$$
and  such that the open ball
of radius $\epsilon_2$ about $x_2$ lies 
entirely in $U_2$.  We continue by induction, and construct a sequence
of points $x_k\in U_k$ and positive $\epsilon_k$ such that
$$d(x_{k-1},x_k)<\frac{\epsilon_{k-1}}{2},\quad \epsilon_k<\frac{\epsilon_{k-1}}{2},$$
and such that the open
ball of radius $\epsilon_k$ lies entirely in $U_k$. 

By construction, for $0\leq j<k$ we have
$$d(x_j,x_k) < \epsilon_j \left(\frac{1}{2} + \cdots +
  \frac{1}{2^{k-j}}\right) < \epsilon_j \leq \frac{\epsilon_0}{2^j}.$$
Hence the
sequence $x_k, \; k=1,2,\ldots$ is Cauchy, and converges by hypothesis
to some $x\in X$.  It is  clear that for every $k$ we have
$$d(x,x_k) \leq \epsilon_k.$$
Moreover it follows that
$$d(x,x_k) \leq d(x,x_{k+1}) + d(x_{k},x_{k+1}) < \epsilon_{k+1} +
\frac{\epsilon_{k}}{2},$$
and hence a fortiori
$$d(x,x_k)<\epsilon_k$$
for every $k$.  By construction then, $x\in
U_k$ for all $k=1,2,\ldots$, as well.  QED
%%%%%
%%%%%
\end{document}
