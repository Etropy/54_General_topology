\documentclass[12pt]{article}
\usepackage{pmmeta}
\pmcanonicalname{Homeomorphism}
\pmcreated{2013-03-22 11:59:35}
\pmmodified{2013-03-22 11:59:35}
\pmowner{rspuzio}{6075}
\pmmodifier{rspuzio}{6075}
\pmtitle{homeomorphism}
\pmrecord{16}{30912}
\pmprivacy{1}
\pmauthor{rspuzio}{6075}
\pmtype{Definition}
\pmcomment{trigger rebuild}
\pmclassification{msc}{54C05}
\pmsynonym{topological equivalence}{Homeomorphism}
\pmsynonym{topologically equivalent}{Homeomorphism}
\pmrelated{Homeotopy}
\pmrelated{CrosscapSlide}
\pmrelated{AlexanderTrick}
\pmrelated{GroupoidCategory}
\pmdefines{homeomorphic}
\pmdefines{autohomeomorphism}
\pmdefines{auto-homeomorphism}
\pmdefines{self-homeomorphism}

\usepackage{amssymb}
\usepackage{amsmath}
\usepackage{amsfonts}
\usepackage{graphicx}
%%%\usepackage{xypic}
\begin{document}
A \emph{homeomorphism} $f$ of topological spaces is a continuous, bijective map such that $f^{-1}$ is also continuous. We also say that two spaces are \emph{homeomorphic} if such a map exists.

If two topological spaces are homeomorphic, they are topologically equivalent --- using the techniques of topology, there is no way of distinguishing one space from the other.

An \emph{autohomeomorphism} (also known as a \emph{self-homeomorphism}) is a 
homeomorphism from a topological space to itself.
%%%%%
%%%%%
%%%%%
\end{document}
