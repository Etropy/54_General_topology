\documentclass[12pt]{article}
\usepackage{pmmeta}
\pmcanonicalname{IdentificationTopology}
\pmcreated{2013-03-22 14:41:26}
\pmmodified{2013-03-22 14:41:26}
\pmowner{rspuzio}{6075}
\pmmodifier{rspuzio}{6075}
\pmtitle{identification topology}
\pmrecord{11}{36299}
\pmprivacy{1}
\pmauthor{rspuzio}{6075}
\pmtype{Definition}
\pmcomment{trigger rebuild}
\pmclassification{msc}{54A99}
\pmsynonym{final topology}{IdentificationTopology}
\pmrelated{InitialTopology}

% this is the default PlanetMath preamble.  as your knowledge
% of TeX increases, you will probably want to edit this, but
% it should be fine as is for beginners.

% almost certainly you want these
\usepackage{amssymb}
\usepackage{amsmath}
\usepackage{amsfonts}

% used for TeXing text within eps files
%\usepackage{psfrag}
% need this for including graphics (\includegraphics)
%\usepackage{graphicx}
% for neatly defining theorems and propositions
\usepackage{amsthm}
% making logically defined graphics
%%%\usepackage{xypic}

% there are many more packages, add them here as you need them

% define commands here
\newtheorem{prop}{Proposition}
\newtheorem{thm}{Theorem}
\newtheorem{ex}{Example}
\newcommand{\real}{\mathbb{R}}
\begin{document}
Let $f$ be a function from a topological space $X$ to a set $Y$.  The \emph{identification topology} on $Y$ with respect to $f$ is defined to be the finest topology on $Y$ such that the function $f$ is continuous.

\begin{thm} Let $f:X\to Y$ be defined as above.  The following are equivalent:
\begin{enumerate}
\item $\mathcal{T}$ is the identification topology on $Y$.
\item $U\subseteq Y$ is open under $\mathcal{T}$ iff $f^{-1}(U)$ is open in $X$.
\end{enumerate}
\end{thm}
\begin{proof}
($1.\Rightarrow 2.$)  If $U$ is open under $\mathcal{T}$, then $f^{-1}(U)$ is open in $X$ as $f$ is continuous under $\mathcal{T}$.  Now, suppose $U$ is not open under $\mathcal{T}$ and $f^{-1}(U)$ is open in $X$.  Let $\mathcal{B}$ be a subbase of $\mathcal{T}$.  Define $\mathcal{B}':=\mathcal{B}\cup \lbrace U\rbrace$.  Then the topology $\mathcal{T}'$ generated by $\mathcal{B}'$ is a strictly finer topology than $\mathcal{T}$ making $f$ continuous, a contradiction.

($2.\Rightarrow 1.$)  Let $\mathcal{T}$ be the topology defined by 2.  Then $f$ is continuous.  Suppose $\mathcal{T}'$ is another topology on $Y$ making $f$ continuous.  Let $U$ be $\mathcal{T}'$-open.  Then $f^{-1}(U)$ is open in $X$, which implies $U$ is $\mathcal{T}$-open.  Thus $\mathcal{T}'\subseteq \mathcal{T}$ and $\mathcal{T}$ is finer than $\mathcal{T}'$.
\end{proof}

\textbf{Remarks}.
\begin{itemize}
\item
$\mathcal{S}=\lbrace f(V)\mid V\mbox{ is open in }X\rbrace$ is a subbasis for $f(X)$, using the subspace topology on $f(X)$ of the identification topology on $Y$.
\item
More generally, let $X_i$ be a family of topological spaces and $f_i:X_i\to Y$ be a family of functions from $X_i$ into $Y$.  The \emph{identification topology} on $Y$ with respect to the family $f_i$ is the finest topology on $Y$ making each $f_i$ a continuous function.  In literature, this topology is also called the \emph{final topology}.
\item
The dual concept of this is the initial topology.
\item
Let $f:X\to Y$ be defined as above.  Define binary relation $\sim$ on $X$ so that $x\sim y$ iff $f(x)=f(y)$.  Clearly $\sim$ is an equivalence relation.  Let $X^*$ be the quotient $X/\sim$.  Then $f$ induces an injective map $f^*:X^*\to Y$ given by $f^*([x])=f(x)$.  Let $Y$ be given the identification topology and $X^*$ the quotient topology (induced by $\sim$), then $f^*$ is continuous.  Indeed, for if $V\subseteq Y$ is open, then $f^{-1}(V)$ is open in $X$.  But then $f^{-1}(V)=\bigcup f^{* -1}(V)$, which implies $f^{* -1}(V)$ is open in $X^*$.  Furthermore, the argument is reversible, so that if $U$ is open in $X^*$, then so is $f^*(U)$ open in $Y$.  Finally, if $f$ is surjective, so is $f^*$, so that $f^*$ is a homeomorphism.
\end{itemize}
%%%%%
%%%%%
\end{document}
