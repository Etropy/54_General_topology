\documentclass[12pt]{article}
\usepackage{pmmeta}
\pmcanonicalname{GeneralizationOfAPseudometric}
\pmcreated{2013-03-22 16:43:06}
\pmmodified{2013-03-22 16:43:06}
\pmowner{CWoo}{3771}
\pmmodifier{CWoo}{3771}
\pmtitle{generalization of a pseudometric}
\pmrecord{6}{38936}
\pmprivacy{1}
\pmauthor{CWoo}{3771}
\pmtype{Definition}
\pmcomment{trigger rebuild}
\pmclassification{msc}{54E35}
\pmsynonym{semipseudometric}{GeneralizationOfAPseudometric}
\pmsynonym{quasipseudometric}{GeneralizationOfAPseudometric}
\pmsynonym{semipseudometric space}{GeneralizationOfAPseudometric}
\pmsynonym{quasipseudometric space}{GeneralizationOfAPseudometric}
\pmrelated{semimetric}
\pmrelated{quasimetric}
\pmrelated{GeneralizationOfAUniformity}
\pmdefines{semi-pseudometric space}
\pmdefines{quasi-pseudometric space}
\pmdefines{semi-pseudometric}
\pmdefines{quasi-pseudometric}

\usepackage{amssymb,amscd}
\usepackage{amsmath}
\usepackage{amsfonts}

% used for TeXing text within eps files
%\usepackage{psfrag}
% need this for including graphics (\includegraphics)
%\usepackage{graphicx}
% for neatly defining theorems and propositions
\usepackage{amsthm}
% making logically defined graphics
%%\usepackage{xypic}
\usepackage{pst-plot}
\usepackage{psfrag}

% define commands here

\begin{document}
Let $X$ be a set.  Let $d:X\times X \to \mathbb{R}$ be a function with the property that $d(x,y)\ge 0$ for all $x,y\in X$.  Then $d$ is a
\begin{enumerate}
\item \emph{semi-pseudometric} if $d(x,y)=d(y,x)$ for all $x,y\in X$,
\item \emph{quasi-pseudometric} if $d(x,z)\le d(x,y)+d(y,z)$ for all $x,y,z\in X$.
\end{enumerate}

$X$ equipped with a function $d$ described above is called a \emph{semi-pseudometric space} or a \emph{quasi-pseudometric} space, depending on whether $d$ is a semi-pseudometric or a quasi-pseudometric.  A pseudometric is the same as a semi-pseudometric that is a quasi-pseudometric at the same time.

If $d$ satisfies the property that $d(x,y)=0$ implies $x=y$, then $d$ is called a \emph{semi-metric} if $d$ is a semi-pseudometric, or a \emph{quasi-metric} if $d$ is a quasi-pseudometric.

%%%%%
%%%%%
\end{document}
