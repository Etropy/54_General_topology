\documentclass[12pt]{article}
\usepackage{pmmeta}
\pmcanonicalname{TheGroupsOfRealNumbers}
\pmcreated{2013-03-22 16:08:50}
\pmmodified{2013-03-22 16:08:50}
\pmowner{Algeboy}{12884}
\pmmodifier{Algeboy}{12884}
\pmtitle{the groups of real numbers}
\pmrecord{5}{38226}
\pmprivacy{1}
\pmauthor{Algeboy}{12884}
\pmtype{Result}
\pmcomment{trigger rebuild}
\pmclassification{msc}{54C30}
\pmclassification{msc}{26-00}
\pmclassification{msc}{12D99}
\pmrelated{ExponentialFunction}
\pmrelated{GroupHomomorphism}
\pmrelated{GroupsInField}

\usepackage{latexsym}
\usepackage{amssymb}
\usepackage{amsmath}
\usepackage{amsfonts}
\usepackage{amsthm}

%%\usepackage{xypic}

%-----------------------------------------------------

%       Standard theoremlike environments.

%       Stolen directly from AMSLaTeX sample

%-----------------------------------------------------

%% \theoremstyle{plain} %% This is the default

\newtheorem{thm}{Theorem}

\newtheorem{coro}[thm]{Corollary}

\newtheorem{lem}[thm]{Lemma}

\newtheorem{lemma}[thm]{Lemma}

\newtheorem{prop}[thm]{Proposition}

\newtheorem{conjecture}[thm]{Conjecture}

\newtheorem{conj}[thm]{Conjecture}

\newtheorem{defn}[thm]{Definition}

\newtheorem{remark}[thm]{Remark}

\newtheorem{ex}[thm]{Example}



%\countstyle[equation]{thm}



%--------------------------------------------------

%       Item references.

%--------------------------------------------------


\newcommand{\exref}[1]{Example-\ref{#1}}

\newcommand{\thmref}[1]{Theorem-\ref{#1}}

\newcommand{\defref}[1]{Definition-\ref{#1}}

\newcommand{\eqnref}[1]{(\ref{#1})}

\newcommand{\secref}[1]{Section-\ref{#1}}

\newcommand{\lemref}[1]{Lemma-\ref{#1}}

\newcommand{\propref}[1]{Prop\-o\-si\-tion-\ref{#1}}

\newcommand{\corref}[1]{Cor\-ol\-lary-\ref{#1}}

\newcommand{\figref}[1]{Fig\-ure-\ref{#1}}

\newcommand{\conjref}[1]{Conjecture-\ref{#1}}


% Normal subgroup or equal.

\providecommand{\normaleq}{\unlhd}

% Normal subgroup.

\providecommand{\normal}{\lhd}

\providecommand{\rnormal}{\rhd}
% Divides, does not divide.

\providecommand{\divides}{\mid}

\providecommand{\ndivides}{\nmid}


\providecommand{\union}{\cup}

\providecommand{\bigunion}{\bigcup}

\providecommand{\intersect}{\cap}

\providecommand{\bigintersect}{\bigcap}










\begin{document}
\begin{prop}
The additive group of real number $\langle \mathbb{R},+\rangle$ is isomorphic
to the multiplicative group of positive real numbers $\langle \mathbb{R}^+,\times\rangle$.
\end{prop}
\begin{proof}
Let $f(x)=e^x$.  This maps the group $\langle \mathbb{R},+\rangle$ to the group
$\langle \mathbb{R}^+,\times\rangle$.  As $f$ has an inverse $f^{-1}(x)=\ln x$
we observe $f$ is invertible.  Furthermore, $f(x+y)=e^{x+y}=e^x e^y=f(x)f(y)$ so
$f$ is a homomorphism.  Thus $f$ is an isomorphism.
\end{proof}

\begin{coro}
The multiplicative group of non-zero real number $\mathbb{R}^\times$ is isomorphic to $\mathbb{Z}_2\oplus \langle \mathbb{R},+\rangle$.
\end{coro}
\begin{proof}
Use the map $f:\mathbb{Z}_2\oplus \langle \mathbb{R},+\rangle\rightarrow \mathbb{R}^\times$ defined by $f(s,r)=(-1)^s e^r$.\footnote{We write $(-1)^s$ to mean $(-1)^{s'}$ for any integer $s'$ representative of the equivalence class of $s$ in
$\mathbb{Z}_2$.}  Then 
\[f((s_1,r_1)+(s_2,r_2))=f(s_1+s_2,r_1+r_2)=(-1)^{s_1+s_2}e^{r_1+r_2}
=(-1)^{s_1} e^{r_1} (-1)^{s_2}e^{r_2}=f(s_1,r_1) f(s_2,r_2)\]
so that $f$ is a homomorphism.  Furthermore, $f^{-1}(r)=(\operatorname{sign} r,\ln |r|)$
is the inverse of $f$ so that $f$ is bijective and thus an isomorphism of groups.
\end{proof}


%%%%%
%%%%%
\end{document}
