\documentclass[12pt]{article}
\usepackage{pmmeta}
\pmcanonicalname{RationalNumbersAreRealNumbers}
\pmcreated{2013-03-22 15:45:49}
\pmmodified{2013-03-22 15:45:49}
\pmowner{matte}{1858}
\pmmodifier{matte}{1858}
\pmtitle{rational numbers are real numbers}
\pmrecord{6}{37719}
\pmprivacy{1}
\pmauthor{matte}{1858}
\pmtype{Result}
\pmcomment{trigger rebuild}
\pmclassification{msc}{54C30}
\pmclassification{msc}{26-00}
\pmclassification{msc}{12D99}
\pmrelated{GroundField}

% this is the default PlanetMath preamble.  as your knowledge
% of TeX increases, you will probably want to edit this, but
% it should be fine as is for beginners.

% almost certainly you want these
\usepackage{amssymb}
\usepackage{amsmath}
\usepackage{amsfonts}
\usepackage{amsthm}

\usepackage{mathrsfs}

% used for TeXing text within eps files
%\usepackage{psfrag}
% need this for including graphics (\includegraphics)
%\usepackage{graphicx}
% for neatly defining theorems and propositions
%
% making logically defined graphics
%%%\usepackage{xypic}

% there are many more packages, add them here as you need them

% define commands here

\newcommand{\sR}[0]{\mathbb{R}}
\newcommand{\sC}[0]{\mathbb{C}}
\newcommand{\sN}[0]{\mathbb{N}}
\newcommand{\sZ}[0]{\mathbb{Z}}

 \usepackage{bbm}
 \newcommand{\Z}{\mathbbmss{Z}}
 \newcommand{\N}{\mathbbmss{N}}
 \newcommand{\C}{\mathbbmss{C}}
 \newcommand{\F}{\mathbbmss{F}}
 \newcommand{\R}{\mathbbmss{R}}
 \newcommand{\Q}{\mathbbmss{Q}}



\newcommand*{\norm}[1]{\lVert #1 \rVert}
\newcommand*{\abs}[1]{| #1 |}



\newtheorem{thm}{Theorem}
\newtheorem{defn}{Definition}
\newtheorem{prop}{Proposition}
\newtheorem{lemma}{Lemma}
\newtheorem{cor}{Corollary}
\begin{document}
%\subsubsection*{Rational numbers are real numbers}
Let us first show that the natural numbers $0,1,2,\ldots$ are 
contained in the real numbers as constructed above. 
Heuristically, this should be clear. We start with $0$. 
By adding $1$ repeatedly we obtain the natural numbers
$$
  0, \quad 0+1, \quad (0+1)+1, \quad ((0+1)+1)+1, \ldots,
$$
To make this precise, let $\N$ be the natural numbers.
(We assume that these exist. For example, all the usual constructions
of $\R$ rely on the existence of the natural numbers.)
Then we can define a map $f\colon \N\to \R$ as
\begin{enumerate}
\item $f(0)=0$, or more precisely, $f(0_\N)=0_\R$,
\item $f(a+1)=f(a)+1$ for $a\in \N$.
\end{enumerate}
By induction on $a$ one can prove that
\begin{eqnarray*}
  f(a + b) &=&  f(a) + f(b), \\
  f(a b)   &=&  f(a) f(b), \quad a,b\in \N
\end{eqnarray*}
and
\begin{eqnarray*}
  f(a) &\ge & 0, \ a\in \N\ \mbox{with equality only when}\ a=0.
\end{eqnarray*}
The last claim follows since $f(a)>0$ for $a=1,2,\ldots$ (by induction), 
and $f(0)=0$. 
It follows that $f$ is an injection: If $a\le b$, then $f(a)=f(b)$ implies
that $f(a)=f(a)+f(b-a)$, so $a=b$. 

To conclude, let us show that
$f(\N)\subset\R$ satisfies the Peano axioms with zero element $f(0)$ and
sucessor operator 
\begin{eqnarray*}
S\colon f(\N) &\to& f(\N) \\
        k &\mapsto & f(f^{-1}(k)+1)
\end{eqnarray*}
First, as $f$ is a bijection,  $x=y$ if and only if $S(x)=S(y)$ 
   is clear. 
Second, if $S(k)=0$ for some $k=f(a)\in f(\N)$, then $a+1=0$; a contradiction.
Lastly, the axiom of induction follows since $\N$ satisfies this axiom.
We have shown that $f(\N)$ are a subset of the real numbers that
behave as the natural numbers. 

From the natural numbers, the integers and rationals can
be defined as
\begin{eqnarray*}
  \Z&=& \N\cup \{-z\in \R : z\in \N\}, \\ 
  \Q&=& \left\{ \frac{a}{b} : a\in \Z, b\in \N\setminus\{0\} \right\}.
\end{eqnarray*}
Mathematically, $\Z$ and $\Q$ are subrings of $\R$ that are
ring isomorphic to the integers and rationals, respectively.

\subsubsection*{Other constructions}
The above construction follows \cite{royden}. However, there are also
other constructions. For example, in \cite{spivak}, natural numbers in $\R$
are defined as follows. First, a set $L\subseteq \R$ is \emph{inductiv{e}} if 
\begin{enumerate}
\item $0\in L$, 
\item if $a\in L$, then $a+1\in L$.
\end{enumerate}
Then the natural numbers are defined as real numbers that are contained in all
inductiv{e} sets.
A third approach is to explicitly exhibit the natural numbers when
constructing the real numbers. For example, in \cite{rudin}, 
it is shown that the rational numbers form a subfield of $\R$
using explicit Dedekind cuts. 


\begin{thebibliography}{9}
\bibitem{royden} H.L. Royden, 
   \emph{Real analysis}, Prentice Hall, 1988.
\bibitem{spivak} M. Spivak,
   \emph{Calculus}, Publish or Perish.
\bibitem{rudin} W. Rudin,
   \emph{Principles of mathematical analysis},
   McGraw-Hill, 1976.
\end{thebibliography}
%%%%%
%%%%%
\end{document}
