\documentclass[12pt]{article}
\usepackage{pmmeta}
\pmcanonicalname{QuotientSpace}
\pmcreated{2013-03-22 12:39:40}
\pmmodified{2013-03-22 12:39:40}
\pmowner{djao}{24}
\pmmodifier{djao}{24}
\pmtitle{quotient space}
\pmrecord{5}{32930}
\pmprivacy{1}
\pmauthor{djao}{24}
\pmtype{Definition}
\pmcomment{trigger rebuild}
\pmclassification{msc}{54B15}
\pmrelated{AdjunctionSpace}
\pmdefines{quotient topology}
\pmdefines{quotient map}

\endmetadata

% this is the default PlanetMath preamble.  as your knowledge
% of TeX increases, you will probably want to edit this, but
% it should be fine as is for beginners.

% almost certainly you want these
\usepackage{amssymb}
\usepackage{amsmath}
\usepackage{amsfonts}

% used for TeXing text within eps files
%\usepackage{psfrag}
% need this for including graphics (\includegraphics)
%\usepackage{graphicx}
% for neatly defining theorems and propositions
%\usepackage{amsthm}
% making logically defined graphics
%%%\usepackage{xypic} 

% there are many more packages, add them here as you need them

% define commands here
\begin{document}
Let $X$ be a topological space, and let $\sim$ be an equivalence relation on $X$. Write $X^*$ for the set of equivalence classes of $X$ under $\sim$. The {\em quotient topology} on $X^*$ is the topology whose open sets are the subsets $U \subset X^*$ such that
$$
\bigcup U \subset X
$$
is an open subset of $X$. The space $X^*$ is called the {\em quotient space} of the space $X$ with respect to $\sim$. It is often written $X/\sim$.

The projection map $\pi: X \longrightarrow X^*$ which sends each element of $X$ to its equivalence class is always a continuous map. In fact, the map $\pi$ satisfies the stronger property that a subset $U$ of $X^*$ is open if and only if the subset $\pi^{-1}(U)$ of $X$ is open. In general, any surjective map $p: X \longrightarrow Y$ that satisfies this stronger property is called a {\em quotient map}, and given such a quotient map, the space $Y$ is always homeomorphic to the quotient space of $X$ under the equivalence relation
$$
x \sim x' \iff p(x) = p(x').
$$

As a set, the construction of a quotient space collapses each of the equivalence classes of $\sim$ to a single point. The topology on the quotient space is then chosen to be the strongest topology such that the projection map $\pi$ is continuous.

For $A \subset X$, one often writes $X/A$ for the quotient space obtained by identifying all the points of $A$ with each other.
%%%%%
%%%%%
\end{document}
