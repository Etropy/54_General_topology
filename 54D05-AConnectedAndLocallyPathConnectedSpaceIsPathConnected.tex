\documentclass[12pt]{article}
\usepackage{pmmeta}
\pmcanonicalname{AConnectedAndLocallyPathConnectedSpaceIsPathConnected}
\pmcreated{2013-03-22 16:50:43}
\pmmodified{2013-03-22 16:50:43}
\pmowner{Mathprof}{13753}
\pmmodifier{Mathprof}{13753}
\pmtitle{a connected and locally path connected space is path connected}
\pmrecord{6}{39090}
\pmprivacy{1}
\pmauthor{Mathprof}{13753}
\pmtype{Theorem}
\pmcomment{trigger rebuild}
\pmclassification{msc}{54D05}

% this is the default PlanetMath preamble.  as your knowledge
% of TeX increases, you will probably want to edit this, but
% it should be fine as is for beginners.

% almost certainly you want these
\usepackage{amssymb}
\usepackage{amsmath}
\usepackage{amsfonts}

% used for TeXing text within eps files
%\usepackage{psfrag}
% need this for including graphics (\includegraphics)
%\usepackage{graphicx}
% for neatly defining theorems and propositions
%\usepackage{amsthm}
% making logically defined graphics
%%%\usepackage{xypic}

% there are many more packages, add them here as you need them

% define commands here

\begin{document}
Theorem. A connected, locally path connected topological space is path connected.

Proof. Let $X$ be the space and fix $p \in X$. Let $C$ be the set of all points in  
$X$ that can be joined to $p$ by a path. $C$ is nonempty so it is enough to show that $C$ is both closed and open. 

To show first that $C$ is open: Let $c$ be in $C$ and choose an open path connected neighborhood $U$ of $c$. If $u \in U$ we can find a path joining $u$ to $c$ and then join that path to a path from $p$ to $c$. Hence $u$ is in $C$.

To show that $C$ is closed: Let $c$ be in $\overline{C} $ and choose an open path connected neighborhood $U$ of $c$. Then $C \cap U \neq \varnothing$. Choose $q \in C \cap U$. Then $c$ can be joined to $q$ by a path and $q$  can be joined to $p$ by a path, so by addition of paths, $p$ can be joined to $c$ by a path, that is, $c \in C$.

%%%%%
%%%%%
\end{document}
