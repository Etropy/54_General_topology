\documentclass[12pt]{article}
\usepackage{pmmeta}
\pmcanonicalname{GeneralizationOfAUniformity}
\pmcreated{2013-03-22 16:43:09}
\pmmodified{2013-03-22 16:43:09}
\pmowner{CWoo}{3771}
\pmmodifier{CWoo}{3771}
\pmtitle{generalization of a uniformity}
\pmrecord{5}{38937}
\pmprivacy{1}
\pmauthor{CWoo}{3771}
\pmtype{Definition}
\pmcomment{trigger rebuild}
\pmclassification{msc}{54E15}
\pmsynonym{semiuniformity}{GeneralizationOfAUniformity}
\pmsynonym{quasiuniformity}{GeneralizationOfAUniformity}
\pmsynonym{semiuniform space}{GeneralizationOfAUniformity}
\pmsynonym{quasiuniform space}{GeneralizationOfAUniformity}
\pmsynonym{semi-uniform}{GeneralizationOfAUniformity}
\pmsynonym{quasi-uniform}{GeneralizationOfAUniformity}
\pmsynonym{semiuniform}{GeneralizationOfAUniformity}
\pmsynonym{quasiuniform}{GeneralizationOfAUniformity}
\pmrelated{GeneralizationOfAPseudometric}
\pmdefines{semi-uniformity}
\pmdefines{quasi-uniformity}
\pmdefines{semi-uniform space}
\pmdefines{quasi-uniform space}

\usepackage{amssymb,amscd}
\usepackage{amsmath}
\usepackage{amsfonts}

% used for TeXing text within eps files
%\usepackage{psfrag}
% need this for including graphics (\includegraphics)
%\usepackage{graphicx}
% for neatly defining theorems and propositions
\usepackage{amsthm}
% making logically defined graphics
%%\usepackage{xypic}
\usepackage{pst-plot}
\usepackage{psfrag}

% define commands here

\begin{document}
Let $X$ be a set.  Let $\mathcal{U}$ be a family of subsets of $X\times X$ such that $\mathcal{U}$ is a filter, and that every element of $\mathcal{U}$ contains the diagonal relation $\Delta$ (reflexive).  Consider the following possible ``axioms'':
\begin{enumerate}
\item for every $U\in \mathcal{U}$, $U^{-1}\in \mathcal{U}$
\item for every $U\in \mathcal{U}$, there is $V\in \mathcal{U}$ such that $V\circ V\in U$,
\end{enumerate}

where $U^{-1}$ is defined as the \PMlinkname{inverse relation}{OperationsOnRelations} of $U$, and $\circ$ is the \PMlinkname{composition of relations}{OperationsOnRelations}.  If $\mathcal{U}$ satisfies Axiom 1, then $\mathcal{U}$ is called a \emph{semi-uniformity}.  If $\mathcal{U}$ satisfies Axiom 2, then $\mathcal{U}$ is called a \emph{quasi-uniformity}.  The underlying set $X$ equipped with $\mathcal{U}$ is called a \emph{semi-uniform space} or a \emph{quasi-uniform space} according to whether $\mathcal{U}$ is a semi-uniformity or a quasi-uniformity.

A semi-pseudometric space is a semi-uniform space.  A quasi-pseudometric space is a quasi-uniform space.

A uniformity is one that satisfies both axioms, which is equivalent to saying that it is both a semi-uniformity and a quasi-uniformity.

\begin{thebibliography}{9}
\bibitem{wp} W. Page, \emph{Topological Uniform Structures}, Wiley, New York 1978.
\end{thebibliography}
%%%%%
%%%%%
\end{document}
