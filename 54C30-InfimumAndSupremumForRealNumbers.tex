\documentclass[12pt]{article}
\usepackage{pmmeta}
\pmcanonicalname{InfimumAndSupremumForRealNumbers}
\pmcreated{2013-03-22 15:41:42}
\pmmodified{2013-03-22 15:41:42}
\pmowner{matte}{1858}
\pmmodifier{matte}{1858}
\pmtitle{infimum and supremum for real numbers}
\pmrecord{6}{37638}
\pmprivacy{1}
\pmauthor{matte}{1858}
\pmtype{Topic}
\pmcomment{trigger rebuild}
\pmclassification{msc}{54C30}
\pmclassification{msc}{26-00}
\pmclassification{msc}{12D99}
\pmrelated{SetsThatDoNotHaveAnInfimum}
\pmrelated{Infimum}
\pmrelated{Supremum}

% this is the default PlanetMath preamble.  as your knowledge
% of TeX increases, you will probably want to edit this, but
% it should be fine as is for beginners.

% almost certainly you want these
\usepackage{amssymb}
\usepackage{amsmath}
\usepackage{amsfonts}
\usepackage{amsthm}

\usepackage{mathrsfs}

% used for TeXing text within eps files
%\usepackage{psfrag}
% need this for including graphics (\includegraphics)
%\usepackage{graphicx}
% for neatly defining theorems and propositions
%
% making logically defined graphics
%%%\usepackage{xypic}

% there are many more packages, add them here as you need them

% define commands here

\newcommand{\sR}[0]{\mathbb{R}}
\newcommand{\sC}[0]{\mathbb{C}}
\newcommand{\sN}[0]{\mathbb{N}}
\newcommand{\sZ}[0]{\mathbb{Z}}

 \usepackage{bbm}
 \newcommand{\Z}{\mathbbmss{Z}}
 \newcommand{\C}{\mathbbmss{C}}
 \newcommand{\F}{\mathbbmss{F}}
 \newcommand{\R}{\mathbbmss{R}}
 \newcommand{\Q}{\mathbbmss{Q}}



\newcommand*{\norm}[1]{\lVert #1 \rVert}
\newcommand*{\abs}[1]{| #1 |}



\newtheorem{thm}{Theorem}
\newtheorem{defn}{Definition}
\newtheorem{prop}{Proposition}
\newtheorem{lemma}{Lemma}
\newtheorem{cor}{Corollary}
\begin{document}
Suppose $A$ is a non-empty subset of $\R$.\, If $A$ is bounded from above, then 
the axioms of the real numbers imply that there exists
a \emph{least upper bound} for  $A$. 
That is, there exists an $m\in \R$ such that
\begin{enumerate}
\item $m$ is an upper bound for $A$, that is, $a\le m$ for all $a\in A$, 
\item if $M$ is another upper bound for $A$, then $m\le M$.
\end{enumerate}
Such a number $m$ is called the \emph{supremum} of $A$,
and it is denoted by $\sup A$. It is easy to see that
there can be only one least upper bound. If $m_1$ and $m_2$ are 
two least upper bounds for $A$. Then $m_1\le m_2$ and $m_2\le m_1$, 
and $m_1=m_2$.

Next, let us consider a set $A$ that is bounded from below. That is, for
some $m\in \R$ we have $m\le a$ for all $a\in A$.\, Then we say that 
$M\in \R$ is a 
a \emph{greatest lower bound} for $A$ if
\begin{enumerate}
\item $M$ is an lower bound for $A$, that is, $M \le a$ for all $a\in A$, 
\item if $m$ is another lower bound for $A$, then $m\le M$.
\end{enumerate}
Such a number $M$ is called the \emph{infimum} of $A$,
and it is denoted by $\inf A$. Just as we proved that the supremum
is unique, one can also show that the infimum is unique. 
The next lemma shows that the infimum exists.

\begin{lemma} Every non-empty set bounded from below has a greatest lower bound. 
\end{lemma}

\begin{proof} Let $m\in \R$ be a lower bound for non-empty set $A$. 
In other words, $m\le a$ for all $a\in A$. Let
$$  
   -A = \{ -a \in \R : a\in A\}.
$$
Let us recall the following result from \PMlinkname{this page}{InequalityForRealNumbers};
if $m$ is an upper(lower) bound for $A$, then $-m$ is a lower(upper)
bound for $-A$. 

Thus $-A$ is bounded from above by $-m$. 
%In fact, if $b\in B$, then 
%$b=-a$ for some $a\in A$, so $-b=a\ge m$, and $b\le -m$.)
It follows that $-A$ has a  least upper bound $\sup (-A)$. 
Now $-\sup (-A)$ is a greatest lower bound for $A$. 
First, by the result, it is a lower bound for $A$. 
Second, if $m$ is a lower bound for $A$,
then $-m$ is a upper bound for $-A$, and 
$\sup (-A)\le -m$, or $m\ge -\sup (-A)$.
\end{proof}

The proof shows that if $A$ is non-empty and bounded from below, 
then 
$$
  \inf A = -\sup (-A).
$$
In consequence, if $A$ is bounded from above, 
then 
$$
  \sup A = -\inf (-A).
$$

In many respects, the supremum and infimum are similar to the maximum
and minimum, or the largest and smallest element in a set. 
However, it is important to notice that the $\inf A$ and $\sup A$
do not need to belong to $A$. (See examples below.)

\subsubsection*{Examples}
\begin{enumerate}

\item 
For example, consider the set of negative real numbers
$$
  A = \{ x\in \R:\,\, x<0\}.
$$
Then\, $\sup A = 0$. Indeed. First, $a < 0$ for all $a \in A$, 
and if $a < b$ for all $a \in A$, then $0 \le b$. 

\item The sequence \,\,
$-(1\!-\!\frac{1}{1}),\,1\!-\!\frac{1}{2},\,-(1\!-\!\frac{1}{3}),\, 1\!-\!\frac{1}{4},\,-(1\!-\!\frac{1}{5}),\,...$\,\,
is not convergent.\, The set\, $A = \{(-1)^n(1-\frac{1}{n}):\,\, n\in\mathbb{Z}_+\}$\, formed by its members has the infimum $-1$ and the supremum 1.

\end{enumerate}
%%%%%
%%%%%
\end{document}
