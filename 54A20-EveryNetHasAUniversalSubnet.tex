\documentclass[12pt]{article}
\usepackage{pmmeta}
\pmcanonicalname{EveryNetHasAUniversalSubnet}
\pmcreated{2013-03-22 17:25:16}
\pmmodified{2013-03-22 17:25:16}
\pmowner{asteroid}{17536}
\pmmodifier{asteroid}{17536}
\pmtitle{every net has a universal subnet}
\pmrecord{8}{39795}
\pmprivacy{1}
\pmauthor{asteroid}{17536}
\pmtype{Theorem}
\pmcomment{trigger rebuild}
\pmclassification{msc}{54A20}
\pmsynonym{Kelley's theorem}{EveryNetHasAUniversalSubnet}
%\pmkeywords{Kelley}
%\pmkeywords{universal net}
%\pmkeywords{ultranet}
\pmrelated{Ultranet}

\endmetadata

% this is the default PlanetMath preamble.  as your knowledge
% of TeX increases, you will probably want to edit this, but
% it should be fine as is for beginners.

% almost certainly you want these
\usepackage{amssymb}
\usepackage{amsmath}
\usepackage{amsfonts}

% used for TeXing text within eps files
%\usepackage{psfrag}
% need this for including graphics (\includegraphics)
%\usepackage{graphicx}
% for neatly defining theorems and propositions
%\usepackage{amsthm}
% making logically defined graphics
%%%\usepackage{xypic}

% there are many more packages, add them here as you need them

% define commands here

\begin{document}
{\bf Theorem - (Kelley's theorem) -} Let $X$ be a non-empty set. Every net $(x_{\alpha})_{\alpha \in \mathcal{A}}$ in $X$ has a \PMlinkname{universal subnet}{Ultranet}. That is, there is a subnet such that for every $E \subseteq X$ either the subnet is eventually in $E$ or eventually in $X-E$.

{\bf Proof :} Let $\mathcal{F}$ be a section filter for the net $(x_{\alpha})_{\alpha \in \mathcal{A}}$.

 Let $\mathcal{D}=\{(\alpha,U):\alpha \in \mathcal{A}\;,\; U \in \mathcal{F},\; x_{\alpha} \in U \}$. $\mathcal{D}$ is a directed set under the order relation given by
\begin{displaymath}
(\alpha,U) \leq (\beta,V) \Longleftrightarrow \begin{cases}  
  \alpha \leq \beta \\
  V \subseteq U
\end{cases}
\end{displaymath}

The map $f:\mathcal{D} \longrightarrow \mathcal{A}$ defined by $f(\alpha,U):=\alpha$ is order preserving and cofinal. Therefore there is a subnet $(y_{(\alpha,U)})_{(\alpha,U) \in \mathcal{D}}$ of $(x_{\alpha})_{\alpha \in \mathcal{A}}$ associated with the map $f$ (that is, $y_{(\alpha,U)} = x_{\alpha}$).

We now prove that $(y_{(\alpha,U)})_{(\alpha,U) \in \mathcal{D}}$ is a \PMlinkescapetext{universal} net.

Let $E \subseteq X$. We have that $(y_{(\alpha,U)})_{(\alpha,U) \in \mathcal{D}}$ is frequently in $E$ or frequently in $X-E$.

Suppose $(y_{(\alpha,U)})_{(\alpha,U) \in \mathcal{D}}$ is frequently in $E$.

Let $A \in \mathcal{F}$ and $S(\alpha):=\{x_{\beta}:\alpha \leq \beta\}$. We have that $S(\alpha) \in \mathcal{F}$ by definition of section filter.

As $\mathcal{F}$ is a filter, $A \cap S(\alpha) \neq \emptyset$ and so there exists $\beta$ with $\alpha \leq \beta$ such that $x_{\beta} \in A$. Hence, $(\beta,A) \in \mathcal{D}$.

As $(y_{(\alpha,U)})_{(\alpha,U) \in \mathcal{D}}$ is frequently in $E$, there exists $(\gamma,B) \in \mathcal{D}$ with $(\beta,A) \leq (\gamma,B)$ such that $y_{(\gamma,B)} \in E$.

Also, $y_{(\gamma,B)}$ is in $B$, and therefore, in $A$. So $A \cap E \neq \emptyset$.

We conclude that $E \cap A \neq \emptyset$ for every $A \in \mathcal{F}$. Therefore, $\mathcal{F} \cup \{E\}$ \PMlinkescapetext{generates} a filter in $X$. As $\mathcal{F}$ is a maximal filter we conclude that $E \in \mathcal{F}$, and consequently, $(\gamma,E) \in \mathcal{D}$.

We can now see that for every $(\delta,C)$ with $(\gamma,E) \leq (\delta,C)$, $y_{(\delta,C)}$ is in $C$ and so is in $E$. Therefore, $(y_{(\alpha,U)})_{(\alpha,U) \in \mathcal{D}}$ is eventually in $E$.

\emph{Remark:} If $(y_{(\alpha,U)})_{(\alpha,U) \in \mathcal{D}}$ is frequently in $X-E$, by an analogous \PMlinkescapetext{argument} we can conclude that it is eventually in $X-E$.

This proves that $(y_{(\alpha,U)})_{(\alpha,U) \in \mathcal{D}}$ is a \PMlinkescapetext{universal} subnet of $(x_{\alpha})_{\alpha \in \mathcal{A}}$. $\square$
%%%%%
%%%%%
\end{document}
