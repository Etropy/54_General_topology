\documentclass[12pt]{article}
\usepackage{pmmeta}
\pmcanonicalname{Ultranet}
\pmcreated{2013-03-22 12:54:35}
\pmmodified{2013-03-22 12:54:35}
\pmowner{asteroid}{17536}
\pmmodifier{asteroid}{17536}
\pmtitle{ultranet}
\pmrecord{9}{33260}
\pmprivacy{1}
\pmauthor{asteroid}{17536}
\pmtype{Definition}
\pmcomment{trigger rebuild}
\pmclassification{msc}{54A20}
\pmsynonym{universal net}{Ultranet}
%\pmkeywords{topology}
\pmrelated{Ultrafilter}
\pmrelated{EveryNetHasAUniversalSubnet}

\endmetadata

% this is the default PlanetMath preamble.  as your knowledge
% of TeX increases, you will probably want to edit this, but
% it should be fine as is for beginners.

% almost certainly you want these
\usepackage{amssymb}
\usepackage{amsmath}
\usepackage{amsfonts}

% used for TeXing text within eps files
%\usepackage{psfrag}
% need this for including graphics (\includegraphics)
%\usepackage{graphicx}
% for neatly defining theorems and propositions
%\usepackage{amsthm}
% making logically defined graphics
%%%\usepackage{xypic} 

% there are many more packages, add them here as you need them

% define commands here
\begin{document}
A net $(x_a)_{a\in A}$ on a set $X$ is said to be an {\bf ultranet} or {\bf universal net} if whenever $E\subseteq X$, $(x_a)$ is either eventually in $E$ or eventually in $X\smallsetminus E$.


{\bf \PMlinkescapetext{Properties}:}
\begin{itemize}
\item It can be shown that every net has a universal subnet.
\item When $X$ is a locally compact topological space, a universal net in $X$ is either convergent or it ``goes to \PMlinkescapetext{infinity}'' (it eventually leaves every compact subset).
\end{itemize}
%%%%%
%%%%%
\end{document}
