\documentclass[12pt]{article}
\usepackage{pmmeta}
\pmcanonicalname{Ultrametric}
\pmcreated{2013-03-22 13:28:28}
\pmmodified{2013-03-22 13:28:28}
\pmowner{Koro}{127}
\pmmodifier{Koro}{127}
\pmtitle{ultrametric}
\pmrecord{21}{34044}
\pmprivacy{1}
\pmauthor{Koro}{127}
\pmtype{Definition}
\pmcomment{trigger rebuild}
\pmclassification{msc}{54E35}
\pmrelated{MetricSpace}
\pmrelated{Valuation}
\pmrelated{UltrametricSpace}

\endmetadata

% this is the default PlanetMath preamble.  as your knowledge
% of TeX increases, you will probably want to edit this, but
% it should be fine as is for beginners.

% almost certainly you want these
\usepackage{amssymb}
\usepackage{amsmath}
\usepackage{amsfonts}

% used for TeXing text within eps files
%\usepackage{psfrag}
% need this for including graphics (\includegraphics)
%\usepackage{graphicx}
% for neatly defining theorems and propositions
%\usepackage{amsthm}
% making logically defined graphics
%%%\usepackage{xypic}

% there are many more packages, add them here as you need them

% define commands here
\begin{document}
Any metric $d: X \times X \to \mathbb{R}$ on a set $X$ must satisfy the triangle inequality:

\begin{equation*}
(\forall x,y,z) \quad d(x,z) \leq d(x,y) + d(y,z)
\end{equation*}

An {\em ultrametric} must additionally satisfy a stronger version of the triangle inequality:


\begin{equation*}
(\forall x,y,z) \quad d(x,z) \leq \max\{ d(x,y),  d(y,z) \}
\end{equation*}

Here is an example of an ultrametric on a space with 5 points, labelled $a,b,c,d,e$:

\begin{equation*}
\begin{array}{c|c|c|c|c|c}
   & a  & b  & c  & d  & e
\\ \hline
 a & 0  & 12 & 4  & 6  & 12
\\ \hline
 b &    & 0  & 12 & 12 & 5
\\ \hline
 c &    &    & 0  & 6  & 12
\\ \hline
 d &    &    &    & 0  & 12
\\ \hline
 e &    &    &    &    & 0
\end{array}
\end{equation*}

In the table above, an entry $n$ in the \PMlinkescapetext{row} for element $x$ and the \PMlinkescapetext{column} for element $y$ indicates that $d(x,y)=n$, where $d$ is the ultrametric.  By symmetry of the ultrametric ($d(x,y)=d(y,x)$), the above table yields all values of $d(x,y)$ for all $x,y \in \{a,b,c,d,e\}$.

The ultrametric condition is equivalent to the ultrametric three point condition:

\begin{equation*}
(\forall x,y,z) \quad x,y,z \textrm{ can be renamed such that } d(x,z) \leq d(x,y) = d(y,z)
\end{equation*}

Ultrametrics can be used to model bifurcating hierarchical systems.\, The distance between nodes in a weight-balanced binary tree is an ultrametric. Similarly, an ultrametric can be modelled by a weight-balanced binary tree, although the choice of tree is not necessarily unique.\, Tree models of ultrametrics are sometimes called \emph{ultrametric trees}.

The metrics induced by non-Archimedean valuations are ultrametrics.

\PMlinkescapeword{induced}
%%%%%
%%%%%
\end{document}
