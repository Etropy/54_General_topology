\documentclass[12pt]{article}
\usepackage{pmmeta}
\pmcanonicalname{MetricSpacesAreHausdorff}
\pmcreated{2013-03-22 14:21:29}
\pmmodified{2013-03-22 14:21:29}
\pmowner{waj}{4416}
\pmmodifier{waj}{4416}
\pmtitle{metric spaces are Hausdorff}
\pmrecord{4}{35838}
\pmprivacy{1}
\pmauthor{waj}{4416}
\pmtype{Proof}
\pmcomment{trigger rebuild}
\pmclassification{msc}{54D10}
\pmclassification{msc}{54E35}
%\pmkeywords{teaching proofs}
\pmrelated{MetricSpace}
\pmrelated{SeparationAxioms}

% this is the default PlanetMath preamble.  as your knowledge
% of TeX increases, you will probably want to edit this, but
% it should be fine as is for beginners.

% almost certainly you want these
\usepackage{amssymb}
\usepackage{amsmath}
\usepackage{amsfonts}

% used for TeXing text within eps files
%\usepackage{psfrag}
% need this for including graphics (\includegraphics)
%\usepackage{graphicx}
% for neatly defining theorems and propositions
\usepackage{amsthm}
% making logically defined graphics
%%%\usepackage{xypic}

% there are many more packages, add them here as you need them

% define commands here
\def\co{\colon\thinspace}
\theoremstyle{definition}
\newtheorem*{thm}{Theorem}
\begin{document}
Suppose we have a space $X$ and a metric $d$ on $X$.  We'd like to show that the metric topology that $d$ gives $X$ is Hausdorff.

Say we've got distinct $x,y\in X$.  Since $d$ is a metric, $d(x,y)\neq 0$.  Then the open balls $B_x = B(x,\frac{d(x,y)}{2})$ and $B_y = B(y, \frac{d(x,y)}{2})$ are open sets in the metric topology which contain $x$ and $y$ respectively.  If we could show $B_x$ and $B_y$ are disjoint, we'd have shown that $X$ is Hausdorff.

We'd like to show that an arbitrary point $z$ can't be in both $B_x$ and $B_y$.  Suppose there is a $z$ in both, and we'll derive a contradiction.  Since $z$ is in these open balls, $d(z,x) < \frac{d(x,y)}{2}$ and $d(z,y) < \frac{d(x,y)}{2}$.  But then $d(z,x) + d(z,y) < d(x,y)$, contradicting the triangle inequality.

So $B_x$ and $B_y$ are disjoint, and $X$ is Hausdorff.$\square$
%%%%%
%%%%%
\end{document}
