\documentclass[12pt]{article}
\usepackage{pmmeta}
\pmcanonicalname{BackgroundOfSeveralOpenProblemsIStumbledUpon}
\pmcreated{2013-04-21 12:18:16}
\pmmodified{2013-04-21 12:18:16}
\pmowner{porton}{9363}
\pmmodifier{porton}{9363}
\pmtitle{Background of several open problems I stumbled upon}
\pmrecord{14}{87296}
\pmprivacy{1}
\pmauthor{porton}{9363}
\pmtype{Topic}
\pmcomment{See the right sidebar of this page for the list of problems.}
\pmclassification{msc}{54J05}
\pmclassification{msc}{54A05}
\pmclassification{msc}{54D99}
\pmclassification{msc}{54E05}
\pmclassification{msc}{54E17}
\pmclassification{msc}{54E99}

\endmetadata

% this is the default PlanetMath preamble.  as your knowledge
% of TeX increases, you will probably want to edit this, but
% it should be fine as is for beginners.

% almost certainly you want these
\usepackage{amssymb}
\usepackage{amsmath}
\usepackage{amsfonts}

% need this for including graphics (\includegraphics)
\usepackage{graphicx}
% for neatly defining theorems and propositions
\usepackage{amsthm}

% making logically defined graphics
%\usepackage{xypic}
% used for TeXing text within eps files
%\usepackage{psfrag}

% there are many more packages, add them here as you need them

% define commands here
\newtheorem{theorem}{Theorem}
\begin{document}
This lists some problem aroused while I wrote my book ``Algebraic General Topology. Volume 1''.

The draft of the book is now \href{http://www.mathematics21.org/algebraic-general-topology.html}{available online}.

Please first read the book and then collaborate with me in solving the below listed problems.

Theorems to prove:
\begin{itemize}
\item Intersection with a staroid product through its upper sets.
\item Intersecting two staroidal products.
\item Multifuncoid has atomic arguments.
\item Every regular paratopological group is Tychonoff.
\end{itemize}

More related materials:
\begin{itemize}
\item Direct products in a category of funcoids.
\item Compact funcoids.
\item Meta-singular numbers.
\end{itemize}

See the right sidebar of this page ("Attached articles") for the list of problems.

TODO: More information should be put to this page.

See also \href{http://researchtrends.wikia.com/wiki/Cartesian_closed_categories_containing_Top_and_Prox_as_subcategories}{this page}. Depsite of the fact that this page is written by me, I can't remember whether that page notation is compatible with the notation in my book. Please analyse it and make its copy on PlanetMath.
\end{document}
