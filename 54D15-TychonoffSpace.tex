\documentclass[12pt]{article}
\usepackage{pmmeta}
\pmcanonicalname{TychonoffSpace}
\pmcreated{2013-03-22 12:12:42}
\pmmodified{2013-03-22 12:12:42}
\pmowner{yark}{2760}
\pmmodifier{yark}{2760}
\pmtitle{Tychonoff space}
\pmrecord{11}{31534}
\pmprivacy{1}
\pmauthor{yark}{2760}
\pmtype{Definition}
\pmcomment{trigger rebuild}
\pmclassification{msc}{54D15}
\pmsynonym{Tikhonov space}{TychonoffSpace}
\pmsynonym{Tychonoff topological space}{TychonoffSpace}
\pmsynonym{Tikhonov topological space}{TychonoffSpace}
\pmsynonym{Tychonov space}{TychonoffSpace}
\pmsynonym{Tychonov topological space}{TychonoffSpace}
%\pmkeywords{topology}
\pmrelated{NormalTopologicalSpace}
\pmrelated{T3Space}
\pmdefines{Tychonoff}
\pmdefines{completely regular}
\pmdefines{completely regular space}
\pmdefines{Tikhonov}
\pmdefines{Tychonov}

\usepackage{amssymb}
\usepackage{amsmath}
\usepackage{amsfonts}
\begin{document}
\PMlinkescapeword{Hausdorff}

A topological space $X$ is said to be \emph{completely regular} if whenever $C\subseteq X$ is closed and $x\in X\setminus C$ then there is a continuous function $f\colon X\to[0,1]$ with $f(x)=0$ and $f(C)\subseteq\{1\}$.

A completely regular space that is also \PMlinkname{$T_0$}{T0Space} (and therefore \PMlinkname{Hausdorff}{T2Space})
is called a \emph{Tychonoff space}, or a \emph{$T_{3\frac{1}{2}}$ space}.

Some authors interchange the meanings of `completely regular' and `$T_{3\frac{1}{2}}$' compared to the above.

It can be proved that a topological space is Tychonoff
if and only if it has a Hausdorff compactification.

%%%%%
%%%%%
%%%%%
\end{document}
