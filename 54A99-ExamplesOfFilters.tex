\documentclass[12pt]{article}
\usepackage{pmmeta}
\pmcanonicalname{ExamplesOfFilters}
\pmcreated{2013-03-22 12:54:31}
\pmmodified{2013-03-22 12:54:31}
\pmowner{Evandar}{27}
\pmmodifier{Evandar}{27}
\pmtitle{examples of filters}
\pmrecord{8}{33259}
\pmprivacy{1}
\pmauthor{Evandar}{27}
\pmtype{Example}
\pmcomment{trigger rebuild}
\pmclassification{msc}{54A99}
\pmclassification{msc}{03E99}
\pmdefines{cofinite filter}
\pmdefines{Fr\'echet filter}

\endmetadata

% this is the default PlanetMath preamble.  as your knowledge
% of TeX increases, you will probably want to edit this, but
% it should be fine as is for beginners.

% almost certainly you want these
\usepackage{amssymb}
\usepackage{amsmath}
\usepackage{amsfonts}

% used for TeXing text within eps files
%\usepackage{psfrag}
% need this for including graphics (\includegraphics)
%\usepackage{graphicx}
% for neatly defining theorems and propositions
%\usepackage{amsthm}
% making logically defined graphics
%%%\usepackage{xypic} 

% there are many more packages, add them here as you need them

% define commands here
\begin{document}
\begin{itemize}
\item
If $X$ is any set and $A\subseteq X$ then $\mathcal{F} = \{ F\subseteq X\colon A\subseteq F\}$ is a fixed filter on $X$; $\mathcal{F}$ is an ultrafilter iff $A$ consists of a single point.
\item
If $X$ is any infinite set, then $\{ F\subseteq X\colon X\smallsetminus F \mbox{is finite }\}$ is a free filter on $X$, called the \emph{cofinite filter}.
\item
The filter on $\mathbb{R}$ generated by the filter base $\{ (n,\infty)\colon n\in\mathbb{N}\}$ is called the \emph{Fr\'echet filter} on $\mathbb{R}$; it is a free filter which does not converge or have any accumulation points.
\item
The filter on $\mathbb{R}$ generated by the filter base $\{ (0,\varepsilon )\colon\varepsilon >0\}$ is a free filter on $\mathbb{R}$ which converges to $0$.
\end{itemize}
%%%%%
%%%%%
\end{document}
