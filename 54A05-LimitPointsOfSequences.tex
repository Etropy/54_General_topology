\documentclass[12pt]{article}
\usepackage{pmmeta}
\pmcanonicalname{LimitPointsOfSequences}
\pmcreated{2013-03-22 14:38:13}
\pmmodified{2013-03-22 14:38:13}
\pmowner{rspuzio}{6075}
\pmmodifier{rspuzio}{6075}
\pmtitle{limit points of sequences}
\pmrecord{7}{36220}
\pmprivacy{1}
\pmauthor{rspuzio}{6075}
\pmtype{Definition}
\pmcomment{trigger rebuild}
\pmclassification{msc}{54A05}
\pmdefines{limit point of a sequence}
\pmdefines{limit point of the sequence}
\pmdefines{accumulation point of a sequence}
\pmdefines{accumulation point of the sequence}

\endmetadata

% this is the default PlanetMath preamble.  as your knowledge
% of TeX increases, you will probably want to edit this, but
% it should be fine as is for beginners.

% almost certainly you want these
\usepackage{amssymb}
\usepackage{amsmath}
\usepackage{amsfonts}

% used for TeXing text within eps files
%\usepackage{psfrag}
% need this for including graphics (\includegraphics)
%\usepackage{graphicx}
% for neatly defining theorems and propositions
%\usepackage{amsthm}
% making logically defined graphics
%%%\usepackage{xypic}

% there are many more packages, add them here as you need them

% define commands here
\begin{document}
In a topological space $X$, a point $x$ is a \emph{limit point of the sequence} $x_0, x_1, \ldots$ if, for every open set containing $x$, there are finitely many indices such that the corresponding elements of the sequence do not belong to the open set. 

A point $x$ is an \emph{accumulation point of the sequence} $x_0, x_1, \ldots$ if, for every open set containing $x$, there are infinitely many indices such that the corresponding elements of the sequence belong to the open set.

It is worth noting that the set of limit points of a sequence can differ from the set of limit points of the set of elements of the sequence.  Likewise the set of accumulation points of a sequence can differ from the set of accumulation points of the set of elements of the sequence.

Reference: L. A. Steen and J. A. Seebach, Jr.  ``Counterxamples in Topology'' Dover Publishing 1970
%%%%%
%%%%%
\end{document}
