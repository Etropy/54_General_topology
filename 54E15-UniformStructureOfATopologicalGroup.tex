\documentclass[12pt]{article}
\usepackage{pmmeta}
\pmcanonicalname{UniformStructureOfATopologicalGroup}
\pmcreated{2013-03-22 12:47:21}
\pmmodified{2013-03-22 12:47:21}
\pmowner{mps}{409}
\pmmodifier{mps}{409}
\pmtitle{uniform structure of a topological group}
\pmrecord{10}{33104}
\pmprivacy{1}
\pmauthor{mps}{409}
\pmtype{Derivation}
\pmcomment{trigger rebuild}
\pmclassification{msc}{54E15}
\pmdefines{right uniformity}
\pmdefines{left uniformity}

% this is the default PlanetMath preamble.  as your knowledge
% of TeX increases, you will probably want to edit this, but
% it should be fine as is for beginners.

% almost certainly you want these
\usepackage{amssymb}
\usepackage{amsmath}
\usepackage{amsfonts}

% used for TeXing text within eps files
%\usepackage{psfrag}
% need this for including graphics (\includegraphics)
%\usepackage{graphicx}
% for neatly defining theorems and propositions
%\usepackage{amsthm}
% making logically defined graphics
%%%\usepackage{xypic}

% there are many more packages, add them here as you need them

% define commands here
\begin{document}
Let $G$ be a topological group. There is a natural uniform structure on $G$ which induces its topology. We define a subset $V$ of the Cartesian product $G \times G$ to be an entourage if and only if it contains a subset of the form
\[ V_N = \{ (x,y) \in G \times G : xy^{-1} \in N \} \]
for some $N$ neighborhood of the identity element. This is called the \emph{right uniformity} of the topological group, with which \PMlinkescapetext{right} multiplication becomes a uniformly continuous map.
The \emph{left uniformity} is defined in a \PMlinkescapetext{similar} fashion, but in general they don't coincide, although they both induce the same topology on $G$.
%%%%%
%%%%%
\end{document}
