\documentclass[12pt]{article}
\usepackage{pmmeta}
\pmcanonicalname{OperatorTopologies}
\pmcreated{2013-03-22 17:22:04}
\pmmodified{2013-03-22 17:22:04}
\pmowner{asteroid}{17536}
\pmmodifier{asteroid}{17536}
\pmtitle{operator topologies}
\pmrecord{18}{39729}
\pmprivacy{1}
\pmauthor{asteroid}{17536}
\pmtype{Definition}
\pmcomment{trigger rebuild}
\pmclassification{msc}{54E99}
\pmclassification{msc}{47L05}
\pmclassification{msc}{46A32}
\pmrelated{OperatorNorm}
\pmdefines{strong operator topology}
\pmdefines{weak operator topology}
\pmdefines{$\sigma$-weak operator topology}
\pmdefines{$\sigma$-strong operator topology}
\pmdefines{strong-* operator topology}
\pmdefines{$\sigma$-strong-* operator topology}
\pmdefines{ultra-strong operator topology}
\pmdefines{ultra-weak operator topology}
\pmdefines{ultra-stro}

% this is the default PlanetMath preamble.  as your knowledge
% of TeX increases, you will probably want to edit this, but
% it should be fine as is for beginners.

% almost certainly you want these
\usepackage{amssymb}
\usepackage{amsmath}
\usepackage{amsfonts}

% used for TeXing text within eps files
%\usepackage{psfrag}
% need this for including graphics (\includegraphics)
%\usepackage{graphicx}
% for neatly defining theorems and propositions
%\usepackage{amsthm}
% making logically defined graphics
%%%\usepackage{xypic}

% there are many more packages, add them here as you need them
%%\usepackage{xypic}
% define commands here

\begin{document}
\PMlinkescapeword{norm}

Let $X$ be a normed vector space and $B(X)$ the space of bounded operators in $X$.  There are several interesting topologies that can be given to $B(X)$.  In what follows, $T_{\alpha}$ denotes a net in $B(X)$ and $T$ denotes an element of $B(X)$.

Note: On 4, 5, 6 and 7, $X$ must be a Hilbert space.

\subsection{1. Norm Topology}
This is the topology induced by the usual operator norm.\\

\begin{displaymath}
 T_{\alpha} \longrightarrow T \text{\emph{in the norm topology}}\;\; \Longleftrightarrow \; \|T_{\alpha} - T \|
 \longrightarrow 0
\end{displaymath}

\subsection{2. Strong Operator Topology}
 This is the topology generated by the family of semi-norms $\|
 \cdot \|_{x}\;, x \in X$ defined by $\|T \|_{x} := \|Tx \|$. That means

\begin{displaymath}
T_{\alpha} \longrightarrow T \text{\emph{in the strong operator topology}}\;\; \Longleftrightarrow\; \|(T_{\alpha}-
 T )x\| \longrightarrow 0 \quad, \forall x \in X
\end{displaymath}

\subsection{3. Weak Operator Topology}
 This is the topology generated by the family of semi-norms
 $\| \cdot \|_{f,x}\;$, where $x \in X$ and $f$ is a linear functional of $X$ (written $f\in X^*$, the dual vector space  of $X$), defined by $\| T \|_{f,x} := |f(Tx)|$. That means

\begin{displaymath}
T_{\alpha} \longrightarrow T \text{\emph{in the weak operator topology}}\;\; \Longleftrightarrow\; \|f((T_{\alpha} -T)x) \| \longrightarrow 0 \quad, \forall x \in X ,\; \forall f\in X^*
\end{displaymath}

$\,$

In case $X$ is an Hilbert space with inner product $\langle \cdot, \cdot \rangle$, we have that
\begin{displaymath}
T_{\alpha} \longrightarrow T \text{\emph{in the weak operator topology}}\;\; \Longleftrightarrow
 \; |\langle (T_{\alpha} - T)x, y \rangle | \longrightarrow 0 \quad, \forall x, y \in X
\end{displaymath}

\subsection{4. $\sigma$-Strong Operator Topology}
In this topology $X$ must be a Hilbert space. Let $K(X)$ denote the space of compact operators on $X$.

The \emph{$\sigma$-strong operator topology} is the topology generated by the family of semi-norms $\|\cdot\|_S\;, S \in K(X)$, defined by $\|T\|_S := \|TS\|$. That means

\begin{displaymath}
 T_{\alpha} \longrightarrow T \text{\emph{in the $\sigma$-strong operator topology}}\;\; \Longleftrightarrow\; \|(T_{\alpha}-T)S\| \longrightarrow 0 \quad, \forall S \in K(X)
\end{displaymath}

$\,$

Equivalently, $T_{\alpha} \longrightarrow T\;\; \Longleftrightarrow \; T_{\alpha}S \longrightarrow TS$ in norm for every $S \in K(X)$.

This topology is also called the \emph{ultra-strong operator topology}.

\subsection{5. $\sigma$-Weak Operator Topology}
In this topology $X$ must be a Hilbert space. Let $B(X)_*$ denote the space of trace-class operators on $X$ and $Tr(S)$ the trace of an operator $S \in B(X)_*$.

The \emph{$\sigma$-weak operator topology} is the topology generated by the family of semi-norms $\{\omega_{S} : S \in B(X)_*\}$ defined by $\omega_{S}(T) := |Tr(TS)|$. That means

\begin{displaymath}
T_{\alpha} \longrightarrow T \text{\emph{in the $\sigma$-weak operator topology}}\;\; \Longleftrightarrow\; |Tr[(T_{\alpha}-T)S]| \longrightarrow 0 \quad, \forall S \in B(X)_*
\end{displaymath}

This topology is also called the \emph{ultra-weak operator topology}.


\subsection{6. Strong-* Operator Topology}
In this topology $X$ must be a Hilbert space. In the following $T^*$ denotes the adjoint operator of $T$.

The \emph{strong-* operator topology} is the topology generated by the family of semi-norms $\|
 \cdot \|_{x}\;, x \in X$ defined by $\|T \|_{x} := \|Tx \|+\|T^*x\|$. That means

\begin{displaymath}
T_{\alpha} \longrightarrow T \text{\emph{in the strong-* operator topology}}\;\; \Longleftrightarrow\; \|(T_{\alpha}-
 T )x\|+\|(T_{\alpha}^*-
 T^* )x\| \longrightarrow 0 \quad, \forall x \in X
\end{displaymath}

Equivalently, $T_{\alpha} \longrightarrow T$ if and only if $T_{\alpha}x \longrightarrow Tx$ and $T_{\alpha}^*x \longrightarrow T^*x$, for every $x \in X$.

\subsection{7. $\sigma$-Strong-* Operator Topology}
In this topology $X$ must be a Hilbert space. Let $K(X)$ denote the space of compact operators on $X$. In the following $T^*$ denotes the adjoint operator of $T$.

The \emph{$\sigma$-strong-* operator topology} is the topology generated by the family of semi-norms $\|
 \cdot \|_S\;, S \in K(X)$ defined by $\|T \|_S := \|TS \|+\|T^*S\|$. That means

\begin{displaymath}
T_{\alpha} \longrightarrow T \text{\emph{in the $\sigma$-strong-* operator topology}}\;\; \Longleftrightarrow\; \|(T_{\alpha}-
 T)S\|+\|(T_{\alpha}^*-
 T^*)S\| \longrightarrow 0 \quad, \forall S \in K(X)
\end{displaymath}

$\,$

Equivalently, $T_{\alpha} \longrightarrow T$ if and only if $T_{\alpha}S \longrightarrow TS$ and $T_{\alpha}^*S \longrightarrow T^*S$ in norm, for every $S \in K(X)$.

This topology is also called \emph{ultra-strong-* operator topology}.


\subsection{Comparison of Operator Topologies}
\begin{itemize}
\item The norm topology is the strongest of the topologies defined above.
\item The weak operator topology is weaker than the strong operator topology, which is weaker than the norm topology.
\item In Hilbert spaces we can summarize the relations of the above topologies in the following diagram. Given two topologies $\mathcal{U},\mathcal{V}$ the notation $\mathcal{U} \rightarrow \mathcal{V}$ means $\mathcal{U}$ is weaker than $\mathcal{V}$:
\begin{displaymath}
\xymatrix{ weak  \ar[r] \ar[d] & strong \ar[r] \ar[d] & \emph{strong-*} \ar[d] \\
\emph{$\sigma$-weak} \ar[r] & \emph{$\sigma$-strong} \ar[r] & \emph{$\sigma$-strong-*} \ar[r] & Norm}
\end{displaymath}
\end{itemize}
%%%%%
%%%%%
\end{document}
