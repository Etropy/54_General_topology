\documentclass[12pt]{article}
\usepackage{pmmeta}
\pmcanonicalname{SecondCountable}
\pmcreated{2013-03-22 12:05:06}
\pmmodified{2013-03-22 12:05:06}
\pmowner{rspuzio}{6075}
\pmmodifier{rspuzio}{6075}
\pmtitle{second countable}
\pmrecord{17}{31162}
\pmprivacy{1}
\pmauthor{rspuzio}{6075}
\pmtype{Definition}
\pmcomment{trigger rebuild}
\pmclassification{msc}{54D70}
\pmsynonym{second axiom of countability}{SecondCountable}
\pmsynonym{completely separable}{SecondCountable}
\pmsynonym{perfectly separable}{SecondCountable}
\pmsynonym{second-countable}{SecondCountable}
\pmrelated{Separable}
\pmrelated{Lindelof}
\pmrelated{EverySecondCountableSpaceIsSeparable}
\pmrelated{LindelofTheorem}
\pmrelated{UrysohnMetrizationTheorem}
\pmrelated{FirstAxiomOfCountability}
\pmrelated{LocallyCompactGroupoids}
\pmrelated{FirstCountable}

\endmetadata

\usepackage{amssymb}
\usepackage{amsmath}
\usepackage{amsfonts}
\usepackage{graphicx}
%%%\usepackage{xypic}
\begin{document}
A topological space is said to be \emph{second \PMlinkescapetext{countable}} if it has a countable \PMlinkname{basis}{BasisTopologicalSpace}. 
It can be shown that a \PMlinkescapetext{second countable} space is both Lindel\"of and separable, although the converses fail.  For instance, the lower limit topology on the real line is both Lindel\"of and separable, but not second countable.
%%%%%
%%%%%
%%%%%
\end{document}
