\documentclass[12pt]{article}
\usepackage{pmmeta}
\pmcanonicalname{Bounded1}
\pmcreated{2013-03-22 14:00:00}
\pmmodified{2013-03-22 14:00:00}
\pmowner{yark}{2760}
\pmmodifier{yark}{2760}
\pmtitle{bounded}
\pmrecord{11}{34826}
\pmprivacy{1}
\pmauthor{yark}{2760}
\pmtype{Definition}
\pmcomment{trigger rebuild}
\pmclassification{msc}{54E35}
\pmrelated{EuclideanDistance}
\pmrelated{MetricSpace}
\pmdefines{bounded interval}

\usepackage{amssymb}
\usepackage{amsmath}
\usepackage{amsfonts}

\newcommand{\Z}{\mathbb{Z}}
\newcommand{\C}{\mathbb{C}}
\newcommand{\R}{\mathbb{R}}
\newcommand{\Q}{\mathbb{Q}}

\begin{document}
\PMlinkescapeword{between}
\PMlinkescapeword{equivalent}

Let $X$ be a subset of $\R$. We say that $X$ is bounded when there exists a real number $M$ such that $|x|<M$ for all $x\in X$. When $X$ is an interval, we speak of a bounded interval.

This can be generalized first to $\R^n$. We say  that $X\subseteq \R^n$ is bounded if there is a real number $M$ such that $\Vert x\Vert<M$ for all $x\in X$ and $\Vert\cdot\Vert$ is the Euclidean distance between $x$ and $y$.

This condition is equivalent to the statement: There is a real number $T$ such that $\Vert x-y\Vert<T$ for all $x,y\in X$.

A further generalization to any metric space $V$ says that $X\subseteq V$ is bounded when there is a real number $M$ such that $d(x,y)<M$ for all $x,y\in X$, where $d$ is the metric on $V$.
%%%%%
%%%%%
\end{document}
