\documentclass[12pt]{article}
\usepackage{pmmeta}
\pmcanonicalname{Ultrafilter}
\pmcreated{2013-03-22 12:13:59}
\pmmodified{2013-03-22 12:13:59}
\pmowner{yark}{2760}
\pmmodifier{yark}{2760}
\pmtitle{ultrafilter}
\pmrecord{11}{31611}
\pmprivacy{1}
\pmauthor{yark}{2760}
\pmtype{Definition}
\pmcomment{trigger rebuild}
\pmclassification{msc}{54A20}
\pmrelated{Filter}
\pmrelated{Ultranet}
\pmrelated{EveryBoundedSequenceHasLimitAlongAnUltrafilter}
\pmrelated{LatticeFilter}
\pmdefines{fixed ultrafilter}
\pmdefines{principal ultrafilter}
\pmdefines{trivial ultrafilter}
\pmdefines{free ultrafilter}
\pmdefines{non-principal ultrafilter}
\pmdefines{nonprincipal ultrafilter}
\pmdefines{uniform ultrafilter}

\endmetadata

\usepackage{amssymb}
\usepackage{amsmath}
\usepackage{amsfonts}

\begin{document}
\PMlinkescapeword{fixed}
\PMlinkescapeword{uniform}

Let $X$ be a set.

\section*{Definitions}

A collection $\mathcal{U}$ of subsets of $X$ is an \emph{ultrafilter} if $\mathcal{U}$ is a filter, and whenever $A\subseteq X$ then either $A\in\mathcal{U}$ or $X\setminus A\in\mathcal{U}$.

Equivalently, an ultrafilter on $X$ is a \PMlinkname{maximal}{MaximalElement} filter on $X$.

More generally, an \PMlinkname{ultrafilter of a lattice}{LatticeFilter}
is a maximal proper filter of the lattice.
This is indeed a generalization, as an ultrafilter on $X$
can then be defined as an ultrafilter of the power set $\mathcal{P}(X)$.

\section*{Types of ultrafilter}

For any $x\in X$ the set $\{A\subseteq X\mid x\in A\}$ is an ultrafilter on $X$.
An ultrafilter formed in this way is called a \emph{fixed ultrafilter},
or a \emph{principal ultrafilter}, or a \emph{trivial ultrafilter}.
Any other ultrafilter on $X$ is called a \emph{free ultrafilter},
or a \emph{non-principal ultrafilter}.
An ultrafilter on a finite set is necessarily fixed.
On any infinite set there are free ultrafilters
(\PMlinkname{in great abundance}{NumberOfUltrafilters}),
but their existence depends on the Axiom of Choice,
and so none can be explicitly constructed.

An ultrafilter $\mathcal{U}$ on $X$ is called a \emph{uniform ultrafilter}
if every member of $\mathcal{U}$ has the same cardinality.
(An ultrafilter on a singleton is uniform,
but this is a degenerate case and is often excluded.
All other uniform ultrafilters are free.)
%%%%%
%%%%%
%%%%%
\end{document}
