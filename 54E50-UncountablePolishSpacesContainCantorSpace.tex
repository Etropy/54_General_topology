\documentclass[12pt]{article}
\usepackage{pmmeta}
\pmcanonicalname{UncountablePolishSpacesContainCantorSpace}
\pmcreated{2013-03-22 18:48:33}
\pmmodified{2013-03-22 18:48:33}
\pmowner{gel}{22282}
\pmmodifier{gel}{22282}
\pmtitle{uncountable Polish spaces contain Cantor space}
\pmrecord{6}{41611}
\pmprivacy{1}
\pmauthor{gel}{22282}
\pmtype{Theorem}
\pmcomment{trigger rebuild}
\pmclassification{msc}{54E50}
%\pmkeywords{Polish space}
%\pmkeywords{Cantor space}
\pmrelated{PolishSpace}

\endmetadata

% almost certainly you want these
\usepackage{amssymb}
\usepackage{amsmath}
\usepackage{amsfonts}

% used for TeXing text within eps files
%\usepackage{psfrag}
% need this for including graphics (\includegraphics)
%\usepackage{graphicx}
% for neatly defining theorems and propositions
\usepackage{amsthm}
% making logically defined graphics
%%%\usepackage{xypic}

% there are many more packages, add them here as you need them

% define commands here
\newtheorem*{theorem*}{Theorem}
\newtheorem*{lemma*}{Lemma}
\newtheorem*{corollary*}{Corollary}
\newtheorem*{definition*}{Definition}
\newtheorem{theorem}{Theorem}
\newtheorem{lemma}{Lemma}
\newtheorem{corollary}{Corollary}
\newtheorem{definition}{Definition}

\begin{document}
\PMlinkescapeword{contains}
\PMlinkescapeword{theorem}
\PMlinkescapeword{subset}
\PMlinkescapeword{continuous}
\PMlinkescapeword{function}
\PMlinkescapeword{implies}
\PMlinkescapeword{homeomorphism}

Cantor space is an example of a compact and uncountable Polish space. In fact, every uncountable Polish space contains Cantor space, as stated by the following theorem.

\begin{theorem*}
Let $X$ be an uncountable Polish space. Then, it contains a subset $S$ which is homeomorphic to Cantor space.
\end{theorem*}

For example, the set $\mathbb{R}$ of real numbers contains the \PMlinkname{Cantor middle thirds set}{CantorSet}. Note that, being homeomorphic to Cantor space, $S$ must be a compact and hence closed subset of $X$.
The result is trivial in the case of Baire space $\mathcal{N}$, in which case we may take $S$ to be the set of all $s\in\mathcal{N}$ satisfying $s_n\in\{1,2\}$ for all $n$.
Then, for any uncountable Polish space $X$ there exists a continuous and one-to-one function $f\colon\mathcal{N}\to X$ (see \PMlinkname{here}{InjectiveImagesOfBaireSpace}). Then $f$ gives a continuous bijection from $S$ to $f(S)$. The \PMlinkname{inverse function theorem}{InverseFunctionTheoremTopologicalSpaces} implies that $f$ is a homeomorphism between $S$ and $f(S)$ and, therefore, $f(S)$ is homeomorphic to Cantor space.

%%%%%
%%%%%
\end{document}
