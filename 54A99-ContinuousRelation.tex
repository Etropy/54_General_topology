\documentclass[12pt]{article}
\usepackage{pmmeta}
\pmcanonicalname{ContinuousRelation}
\pmcreated{2013-03-22 17:05:39}
\pmmodified{2013-03-22 17:05:39}
\pmowner{CWoo}{3771}
\pmmodifier{CWoo}{3771}
\pmtitle{continuous relation}
\pmrecord{12}{39389}
\pmprivacy{1}
\pmauthor{CWoo}{3771}
\pmtype{Definition}
\pmcomment{trigger rebuild}
\pmclassification{msc}{54A99}

\usepackage{amssymb,amscd}
\usepackage{amsmath}
\usepackage{amsfonts}
\usepackage{mathrsfs}

% used for TeXing text within eps files
%\usepackage{psfrag}
% need this for including graphics (\includegraphics)
%\usepackage{graphicx}
% for neatly defining theorems and propositions
\usepackage{amsthm}
% making logically defined graphics
%%\usepackage{xypic}
\usepackage{pst-plot}
\usepackage{psfrag}

% define commands here
\newtheorem{prop}{Proposition}
\newtheorem{thm}{Theorem}
\newtheorem{ex}{Example}
\newcommand{\real}{\mathbb{R}}
\newcommand{\pdiff}[2]{\frac{\partial #1}{\partial #2}}
\newcommand{\mpdiff}[3]{\frac{\partial^#1 #2}{\partial #3^#1}}
\newcommand{\up}{\uparrow\!\!}
\newcommand{\down}{\downarrow\!\!}
\begin{document}
The idea of a continuous relation is neither as old nor as well-established as the idea of a continuous function.
Different authors use somewhat different definitions.  The present article is based on the following definition:

Let $X$ and $Y$ be topological spaces and $R$ a relation between $X$ and $Y$ ($R$ is a subset of $X\times Y$).  $R$ is said to be \PMlinkescapetext{\emph{continuous}} if 
\begin{center} for any open subset $V$ of $Y$, $R^{-1}(V)$ is open in $X$.\end{center}  
Here $R^{-1}(V)$ is the inverse image of $V$ under $R$, and is defined as $$R^{-1}(V):=\lbrace x\in X\mid xRy\mbox{ for some }y\in V\rbrace.$$  Equivalently, $R$ is a continuous relation if for any open set $V$ of $Y$, the set $\pi_X((X\times V)\cap R)$ is open in $X$, where $\pi_X$ is the projection map $X\times  Y\to X$.

\textbf{Remark}.  Continuous relations are generalization of continuous functions: if a continuous relation is also a function, then it is a continuous function.

Some examples.
\begin{itemize}
\item Let $X$ be an ordered space.  Then the partial order $\le$ is continuous iff for every open subset $A$ of $X$, its lower set $\down A$ is also open in $X$.

In particular, in $\mathbb{R}$, the usual linear ordering $\le$ on $\mathbb{R}$ is continuous.  To see this, let $A$ be an open subset of $\mathbb{R}$.  If $A=\varnothing$, then $\down A=\varnothing$ as well, and so is open.  Suppose now that $A$ is non-empty and deal with the case when $A$ is not bounded from above.  If $r\in \mathbb{R}$, then there is $a\in A$ such that $r\le a$, so that $r\in \down A$, which implies $\down A=\mathbb{R}$.  Hence $\down A$ is open.  If $A$ is bounded from above, then $A$ has a supremum (since $\mathbb{R}$ is Dedekind complete), say $x$.  Since $A$ is open, $x\notin A$ (or else $x\in (a,b)\subseteq A$, implying $x<\frac{x+b}{2}\in (a,b)$, contradicting the fact that $x$ is the least upper bound of $A$).  So $\down A=(-\infty,x)$, which is open also.  Therefore, $\le$ is a continuous relation on $\mathbb{R}$.
\item Again, we look at the space $\mathbb{R}$ with its usual interval topology.  The relation this time is $R=\lbrace (x,y)\in \mathbb{R}^2 \mid x^2+y^2=1\rbrace$.  This is not a continuous relation.  Take $A=(-2,2)$, which is open.  But then $R^{-1}(A)=[-1,1]$, which is closed.
\item Now, let $X$ be a locally connected topological space.  For any $x,y\in X$, define $x\sim y$ iff $x$ and $y$ belong to the same connected component of $X$.  Let $A$ be an open subset of $X$.  Then $B=\sim^{-1}(A)$ is the union of all connected components containing points of $A$.  Since (it can be shown) each connected component is open, so is their union, and hence $B$ is open. Thus $\sim$ is a continuous relation.
\item If $R$ is symmetric, then $R$ is continuous iff $R^{-1}$ is.  In particular, in a topological space $X$, an equivalence relation $\sim$ on $X$ is continuous iff the projection $p$ of $X$ onto the quotient space $X/\sim$ is an open mapping.
\end{itemize}

\textbf{Remark}.  Alternative definitions:
One apparently common definition (as described by Wyler) is to require inverse images of open sets to be open and inverse
images of closed sets to be closed (making the relation upper and lower semi-continuous).  Wyler suggests the following
definition:
If $r\colon e\to f$ is a relation between topological spaces $E$ and $F$, then $r$ is continuous iff for each topological space $A$, and functions $f\colon A\to E$ and $g\colon A\to F$ such that $f(u) \mathrel{r} g(u)$ for all $u\in A$, continuity of $f$ implies continuity of $g$.

\begin{thebibliography}{8}
\bibitem{tsb} T.~S.~Blyth, \emph{Lattices and Ordered Algebraic Structures}, Springer, New York (2005).
\bibitem{kelley} J.~L.~Kelley, \emph{General Topology}, D.~van Nostrand Company, Inc., 1955.
\bibitem{owyler} Oswald Wyler, \emph{\PMlinkexternal{\emph{A Characterization of Regularity in Topology}}{http://links.jstor.org/sici?sici=0002-9939\%28197108\%2929\%3A3\%3C588\%3AACORIT\%3E2.0.CO\%3B2-3}}
Proceedings of the American Mathematical Society, Vol.~29, No. 3. (Aug., 1971), pp. 588-590.

\end{thebibliography}
%%%%%
%%%%%
\end{document}
