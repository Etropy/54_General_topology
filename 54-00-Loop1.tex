\documentclass[12pt]{article}
\usepackage{pmmeta}
\pmcanonicalname{Loop1}
\pmcreated{2013-03-22 12:16:21}
\pmmodified{2013-03-22 12:16:21}
\pmowner{nerdy2}{62}
\pmmodifier{nerdy2}{62}
\pmtitle{loop}
\pmrecord{5}{31706}
\pmprivacy{1}
\pmauthor{nerdy2}{62}
\pmtype{Definition}
\pmcomment{trigger rebuild}
\pmclassification{msc}{54-00}

\endmetadata

% this is the default PlanetMath preamble.  as your knowledge
% of TeX increases, you will probably want to edit this, but
% it should be fine as is for beginners.

% almost certainly you want these
\usepackage{amssymb}
\usepackage{amsmath}
\usepackage{amsfonts}

% used for TeXing text within eps files
%\usepackage{psfrag}
% need this for including graphics (\includegraphics)
%\usepackage{graphicx}
% for neatly defining theorems and propositions
%\usepackage{amsthm}
% making logically defined graphics
%%%\usepackage{xypic} 

% there are many more packages, add them here as you need them

% define commands here
\begin{document}
A {\em loop} based at $x_0$ in a topological space $X$ is simply a continuous map $f : [0,1]\to X$ with $f(0) = f(1) = x_0$.

The collection of all such loops, modulo homotopy equivalence, forms a group known as the fundamental group.

More generally, the space of loops in $X$ based at $x_0$ with the compact-open topology, represented by $\Omega_{x_0}$, is known as the loop space of $X$.  And one has the homotopy groups $\pi_n(X,x_0) = \pi_{n-1}(\Omega_{x_0},\iota)$, where $\pi_n$ represents the higher homotopy groups, and $\iota$ is the basepoint in $\Omega_{x_0}$ consisting of the constant loop at $x_0$.
%%%%%
%%%%%
\end{document}
