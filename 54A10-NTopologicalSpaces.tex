\documentclass[12pt]{article}
\usepackage{pmmeta}
\pmcanonicalname{NTopologicalSpaces}
\pmcreated{2013-06-15 11:20:55}
\pmmodified{2013-06-15 11:20:55}
\pmowner{kamranmetric}{1000391}
\pmmodifier{kamranmetric}{1000391}
\pmtitle{N-Topological Spaces}
\pmrecord{1}{}
\pmprivacy{1}
\pmauthor{kamranmetric}{1000391}
\pmtype{Definition}
\pmclassification{msc}{54A10}

\endmetadata

% this is the default PlanetMath preamble.  as your knowledge
% of TeX increases, you will probably want to edit this, but
% it should be fine as is for beginners.

% almost certainly you want these
\usepackage{amssymb}
\usepackage{amsmath}
\usepackage{amsfonts}

% need this for including graphics (\includegraphics)
\usepackage{graphicx}
% for neatly defining theorems and propositions
\usepackage{amsthm}

% making logically defined graphics
%\usepackage{xypic}
% used for TeXing text within eps files
%\usepackage{psfrag}

% there are many more packages, add them here as you need them

% define commands here

\begin{document}
\documentclass[10pt, oneside,reqno]{amsart}
\usepackage{amsmath, amssymb, amsfonts}
\usepackage{amsthm}
\usepackage{hyperref}
\begin{document}
A non-empty set $X$ equipped with $N$ arbitrary topologies $\mathfrak{J}_1$,$\mathfrak{J}_2$,....., $\mathfrak{J}_N$ is called an N-topological space and denoted by $(X,\mathfrak{J}_1,.....,\mathfrak{J}_N)$.\\
Reference:   Khan, K. A., On the possibility of N-topological spaces, International Journal of Mathematical Archive (IJMA) 3 (2012), No.7, 2520-2523
\end{document}
\end{document}
