\documentclass[12pt]{article}
\usepackage{pmmeta}
\pmcanonicalname{HausdorffSpace}
\pmcreated{2013-03-22 12:18:18}
\pmmodified{2013-03-22 12:18:18}
\pmowner{yark}{2760}
\pmmodifier{yark}{2760}
\pmtitle{Hausdorff space}
\pmrecord{23}{31855}
\pmprivacy{1}
\pmauthor{yark}{2760}
\pmtype{Definition}
\pmcomment{trigger rebuild}
\pmclassification{msc}{54D10}
\pmsynonym{Hausdorff topological space}{HausdorffSpace}
\pmsynonym{T2 space}{HausdorffSpace}
\pmrelated{SeparationAxioms}
\pmrelated{T1Space}
\pmrelated{T0Space}
\pmrelated{T3Space}
\pmrelated{RegularSpace}
\pmrelated{MetricSpace}
\pmrelated{NormalTopologicalSpace}
\pmrelated{ASpaceMathnormalXIsHausdorffIfAndOnlyIfDeltaXIsClosed}
\pmrelated{SierpinskiSpace}
\pmrelated{HausdorffSpaceNotCompletelyHausdorff}
\pmrelated{Tychonoff}
\pmrelated{PropertyThatCompactSetsInASpaceAreClosedLies}
\pmdefines{Hausdorff}
\pmdefines{Hausdorff topology}
\pmdefines{T2}
\pmdefines{T2 topology}
\pmdefines{T2 axiom}

\endmetadata

\usepackage{amssymb}
\usepackage{amsmath}
\usepackage{amsfonts}

\def\emptyset{\varnothing}
\begin{document}
A topological space $(X,\tau)$ is said to be $T_2$ 
(or said to satisfy the $T_2$ axiom) if given 
distinct $x,y\in X$, there exist disjoint 
open sets $U,V\in\tau$ (that is, $U\cap V=\emptyset$) 
such that $x\in U$ and $y\in V$.

A $T_2$ space is also known as a \emph{Hausdorff space}.
A \emph{Hausdorff topology} for a set $X$ is a topology 
$\tau$ such that $(X,\tau)$ is a Hausdorff space.

\subsubsection*{Properties}
The following properties are equivalent:
\begin{enumerate}
\item $X$ is a Hausdorff space. 
\item The set 
$$
\Delta=\{(x,y)\in X\times X:x=y\}
$$
is closed in the product topology of $X\times X$.
\item For all $x\in X$, we have
$$
  \{x\} = \bigcap \{A : A\subseteq X\ \mbox{closed}, \mbox{$\exists$ open set}\  U\ \mbox{such that}\ x\in U\subseteq A\}.
$$
\end{enumerate}

Important examples of Hausdorff spaces are metric spaces, manifolds, 
and topological vector spaces.
%%%%%
%%%%%
\end{document}
