\documentclass[12pt]{article}
\usepackage{pmmeta}
\pmcanonicalname{VanishAtInfinity}
\pmcreated{2013-03-22 17:50:57}
\pmmodified{2013-03-22 17:50:57}
\pmowner{asteroid}{17536}
\pmmodifier{asteroid}{17536}
\pmtitle{vanish at infinity}
\pmrecord{6}{40321}
\pmprivacy{1}
\pmauthor{asteroid}{17536}
\pmtype{Definition}
\pmcomment{trigger rebuild}
\pmclassification{msc}{54D45}
\pmclassification{msc}{54C35}
\pmsynonym{zero at infinity}{VanishAtInfinity}
\pmsynonym{vanishes at infinity}{VanishAtInfinity}
\pmrelated{RegularAtInfinity}
\pmrelated{ApplicationsOfUrysohnsLemmaToLocallyCompactHausdorffSpaces}
\pmdefines{$C_0$}

\endmetadata

% this is the default PlanetMath preamble.  as your knowledge
% of TeX increases, you will probably want to edit this, but
% it should be fine as is for beginners.

% almost certainly you want these
\usepackage{amssymb}
\usepackage{amsmath}
\usepackage{amsfonts}

% used for TeXing text within eps files
%\usepackage{psfrag}
% need this for including graphics (\includegraphics)
%\usepackage{graphicx}
% for neatly defining theorems and propositions
%\usepackage{amsthm}
% making logically defined graphics
%%%\usepackage{xypic}

% there are many more packages, add them here as you need them

% define commands here

\begin{document}
Let $X$ be a locally compact space. A function $f:X \longrightarrow \mathbb{C}$ is said to \emph{vanish at infinity} if, for every $\epsilon > 0$, there is a compact set $K \subseteq X$ such that $|f(x)|<\epsilon$ for every $x \in X-K$, where $\|\cdot\|$ denotes the standard \PMlinkname{norm}{Norm2} on $\mathbb{C}$. 

If $X$ is non-compact, let $X \cup \{\infty\}$ be the one-point compactification of $X$. The above definition can be rephrased as:  The extension of $f$ to $X \cup \{\infty\}$ satisfying $f(\infty)=0$ is continuous at the point $\infty$.

The set of continuous functions $X \longrightarrow \mathbb{C}$ that vanish at infinity is an algebra over the complex field and is usually denoted by $C_0(X)$.

\subsubsection{Remarks}
\begin{itemize}
\item When $X$ is compact, all functions $X \longrightarrow \mathbb{C}$ vanish at infinity.  Hence, $C_0(X) = C(X)$.
\end{itemize}
%%%%%
%%%%%
\end{document}
