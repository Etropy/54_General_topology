\documentclass[12pt]{article}
\usepackage{pmmeta}
\pmcanonicalname{UnionOfNondisjointConnectedSetsIsConnected}
\pmcreated{2013-03-22 15:17:50}
\pmmodified{2013-03-22 15:17:50}
\pmowner{matte}{1858}
\pmmodifier{matte}{1858}
\pmtitle{union of non-disjoint connected sets is connected}
\pmrecord{8}{37095}
\pmprivacy{1}
\pmauthor{matte}{1858}
\pmtype{Theorem}
\pmcomment{trigger rebuild}
\pmclassification{msc}{54D05}

\endmetadata

% this is the default PlanetMath preamble.  as your knowledge
% of TeX increases, you will probably want to edit this, but
% it should be fine as is for beginners.

% almost certainly you want these
\usepackage{amssymb}
\usepackage{amsmath}
\usepackage{amsfonts}
\usepackage{amsthm}

\usepackage{mathrsfs}

% used for TeXing text within eps files
%\usepackage{psfrag}
% need this for including graphics (\includegraphics)
%\usepackage{graphicx}
% for neatly defining theorems and propositions
%
% making logically defined graphics
%%%\usepackage{xypic}

% there are many more packages, add them here as you need them

% define commands here

\newcommand{\sR}[0]{\mathbb{R}}
\newcommand{\sC}[0]{\mathbb{C}}
\newcommand{\sN}[0]{\mathbb{N}}
\newcommand{\sZ}[0]{\mathbb{Z}}

 \usepackage{bbm}
 \newcommand{\Z}{\mathbbmss{Z}}
 \newcommand{\C}{\mathbbmss{C}}
 \newcommand{\F}{\mathbbmss{F}}
 \newcommand{\R}{\mathbbmss{R}}
 \newcommand{\Q}{\mathbbmss{Q}}



\newcommand*{\norm}[1]{\lVert #1 \rVert}
\newcommand*{\abs}[1]{| #1 |}



\newtheorem{thm}{Theorem}
\newtheorem{defn}{Definition}
\newtheorem{prop}{Proposition}
\newtheorem{lemma}{Lemma}
\newtheorem{cor}{Corollary}
\begin{document}
\begin{thm} Suppose $A,B$ are connected sets in a topological 
  space $X$. If $A,B$ are not disjoint, then $A\cup B$ is connected.
\end{thm}

\begin{proof} By assumption, we have two implications.
First, if $U,V$ are open in $A$ and $U\cup V=A$, then $U\cap V\neq \emptyset$.
Second, if $U,V$ are open in $B$ and $U\cup V=B$, then $U\cap V\neq \emptyset$.
To prove that $A\cup B$ is connected, suppose $U,V$ are open in $A\cup B$
and $U\cup V=A\cup B$.
Then 
\begin{eqnarray*}
  U\cup V &=& ((U\cup V )\cap A) \,\cup \,((U\cup V)\cap B) \\
&=& (U\cap A) \cup (V\cap A) \cup (U\cap B)\cup (V\cap B)
\end{eqnarray*}
Let us show that $U\cap A$ and $V\cap A$ are open in $A$. 
To do this, we use \PMlinkname{this result}{SubspaceOfASubspace}
  and notation from that entry too. 
For example, as $U\in \tau_{A\cup B, X}$, $U\cap A \in \tau_{A,A\cup B, X}=\tau_{A,X}$, 
  and so $U\cap A$, $V\cap A$ are open in $A$. 
Since $(U\cap A)\cup (V\cap A)=A$, it follows that 
$$
  \emptyset \neq (U\cap A)\cap (V\cap A) =(U\cap V)\cap A.
$$
If $U\cap V=\emptyset$, then this is a contradition, so 
   $A\cup B$ must be connected.
\end{proof}
%%%%%
%%%%%
\end{document}
