\documentclass[12pt]{article}
\usepackage{pmmeta}
\pmcanonicalname{CofiniteAndCocountableTopologies}
\pmcreated{2013-03-22 13:03:30}
\pmmodified{2013-03-22 13:03:30}
\pmowner{yark}{2760}
\pmmodifier{yark}{2760}
\pmtitle{cofinite and cocountable topologies}
\pmrecord{21}{33464}
\pmprivacy{1}
\pmauthor{yark}{2760}
\pmtype{Definition}
\pmcomment{trigger rebuild}
\pmclassification{msc}{54B99}
\pmrelated{FiniteComplementTopology}
\pmdefines{cofinite topology}
\pmdefines{cocountable topology}
\pmdefines{cofinite}
\pmdefines{cocountable}

\endmetadata

\usepackage{amssymb}
\usepackage{amsmath}
\usepackage{amsfonts}

\def\emptyset{\varnothing}
\begin{document}
\PMlinkescapeword{words}

The \emph{cofinite topology} on a set $X$
is defined to be the topology $\mathcal{T}$ where
\[
  \mathcal{T} = \{A \subseteq X \mid
  X \setminus A \hbox{ is finite, or } A=\emptyset\}.
\]
In other words, the closed sets in the cofinite topology are $X$ and the finite subsets of $X$.

Analogously, the \emph{cocountable topology} on $X$
is defined to be the topology
in which the closed sets are $X$ and the countable subsets of $X$.

The cofinite topology on $X$ is the coarsest \PMlinkname{$T_1$ topology}{T1Space} on $X$.

The cofinite topology on a finite set $X$ is the discrete topology.
Similarly, the cocountable topology on a countable set $X$ is the discrete topology.

A set $X$ together with the cofinite topology forms a compact topological space.
%%%%%
%%%%%
\end{document}
