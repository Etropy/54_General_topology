\documentclass[12pt]{article}
\usepackage{pmmeta}
\pmcanonicalname{varepsilonnet}
\pmcreated{2013-03-22 13:37:54}
\pmmodified{2013-03-22 13:37:54}
\pmowner{Koro}{127}
\pmmodifier{Koro}{127}
\pmtitle{$\varepsilon$-net}
\pmrecord{4}{34280}
\pmprivacy{1}
\pmauthor{Koro}{127}
\pmtype{Definition}
\pmcomment{trigger rebuild}
\pmclassification{msc}{54E35}
\pmrelated{Cover}

% this is the default PlanetMath preamble.  as your knowledge
% of TeX increases, you will probably want to edit this, but
% it should be fine as is for beginners.

% almost certainly you want these
\usepackage{amssymb}
\usepackage{amsmath}
\usepackage{amsfonts}

% used for TeXing text within eps files
%\usepackage{psfrag}
% need this for including graphics (\includegraphics)
%\usepackage{graphicx}
% for neatly defining theorems and propositions
%\usepackage{amsthm}
% making logically defined graphics
%%%\usepackage{xypic}

% there are many more packages, add them here as you need them

% define commands here
\begin{document}
{\bf Definition}
Suppose $X$ is a metric space with a metric $d$, and suppose
$S$ is a subset of $X$. Let $\varepsilon$ be a positive real number.
A subset $N\subset S$ is an $\varepsilon$\emph{-net
for $S$} if, for all $x\in S$, there is an $y\in N$,
such that $d(x,y)<\varepsilon$.

For any $\varepsilon>0$ and $S\subset X$, the set $S$ is trivially an 
$\varepsilon$-net for itself. 

{\bf Theorem} 
Suppose $X$ is a metric space with a metric $d$, and suppose
$S$ is a subset of $X$. Let $\varepsilon$ be a positive real number.
Then $N$ is an $\varepsilon$-net for $S$, if and only if
$$\{ B_\varepsilon(y) \mid y\in N \}$$
is a cover for $S$. (Here $B_\varepsilon(x)$ is
the open ball with center $x$ and radius $\varepsilon$.)

\emph{Proof.} Suppose $N$ is an $\varepsilon$-net for $S$. 
If $x\in S$, there is an $y\in N$ such that $x\in B_\varepsilon(y)$. 
Thus, $x$ is covered by some set in $\{ B_\varepsilon(x) \mid x\in N \}$.
Conversely, suppose $\{ B_\varepsilon(y) \mid y\in N \}$ is
a cover for $S$, and suppose $x\in S$. By assumption,
there is an $y\in N$, such that $x\in B_\varepsilon(y)$.
Hence $d(x,y)<\varepsilon$ with $y\in N$.
$\Box$

{\bf Example}
In $X=\mathbb{R}^2$ with the usual
Cartesian metric, the set
$$ N = \{(a,b) \mid a,b\in \mathbb{Z} \}$$
is an $\varepsilon$-net for $X$ assuming that
$\varepsilon> \sqrt{2}/2$. $\Box$

The above definition and example can be found in \cite{bachman}, page 64-65.

\begin{thebibliography}{9}
\bibitem{bachman}
G. Bachman, L. Narici,
\emph{Functional analysis},
Academic Press, 1966.
\end{thebibliography}
%%%%%
%%%%%
\end{document}
