\documentclass[12pt]{article}
\usepackage{pmmeta}
\pmcanonicalname{Separated}
\pmcreated{2013-03-22 15:16:34}
\pmmodified{2013-03-22 15:16:34}
\pmowner{matte}{1858}
\pmmodifier{matte}{1858}
\pmtitle{separated}
\pmrecord{15}{37064}
\pmprivacy{1}
\pmauthor{matte}{1858}
\pmtype{Definition}
\pmcomment{trigger rebuild}
\pmclassification{msc}{54-00}
\pmclassification{msc}{54D05}

% this is the default PlanetMath preamble.  as your knowledge
% of TeX increases, you will probably want to edit this, but
% it should be fine as is for beginners.

% almost certainly you want these
\usepackage{amssymb}
\usepackage{amsmath}
\usepackage{amsfonts}
\usepackage{amsthm}

\usepackage{mathrsfs}

% used for TeXing text within eps files
%\usepackage{psfrag}
% need this for including graphics (\includegraphics)
%\usepackage{graphicx}
% for neatly defining theorems and propositions
%
% making logically defined graphics
%%%\usepackage{xypic}

% there are many more packages, add them here as you need them

% define commands here

\newcommand{\sR}[0]{\mathbb{R}}
\newcommand{\sC}[0]{\mathbb{C}}
\newcommand{\sN}[0]{\mathbb{N}}
\newcommand{\sZ}[0]{\mathbb{Z}}

 \usepackage{bbm}
 \newcommand{\Z}{\mathbbmss{Z}}
 \newcommand{\C}{\mathbbmss{C}}
 \newcommand{\F}{\mathbbmss{F}}
 \newcommand{\R}{\mathbbmss{R}}
 \newcommand{\Q}{\mathbbmss{Q}}



\newcommand*{\norm}[1]{\lVert #1 \rVert}
\newcommand*{\abs}[1]{| #1 |}



\newtheorem{thm}{Theorem}
\newtheorem{defn}{Definition}
\newtheorem{prop}{Proposition}
\newtheorem{lemma}{Lemma}
\newtheorem{cor}{Corollary}
\begin{document}
{\bf Definition} 
Suppose $A$ and $B$ are subsets of a topological space
$X$. Then $A$ and $B$ are {\bf separated} provided that
\[
\begin{array}{ccc}
\overline{A}\cap B &=& \emptyset, \\
A\cap \overline{B} &=& \emptyset,
\end{array}
\]
where $\overline{A}$ is the \PMlinkname{closure operator}{Closure} in $X$.

\subsubsection*{Properties}
\begin{enumerate}
\item If $A,B$ are separated in $X$, and $f\colon X\to Y$ is a homeomorphism, 
then $f(A)$ and $f(B)$ are separated in $Y$. 
\end{enumerate}

\subsubsection*{Examples}
\begin{enumerate}
\item On $\R$, the intervals $(0,1)$ and $(1,2)$ are separated.
\item If $d(x,y)\ge r+s$, then the open balls $B_r(x)$ and $B_s(y)$ are 
  separated \PMlinkname{(proof.)}{WhenAreBallsSeparated}.
\item If $A$ is a clopen set, then $A$ and $A^\complement$ are separated.
This follows since $\overline{S}=S$ when $S$ is a closed set.
\end{enumerate}

\subsubsection*{Remarks}
The above definition follows \cite{kelley}. In
\cite{jameson}, separated sets are called
{\bf strongly disjoin{t}} sets.

\begin{thebibliography}{9}
\bibitem{kelley}
J.L. Kelley, \emph{General Topology}, D. van Nostrand Company, Inc., 1955.
\bibitem{jameson} G.J. Jameson, \emph{Topology and Normed Spaces},
Chapman and Hall, 1974.
\end{thebibliography}
%%%%%
%%%%%
\end{document}
