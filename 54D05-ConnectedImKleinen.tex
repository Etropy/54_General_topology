\documentclass[12pt]{article}
\usepackage{pmmeta}
\pmcanonicalname{ConnectedImKleinen}
\pmcreated{2013-03-22 15:59:00}
\pmmodified{2013-03-22 15:59:00}
\pmowner{Mathprof}{13753}
\pmmodifier{Mathprof}{13753}
\pmtitle{connected im kleinen}
\pmrecord{8}{38000}
\pmprivacy{1}
\pmauthor{Mathprof}{13753}
\pmtype{Definition}
\pmcomment{trigger rebuild}
\pmclassification{msc}{54D05}
%\pmkeywords{connected}
%\pmkeywords{locally connected}

\endmetadata

% this is the default PlanetMath preamble.  as your knowledge
% of TeX increases, you will probably want to edit this, but
% it should be fine as is for beginners.

% almost certainly you want these
\usepackage{amssymb}
\usepackage{amsmath}
\usepackage{amsfonts}

% used for TeXing text within eps files
%\usepackage{psfrag}
% need this for including graphics (\includegraphics)
%\usepackage{graphicx}
% for neatly defining theorems and propositions
%\usepackage{amsthm}
% making logically defined graphics
%%%\usepackage{xypic}

% there are many more packages, add them here as you need them

% define commands here

\begin{document}
A topological space $ X$ is {\it connected im kleinen  at a point} $ x $ if every open set $ U$ containing $ x$ contains an  open set  $ V$ containing $ x$ such that if $ y$ is a point of $ V$, then there is a connected subset of $ U$ containing $\{x,y\}$. 
\\ Another way to say this is that $X$ is connected im kleinen at a point $x$ if $x$ has a neighborhood base of connected sets (not necessarily open).

A locally connected space is connected im kleinen at each point. 

A space can be connected im kleinen at a point but not locally connected at the point. 


If a topological space is connected im kleinen at each point, then it is locally 
connected. 

 \begin{thebibliography}{9}
\bibitem{willard} S. Willard, \emph{General Topology},
Addison-Wesley, Publishing Company, 1970.
\bibitem{hocking} J.G. Hocking, G.S. Young, \emph{Topology}, Dover Pubs,
1988, republication of 1961 Addison-Wesley edition.
\end{thebibliography}
%%%%%
%%%%%
\end{document}
