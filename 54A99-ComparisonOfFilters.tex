\documentclass[12pt]{article}
\usepackage{pmmeta}
\pmcanonicalname{ComparisonOfFilters}
\pmcreated{2013-03-22 14:41:38}
\pmmodified{2013-03-22 14:41:38}
\pmowner{rspuzio}{6075}
\pmmodifier{rspuzio}{6075}
\pmtitle{comparison of filters}
\pmrecord{5}{36303}
\pmprivacy{1}
\pmauthor{rspuzio}{6075}
\pmtype{Definition}
\pmcomment{trigger rebuild}
\pmclassification{msc}{54A99}
\pmclassification{msc}{03E99}
\pmdefines{finer}
\pmdefines{coarser}
\pmdefines{strictly finer}
\pmdefines{strictly coarser}
\pmdefines{comparable}

\endmetadata

% this is the default PlanetMath preamble.  as your knowledge
% of TeX increases, you will probably want to edit this, but
% it should be fine as is for beginners.

% almost certainly you want these
\usepackage{amssymb}
\usepackage{amsmath}
\usepackage{amsfonts}

% used for TeXing text within eps files
%\usepackage{psfrag}
% need this for including graphics (\includegraphics)
%\usepackage{graphicx}
% for neatly defining theorems and propositions
%\usepackage{amsthm}
% making logically defined graphics
%%%\usepackage{xypic}

% there are many more packages, add them here as you need them

% define commands here
\begin{document}
Let $\mathbb{F}_1$ and $\mathbb{F}_2$ be two filters on the same set.  The following terminology is commonly used to describe the relation of $\mathbb{F}_1$ to $\mathbb{F}_2$:

$\mathbb{F}_2$ is said to be \emph{finer} than $\mathbb{F}_1$ if $\mathbb{F}_1 \subseteq \mathbb{F}_2$.

$\mathbb{F}_2$ is said to be \emph{coarser} than $\mathbb{F}_1$ if $\mathbb{F}_1 \supseteq \mathbb{F}_2$.

$\mathbb{F}_2$ is said to be \emph{strictly finer} than $\mathbb{F}_1$ if $\mathbb{F}_1 \subset \mathbb{F}_2$.

$\mathbb{F}_2$ is said to be \emph{strictly coarser} than $\mathbb{F}_1$ if $\mathbb{F}_1 \supset \mathbb{F}_2$.

$\mathbb{F}_1$ and $\mathbb{F}_2$ are said to be \emph{comparable} if either $\mathbb{F}_1 \subseteq \mathbb{F}_2$ or $\mathbb{F}_1 \supseteq \mathbb{F}_2$.
%%%%%
%%%%%
\end{document}
