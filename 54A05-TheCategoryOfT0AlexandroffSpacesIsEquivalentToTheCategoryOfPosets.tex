\documentclass[12pt]{article}
\usepackage{pmmeta}
\pmcanonicalname{TheCategoryOfT0AlexandroffSpacesIsEquivalentToTheCategoryOfPosets}
\pmcreated{2013-03-22 18:46:04}
\pmmodified{2013-03-22 18:46:04}
\pmowner{joking}{16130}
\pmmodifier{joking}{16130}
\pmtitle{the category of T0 Alexandroff spaces is equivalent to the category of posets}
\pmrecord{8}{41549}
\pmprivacy{1}
\pmauthor{joking}{16130}
\pmtype{Theorem}
\pmcomment{trigger rebuild}
\pmclassification{msc}{54A05}

% this is the default PlanetMath preamble.  as your knowledge
% of TeX increases, you will probably want to edit this, but
% it should be fine as is for beginners.

% almost certainly you want these
\usepackage{amssymb}
\usepackage{amsmath}
\usepackage{amsfonts}

% used for TeXing text within eps files
%\usepackage{psfrag}
% need this for including graphics (\includegraphics)
%\usepackage{graphicx}
% for neatly defining theorems and propositions
%\usepackage{amsthm}
% making logically defined graphics
%%%\usepackage{xypic}

% there are many more packages, add them here as you need them

% define commands here

\begin{document}
Let $\mathcal{AT}$ be the category of all $\mathrm{T}_{0}$, Alexandroff spaces and continuous maps between them. Furthermore let $\mathcal{POSET}$ be the category of all posets and order preserving maps.

\textbf{Theorem.} The categories $\mathcal{AT}$ and $\mathcal{POSET}$ are equivalent.

\textit{Proof.} Consider two functors:
$$T:\mathcal{AT}\to\mathcal{POSET};$$
$$S:\mathcal{POSET}\to \mathcal{AT},$$
such that $T(X,\tau)=(X,\leq)$, where $\leq$ is an induced partial order on an Alexandroff space and $T(f)=f$ for continuous map. Analogously, let $S(X,\leq)=(X,\tau)$, where $\tau$ is an induced Alexandroff topology on a poset and $S(f)=f$ for order preserving maps. One can easily show that $T$ and $S$ are well defined. Furthermore, it is easy to verify that equalities $T\circ S=1_{\mathcal{POSET}}$ and $S\circ T=1_{\mathcal{AT}}$ hold, which completes the proof. $\square$

\textbf{Remark.} Of course every finite topological space is Alexandroff, thus we have very nice ,,interpretation'' of finite $\mathrm{T}_{0}$ spaces - finite posets (since functors $T$ and $S$ do not change set-theoretic properties of underlying sets such as finitness).
%%%%%
%%%%%
\end{document}
