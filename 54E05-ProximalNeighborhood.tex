\documentclass[12pt]{article}
\usepackage{pmmeta}
\pmcanonicalname{ProximalNeighborhood}
\pmcreated{2013-03-22 16:58:25}
\pmmodified{2013-03-22 16:58:25}
\pmowner{CWoo}{3771}
\pmmodifier{CWoo}{3771}
\pmtitle{proximal neighborhood}
\pmrecord{7}{39250}
\pmprivacy{1}
\pmauthor{CWoo}{3771}
\pmtype{Definition}
\pmcomment{trigger rebuild}
\pmclassification{msc}{54E05}
\pmsynonym{proximity neighborhood}{ProximalNeighborhood}
\pmsynonym{$\delta$-neighborhood}{ProximalNeighborhood}

\endmetadata

\usepackage{amssymb,amscd}
\usepackage{amsmath}
\usepackage{amsfonts}
\usepackage{mathrsfs}

% used for TeXing text within eps files
%\usepackage{psfrag}
% need this for including graphics (\includegraphics)
%\usepackage{graphicx}
% for neatly defining theorems and propositions
\usepackage{amsthm}
% making logically defined graphics
%%\usepackage{xypic}
\usepackage{pst-plot}
\usepackage{psfrag}

% define commands here
\newtheorem{prop}{Proposition}
\newtheorem{thm}{Theorem}
\newtheorem{ex}{Example}
\newcommand{\real}{\mathbb{R}}
\newcommand{\pdiff}[2]{\frac{\partial #1}{\partial #2}}
\newcommand{\mpdiff}[3]{\frac{\partial^#1 #2}{\partial #3^#1}}
\begin{document}
Let $X$ be a set and $P(X)$ its power set.  Let $\ll$ be a binary relation on $P(X)$ satisfying the

following conditions, for any $A,B\subseteq X$:
\begin{enumerate}
\item $X\ll X$,
\item $A\ll B$ implies $A\subseteq B$,
\item $A\ll B$ and $C\ll D$ imply $A\cap C\ll B\cap D$,
\item $A\ll B$ implies $B'\ll A'$ ($'$ is the complement operator)
\item $A\subseteq B\ll C\subseteq D$, then $A\ll D$, and
\item if $A\ll B$, then there is $C\subseteq X$, such that $A\ll C\ll B$.
\end{enumerate}
By 1 and 4, it is easy to see that $\varnothing\ll \varnothing$.  Also, 3 and 4 show that $A\cup C\ll B\cup D$ whenever $A\ll B$ and $C\ll D$.  So $\ll$ is a topogenous order, which means $\ll$ is transitive and anti-symmetric.  Under this order relation, we say that $B$ is a \emph{proximal neighborhood} of $A$ if $A\ll B$.

The reason why we call $B$ a ``proximal'' neighborhood is due to the following:

\begin{thm} Let $X$ be a set.  The following are true.
\begin{itemize}  
\item Let $\ll$ be defined as above.  Define a new relation $\delta$ on $P(X)$: $A\delta'B'$ iff $A\ll B$.  Then $\delta$ so defined is a proximity relation, turning $X$ into a proximity space.
\item Conversely, let $(X,\delta)$ is a proximity space.  Define a new relation $\ll$ on $P(X)$: $A\ll B$ iff $A\delta'B'$.  Then $\ll$ satisfies the six properties above.
\end{itemize}
\end{thm}
\begin{proof}
Suppose first that $X$ and $\ll$ are defined as above.  We will verify the individual nearness relation axioms of $\delta$ by proving their contrapositives in each case, except the last axiom:
\begin{enumerate}
\item if $A\delta'B$, then $A\ll B'$, or $A\subseteq B'$, so $A\cap B=\varnothing$;
\item suppose either $A=\varnothing$ or $B=\varnothing$.  In either case, $A\ll B'$, which means $A\delta' B$;
\item if $A\delta' B$, then $A\ll B'$, so $B''\ll A'$, or $B\ll A'$, or $B\delta' A$;
\item if $A_1\delta' B$ and $A_2\delta' B$, then $A_1\ll B$ and $A_2\ll B$, so $(A_1\cup A_2)\ll B$, or $(A_1\cup A_2)\delta' B$;
\item if $A\delta' B$, then $A\ll B'$.  So there is $D\subseteq X$ with $A\ll D$ and $D\ll B'$.  Let $C=D'$.  Then $A\ll C'$ and $C'\ll B'$, or $A\delta' C$ and $C'\delta' B$.
\end{enumerate}

Next, suppose $(X,\delta)$ is a proximity space.  We now verify the six properties of $\ll$ above.
\begin{enumerate}
\item since $X\delta' \varnothing$, $X\ll \varnothing'$, or $X\ll X$; 
\item suppose $A\delta' B'$, then if $x\in A$, we have $x\delta'B'$, implying $x\cap B'=\varnothing$, or $x\in B$;
\item if $A\ll B$ and $C\ll D$, then $A\delta' B'$ and $C\delta' D'$, which means $A\delta' (B'\cup D')$ and $C\delta' (B'\cup D')$, which together imply $(A\cap C)\delta' (B'\cup D')$, or $(A\cap C)\delta (B\cap D)'$, or $A\cap C\ll B\cap D$;
\item if $A\ll B$, then $A\delta' B'$, so $B'\delta' A$ (as $\delta$ is symmetric, so is its complement), which is the same as $B'\delta' A''$, or $B'\ll A'$;
\item if $A\delta D'$, then $B\delta C'$ (since $A\subseteq B$ and $D'\subseteq C'$), so $B\ll' C$, a contradiction;
\item if $A\ll B$, then $A\delta'B'$, so there is $D\subseteq X$ with $A\delta'D$ and $D'\delta'B'$.  Define $C=D'$, then $A\ll C$ and $C\ll B$, as desired.
\end{enumerate}
This completes the proof.
\end{proof}

Because of the above, we see that a proximity space can be equivalently defined using the proximal neighborhood concept.  To emphasize its relationship with $\delta$, a proximal neighborhood is also called a $\delta$-neighbhorhood.

Furthermore, we have 
\begin{thm}
if $B$ is a proximal neighborhood of $A$ in a proximity space $(X,\delta)$, then $B$ is a (topological) neighborhood of $A$ under the topology $\tau(\delta)$ induced by the proximity relation $\delta$.  In other words, if $A\ll B$, then $A\subseteq B^{\circ}$ and $A^c\subseteq B$, where $^{\circ}$ and $^c$ denote the interior and closure operators.
\end{thm}
\begin{proof}
Since $A\delta'B'$, then $x\delta'B'$ whenever $x\in A$, which is the contrapositive of the statement: $x\in A'$ whenever $x\delta B'$, which is equivalent to $B'^c\subseteq A'$, or $A\subseteq B^{\circ}$.  Furthermore, if $x\notin B$, then $x\in B'$.  But $A\delta'B'$ b assumption.  This implies $x\delta' A$, which means $x\notin A^c$.  Therefore $A^c\subseteq B$.
\end{proof}
\textbf{Remark}.  However, not every $\tau(\delta)$-neighborhood is a $\delta$-neighborhood.
%%%%%
%%%%%
\end{document}
