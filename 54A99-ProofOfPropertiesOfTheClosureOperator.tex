\documentclass[12pt]{article}
\usepackage{pmmeta}
\pmcanonicalname{ProofOfPropertiesOfTheClosureOperator}
\pmcreated{2013-03-22 14:12:19}
\pmmodified{2013-03-22 14:12:19}
\pmowner{archibal}{4430}
\pmmodifier{archibal}{4430}
\pmtitle{proof of properties of the closure operator}
\pmrecord{4}{35638}
\pmprivacy{1}
\pmauthor{archibal}{4430}
\pmtype{Proof}
\pmcomment{trigger rebuild}
\pmclassification{msc}{54A99}

\endmetadata

% this is the default PlanetMath preamble.  as your knowledge
% of TeX increases, you will probably want to edit this, but
% it should be fine as is for beginners.

% almost certainly you want these
\usepackage{amssymb}
\usepackage{amsmath}
\usepackage{amsfonts}

% used for TeXing text within eps files
%\usepackage{psfrag}
% need this for including graphics (\includegraphics)
%\usepackage{graphicx}
% for neatly defining theorems and propositions
%\usepackage{amsthm}
% making logically defined graphics
%%%\usepackage{xypic}

% there are many more packages, add them here as you need them

% define commands here

\newtheorem{theorem}{Theorem}
\newtheorem{defn}{Definition}
\newtheorem{prop}{Proposition}
\newtheorem{lemma}{Lemma}
\newtheorem{cor}{Corollary}
\begin{document}
Recall that the closure of a set $A$ in a topological space $X$ is defined to be the intersection of all closed sets containing it.

\begin{description}

\item[$A\subset \overline{A}$]:  By definition 
\[
\overline{A}=\bigcap_{C\supseteq A, \ C\text{ closed}} C,
\]
but since for every $C$ we have $A\subseteq C$, we immediately find 
\[
A\subseteq \bigcap_{C\supseteq A, \ C\text{ closed}} C.
\]

\item[$\overline{A}$ is closed]: Recall that the intersection of any number of closed sets is closed, so the closure is itself closed.

\item[$\overline{\emptyset} = \emptyset$, $\overline{X} = X$, and $\overline{\overline{A}} = \overline{A}$]: 
If $C$ is any closed set, then  
\[
\overline{C} = \bigcap_{C'\supseteq C, \ C'\text{ closed}} C' = C \cap \bigcap_{C'\supsetneq C, \ C'\text{ closed}} C' = C.
\] 

\item[$\overline{A\cup B} = \overline{A} \cup \overline{B}$]: First write down the definition:
\begin{align*}
\overline{A} \cup \overline{B} 
& = \bigcap_{C\supseteq A, \ C\text{ closed}} C \cup \bigcap_{D\supseteq B, 
\ D\text{ closed}} D, \\
\intertext{then apply DeMorgan's law to get}
& = \bigcap_{C\supseteq A, D\supseteq B, \ C, D\text{ closed}}(C\cup D), \\
\intertext{but for every such pair $C$, $D$, we have that $E = C\cup D$ is a closed set containing $A\cup B$.  Conversely, every closed set $E$ containing $A\cup B$ is obtained from such a pair --- just take $(E,E)$ to be the pair.  Thus}
& = \bigcap_{E\supseteq A\cup B, \ E\text{ closed}}(E) \\
& = \overline{A\cup B}.
\end{align*}

\item[$\overline{A\cap B} \subset \overline{A}\cap \overline{B}$]:
\begin{align*}
\overline{A} \cap \overline{B} 
& = \bigcap_{C\supseteq A, C\text{ closed}} C \cap \bigcap_{D\supseteq B, 
\ D\text{ closed}} D, \\
& = \bigcap_{C\supseteq A, D\supseteq B, \ C, D\text{ closed}}(C\cap D), \\
\intertext{but for every such pair $C$, $D$, we have that $E = C\cap D$ is a closed set containing $A\cap B$.  However, some closed sets may not arise in this way, so we do not have equality.  Thus}
& \supseteq \bigcap_{E\supseteq A\cap B, \ E\text{ closed}}(E) \\
& = \overline{A\cap B}.
\end{align*}
so we have
\[
\overline{A} \cap \overline{B} \supseteq \overline{A\cap B}.
\]

\item[$\overline{A} = A\cup A'$ where $A'$ is the set of all limit points of $A$]:
Let $a$ be a limit point of $A$, and let $C$ be a closed set containing $A$.  If $a$ is not in $C$, then $X\setminus C$ is an open set containing $a$ but not meeting $C$, which implies that $X\setminus C$ does not meet $A$, which contradicts the fact that $a$ was a limit point of $A$.  Conversely, suppose that $a$ is not a limit point of $A$, and that $a$ is not in $A$.  Then there is some open neighborhood $U$ of $a$ which does not meet $A$.  But then $X\setminus U$ is a closed set containing $A$ but not containing $a$, so $a\notin\overline{A}$. 

\end{description}
%%%%%
%%%%%
\end{document}
