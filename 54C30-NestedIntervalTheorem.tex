\documentclass[12pt]{article}
\usepackage{pmmeta}
\pmcanonicalname{NestedIntervalTheorem}
\pmcreated{2013-03-22 17:27:12}
\pmmodified{2013-03-22 17:27:12}
\pmowner{pahio}{2872}
\pmmodifier{pahio}{2872}
\pmtitle{nested interval theorem}
\pmrecord{11}{39835}
\pmprivacy{1}
\pmauthor{pahio}{2872}
\pmtype{Theorem}
\pmcomment{trigger rebuild}
\pmclassification{msc}{54C30}
\pmclassification{msc}{26-00}

% this is the default PlanetMath preamble.  as your knowledge
% of TeX increases, you will probably want to edit this, but
% it should be fine as is for beginners.

% almost certainly you want these
\usepackage{amssymb}
\usepackage{amsmath}
\usepackage{amsfonts}

% used for TeXing text within eps files
%\usepackage{psfrag}
% need this for including graphics (\includegraphics)
%\usepackage{graphicx}
% for neatly defining theorems and propositions
 \usepackage{amsthm}
% making logically defined graphics
%%%\usepackage{xypic}

% there are many more packages, add them here as you need them

% define commands here

\theoremstyle{definition}
\newtheorem*{thmplain}{Theorem}
\newtheorem{prop}{Proposition}
\begin{document}
\begin{prop} If 
$$[a_1,\,b_1] \;\supseteq\; [a_2,\,b_2] \;\supseteq\; [a_3,\,b_3] \;\supseteq\ldots$$
is a sequence of nested closed intervals, then 
\begin{align*}
\bigcap_{n=1}^\infty [a_n,\,b_n] \;\neq\; \varnothing.
\end{align*}
If also\, $\displaystyle\lim_{n\to\infty}(b_n\!-\!a_n) = 0$,\, then the infinite 
intersection consists of a unique real number.
\end{prop}

\begin{proof}  There are two consequences to nesting of intervals: $[a_m,\,b_m]\subseteq[a_n,\,b_n]$ for $n\le m$:
\begin{enumerate}
\item first of all, we have the inequality $a_n\le a_m$ for $n\le m$, which means that the sequence $a_1, a_2, \ldots, a_n, \ldots$ is nondecreasing;
\item in addition, we also have two inequalities: $a_m\le b_n$ and $a_n\le b_m$.  In either case, we have that $a_i\le b_j$ for all $i,j$.  This means that the sequence $a_1, a_2, \ldots, a_n, \ldots$ is bounded from above by all $b_i$, where $i=1,2,\ldots$.  
\end{enumerate}
Therefore, the limit of the sequence $(a_i)$ exists, and is just the supremum, say $a$ (see proof \PMlinkname{here}{NondecreasingSequenceWithUpperBound}).  Similarly the sequence $(b_i)$ is nonincreasing and bounded from below by all $a_i$, where $i=1,2,\ldots$, and hence has an infimum $b$.  

Now, as the supremum of $(a_i)$, $a\le b_i$ for all $i$.  But because $b$ is the infimum of $(b_i)$, $a\le b$.  Therefore, the interval $[a,b]$ is non-empty (containing at least one of $a,b$).  Since $a_i\le a\le b\le b_i$, every interval $[a_i,b_i]$ contains the interval $[a,b]$, so their intersection also contains $[a,b]$, hence is non-empty.

If $c$ is a point outside of $[a,b]$, say $c<a$, then there is some $a_i$, such that $c<a_i$ (by the definition of the supremum $a$), and hence $c\notin [a_i,b_i]$.  This shows that the intersection actually coincides with $[a,b]$.

Now, since $\displaystyle\lim_{n\to\infty}(b_n-a_n) = 0$, we have that $b-a=\displaystyle\lim_{n\to\infty}b_n - \displaystyle\lim_{n\to\infty} a_n = \displaystyle\lim_{n\to\infty}(b_n-a_n) = 0$.  So $a=b$.  This means that the intersection of the nested intervals contains a single point $a$.
\end{proof}

\textbf{Remark}.\, This result is called the \emph{nested interval theorem}.
It is a restatement of the \emph{finite intersection property}
for the compact set \,$[a_1,\,b_1]$.\, The result may also be proven by elementary methods:
namely, any number lying in between the supremum of all the $a_n$ and the infimum of all the $b_n$
will be in all the nested intervals.

%%%%%
%%%%%
\end{document}
