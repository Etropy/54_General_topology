\documentclass[12pt]{article}
\usepackage{pmmeta}
\pmcanonicalname{ProofOfTychonoffsTheoremInFiniteCase}
\pmcreated{2013-03-22 15:26:27}
\pmmodified{2013-03-22 15:26:27}
\pmowner{stevecheng}{10074}
\pmmodifier{stevecheng}{10074}
\pmtitle{proof of Tychonoff's theorem in finite case}
\pmrecord{4}{37288}
\pmprivacy{1}
\pmauthor{stevecheng}{10074}
\pmtype{Proof}
\pmcomment{trigger rebuild}
\pmclassification{msc}{54D30}

% this is the default PlanetMath preamble.  as your knowledge
% of TeX increases, you will probably want to edit this, but
% it should be fine as is for beginners.

% almost certainly you want these
\usepackage{amssymb}
\usepackage{amsmath}
\usepackage{amsfonts}

% used for TeXing text within eps files
%\usepackage{psfrag}
% need this for including graphics (\includegraphics)
%\usepackage{graphicx}
% for neatly defining theorems and propositions
\usepackage{amsthm}
% making logically defined graphics
%%%\usepackage{xypic}

% there are many more packages, add them here as you need them
\usepackage{enumerate}

% define commands here
\newcommand{\real}{\mathbb{R}}
\newcommand{\rat}{\mathbb{Q}}
\newcommand{\nat}{\mathbb{N}}

\providecommand{\abs}[1]{\lvert#1\rvert}
\providecommand{\absW}[1]{\left\lvert#1\right\rvert}
\providecommand{\absB}[1]{\Bigl\lvert#1\Bigr\rvert}
\providecommand{\norm}[1]{\lVert#1\rVert}
\providecommand{\normW}[1]{\left\lVert#1\right\rVert}
\providecommand{\normB}[1]{\Bigl\lVert#1\Bigr\rVert}
\providecommand{\defnterm}[1]{\emph{#1}}
\begin{document}
(The finite case of Tychonoff's Theorem is of course a subset of the infinite case,
but the proof is substantially easier, so that is why it is presented here.)

To prove that $X_1 \times \dotsm \times X_n$ is compact
if the $X_i$ are compact, it suffices (by induction) to prove that $X \times Y$ is compact
when $X$ and $Y$ are.  It also suffices to prove that 
a finite subcover can be extracted from every open cover of $X \times Y$
by only the \emph{basis sets} of the form $U \times V$, where $U$ is open in $X$ and $V$ is open in $Y$.

\begin{proof}
The proof is by the straightforward strategy of composing a finite subcover
from a lower-dimensional subcover.  Let the open cover $\mathcal{C}$ of $X \times Y$
by basis sets be given.  

The set $X \times \{ y \}$ is compact, because it is the image of a 
continuous embedding of the compact set $X$.
Hence $X \times \{ y \}$ has a finite subcover in $\mathcal{C}$: label the subcover
by $\mathcal{S}^y = \{ U_1^y \times V_1^y, \dotsc, U_{k_y}^y \times V_{k_y}^y \}$.
Do this for each $y \in Y$.

To get the desired subcover of $X \times Y$, we need to pick a finite number
of $y \in Y$.  Consider $V^y = \bigcap_{i=1}^{k_y} V_i^y$.
This is a finite intersection of open sets, so $V^y$ is open in $Y$.
The collection $\{ V^y : y \in Y\}$ is an open covering of $Y$, so pick
a finite subcover $ V^{y_1}, \dotsc, V^{y_l}$.
Then $\bigcup_{j=1}^l \mathcal{S}^{y_j}$ is a finite subcover
of $X \times Y$.
\end{proof}
%%%%%
%%%%%
\end{document}
