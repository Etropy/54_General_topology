\documentclass[12pt]{article}
\usepackage{pmmeta}
\pmcanonicalname{QuasimetricSpace}
\pmcreated{2013-03-22 14:40:21}
\pmmodified{2013-03-22 14:40:21}
\pmowner{mathcam}{2727}
\pmmodifier{mathcam}{2727}
\pmtitle{quasimetric space}
\pmrecord{8}{36274}
\pmprivacy{1}
\pmauthor{mathcam}{2727}
\pmtype{Definition}
\pmcomment{trigger rebuild}
\pmclassification{msc}{54E35}
\pmsynonym{quasi-metric space}{QuasimetricSpace}
\pmrelated{PseudometricSpace}
\pmrelated{MetricSpace}
\pmrelated{GeneralizationOfAPseudometric}
\pmdefines{quasimetric}
\pmdefines{quasi-metric}

\endmetadata

% this is the default PlanetMath preamble.  as your knowledge
% of TeX increases, you will probably want to edit this, but
% it should be fine as is for beginners.

% almost certainly you want these
\usepackage{amssymb}
\usepackage{amsmath}
\usepackage{amsfonts}
\usepackage{amsthm}

% used for TeXing text within eps files
%\usepackage{psfrag}
% need this for including graphics (\includegraphics)
%\usepackage{graphicx}
% for neatly defining theorems and propositions
%\usepackage{amsthm}
% making logically defined graphics
%%%\usepackage{xypic}

% there are many more packages, add them here as you need them

% define commands here

\newcommand{\mc}{\mathcal}
\newcommand{\mb}{\mathbb}
\newcommand{\mf}{\mathfrak}
\newcommand{\ol}{\overline}
\newcommand{\ra}{\rightarrow}
\newcommand{\la}{\leftarrow}
\newcommand{\La}{\Leftarrow}
\newcommand{\Ra}{\Rightarrow}
\newcommand{\nor}{\vartriangleleft}
\newcommand{\Gal}{\text{Gal}}
\newcommand{\GL}{\text{GL}}
\newcommand{\Z}{\mb{Z}}
\newcommand{\R}{\mb{R}}
\newcommand{\Q}{\mb{Q}}
\newcommand{\C}{\mb{C}}
\newcommand{\<}{\langle}
\renewcommand{\>}{\rangle}
\begin{document}
A {\em quasimetric space} $(X,d)$ is a set $X$ together with a non-negative real-valued function $d: X \times X \longrightarrow \mathbb{R}$ (called a {\em quasimetric}) such that, for every $x,y,z \in X$,
\begin{itemize}
\item $d(x,y)\geq 0$ with equality if and only if $x=y$.
\item $d(x,z) \leq d(x,y) + d(y,z)$
\end{itemize}

In other words, a quasimetric space is a generalization of a metric space in which we drop the requirement that, for two points $x$ and $y$, the ``distance'' between $x$ and $y$ is the same as the ``distance'' between $y$ and $x$ (i.e. the symmetry axiom of metric spaces).

Some properties:
\begin{itemize}
\item If $(X,d)$ is a quasimetric space, we can form a metric space $(X,d')$ where $d'$ is defined for all $x,y\in X$ by
\begin{align*}
d'(x,y) = \frac{1}{2}(d(x,y)+d(y,x)).
\end{align*}
\item Every metric space is trivially a quasimetric space.
\item A quasimetric that is \PMlinkescapetext{symmetric} (i.e. \PMlinkescapetext{satisfies} $d(x,y)=d(y,x)$ for all $x,y\in X$ is a metric.
\end{itemize}

\begin{thebibliography}{9}
\bibitem{steen} L.A. Steen, J.A.Seebach, Jr.,
\emph{Counterexamples in topology},
Holt, Rinehart and Winston, Inc., 1970.
\bibitem{shen}
Z. Shen, \emph{Lectures of Finsler geometry}, World Sientific, 2001.
\end{thebibliography}
%%%%%
%%%%%
\end{document}
