\documentclass[12pt]{article}
\usepackage{pmmeta}
\pmcanonicalname{FiniteComplementTopology}
\pmcreated{2013-03-22 14:37:54}
\pmmodified{2013-03-22 14:37:54}
\pmowner{mathcam}{2727}
\pmmodifier{mathcam}{2727}
\pmtitle{finite complement topology}
\pmrecord{8}{36213}
\pmprivacy{1}
\pmauthor{mathcam}{2727}
\pmtype{Definition}
\pmcomment{trigger rebuild}
\pmclassification{msc}{54A05}
\pmsynonym{cofinite topology}{FiniteComplementTopology}
\pmrelated{CofiniteTopology}
\pmrelated{CofiniteAndCocountableTopology}

\endmetadata

% this is the default PlanetMath preamble.  as your knowledge
% of TeX increases, you will probably want to edit this, but
% it should be fine as is for beginners.

% almost certainly you want these
\usepackage{amssymb}
\usepackage{amsmath}
\usepackage{amsfonts}
\usepackage{amsthm}

% used for TeXing text within eps files
%\usepackage{psfrag}
% need this for including graphics (\includegraphics)
%\usepackage{graphicx}
% for neatly defining theorems and propositions
%\usepackage{amsthm}
% making logically defined graphics
%%%\usepackage{xypic}

% there are many more packages, add them here as you need them

% define commands here

\newcommand{\mc}{\mathcal}
\newcommand{\mb}{\mathbb}
\newcommand{\mf}{\mathfrak}
\newcommand{\ol}{\overline}
\newcommand{\ra}{\rightarrow}
\newcommand{\la}{\leftarrow}
\newcommand{\La}{\Leftarrow}
\newcommand{\Ra}{\Rightarrow}
\newcommand{\nor}{\vartriangleleft}
\newcommand{\Gal}{\text{Gal}}
\newcommand{\GL}{\text{GL}}
\newcommand{\Z}{\mb{Z}}
\newcommand{\R}{\mb{R}}
\newcommand{\Q}{\mb{Q}}
\newcommand{\C}{\mb{C}}
\newcommand{\<}{\langle}
\renewcommand{\>}{\rangle}
\begin{document}
Let $X$ be a set.  We can define the \emph{finite complement topology} on $X$ by declaring a subset $U\subset X$ to be open if $X\backslash U$ is finite, or if $U$ is all of $X$ or the empty set.  Note that this is equivalent to defining a topology by defining the closed sets in $X$ to be all finite sets (and $X$ itself).

If $X$ is finite, the finite complement topology on $X$ is clearly the discrete topology, as the complement of \emph{any} subset is finite.

If $X$ is countably infinite (or larger), the finite complement topology gives a standard example of a space that is not Hausdorff (each open set must contain all but finitely many points, so any two open sets must intersect).

In general, the finite complement topology on an infinite set satisfies strong compactness conditions (compact, \PMlinkname{$\sigma$-compact}{SigmaCompact}, sequentially compact, etc.) since each open set in a cover contains "almost all'' of the points of $X$.  On the other hand, the finite complement topology fails all but the simplest of separation axioms since, as above, $X$ is hyperconnected under this topology.

The finite complement topology is the coarsest T1-topology on a given set.
%%%%%
%%%%%
\end{document}
