\documentclass[12pt]{article}
\usepackage{pmmeta}
\pmcanonicalname{LocallyConnected}
\pmcreated{2013-03-22 12:38:48}
\pmmodified{2013-03-22 12:38:48}
\pmowner{djao}{24}
\pmmodifier{djao}{24}
\pmtitle{locally connected}
\pmrecord{5}{32912}
\pmprivacy{1}
\pmauthor{djao}{24}
\pmtype{Definition}
\pmcomment{trigger rebuild}
\pmclassification{msc}{54D05}
\pmrelated{ConnectedSet}
\pmrelated{ConnectedSpace}
\pmrelated{PathConnected}
\pmrelated{SemilocallySimplyConnected}
\pmdefines{locally path connected}

% this is the default PlanetMath preamble.  as your knowledge
% of TeX increases, you will probably want to edit this, but
% it should be fine as is for beginners.

% almost certainly you want these
\usepackage{amssymb}
\usepackage{amsmath}
\usepackage{amsfonts}

% used for TeXing text within eps files
%\usepackage{psfrag}
% need this for including graphics (\includegraphics)
%\usepackage{graphicx}
% for neatly defining theorems and propositions
%\usepackage{amsthm}
% making logically defined graphics
%%%\usepackage{xypic} 

% there are many more packages, add them here as you need them

% define commands here
\begin{document}
A topological space $X$ is {\em locally connected} at a point $x \in X$ if every neighborhood $U$ of $x$ contains a connected neighborhood $V$ of $x$. The space $X$ is {\em locally connected} if it is locally connected at every point $x \in X$.

A topological space $X$ is {\em locally path connected} at a point $x \in X$ if every neighborhood $U$ of $x$ contains a path connected neighborhood $V$ of $x$. The space $X$ is {\em locally path connected} if it is locally path connected at every point $x \in X$.
%%%%%
%%%%%
\end{document}
