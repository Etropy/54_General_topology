\documentclass[12pt]{article}
\usepackage{pmmeta}
\pmcanonicalname{ProofOfLebesgueNumberLemma}
\pmcreated{2013-03-22 13:09:20}
\pmmodified{2013-03-22 13:09:20}
\pmowner{scanez}{1021}
\pmmodifier{scanez}{1021}
\pmtitle{proof of Lebesgue number lemma}
\pmrecord{7}{33596}
\pmprivacy{1}
\pmauthor{scanez}{1021}
\pmtype{Proof}
\pmcomment{trigger rebuild}
\pmclassification{msc}{54E45}

% this is the default PlanetMath preamble.  as your knowledge
% of TeX increases, you will probably want to edit this, but
% it should be fine as is for beginners.

% almost certainly you want these
\usepackage{amssymb}
\usepackage{amsmath}
\usepackage{amsfonts}

% used for TeXing text within eps files
%\usepackage{psfrag}
% need this for including graphics (\includegraphics)
%\usepackage{graphicx}
% for neatly defining theorems and propositions
%\usepackage{amsthm}
% making logically defined graphics
%%%\usepackage{xypic}

% there are many more packages, add them here as you need them

% define commands here
\begin{document}
By way of contradiction, suppose that no Lebesgue number existed. Then there exists an open cover $\mathcal{U}$ of $X$ such that for all $\delta > 0$ there exists an $x \in X$ such that no $U \in \mathcal{U}$ contains $B_\delta(x)$ (the open ball of radius $\delta$ around $x$). Specifically, for each $n \in \mathbb{N}$, since $1/n > 0$ we can choose an $x_n \in X$ such that no $U \in \mathcal{U}$ contains $B_{1/n}(x_n)$. Now, $X$ is compact so there exists a subsequence $(x_{n_k})$ of the sequence of points $(x_n)$ that converges to
some $y \in X$. Also, $\mathcal{U}$ being an open cover of $X$ implies that there exists $\lambda > 0$ and $U \in \mathcal{U}$ such that $B_\lambda(y) \subseteq U$. Since the sequence $(x_{n_k})$ converges to $y$, for $k$ large enough it is true that $d(x_{n_k},y) < \lambda/2$ ($d$ is the metric on $X$) and $1/n_k < \lambda/2$. Thus after an application of the triangle inequality, it follows that
\begin{displaymath}
	B_{1/n_k}(x_{n_k}) \subseteq B_\lambda(y) \subseteq U,
\end{displaymath}
contradicting the assumption that no $U \in \mathcal{U}$ contains $B_{1/n}(x_n)$. Hence a Lebesgue number for $\mathcal{U}$ does exist.
%%%%%
%%%%%
\end{document}
