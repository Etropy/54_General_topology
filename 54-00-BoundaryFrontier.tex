\documentclass[12pt]{article}
\usepackage{pmmeta}
\pmcanonicalname{BoundaryFrontier}
\pmcreated{2013-03-22 13:34:46}
\pmmodified{2013-03-22 13:34:46}
\pmowner{yark}{2760}
\pmmodifier{yark}{2760}
\pmtitle{boundary / frontier}
\pmrecord{17}{34200}
\pmprivacy{1}
\pmauthor{yark}{2760}
\pmtype{Definition}
\pmcomment{trigger rebuild}
\pmclassification{msc}{54-00}
\pmsynonym{boundary}{BoundaryFrontier}
\pmsynonym{frontier}{BoundaryFrontier}
\pmsynonym{topological boundary}{BoundaryFrontier}
\pmrelated{ExtendedBoundary}
\pmrelated{Interior}

\usepackage{amssymb}
\usepackage{amsmath}
\usepackage{amsfonts}

\def\R{\mathbb{R}}
\DeclareMathOperator{\bd}{bd}
\DeclareMathOperator{\fr}{fr}
\begin{document}
\PMlinkescapeword{term}

{\bf Definition.}
Let $X$ be a topological space and let $A$ be a subset 
of $X$. The \emph{boundary} (or \emph{frontier}) of $A$ is the set
$\partial A = \overline{A}\cap \overline{X\backslash A}$,
where the overline denotes the closure of a set.
Instead of $\partial A$, many authors use some other notation
such as $\bd(A)$, $\fr(A)$, $A^b$ or $\beta(A)$.
Note that the $\partial$ symbol is also used for other meanings of `boundary'.

From the definition, it follows that the boundary of any set is a closed set.
It also follows that $\partial A = \partial(X\backslash A)$,
and $\partial X=\varnothing=\partial\varnothing$.

The term `boundary' (but not `frontier') is used in a different sense for topological manifolds: the boundary $\partial M$ of a topological $n$-manifold $M$ is the set of points in $M$ that do not have a neighbourhood homeomorphic to $\R^n$. (Some authors define topological manifolds in such a way that they necessarily have empty boundary.)
For example, the boundary of the topological $1$-manifold $[0,1]$ is $\{0,1\}$.
%%%%%
%%%%%
\end{document}
