\documentclass[12pt]{article}
\usepackage{pmmeta}
\pmcanonicalname{GraphTopology}
\pmcreated{2013-03-22 13:37:03}
\pmmodified{2013-03-22 13:37:03}
\pmowner{mps}{409}
\pmmodifier{mps}{409}
\pmtitle{graph topology}
\pmrecord{10}{34250}
\pmprivacy{1}
\pmauthor{mps}{409}
\pmtype{Definition}
\pmcomment{trigger rebuild}
\pmclassification{msc}{54H99}
\pmclassification{msc}{05C62}
\pmclassification{msc}{05C10}
\pmsynonym{one-dimensional CW complex}{GraphTopology}
%\pmkeywords{graph topology construction}
%\pmkeywords{one-dimensional CW complex}
\pmrelated{GraphTheory}
\pmrelated{Graph}
\pmrelated{ConnectedGraph}
\pmrelated{QuotientSpace}
\pmrelated{Realization}
\pmrelated{RSupercategory}
\pmrelated{CWComplexDefinitionRelatedToSpinNetworksAndSpinFoams}

\endmetadata

% this is the default PlanetMath preamble.  as your knowledge
% of TeX increases, you will probably want to edit this, but
% it should be fine as is for beginners.

% almost certainly you want these
\usepackage{amssymb}
\usepackage{amsmath}
\usepackage{amsfonts}

% used for TeXing text within eps files
%\usepackage{psfrag}
% need this for including graphics (\includegraphics)
%\usepackage{graphicx}
% for neatly defining theorems and propositions
%\usepackage{amsthm}
% making logically defined graphics
%%%\usepackage{xypic}

% there are many more packages, add them here as you need them

% define commands here
\def\sse{\subseteq}
\def\bigtimes{\mathop{\mbox{\Huge $\times$}}}
\def\impl{\Rightarrow}
\begin{document}
A graph $(V,E)$ is identified by its vertices $V=\{v_1,v_2,\ldots\}$ and its
edges $E=\{\{v_i,v_j\},\{v_k,v_l\},\ldots\}$. A graph also admits a natural
topology, called the \emph{graph topology}, by identifying every edge
$\{v_i,v_j\}$ with the unit interval $I=[0,1]$ and gluing them together at
coincident vertices.

This construction can be easily realized in the framework of simplicial
complexes. We can form a simplicial complex $G=\left\{\{v\}\mid v\in V\right\} \cup
E$. And the desired topological realization of the graph is just the
geometric realization $|G|$ of $G$.

Viewing a graph as a topological space has several advantages:
\begin{itemize}
\item The notion of graph isomorphism becomes that of simplicial (or cell) \PMlinkname{complex}{CWComplex} isomorphism.
\item The notion of a connected graph coincides with topological
  \PMlinkname{connectedness}{ConnectedSpace}.
\item A connected graph is a tree if and only if its fundamental group is trivial.
\end{itemize}

{\bf Remark:}
A graph is/can be regarded as a one-dimensional $CW$-complex.
%%%%%
%%%%%
\end{document}
