\documentclass[12pt]{article}
\usepackage{pmmeta}
\pmcanonicalname{Hemimetric}
\pmcreated{2013-03-22 14:24:12}
\pmmodified{2013-03-22 14:24:12}
\pmowner{Koro}{127}
\pmmodifier{Koro}{127}
\pmtitle{hemimetric}
\pmrecord{5}{35903}
\pmprivacy{1}
\pmauthor{Koro}{127}
\pmtype{Definition}
\pmcomment{trigger rebuild}
\pmclassification{msc}{54E25}

\endmetadata

% this is the default PlanetMath preamble.  as your knowledge
% of TeX increases, you will probably want to edit this, but
% it should be fine as is for beginners.

% almost certainly you want these
\usepackage{amssymb}
\usepackage{amsmath}
\usepackage{amsfonts}
\usepackage{mathrsfs}

% used for TeXing text within eps files
%\usepackage{psfrag}
% need this for including graphics (\includegraphics)
%\usepackage{graphicx}
% for neatly defining theorems and propositions
%\usepackage{amsthm}
% making logically defined graphics
%%%\usepackage{xypic}

% there are many more packages, add them here as you need them

% define commands here
\newcommand{\C}{\mathbb{C}}
\newcommand{\R}{\mathbb{R}}
\newcommand{\N}{\mathbb{N}}
\newcommand{\Z}{\mathbb{Z}}
\newcommand{\Per}{\operatorname{Per}}
\begin{document}
A \emph{hemimetric} on a set $X$ is a function
$d\colon X\times X\to \R$ such that
\begin{enumerate}
\item $d(x,y)\geq 0$;
\item $d(x,z) \leq d(x,y) + d(y,z)$;
\item $d(x,x) = 0$;
\end{enumerate}
for all $x,y,z\in X$.

Hence, essentially $d$ is a metric which fails to satisfy symmetry and the property that distinct points have positive distance.
A hemimetric induces a topology on $X$ in the same way that a metric does, a basis of open sets being
$$\{B(x,r): x\in X, r>0\},$$
where $B(x,r)=\{y\in X : d(x,y)<r\}$ is the $r$-ball centered at $x$.
%%%%%
%%%%%
\end{document}
