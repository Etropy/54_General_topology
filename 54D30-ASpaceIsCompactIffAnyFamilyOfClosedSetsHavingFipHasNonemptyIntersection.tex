\documentclass[12pt]{article}
\usepackage{pmmeta}
\pmcanonicalname{ASpaceIsCompactIffAnyFamilyOfClosedSetsHavingFipHasNonemptyIntersection}
\pmcreated{2013-03-22 13:34:10}
\pmmodified{2013-03-22 13:34:10}
\pmowner{CWoo}{3771}
\pmmodifier{CWoo}{3771}
\pmtitle{a space is compact iff any family of closed sets having fip has non-empty intersection}
\pmrecord{20}{34181}
\pmprivacy{1}
\pmauthor{CWoo}{3771}
\pmtype{Theorem}
\pmcomment{trigger rebuild}
\pmclassification{msc}{54D30}

\endmetadata

% this is the default PlanetMath preamble.  as your knowledge
% of TeX increases, you will probably want to edit this, but
% it should be fine as is for beginners.

% almost certainly you want these
\usepackage{amssymb}
\usepackage{amsmath}
\usepackage{amsfonts}

% used for TeXing text within eps files
%\usepackage{psfrag}
% need this for including graphics (\includegraphics)
%\usepackage{graphicx}
% for neatly defining theorems and propositions
%\usepackage{amsthm}
% making logically defined graphics
%%%\usepackage{xypic}

% there are many more packages, add them here as you need them

% define commands here
\begin{document}
{\bf Theorem. }
A topological space is compact if and only if any collection of its closed sets having the finite intersection property has non-empty intersection.

The above theorem is essentially
the definition of a compact space rewritten using de Morgan's laws.
The usual definition of a compact space is based on open sets and 
unions. The above characterization, on the other hand, is written 
using closed sets and intersections. 

\emph{Proof.} Suppose $X$ is compact, i.e., any collection of open subsets
that cover $X$ has a finite collection that also cover $X$. Further, suppose
$\{F_i\}_{i\in I}$ is an arbitrary collection of closed subsets
with the finite intersection property. We claim that $\cap_{i\in I} F_i$
is non-empty. 
Suppose otherwise, i.e., suppose $\cap_{i\in I} F_i=\emptyset$. Then, 
\begin{eqnarray*}
X&=&\left(\bigcap_{i\in I} F_i\right)^c\\
 &=&\bigcup_{i\in I} F_i^c.
\end{eqnarray*}
(Here, the complement of a set $A$ in $X$ is written as $A^c$.)
Since each $F_i$ is closed, the collection $\{F_i^c\}_{i\in I}$
is an open cover for $X$. By compactness, there is  a 
finite subset $J\subset I$ such
that $X=\cup_{i\in J} F_i^c$. But then 
$X=(\cap_{i\in J} F_i)^c$, so $\cap_{i\in J} F_i=\emptyset$, which 
contradicts the finite intersection property of $\{F_i\}_{i\in I}$. 

The proof in the other direction is analogous. 
Suppose $X$ has the finite intersection property.
To prove that
$X$ is compact, let $\{F_i\}_{i\in I}$ be a collection of open sets
in $X$ that cover $X$. We claim that this collection contains a finite subcollection
of sets that also cover $X$. 
The proof is by contradiction. 
Suppose
that $X\neq \cup_{i\in J} F_i$ holds for all finite $J\subset I$.
Let us first show that the collection of closed subsets 
$\{F_i^c\}_{i\in I}$ has the finite intersection property. 
If $J$ is a finite subset of $I$, then 
\begin{eqnarray*}
\bigcap_{i\in J} F^c_i &=& \Big(\bigcup_{i\in J} F_i\Big)^c \neq \emptyset,
\end{eqnarray*}
where the last assertion follows since $J$ was finite. 
Then, since $X$ has the finite intersection property, 
\begin{eqnarray*}
\emptyset &\neq& \bigcap_{i\in I} F_i^c = \Big(\bigcup_{i\in I} F_i\Big)^c.
\end{eqnarray*}
This contradicts the assumption that $\{F_i\}_{i\in I}$ is a cover for $X$. 
$\Box$

\begin{thebibliography}{9}
\bibitem{edwards} R.E. Edwards, \emph{Functional Analysis: Theory and Applications}, Dover Publications, 1995.
\end{thebibliography}
%%%%%
%%%%%
\end{document}
