\documentclass[12pt]{article}
\usepackage{pmmeta}
\pmcanonicalname{TychonoffsTheorem}
\pmcreated{2013-03-22 12:05:14}
\pmmodified{2013-03-22 12:05:14}
\pmowner{matte}{1858}
\pmmodifier{matte}{1858}
\pmtitle{Tychonoff's theorem}
\pmrecord{12}{31168}
\pmprivacy{1}
\pmauthor{matte}{1858}
\pmtype{Theorem}
\pmcomment{trigger rebuild}
\pmclassification{msc}{54D30}
\pmsynonym{Tichonov's theorem}{TychonoffsTheorem}
\pmrelated{Compact}

\endmetadata

\usepackage{amssymb}
\usepackage{amsmath}
\usepackage{amsfonts}
\begin{document}
Let $(X_i)_{i\in I}$ be a family of nonempty topological spaces. The product space (see product topology)
$$\prod_{i\in I}X_i$$
is compact if and only if each of the spaces $X_i$ is compact.

Not surprisingly, if $I$ is infinite, the proof requires the Axiom of Choice. Conversely, one can show that Tychonoff's theorem implies that any product of nonempty sets is nonempty, which is one form of the Axiom of Choice.
%%%%%
%%%%%
%%%%%
\end{document}
