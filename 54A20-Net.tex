\documentclass[12pt]{article}
\usepackage{pmmeta}
\pmcanonicalname{Net}
\pmcreated{2013-03-22 12:54:03}
\pmmodified{2013-03-22 12:54:03}
\pmowner{yark}{2760}
\pmmodifier{yark}{2760}
\pmtitle{net}
\pmrecord{12}{33250}
\pmprivacy{1}
\pmauthor{yark}{2760}
\pmtype{Definition}
\pmcomment{trigger rebuild}
\pmclassification{msc}{54A20}
\pmsynonym{Moore-Smith sequence}{Net}
\pmrelated{Filter}
\pmrelated{NetsAndClosuresOfSubspaces}
\pmrelated{ContinuityAndConvergentNets}
\pmrelated{CompactnessAndConvergentSubnets}
\pmrelated{AccumulationPointsAndConvergentSubnets}
\pmrelated{TestingForContinuityViaNets}
\pmdefines{subnet}
\pmdefines{Moore-Smith convergence}
\pmdefines{cluster point}

\endmetadata

%\usepackage{amssymb}
%\usepackage{amsmath}
%\usepackage{amsfonts}

\begin{document}
\PMlinkescapeword{sequences}

Let $X$ be a set.  A \emph{net} is a map from a directed set to $X$.
In other words, it is a pair $(A,\gamma)$
where $A$ is a directed set and $\gamma$ is a map from $A$ to $X$.
If $a\in A$ then $\gamma(a)$ is normally written $x_a$,
and then the net is written $(x_a)_{a\in A}$,
or simply $(x_a)$ if the direct set $A$ is understood.

Now suppose $X$ is a topological space, $A$ is a directed set,
and $(x_a)_{a\in A}$ is a net.  Let $x\in X$.
Then $(x_a)$ is said to \emph{converge} to $x$ if
whenever $U$ is an open neighbourhood of $x$,
there is some $b \in A$ such that $x_a \in U$ whenever $a \geq b$.

Similarly, $x$ is said to be an \emph{accumulation point} (or \emph{cluster point})
of $(x_a)$ if whenever $U$ is an open neighbourhood of $x$ and $b \in A$
there is $a \in A$ such that $a \geq b$ and $x_a \in U$.

Nets are sometimes called \emph{Moore--Smith sequences},
in which case convergence of nets may be called \emph{Moore--Smith convergence}.

If $B$ is another directed set,
and $\delta\colon B\rightarrow A$ is an increasing map
such that $\delta(B)$ is cofinal in $A$,
then the pair $(B, \gamma\circ\delta)$
is said to be a \emph{subnet} of $(A,\gamma)$.
Alternatively, a subnet of a net $(x_\alpha)_{\alpha\in A}$
is sometimes defined to be a net $(x_{\alpha_\beta})_{\beta\in B}$
such that for each $\alpha_0\in A$
there exists a $\beta_0\in B$
such that $\alpha_\beta\geq\alpha_0$ for all $\beta\geq\beta_0$.

Nets are a generalisation of \PMlinkname{sequences}{Sequence},
and in many respects they work better in arbitrary topological spaces
than sequences do. For example:

\begin{itemize}
\item
If $X$ is Hausdorff then any net in $X$ converges to at most one point.
\item
If $Y$ is a subspace of $X$
then $x\in\overline{Y}$ if and only if there is a net in $Y$ converging to $x$.
\item
if $X'$ is another topological space
and $f\colon X\rightarrow X'$ is a map,
then $f$ is continuous at $x$ if and only if
whenever $(x_a)$ is a net converging to $x$,
$(f(x_a))$ is a net converging to $f(x)$.
\item
$X$ is compact if and only if every net has a convergent subnet.
\end{itemize}
%%%%%
%%%%%
\end{document}
