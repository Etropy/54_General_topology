\documentclass[12pt]{article}
\usepackage{pmmeta}
\pmcanonicalname{SubspaceTopology}
\pmcreated{2013-03-22 11:53:22}
\pmmodified{2013-03-22 11:53:22}
\pmowner{djao}{24}
\pmmodifier{djao}{24}
\pmtitle{subspace topology}
\pmrecord{8}{30499}
\pmprivacy{1}
\pmauthor{djao}{24}
\pmtype{Definition}
\pmcomment{trigger rebuild}
\pmclassification{msc}{54B05}
\pmclassification{msc}{15A66}
\pmclassification{msc}{11E88}
\pmsynonym{relative topology}{SubspaceTopology}
\pmdefines{topological subspace}
\pmdefines{subspace}

\usepackage{amssymb}
\usepackage{amsmath}
\usepackage{amsfonts}
\usepackage{graphicx}
%%%%\usepackage{xypic}
\begin{document}
Let $X$ be a topological space, and let $Y \subset X$ be a subset. The {\em subspace topology} on $Y$ is the topology whose open sets are those subsets of $Y$ which equal $U \cap Y$ for some open set $U \subset X$.

In this context, the topological space $Y$ obtained by taking the subspace topology is called a {\em topological subspace}, or simply {\em subspace}, of $X$.
%%%%%
%%%%%
%%%%%
%%%%%
\end{document}
