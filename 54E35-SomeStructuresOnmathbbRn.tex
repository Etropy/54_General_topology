\documentclass[12pt]{article}
\usepackage{pmmeta}
\pmcanonicalname{SomeStructuresOnmathbbRn}
\pmcreated{2013-03-22 14:03:47}
\pmmodified{2013-03-22 14:03:47}
\pmowner{drini}{3}
\pmmodifier{drini}{3}
\pmtitle{some structures on $\mathbb{R}^n$}
\pmrecord{10}{35419}
\pmprivacy{1}
\pmauthor{drini}{3}
\pmtype{Definition}
\pmcomment{trigger rebuild}
\pmclassification{msc}{54E35}
\pmclassification{msc}{53A99}

\usepackage{graphicx}
%%%\usepackage{xypic} 
\usepackage{bbm}
\newcommand{\sZ}{\mathbbmss{Z}}
\newcommand{\sC}{\mathbbmss{C}}
\newcommand{\sR}{\mathbbmss{R}}
\newcommand{\sQ}{\mathbbmss{Q}}
\newcommand{\mathbb}[1]{\mathbbmss{#1}}
\newcommand{\figura}[1]{\begin{center}\includegraphics{#1}\end{center}}
\newcommand{\figuraex}[2]{\begin{center}\includegraphics[#2]{#1}\end{center}}
\begin{document}
Let $n\in \{1,2,\ldots\}$. Then, as a set, $\sR^n$ is the $n$-fold Cartesian 
product of the real numbers. 

\subsubsection{Vector space structure of $\sR^n$}
If $u=(u_1, \ldots, u_n)$ and 
$v=(v_1, \ldots, v_n)$ are points in $\sR^n$, we define their sum
as
$$ u+v=(u_1+v_1, \ldots, u_n+v_n).$$
Also, if $\lambda$ is a scalar (real number), then scalar multiplication
is defined as 
$$ \lambda \cdot u = (\lambda u_1, \ldots, \lambda u_n).$$
With these operations, $\sR^n$ becomes a vector space (over $\sR$) with dimension $n$.
In other words, with this structure, we can talk about, vectors, lines, subspaces of different dimension. 


\subsubsection{Inner product for $\sR^n$}
For $u$ and $v$ as above, we define the inner product as
$$ \langle u,v \rangle = u_1 v_1 + \cdots + u_n v_n.$$
With this product, $\sR^n$ is called an Euclidean space. 

We have also an induced norm $\left\Vert u \right\Vert =  \sqrt{\langle u,u \rangle}$, which gives
$\sR^n$ the structure of a normed space (and thus metric space).
This inner product let us talk about length, angle between vectors, orthogonal vectors.

\subsubsection{Topology for $\sR^n$}
The usual topology for $\sR^n$ is the topology induced by the metric
$$d(x,y) = \Vert x-y\Vert.$$
As a basis for the topology induced by the above norm, one can take
 open balls $B(x,r)=\{ y\in \sR^n \mid \left\Vert x-y \right\Vert < r\}$ where $r>0$
and $x\in\sR^n$.


Properties of the topological space $\sR^n$ are:
\begin{enumerate}
\item $\sR^n$ is second countable, i.e.,  $\sR^n$ has  a countable basis.
\item (Heine-Borel theorem) 
A set in $\sR^n$ is compact if and only if it is closed and bounded. 
\item Since $\sR^n$ is a metric space, $\sR^n$ is a Hausdorff space. 
\end{enumerate}
%%%%%
%%%%%
\end{document}
