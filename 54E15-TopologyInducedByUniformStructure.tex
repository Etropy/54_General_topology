\documentclass[12pt]{article}
\usepackage{pmmeta}
\pmcanonicalname{TopologyInducedByUniformStructure}
\pmcreated{2013-03-22 12:46:44}
\pmmodified{2013-03-22 12:46:44}
\pmowner{Mathprof}{13753}
\pmmodifier{Mathprof}{13753}
\pmtitle{topology induced by uniform structure}
\pmrecord{7}{33091}
\pmprivacy{1}
\pmauthor{Mathprof}{13753}
\pmtype{Derivation}
\pmcomment{trigger rebuild}
\pmclassification{msc}{54E15}
\pmrelated{UniformNeighborhood}
\pmdefines{uniform topology}

\endmetadata

% this is the default PlanetMath preamble.  as your knowledge
% of TeX increases, you will probably want to edit this, but
% it should be fine as is for beginners.

% almost certainly you want these
\usepackage{amssymb}
\usepackage{amsmath}
\usepackage{amsfonts}

% used for TeXing text within eps files
%\usepackage{psfrag}
% need this for including graphics (\includegraphics)
%\usepackage{graphicx}
% for neatly defining theorems and propositions
%\usepackage{amsthm}
% making logically defined graphics
%%%\usepackage{xypic}

% there are many more packages, add them here as you need them

% define commands here
\begin{document}
Let $\mathcal{U}$ be a uniform structure on a set $X$. We define a subset $A$ to be open if and only if for each $x \in A$ there exists an entourage $U \in \mathcal{U}$ such that whenever $(x,y) \in U$, then $y \in A$.

Let us verify that this defines a topology on $X$.

Clearly, the subsets $\emptyset$ and $X$ are open. If $A$ and $B$ are two open sets, then for each $x \in A \cap B$, there exist an entourage $U$ such that, whenever $(x,y) \in U$, then $y \in A$, and an entourage $V$ such that, whenever $(x,y) \in V$, then $y \in B$. Consider the entourage $U \cap V$: whenever $(x,y) \in U\cap V$, then $y \in A \cap B$, hence $A \cap B$ is open.

Suppose $\mathcal{F}$ is an arbitrary family of open subsets. For each $x \in \bigcup \mathcal{F}$, there exists $A \in \mathcal{F}$ such that $x \in A$. Let $U$ be the entourage whose existence is granted by the definition of open set. We have that whenever $(x,y) \in U$, then $y \in A$; hence $y \in \bigcup \mathcal{F}$, which concludes the proof.
%%%%%
%%%%%
\end{document}
