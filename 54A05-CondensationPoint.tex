\documentclass[12pt]{article}
\usepackage{pmmeta}
\pmcanonicalname{CondensationPoint}
\pmcreated{2013-03-22 16:40:45}
\pmmodified{2013-03-22 16:40:45}
\pmowner{sauravbhaumik}{15615}
\pmmodifier{sauravbhaumik}{15615}
\pmtitle{condensation point}
\pmrecord{9}{38887}
\pmprivacy{1}
\pmauthor{sauravbhaumik}{15615}
\pmtype{Definition}
\pmcomment{trigger rebuild}
\pmclassification{msc}{54A05}

\endmetadata

% this is the default PlanetMath preamble.  as your knowledge
% of TeX increases, you will probably want to edit this, but
% it should be fine as is for beginners.

% almost certainly you want these
\usepackage{amssymb}
\usepackage{amsmath}
\usepackage{amsfonts}
\usepackage{mathrsfs}
% used for TeXing text within eps files
%\usepackage{psfrag}
% need this for including graphics (\includegraphics)
%\usepackage{graphicx}
% for neatly defining theorems and propositions
%\usepackage{amsthm}
% making logically defined graphics
%%%\usepackage{xypic}

% there are many more packages, add them here as you need them

% define commands here

\begin{document}
Let $X$ be a topological space and $A\subset X$. A point $x\in X$ is called a {\em condensation point} of $A$ if every open neighbourhood of $x$ contains \emph{uncountably} many points of $A$.

For example, if $X=\mathbb R$ and $A$ any subset, then any accumulation point of $A$ is automatically a condensation point. But if $X=\mathbb Q$ and $A$ any subset, then $A$ does not have any condensation points at all. 

We have further classifications of {\em condensation point} where the topological space is an ordered field. Namely,
\begin{enumerate}
\item {\em unilateral condensation point:} $x$ is a condensation point of $A$ and there is a positive $\epsilon$ with either $(x-\epsilon,x)\cap A$ countable or $(x,x+\epsilon)\cap A$ countable.
\item {\em bilateral condensation point:} For all $\epsilon>0$, we have both $(x-\epsilon,x)\cap A$ and $(x,x+\epsilon)\cap A$ uncountable.


If $\kappa$ is any cardinal (i.e. an ordinal which is the least among all ordinals of the same cardinality as itself), then a $\kappa$-condensation point can be defined similarly. 
\end{enumerate}

%%%%%
%%%%%
\end{document}
