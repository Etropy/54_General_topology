\documentclass[12pt]{article}
\usepackage{pmmeta}
\pmcanonicalname{BoxTopology}
\pmcreated{2013-03-22 12:46:55}
\pmmodified{2013-03-22 12:46:55}
\pmowner{yark}{2760}
\pmmodifier{yark}{2760}
\pmtitle{box topology}
\pmrecord{9}{33095}
\pmprivacy{1}
\pmauthor{yark}{2760}
\pmtype{Definition}
\pmcomment{trigger rebuild}
\pmclassification{msc}{54A99}
\pmsynonym{box product topology}{BoxTopology}
\pmrelated{ProductTopology}
\pmdefines{box product}

\usepackage{amssymb}
\usepackage{amsmath}
\usepackage{amsfonts}

\def\T{{\mathcal T}}
\def\B{{\mathcal B}}
\def\S{{\mathcal S}}
\def\bigtimes{\mathop{\mbox{\Huge $\times$}}}
\begin{document}
\PMlinkescapeword{index}

Let $\{ (X_\alpha,\T_\alpha) \}_{\alpha\in A}$ 
be a family of topological spaces. 
Let $Y$ denote the generalized Cartesian product of the sets $X_\alpha$,
that is
\[
  Y = \prod_{\alpha\in A} X_\alpha.
\]
Let $\B$ denote the set of all products of open sets of the corresponding
spaces, that is
\[
  \B = \left\{ \prod_{\alpha\in A} U_\alpha \,\Biggm|\,
       U_\alpha\in\T_\alpha \text{ for all } \alpha\in A \right\}.
\]

Now we can construct the \emph{box product} $(Y,\S)$, where $\S$, 
referred to as the {\em box topology}, 
is the topology \PMlinkescapetext{generated by} the base $\B$.

When $A$ is a \PMlinkname{finite}{Finite} set, 
the box topology coincides with the product topology.

\section*{Example}

As an example,
the box product of two topological spaces $(X_0,\T_0)$ and $(X_1,\T_1)$
is $(X_0\times X_1,\S)$,
where the box topology $\S$ (which is the same as the product topology)
consists of all sets of the form
$\bigcup_{i\in I}(U_i\times V_i)$,
where $I$ is some index set
and for each $i\in I$ we have $U_i\in\T_0$ and $V_i\in\T_1$.
%%%%%
%%%%%
\end{document}
