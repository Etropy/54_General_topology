\documentclass[12pt]{article}
\usepackage{pmmeta}
\pmcanonicalname{NetsAndClosuresOfSubspaces}
\pmcreated{2013-03-22 17:18:34}
\pmmodified{2013-03-22 17:18:34}
\pmowner{azdbacks4234}{14155}
\pmmodifier{azdbacks4234}{14155}
\pmtitle{nets and closures of subspaces}
\pmrecord{11}{39658}
\pmprivacy{1}
\pmauthor{azdbacks4234}{14155}
\pmtype{Theorem}
\pmcomment{trigger rebuild}
\pmclassification{msc}{54A20}
%\pmkeywords{net}
%\pmkeywords{closure}
%\pmkeywords{subspace}
\pmrelated{Net}
\pmrelated{DirectedSet}
\pmrelated{PartialOrder}

\endmetadata

%%packages
\usepackage{amssymb}
\usepackage{amsmath}
\usepackage{amsfonts}
\usepackage{amsthm}
%%theorem environments
\theoremstyle{plain}
\newtheorem*{theorem*}{Theorem}
\newtheorem*{lemma*}{Lemma}
\newtheorem*{corollary*}{Corollary}
\newtheorem*{proposition*}{Proposition}
%math operators and commands
\newcommand{\set}[1]{\{#1\}}
\newcommand{\medset}[1]{\left\{#1\right\}}
\newcommand{\bigset}[1]{\bigg\{#1\bigg\}}
\newcommand{\abs}[1]{\vert#1\vert}
\newcommand{\medabs}[1]{\left\vert#1\right\vert}
\newcommand{\bigabs}[1]{\bigg\vert#1\bigg\vert}
\newcommand{\norm}[1]{\Vert#1\Vert}
\newcommand{\mednorm}[1]{\left\Vert#1\right\Vert}
\newcommand{\bignorm}[1]{\bigg\Vert#1\bigg\Vert}
\DeclareMathOperator{\Aut}{Aut}
\DeclareMathOperator{\End}{End}
\DeclareMathOperator{\Inn}{Inn}
\DeclareMathOperator{\supp}{supp}

\begin{document}
\begin{theorem*}
A point of a topological space is in the closure of a subspace if and only if there is a net of points of the subspace converging to the point.
\end{theorem*}
\begin{proof}
Let $X$ be a topological space, $x$ a point of $X$, and $A$ a subspace of $X$.
Suppose first that $x\in\bar{A}$, and let $\mathcal{U}$ be the collection of neighborhoods of $x$, \PMlinkid{partially ordered}{123} by reverse \PMlinkescapetext{inclusion}. For each $U\in\mathcal{U}$, select a point $x_U\in U\cap A$ (such a point is guaranteed to exist because $x\in\bar{A}$); then $(x_U)_{U\in\mathcal{U}}$ is a net of points in $A$, and we claim that $x_U\rightarrow x$. To see this, let $V$ be a neighborhood of $x$ in $X$, and note that, by construction, $x_V\in V$; furthermore, if $U\in\mathcal{U}$ satisfies $V\supset U$, then because $x_U\in U$, $x_U\in V$. It follows that $x_U\rightarrow x$.
Conversely, suppose there exists a net $(x_\alpha)_{\alpha\in J}$ of points of $A$ converging to $x$, and let $U\subset X$ be a neighborhood of $x$. Since $x_\alpha\rightarrow x$, there exists $\beta\in J$ such that $x_\alpha\in U$ whenever $\beta\preceq\alpha$. Because $x_\alpha\in A$ for each $\alpha\in J$ by hypothesis, we may conclude that $U\cap A\neq\emptyset$, hence that $x\in\bar{A}$.
\end{proof}
The forward implication of the preceding \PMlinkescapetext{theorem} is a generalization of the result that a point of a topological space is in the closure of a subspace if there is a \emph{sequence} of points of the subspace converging to the point, as a sequence is just a net with the positive integers as its \PMlinkid{domain}{360}; however, the converse (if a point is in the closure of a subspace then there exists a sequence of points of the subspace converging to the point) requires the additional condition that the ambient topological space be first countable. 
%%%%%
%%%%%
\end{document}
