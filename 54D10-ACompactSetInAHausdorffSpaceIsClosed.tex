\documentclass[12pt]{article}
\usepackage{pmmeta}
\pmcanonicalname{ACompactSetInAHausdorffSpaceIsClosed}
\pmcreated{2013-03-22 13:34:31}
\pmmodified{2013-03-22 13:34:31}
\pmowner{mathcam}{2727}
\pmmodifier{mathcam}{2727}
\pmtitle{a compact set in a Hausdorff space is closed}
\pmrecord{6}{34194}
\pmprivacy{1}
\pmauthor{mathcam}{2727}
\pmtype{Theorem}
\pmcomment{trigger rebuild}
\pmclassification{msc}{54D10}
\pmclassification{msc}{54D30}
\pmrelated{ClosedSubsetsOfACompactSetAreCompact}

% this is the default PlanetMath preamble.  as your knowledge
% of TeX increases, you will probably want to edit this, but
% it should be fine as is for beginners.

% almost certainly you want these
\usepackage{amssymb}
\usepackage{amsmath}
\usepackage{amsfonts}

% used for TeXing text within eps files
%\usepackage{psfrag}
% need this for including graphics (\includegraphics)
%\usepackage{graphicx}
% for neatly defining theorems and propositions
%\usepackage{amsthm}
% making logically defined graphics
%%%\usepackage{xypic}

% there are many more packages, add them here as you need them

% define commands here
\begin{document}
\newcommand{\comp}[0]{\complement}

{\bf Theorem.} A compact set in a Hausdorff space is closed.

\emph{Proof.}
Let $A$ be a compact set in a Hausdorff space $X$.
The case when $A$ is empty is trivial, so let us
assume that $A$ is non-empty.
Using \PMlinkname{this theorem}{APointAndACompactSetInAHausdorffSpaceHaveDisjointOpenNeighborhoods}, 
it follows that each point
$y$ in $A^{\comp}$ has a neighborhood $U_y$, which
is disjoint to $A$. (Here, we denote the complement of $A$
by $A^{\comp}$.)
We can therefore write
\begin{eqnarray*}
A^{\comp} &=& \bigcup_{y\in A^{\comp}} U_y.
\end{eqnarray*}
Since an arbitrary union of open sets is open, it follows that $A$ is
closed. $\Box$


{\bf Note.} \\
The above theorem can, for instance, be found in \cite{kelley} (page 141),
or \cite{singer} (Section 2.1, Theorem 2).

\begin{thebibliography}{9}
\bibitem{kelley}
J.L. Kelley,
\emph{General Topology},
D. van Nostrand Company, Inc., 1955.
\bibitem{singer}
I.M. Singer, J.A.Thorpe,
\emph{Lecture Notes on Elementary Topology and Geometry},
Springer-Verlag, 1967.
\end{thebibliography}
%%%%%
%%%%%
\end{document}
