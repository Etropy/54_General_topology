\documentclass[12pt]{article}
\usepackage{pmmeta}
\pmcanonicalname{CompleteUniformSpace}
\pmcreated{2013-03-22 16:41:40}
\pmmodified{2013-03-22 16:41:40}
\pmowner{CWoo}{3771}
\pmmodifier{CWoo}{3771}
\pmtitle{complete uniform space}
\pmrecord{7}{38907}
\pmprivacy{1}
\pmauthor{CWoo}{3771}
\pmtype{Definition}
\pmcomment{trigger rebuild}
\pmclassification{msc}{54E15}
\pmsynonym{semicomplete}{CompleteUniformSpace}
\pmsynonym{semi-complete}{CompleteUniformSpace}
\pmrelated{Complete}
\pmdefines{Cauchy filter}
\pmdefines{Cauchy sequence}
\pmdefines{sequentially complete}
\pmdefines{complete uniformity}

\usepackage{amssymb,amscd}
\usepackage{amsmath}
\usepackage{amsfonts}

% used for TeXing text within eps files
%\usepackage{psfrag}
% need this for including graphics (\includegraphics)
%\usepackage{graphicx}
% for neatly defining theorems and propositions
%\usepackage{amsthm}
% making logically defined graphics
%%\usepackage{xypic}
\usepackage{pst-plot}
\usepackage{psfrag}

% define commands here

\begin{document}
\PMlinkescapeword{complete}

Let $X$ be a uniform space with uniformity $\mathcal{U}$.  A filter $\mathcal{F}$ on $X$ is said to be a \emph{Cauchy filter} if for each entourage $V$ in $\mathcal{U}$, there is an $F\in \mathcal{F}$ such that $F\times F\subseteq V$.  

We say that $X$ is \emph{complete} if every Cauchy filter is a convergent filter in the topology $T_{\mathcal{U}}$ \PMlinkname{induced}{TopologyInducedByUniformStructure} by $\mathcal{U}$.  $\mathcal{U}$ in this case is called a \emph{complete uniformity}.

A \emph{Cauchy sequence} $\lbrace x_i\rbrace$ in a uniform space $X$ is a sequence in $X$ whose section filter is a Cauchy filter.  A Cauchy sequence is said to be convergent if its section filter is convergent.  $X$ is said to be \emph{sequentially complete} if every Cauchy sequence converges (every section filter of it converges).

\textbf{Remark}.  This is a generalization of the concept of completeness in a metric space, as a metric space is a uniform space.  As we see above, in the course of this generalization, two notions of completeness emerge: that of completeness and sequentially completeness.  Clearly, completeness always imply sequentially completeness.  In the context of a metric space, or a metrizable uniform space, the two notions are indistinguishable: sequentially completeness also implies completeness.
%%%%%
%%%%%
\end{document}
