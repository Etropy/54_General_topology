\documentclass[12pt]{article}
\usepackage{pmmeta}
\pmcanonicalname{ProductTopologyPreservesTheHausdorffProperty}
\pmcreated{2013-03-22 13:39:40}
\pmmodified{2013-03-22 13:39:40}
\pmowner{archibal}{4430}
\pmmodifier{archibal}{4430}
\pmtitle{product topology preserves the Hausdorff property}
\pmrecord{7}{34317}
\pmprivacy{1}
\pmauthor{archibal}{4430}
\pmtype{Theorem}
\pmcomment{trigger rebuild}
\pmclassification{msc}{54B10}
\pmclassification{msc}{54D10}

% this is the default PlanetMath preamble.  as your knowledge
% of TeX increases, you will probably want to edit this, but
% it should be fine as is for beginners.

% almost certainly you want these
\usepackage{amssymb}
\usepackage{amsmath}
\usepackage{amsfonts}

% used for TeXing text within eps files
%\usepackage{psfrag}
% need this for including graphics (\includegraphics)
%\usepackage{graphicx}
% for neatly defining theorems and propositions
%\usepackage{amsthm}
% making logically defined graphics
%%%\usepackage{xypic}

% there are many more packages, add them here as you need them

% define commands here

\newcommand{\sR}[0]{\mathbb{R}}
\newcommand{\sC}[0]{\mathbb{C}}
\newcommand{\sN}[0]{\mathbb{N}}
\newcommand{\sZ}[0]{\mathbb{Z}}
\begin{document}
{\bf Theorem} Suppose $\{X_\alpha\}_{\alpha\in A}$ is a collection of
Hausdorff spaces. Then the
 generalized Cartesian product
 $ \prod_{\alpha\in A} X_\alpha $
equipped with the product topology is a Hausdorff space.

\emph{Proof.} Let $Y=\prod_{\alpha\in A} X_\alpha$, and
let $x,y$ be distinct points in $Y$. Then there is an index $\beta \in A$
such that $x(\beta)$ and $y(\beta)$ are distinct points in
the Hausdorff space $X_\beta$. It follows that there are open sets
$U$ and $V$ in $X_\beta$ such that $x(\beta)\in U$, $y(\beta) \in V$,
and $U\cap V = \emptyset$.
Let $\pi_\beta$ be the projection operator $Y\to X_\beta$ defined
\PMlinkname{here}{GeneralizedCartesianProduct}. By the definition of
the product topology, $\pi_\beta$ is continuous, so
$\pi_\beta^{-1}(U)$ and $\pi_\beta^{-1}(V)$ are open sets in $Y$. Also,
since the 
\PMlinkname{preimage commutes with set operations}{InverseImageCommutesWithSetOperations}, 
we have that
\begin{eqnarray*}
\pi_\beta^{-1}(U) \cap \pi_\beta^{-1}(V) &=& \pi_\beta^{-1} \big(U \cap V\big) \\
 &=& \emptyset.
\end{eqnarray*}
Finally, since $x(\beta)\in U$, i.e., $\pi_\beta(x)\in U$,
it follows
that $x\in \pi_\beta^{-1}(U)$. Similarly,  $y\in \pi_\beta^{-1}(V)$.
We have shown that $U$ and $V$ are open disjoint neighborhoods of
$x$ respectively $y$. In other words, $Y$ is a Hausdorff space.
$\Box$
%%%%%
%%%%%
\end{document}
