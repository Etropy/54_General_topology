\documentclass[12pt]{article}
\usepackage{pmmeta}
\pmcanonicalname{Compactification}
\pmcreated{2013-03-22 12:15:42}
\pmmodified{2013-03-22 12:15:42}
\pmowner{Evandar}{27}
\pmmodifier{Evandar}{27}
\pmtitle{compactification}
\pmrecord{8}{31654}
\pmprivacy{1}
\pmauthor{Evandar}{27}
\pmtype{Definition}
\pmcomment{trigger rebuild}
\pmclassification{msc}{54D35}
\pmsynonym{Hausdorff compactification}{Compactification}
\pmrelated{Compact}
\pmrelated{AlexandrovOnePointCompactification}

\endmetadata

% this is the default PlanetMath preamble.  as your knowledge
% of TeX increases, you will probably want to edit this, but
% it should be fine as is for beginners.

% almost certainly you want these
\usepackage{amssymb}
\usepackage{amsmath}
\usepackage{amsfonts}

% used for TeXing text within eps files
%\usepackage{psfrag}
% need this for including graphics (\includegraphics)
%\usepackage{graphicx}
% for neatly defining theorems and propositions
%\usepackage{amsthm}
% making logically defined graphics
%%%%\usepackage{xypic} 

% there are many more packages, add them here as you need them

% define commands here
\begin{document}
Let $X$ be a topological space.  A (Hausdorff) compactification of $X$ is a pair $(K,h)$ where $K$ is a Hausdorff topological space and $h:X\rightarrow K$ is a continuous function such that
\begin{itemize}
\item  $K$ is compact
\item  $h$ is a homeomorphism between $X$ and $h(X)$
\item  $\overline{h(X)}^K=K$ where $\overline{A}^K$ denotes closure in $K$ for any subset $A$ of $K$
\end{itemize}

$h$ is often considered to be the inclusion map, so that $X\subseteq K$ with $\overline{X}^K=K$.
%%%%%
%%%%%
%%%%%
\end{document}
