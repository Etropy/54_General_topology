\documentclass[12pt]{article}
\usepackage{pmmeta}
\pmcanonicalname{AlternativeCharacterizationOfUltrafilter}
\pmcreated{2013-03-22 14:42:20}
\pmmodified{2013-03-22 14:42:20}
\pmowner{yark}{2760}
\pmmodifier{yark}{2760}
\pmtitle{alternative characterization of ultrafilter}
\pmrecord{13}{36324}
\pmprivacy{1}
\pmauthor{yark}{2760}
\pmtype{Theorem}
\pmcomment{trigger rebuild}
\pmclassification{msc}{54A20}

\endmetadata

\usepackage{amssymb}
\usepackage{amsmath}
\usepackage{amsfonts}

\def\F{\mathcal{F}}
\begin{document}
\PMlinkescapeword{finite}

Let $X$ be a set.
A filter $\F$ over $X$ is an ultrafilter if and only if
it satisfies the following condition:
if $A \coprod B = X$ (see disjoint union),
then either $A \in \F$ or $B \in \F$.

This result can be generalized somewhat:
a filter $\F$ over $X$ is an ultrafilter if and only if
it satisfies the following condition:
if $A \cup B = X$ (see union), then either $A \in \F$ or $B \in \F$.

This theorem can be extended to
the following two propositions about finite unions:
\begin{enumerate}
\item A filter $\F$ over $X$ is an ultrafilter if and only if,
whenever $A_1,\dots,A_n$ are subsets of $X$ such that $\coprod_{i=1}^n A_i = X$
then there exists exactly one $i$ such that $A_i \in \F$.
\item A filter $\F$ over $X$ is an ultrafilter if and only if,
whenever $A_1,\dots,A_n$ are subsets of $X$ such that $\bigcup_{i=1}^n A_i = X$
then there exists an $i$ such that $A_i \in \F$.
\end{enumerate}
%%%%%
%%%%%
\end{document}
