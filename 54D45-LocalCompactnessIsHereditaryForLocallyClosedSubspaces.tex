\documentclass[12pt]{article}
\usepackage{pmmeta}
\pmcanonicalname{LocalCompactnessIsHereditaryForLocallyClosedSubspaces}
\pmcreated{2013-03-22 17:36:33}
\pmmodified{2013-03-22 17:36:33}
\pmowner{asteroid}{17536}
\pmmodifier{asteroid}{17536}
\pmtitle{local compactness is hereditary for locally closed subspaces}
\pmrecord{5}{40025}
\pmprivacy{1}
\pmauthor{asteroid}{17536}
\pmtype{Theorem}
\pmcomment{trigger rebuild}
\pmclassification{msc}{54D45}
\pmrelated{LocallyCompactHausdorffSpace}

\endmetadata

% this is the default PlanetMath preamble.  as your knowledge
% of TeX increases, you will probably want to edit this, but
% it should be fine as is for beginners.

% almost certainly you want these
\usepackage{amssymb}
\usepackage{amsmath}
\usepackage{amsfonts}

% used for TeXing text within eps files
%\usepackage{psfrag}
% need this for including graphics (\includegraphics)
%\usepackage{graphicx}
% for neatly defining theorems and propositions
%\usepackage{amsthm}
% making logically defined graphics
%%%\usepackage{xypic}

% there are many more packages, add them here as you need them

% define commands here

\begin{document}
{\bf Theorem - } Let $X$ be a locally compact space and $Y \subseteq X$ a subspace. If $Y$ is locally closed in $X$ then $Y$ is also locally compact.


The converse of this theorem is also true with the additional assumption that $X$ is Hausdorff.

{\bf Theorem 2 -} Let $X$ be a \PMlinkname{locally compact Hausdorff space}{LocallyCompactHausdorffSpace} and $Y \subseteq X$ a subspace. If $Y$ is locally compact then $Y$ is locally closed in $X$.
%%%%%
%%%%%
\end{document}
