\documentclass[12pt]{article}
\usepackage{pmmeta}
\pmcanonicalname{FsigmaSet}
\pmcreated{2013-03-22 14:37:59}
\pmmodified{2013-03-22 14:37:59}
\pmowner{rspuzio}{6075}
\pmmodifier{rspuzio}{6075}
\pmtitle{$F_\sigma$ set}
\pmrecord{9}{36215}
\pmprivacy{1}
\pmauthor{rspuzio}{6075}
\pmtype{Definition}
\pmcomment{trigger rebuild}
\pmclassification{msc}{54A05}
\pmrelated{G_DeltaSet}
\pmrelated{G_deltaSet}
\pmrelated{PavedSet}
\pmrelated{PavedSpace}

\endmetadata

% this is the default PlanetMath preamble.  as your knowledge
% of TeX increases, you will probably want to edit this, but
% it should be fine as is for beginners.

% almost certainly you want these
\usepackage{amssymb}
\usepackage{amsmath}
\usepackage{amsfonts}

% used for TeXing text within eps files
%\usepackage{psfrag}
% need this for including graphics (\includegraphics)
%\usepackage{graphicx}
% for neatly defining theorems and propositions
%\usepackage{amsthm}
% making logically defined graphics
%%%\usepackage{xypic}

% there are many more packages, add them here as you need them

% define commands here
\begin{document}
A subset of a topological space is called a $F_\sigma$ set if it equals the union of a countable collection of closed sets.  

The complement of a $F_\sigma$ set is a \PMlinkname{$G_\delta$ set}{G_DeltaSet}.

For instance, the $X$ set of all points $(x,y)$ in the plane such that either $y = 0$ or $x/y$ is rational is an $F_\sigma$ set because it can be expressed as the union of a countable set of lines:
 $$X = \{(x,0) \mid x \in \mathbb{R} \} \cup \bigcup_{r \in \mathbb{Q}} \{(ry,y) \mid y \in \mathbb{R}\}$$
%%%%%
%%%%%
\end{document}
