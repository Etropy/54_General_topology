\documentclass[12pt]{article}
\usepackage{pmmeta}
\pmcanonicalname{PropertiesOfTheClosureOperator}
\pmcreated{2013-03-22 15:17:05}
\pmmodified{2013-03-22 15:17:05}
\pmowner{matte}{1858}
\pmmodifier{matte}{1858}
\pmtitle{properties of the closure operator}
\pmrecord{11}{37075}
\pmprivacy{1}
\pmauthor{matte}{1858}
\pmtype{Theorem}
\pmcomment{trigger rebuild}
\pmclassification{msc}{54A99}

% this is the default PlanetMath preamble.  as your knowledge
% of TeX increases, you will probably want to edit this, but
% it should be fine as is for beginners.

% almost certainly you want these
\usepackage{amssymb}
\usepackage{amsmath}
\usepackage{amsfonts}
\usepackage{amsthm}

\usepackage{mathrsfs}

% used for TeXing text within eps files
%\usepackage{psfrag}
% need this for including graphics (\includegraphics)
%\usepackage{graphicx}
% for neatly defining theorems and propositions
%
% making logically defined graphics
%%%\usepackage{xypic}

% there are many more packages, add them here as you need them

% define commands here

\newcommand{\sR}[0]{\mathbb{R}}
\newcommand{\sC}[0]{\mathbb{C}}
\newcommand{\sN}[0]{\mathbb{N}}
\newcommand{\sZ}[0]{\mathbb{Z}}

 \usepackage{bbm}
 \newcommand{\Z}{\mathbbmss{Z}}
 \newcommand{\C}{\mathbbmss{C}}
 \newcommand{\F}{\mathbbmss{F}}
 \newcommand{\R}{\mathbbmss{R}}
 \newcommand{\Q}{\mathbbmss{Q}}



\newcommand*{\norm}[1]{\lVert #1 \rVert}
\newcommand*{\abs}[1]{| #1 |}



\newtheorem{thm}{Theorem}
\newtheorem{defn}{Definition}
\newtheorem{prop}{Proposition}
\newtheorem{lemma}{Lemma}
\newtheorem{cor}{Corollary}
\begin{document}
Suppose $X$ is a topological space, and let $\overline{A}$ be the 
closure of $A$ in $X$. 
Then the following properties hold:

\begin{enumerate}
\item $\overline{A}=A\cup A'$ where $A'$ is the derived set of $A$. 
\item $A\subseteq \overline{A}$, and $A=\overline{A}$ if and only if $A$
  is closed
\item $\overline{A}=\emptyset$ if and only if $A=\emptyset$. 
\item If $Y$ is another topological space, then $f\colon X \to Y$ is a continuous map,
  if and only if $f(\overline{A}) \subseteq \overline{f(A)}$ for all $A\subseteq X$. If $f$ is also a homeomorphism,
  then $f(\overline{A}) = \overline{f(A)}$.
\end{enumerate}
%%%%%
%%%%%
\end{document}
