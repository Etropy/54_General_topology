\documentclass[12pt]{article}
\usepackage{pmmeta}
\pmcanonicalname{BanachFixedPointTheorem}
\pmcreated{2013-03-22 12:31:10}
\pmmodified{2013-03-22 12:31:10}
\pmowner{mathwizard}{128}
\pmmodifier{mathwizard}{128}
\pmtitle{Banach fixed point theorem}
\pmrecord{21}{32758}
\pmprivacy{1}
\pmauthor{mathwizard}{128}
\pmtype{Theorem}
\pmcomment{trigger rebuild}
\pmclassification{msc}{54A20}
\pmclassification{msc}{47H10}
\pmclassification{msc}{54H25}
\pmsynonym{contraction principle}{BanachFixedPointTheorem}
\pmsynonym{contraction mapping theorem}{BanachFixedPointTheorem}
\pmsynonym{method of successive approximations}{BanachFixedPointTheorem}
\pmsynonym{Banach-Caccioppoli fixed point theorem}{BanachFixedPointTheorem}
\pmrelated{FixedPoint}
\pmdefines{contraction mapping}
\pmdefines{contraction operator}

\endmetadata

\usepackage{amssymb, amsmath, amsthm, alltt, setspace}
\newtheorem{thm}{Theorem}

\theoremstyle{definition}
\newtheorem*{defn}{Definition}
\theoremstyle{definition}
\newtheorem*{rem}{Remark}

\theoremstyle{definition}
\newtheorem*{nott}{Notation}

\newtheorem{lemma}{Lemma}
\newtheorem{cor}{Corollary}
\newtheorem*{eg}{Example}
\newtheorem*{ex}{Exercise}
\newtheorem*{prop}{Proposition}


\newcommand{\RR}{\mathbb{R}}
\newcommand{\QQ}{\mathbb{Q}}
\newcommand{\ZZ}{\mathbb{Z}}
\newcommand{\NN}{\mathbb{N}}
\newcommand{\leftbb}{[ \! [}
\newcommand{\rightbb}{] \! ]}
\newcommand{\bt}{\begin{thm}}
\newcommand{\et}{\end{thm}}
\newcommand{\Rel}{\mathbf{R}}
\newcommand{\er}{\thicksim}
\newcommand{\sqle}{\sqsubseteq}
\newcommand{\floor}[1]{\lfloor{#1}\rfloor}
\newcommand{\ceil}[1]{\lceil{#1}\rceil}
\begin{document}
\PMlinkescapeword{property}
\PMlinkescapeword{contraction}
\PMlinkescapeword{contractions}

Let $(X,d)$ be a complete metric space.  A function $T:X \to X$ is said to be a \emph{contraction mapping} if there is a \PMlinkescapeword{constant}constant $q$ with $0 \leq q < 1$ such that
\[
  d(Tx,Ty)\leq q\cdot d(x,y)
\]
for all $x,y\in X$.  Contractions have an important property.

\begin{thm}[Banach \PMlinkescapetext{Fixed Point} Theorem]
Every contraction has a unique \PMlinkid{fixed point}{2777}.
\end{thm}

There is an estimate to this fixed point that can be useful in applications.  Let $T$ be a contraction mapping on $(X,d)$ with constant $q$ and unique fixed point $x^* \in X$.  For any $x_0 \in X$, define recursively the following sequence
\begin{eqnarray*}
  x_1      &:=& Tx_0   \\
  x_2      &:=& Tx_1   \\
         &\vdots&      \\
  x_{n+1}  &:=& Tx_n.
\end{eqnarray*}
The following inequality then holds:
\[
  d(x^*,x_n)\leq \frac{q^n}{1-q}d(x_1,x_0).
\]
So the sequence $(x_n)$ converges to $x^*$.  This \PMlinkescapetext{recursive} estimate is occasionally responsible for this result being known as \emph{the method of successive approximations}.
%%%%%
%%%%%
\end{document}
