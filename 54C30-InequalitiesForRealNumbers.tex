\documentclass[12pt]{article}
\usepackage{pmmeta}
\pmcanonicalname{InequalitiesForRealNumbers}
\pmcreated{2013-03-22 13:58:16}
\pmmodified{2013-03-22 13:58:16}
\pmowner{mathcam}{2727}
\pmmodifier{mathcam}{2727}
\pmtitle{inequalities for real numbers}
\pmrecord{12}{34742}
\pmprivacy{1}
\pmauthor{mathcam}{2727}
\pmtype{Definition}
\pmcomment{trigger rebuild}
\pmclassification{msc}{54C30}
\pmclassification{msc}{26-00}
\pmclassification{msc}{12D99}
\pmrelated{SummedNumeratorAndSummedDenominator}
\pmdefines{strict inequality}
\pmdefines{inequality}

\endmetadata

% this is the default PlanetMath preamble.  as your knowledge
% of TeX increases, you will probably want to edit this, but
% it should be fine as is for beginners.

% almost certainly you want these
\usepackage{amssymb}
\usepackage{amsmath}
\usepackage{amsfonts}

% used for TeXing text within eps files
%\usepackage{psfrag}
% need this for including graphics (\includegraphics)
%\usepackage{graphicx}
% for neatly defining theorems and propositions
\usepackage{amsthm}
% making logically defined graphics
%%%\usepackage{xypic}

% there are many more packages, add them here as you need them

% define commands here

\newcommand{\sR}[0]{\mathbb{R}}
\newcommand{\sC}[0]{\mathbb{C}}
\newcommand{\sN}[0]{\mathbb{N}}
\newcommand{\sZ}[0]{\mathbb{Z}}

 \usepackage{bbm}
 \newcommand{\Z}{\mathbbmss{Z}}
 \newcommand{\C}{\mathbbmss{C}}
 \newcommand{\F}{\mathbbmss{F}}
 \newcommand{\N}{\mathbbmss{N}}
 \newcommand{\R}{\mathbbmss{R}}
 \newcommand{\Q}{\mathbbmss{Q}}

\newtheorem{thm}{Theorem}
\newtheorem{defn}{Definition}
\newtheorem{prop}{Proposition}
\newtheorem{lemma}{Lemma}
\newtheorem{cor}{Corollary}
\begin{document}
Suppose $a$ is a real number. 
\begin{enumerate}
\item If $a<0$ then $a$ is a \emph{negativ{e} number}.
\item If $a>0$ then $a$ is a \emph{positiv{e} number}.
\item If $a\le 0$ then $a$ is a \emph{non-positiv{e} number}.
\item If $a\ge 0$ then $a$ is a \emph{non-negativ{e} number}.
\end{enumerate}
The first two inequalities are also called {\bf strict inequalities}.\\
The second two inequalities are also called {\bf loose inequalities}.

\subsubsection*{Properties}
Suppose $a$ and $b$ are real numbers.
\begin{enumerate}
\item If $a>b$, then $-a<-b$. If $a<b$, then $-a>-b$.
\item If $a\ge b$, then $-a\le -b$. If $a\le b$, then $-a\ge -b$.
\end{enumerate}

\begin{lemma}\, $0<a$\, iff\, $-a<0$.
\end{lemma}

\begin{proof}  If $0<a$, then adding $-a$ on both sides of the inequality gives $-a=-a+0<-a+a=0$.\, This process can also be reversed.
\end{proof}


\begin{lemma} For any $a\in \R$, either $a=0$ or $0<a^2$.
\end{lemma}

\begin{proof}
Suppose\, $a\ne 0$, then by trichotomy, we have either\, $0<a$\, or\, $a<0$, but not both.\,  If\, $0<a$,\, then\, $0=0\cdot a<a\cdot a=a^2$.\,  On the other hand, if\, $-(-a)=a<0$, then\, $0<-a$\, by the previous lemma.\,  Then repeating the previous \PMlinkescapetext{argument},\, $0 = 0\cdot(-a) < (-a)(-a)=a^2$.  
\end{proof}

Three direct consequences follow:
\begin{cor}\, $0<1$ \end{cor}
\begin{cor} For any $a\in \R$, $0<1+a^2$. \end{cor}
\begin{cor} There is no real solution for $x$ in the equation $1+x^2=0$. 
\end{cor}

\subsubsection*{Inequality for a converging sequence}
Suppose $a_0,a_1,\ldots$ is a sequence of real numbers converging to a real 
number $a$. 
\begin{enumerate}
\item If $a_i < b$ or $a_i \le b$ 
for some real number $b$ for each $i$, then $a\le b$.
\item If $a_i > b$ or $a_i \ge b$ 
for some real number $b$ for each $i$, then $a\ge b$.
\end{enumerate}
%%%%%
%%%%%
\end{document}
