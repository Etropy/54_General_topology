\documentclass[12pt]{article}
\usepackage{pmmeta}
\pmcanonicalname{ClopenSubset}
\pmcreated{2013-03-22 13:25:29}
\pmmodified{2013-03-22 13:25:29}
\pmowner{mathcam}{2727}
\pmmodifier{mathcam}{2727}
\pmtitle{clopen subset}
\pmrecord{14}{33978}
\pmprivacy{1}
\pmauthor{mathcam}{2727}
\pmtype{Definition}
\pmcomment{trigger rebuild}
\pmclassification{msc}{54D05}
\pmsynonym{clopen set}{ClopenSubset}
\pmsynonym{clopen}{ClopenSubset}
\pmsynonym{closed and open}{ClopenSubset}
\pmrelated{IdentityTheorem}

%\documentclass{amsart}
\usepackage{amsmath}
%\usepackage[all,poly,knot,dvips]{xy}
%\usepackage{pstricks,pst-poly,pst-node,pstcol}


\usepackage{amssymb,latexsym}

\usepackage{amsthm,latexsym}
\usepackage{eucal,latexsym}

% THEOREM Environments --------------------------------------------------

\newtheorem{thm}{Theorem}
 \newtheorem*{mainthm}{Main~Theorem}
 \newtheorem{cor}[thm]{Corollary}
 \newtheorem{lem}[thm]{Lemma}
 \newtheorem{prop}[thm]{Proposition}
 \newtheorem{claim}[thm]{Claim}
 \theoremstyle{definition}
 \newtheorem{defn}[thm]{Definition}
 \theoremstyle{remark}
 \newtheorem{rem}[thm]{Remark}
 \numberwithin{equation}{subsection}


%---------------------  Greek letters, etc ------------------------- 

\newcommand{\CA}{\mathcal{A}}
\newcommand{\CC}{\mathcal{C}}
\newcommand{\CM}{\mathcal{M}}
\newcommand{\CP}{\mathcal{P}}
\newcommand{\CS}{\mathcal{S}}
\newcommand{\BC}{\mathbb{C}}
\newcommand{\BN}{\mathbb{N}}
\newcommand{\BR}{\mathbb{R}}
\newcommand{\BZ}{\mathbb{Z}}
\newcommand{\FF}{\mathfrak{F}}
\newcommand{\FL}{\mathfrak{L}}
\newcommand{\FM}{\mathfrak{M}}
\newcommand{\Ga}{\alpha}
\newcommand{\Gb}{\beta}
\newcommand{\Gg}{\gamma}
\newcommand{\GG}{\Gamma}
\newcommand{\Gd}{\delta}
\newcommand{\GD}{\Delta}
\newcommand{\Ge}{\varepsilon}
\newcommand{\Gz}{\zeta}
\newcommand{\Gh}{\eta}
\newcommand{\Gq}{\theta}
\newcommand{\GQ}{\Theta}
\newcommand{\Gi}{\iota}
\newcommand{\Gk}{\kappa}
\newcommand{\Gl}{\lambda}
\newcommand{\GL}{\Lamda}
\newcommand{\Gm}{\mu}
\newcommand{\Gn}{\nu}
\newcommand{\Gx}{\xi}
\newcommand{\GX}{\Xi}
\newcommand{\Gp}{\pi}
\newcommand{\GP}{\Pi}
\newcommand{\Gr}{\rho}
\newcommand{\Gs}{\sigma}
\newcommand{\GS}{\Sigma}
\newcommand{\Gt}{\tau}
\newcommand{\Gu}{\upsilon}
\newcommand{\GU}{\Upsilon}
\newcommand{\Gf}{\varphi}
\newcommand{\GF}{\Phi}
\newcommand{\Gc}{\chi}
\newcommand{\Gy}{\psi}
\newcommand{\GY}{\Psi}
\newcommand{\Gw}{\omega}
\newcommand{\GW}{\Omega}
\newcommand{\Gee}{\epsilon}
\newcommand{\Gpp}{\varpi}
\newcommand{\Grr}{\varrho}
\newcommand{\Gff}{\phi}
\newcommand{\Gss}{\varsigma}

\def\co{\colon\thinspace}
\begin{document}
A subset of a topological space $X$ is called \emph{clopen} if it is both
open and closed.

\begin{thm}
  The clopen subsets form a Boolean algebra under the operation of
  union, intersection and complement. In other words:
  \begin{itemize}
  \item $X$ and $\emptyset$ are clopen,
  \item the complement of a clopen set is clopen,
  \item finite unions and intersections of clopen sets are clopen.
  \end{itemize}
\end{thm}

\begin{proof}
    The first follows by the definition of a topology, the second by
    noting that complements of open sets are closed, and vice versa,
    and the third by noting that this property holds for both open
    and closed sets.
\end{proof}


One application of clopen sets is that they can be used to
describe connectness.  In particular, a topological space is
connected if and only if its only clopen subsets are itself and
the empty set.
%(\PMlinkname{proof}{CharacterizationsOfConnectedness})

If a space has finitely many connected components then each
connected component is clopen. This may not be the case if there 
are infinitely many components, as the case of the rational numbers 
demonstrates.
%%%%%
%%%%%
\end{document}
