\documentclass[12pt]{article}
\usepackage{pmmeta}
\pmcanonicalname{CoerciveFunction}
\pmcreated{2013-03-22 15:20:13}
\pmmodified{2013-03-22 15:20:13}
\pmowner{paolini}{1187}
\pmmodifier{paolini}{1187}
\pmtitle{coercive function}
\pmrecord{5}{37154}
\pmprivacy{1}
\pmauthor{paolini}{1187}
\pmtype{Definition}
\pmcomment{trigger rebuild}
\pmclassification{msc}{54A05}
\pmsynonym{coercive}{CoerciveFunction}
\pmsynonym{coercitive}{CoerciveFunction}
\pmsynonym{coercitive function}{CoerciveFunction}

% this is the default PlanetMath preamble.  as your knowledge
% of TeX increases, you will probably want to edit this, but
% it should be fine as is for beginners.

% almost certainly you want these
\usepackage{amssymb}
\usepackage{amsmath}
\usepackage{amsfonts}

% used for TeXing text within eps files
%\usepackage{psfrag}
% need this for including graphics (\includegraphics)
%\usepackage{graphicx}
% for neatly defining theorems and propositions
\usepackage{amsthm}
% making logically defined graphics
%%%\usepackage{xypic}

% there are many more packages, add them here as you need them

% define commands here
\newcommand{\R}{\mathbb R}
\newtheorem{theorem}{Theorem}
\newtheorem{proposition}{Proposition}
\newtheorem{definition}{Definition}
\theoremstyle{remark}
\newtheorem{example}{Example}
\begin{document}
\begin{definition}[coercive function]
Let $X$ and $Y$ be topological spaces.
A function $f\colon X\to Y$ is said to be \emph{coercive} if for every compact set $J\subset Y$ there exists a compact set $K\subset X$ such that
\[
  F(X\setminus K) \subset Y\setminus J.
\]
\end{definition}

The general definition given above has a clear sense when specialized to the Euclidean spaces, as shown in the following result.

\begin{proposition}[coercive functions on $\R^n$]
A function $f\colon \R^n \to \R^m$ is coercive if and only if 
\[
  \lim_{|x|\to +\infty} |f(x)| = +\infty.
\]
\end{proposition}
%%%%%
%%%%%
\end{document}
