\documentclass[12pt]{article}
\usepackage{pmmeta}
\pmcanonicalname{ProofOfInjectiveImagesOfBaireSpace}
\pmcreated{2013-03-22 18:47:15}
\pmmodified{2013-03-22 18:47:15}
\pmowner{gel}{22282}
\pmmodifier{gel}{22282}
\pmtitle{proof of injective images of Baire space}
\pmrecord{5}{41583}
\pmprivacy{1}
\pmauthor{gel}{22282}
\pmtype{Proof}
\pmcomment{trigger rebuild}
\pmclassification{msc}{54E50}
%\pmkeywords{Baire space}
%\pmkeywords{Polish space}
%\pmkeywords{condensation point}

% almost certainly you want these
\usepackage{amssymb}
\usepackage{amsmath}
\usepackage{amsfonts}

% used for TeXing text within eps files
%\usepackage{psfrag}
% need this for including graphics (\includegraphics)
%\usepackage{graphicx}
% for neatly defining theorems and propositions
\usepackage{amsthm}
% making logically defined graphics
%%%\usepackage{xypic}

% there are many more packages, add them here as you need them

% define commands here
\newtheorem*{theorem*}{Theorem}
\newtheorem*{lemma*}{Lemma}
\newtheorem*{corollary*}{Corollary}
\newtheorem*{definition*}{Definition}
\newtheorem{theorem}{Theorem}
\newtheorem{lemma}{Lemma}
\newtheorem{corollary}{Corollary}
\newtheorem{definition}{Definition}

\begin{document}
\PMlinkescapeword{function}
\PMlinkescapeword{inverse}
\PMlinkescapeword{closed}
\PMlinkescapeword{difference}
\PMlinkescapeword{topology}
\PMlinkescapeword{generated by}
\PMlinkescapeword{points}
\PMlinkescapeword{contain}
\PMlinkescapeword{union}
\PMlinkescapeword{open}
\PMlinkescapeword{necessary}
\PMlinkescapeword{terms}
\PMlinkescapeword{covering}
\PMlinkescapeword{lemma}
\PMlinkescapeword{integers}
\PMlinkescapeword{contained}
\PMlinkescapeword{positive}
\PMlinkescapeword{contains}
\PMlinkescapeword{equation}

We show that, for an uncountable Polish space $X$, there exists a continuous and one-to-one function $f\colon\mathcal{N}\rightarrow X$ such that $X\setminus f(\mathcal{N})$ is countable. Furthermore, the inverse of $f$ defined on $f(\mathcal{N})$ is Borel measurable.

The construction of the function relies on the following result. We let $d$ be a complete metric on $X$ with respect to which the diameter of any subset is defined.

\begin{lemma*}
Let $S$ be an uncountable subset of $X$ which can be written as the difference of two closed sets, and choose any $\epsilon>0$.

Then, there exists a sequence $S_1,S_2,\ldots$ of pairwise disjoint and uncountable subsets of $S$ with diameter no more than $\epsilon$ and such that,
\begin{enumerate}
\item for each $n>0$, $\bigcup_{k\le n}S_k$ is closed.
\item $S\setminus\bigcup_nS_n$ is countable.
\end{enumerate}
\end{lemma*}
\begin{proof}
As $S=A\setminus B$ is the difference of closed sets, it is a Polish space under the subspace topology.
In fact, if $B$ is nonempty, the topology is generated by the complete metric
\begin{equation*}
d_S(x,y)=d(x,y)+\sup\left\{|d(x,z)^{-1}-d(y,z)^{-1}|\colon z\in B\right\},
\end{equation*}
otherwise we may take $d_S=d$. In either case, $d_S\ge d$, so it is enough to choose the sets $S_n$ to have diameter no more than $2^{-n}$ with respect to $d_S$. Note also that any bounded and closed set with respect to this metric is also closed as a subset of $X$.

Let $S^c$ be the condensation points of $S$, which are the points whose neighborhoods all contain uncountably many points of $S$. Then, $S\setminus S^c$ is a union of countably many countable and open subsets of $S$, so is countable and open. Hence, $S^c$ is uncountable and closed in $S$, and every open subset is uncountable.
Choosing any $p\in S^c$ then $S^c\setminus p$ will not be a closed subset of $X$. So, replacing $S$ by $S\setminus\{p\}$ if necessary, we may suppose that $S^c$ is not closed as a subset of $X$ and, therefore is not compact.

So, for some $\delta>0$, $S^c$ cannot be covered by finitely many sets with $d_S$-diameter no more than $\delta$ (see \PMlinkname{here}{ProofThatAMetricSpaceIsCompactIfAndOnlyIfItIsCompleteAndTotallyBounded}). By separability, there is a sequence $T_1,T_2,\ldots$ of open balls in $S^c$ with diameter less than $\min(\epsilon,\delta)$, and covering $S^c$. Writing $\bar T_n$ for the $d_S$-closure of $T_n$ and eliminating any terms such that $T_n\subseteq\bigcup_{k<n}\bar T_k$, then $S_n\equiv\bar T_n\setminus \bigcup_{k< n}\bar T_k$ have nonempty interior and hence are uncountable, and
\begin{equation*}
S\setminus\bigcup_nS_n=S\setminus S^c
\end{equation*}
is countable, as required.
\end{proof}

Note that $S_n=\bigcup_{k\le n}S_k\setminus\bigcup_{k\le n-1}S_k$ is also a difference of closed sets. So, the lemma allows us to inductively choose sets $C(n_1,\ldots,n_k)\subseteq X$ for integers $k\ge 0$ and $n_1,\ldots,n_k\ge 1$ such that $C()=X$ and the following are satisfied.
\begin{enumerate}
\item $C(n_1,\ldots,n_k,m)$ are uncountable, contained in $C(n_1,\ldots,n_k)$, and pairwise disjoint as $m$ runs through the positive integers.
\item $\bigcup_{j\le m}C(n_1,\ldots,n_k,j)$ is closed for all $m\ge 1$.
\item $C(n_1,\ldots,n_k)\setminus\bigcup_mC(n_1,\ldots,n_k,m)$ is countable and, for $k\ge 1$, has diameter no more than $2^{-k}$.
\end{enumerate}

For any $n\in\mathcal{N}$ we may choose a sequence $x_k\in C(n_1,\ldots,n_k)$. Since, for $k\ge 1$, this set has diameter no more than $2^{-k}$, then $d(x_j,x_k)\le 2^{-k}$ whenever $j\ge k\ge 1$. So, the sequence is \PMlinkname{Cauchy}{CauchySequence} and hence has a limit $x$.
Furthermore, as $x_j$ is contained in the closed set
\begin{equation*}
\bigcup_{m\le n_{k+1}}C(n_1,\ldots,n_k,m)
\end{equation*}
for $j>k$, then $x$ must also be contained in it and hence is in $C(n_1,\ldots,n_k)$. So
\begin{equation*}
C(n)\equiv\bigcap_{k=0}^\infty C(n_1,\ldots,n_k)
\end{equation*}
contains $x$ and is nonempty. Furthermore, as it has zero diameter, it is a singleton. So $f\colon\mathcal{N}\rightarrow X$ is uniquely defined by $f(n)\in C(n)$.


Given any $m,n\in \mathcal{N}$ such that $m_j=n_j$ for $j\le k$, then $f(m)$ and $f(n)$ are both in the set $C(m_1,\ldots,m_k)$, which has diameter no more than $2^{-k}$. So, $d(f(m),f(n))\le 2^{-k}$ and $f$ is continuous.

If $m$ and $n$ are distinct elements of $\mathcal{N}$ and $k$ is the smallest integer such that $m_k\not= n_k$, then $f(m)$ and $f(n)$ are in the disjoint sets $C(m_1,\ldots,m_k)$ and $C(n_1,\ldots,n_k)$ respectively. So $f(m)\not= f(n)$, and $f$ is one-to-one.

Now let $A$ be the countable set
\begin{equation*}
A = \bigcup_k\bigcup_{n_1,\ldots,n_k}\left( C(n_1,\ldots,n_k)\setminus\bigcup_mC(n_1,\ldots,n_k,m)\right)\subseteq X.
\end{equation*}
Also define
\begin{equation*}
\mathcal{N}(n_1,\ldots,n_k)\equiv\left\{m\in\mathcal{N}\colon m_j=n_j\text{ for }j\le k\right\}.
\end{equation*}
These sets form a basis for the topology on $\mathcal{N}$. Clearly,
\begin{equation*}
f\left(\mathcal{N}(n_1,\ldots,n_k)\right)\subseteq C(n_1,\ldots,n_k)\setminus A.
\end{equation*}
Choosing any $x\in C(n_1,\ldots,n_k)\setminus A$ then we can inductively find $n_j$ for $j>k$ such that $x\in C(n_1,\ldots,n_j)$. Then, setting $n=(n_1,n_2,\ldots)$ gives $f(n)=x$. This shows that
\begin{equation}\label{eq:1}
f\left(\mathcal{N}(n_1,\ldots,n_k)\right)=C(n_1,\ldots,n_k)\setminus A.
\end{equation}
In particular, $f(\mathcal{N})=X\setminus A$ and, therefore $X\setminus f(\mathcal{N})$ is countable.
Finally, as $C(n_1,\ldots,n_k)$ is a difference of closed sets, it is Borel, and equation (\ref{eq:1}) shows that the inverse of $f$ is Borel measurable.

%%%%%
%%%%%
\end{document}
