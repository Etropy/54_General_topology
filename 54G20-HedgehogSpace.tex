\documentclass[12pt]{article}
\usepackage{pmmeta}
\pmcanonicalname{HedgehogSpace}
\pmcreated{2013-03-22 14:50:02}
\pmmodified{2013-03-22 14:50:02}
\pmowner{mathcam}{2727}
\pmmodifier{mathcam}{2727}
\pmtitle{hedgehog space}
\pmrecord{10}{36500}
\pmprivacy{1}
\pmauthor{mathcam}{2727}
\pmtype{Definition}
\pmcomment{trigger rebuild}
\pmclassification{msc}{54G20}

% this is the default PlanetMath preamble.  as your knowledge
% of TeX increases, you will probably want to edit this, but
% it should be fine as is for beginners.

% almost certainly you want these
\usepackage{amssymb}
\usepackage{amsmath}
\usepackage{amsfonts}
\usepackage{amsthm}

% used for TeXing text within eps files
%\usepackage{psfrag}
% need this for including graphics (\includegraphics)
%\usepackage{graphicx}
% for neatly defining theorems and propositions
%\usepackage{amsthm}
% making logically defined graphics
%%%\usepackage{xypic}

% there are many more packages, add them here as you need them

% define commands here

\newcommand{\mc}{\mathcal}
\newcommand{\mb}{\mathbb}
\newcommand{\mf}{\mathfrak}
\newcommand{\ol}{\overline}
\newcommand{\ra}{\rightarrow}
\newcommand{\la}{\leftarrow}
\newcommand{\La}{\Leftarrow}
\newcommand{\Ra}{\Rightarrow}
\newcommand{\nor}{\vartriangleleft}
\newcommand{\Gal}{\text{Gal}}
\newcommand{\GL}{\text{GL}}
\newcommand{\Z}{\mb{Z}}
\newcommand{\R}{\mb{R}}
\newcommand{\Q}{\mb{Q}}
\newcommand{\C}{\mb{C}}
\newcommand{\<}{\langle}
\renewcommand{\>}{\rangle}
\begin{document}
For any cardinal number $K$, we can form a topological space, called the \emph{$K$-hedgehog space}, consisting of the disjoint union of $K$ real unit intervals identified at the origin.  Each unit interval is referred to as one of the hedgehog's ``spines.''

The hedgehog space admits a somewhat surprising metric, by defining $d(x,y)=|x-y|$ if $x$ and $y$ lie in the same spine, and by $d(x,y)=x+y$ if $x$ and $y$ lie in different spines.

The hedgehog space is an example of a Moore space, and satisfies many strong normality and compactness properties.

\begin{thebibliography}{1}
\bibitem{steen} L.A. Steen, J.A.Seebach, Jr.,
\emph{Counterexamples in topology},
Holt, Rinehart and Winston, Inc., 1970.
\end{thebibliography}

%%%%%
%%%%%
\end{document}
