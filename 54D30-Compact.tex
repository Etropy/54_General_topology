\documentclass[12pt]{article}
\usepackage{pmmeta}
\pmcanonicalname{Compact}
\pmcreated{2013-03-22 11:53:35}
\pmmodified{2013-03-22 11:53:35}
\pmowner{djao}{24}
\pmmodifier{djao}{24}
\pmtitle{compact}
\pmrecord{11}{30503}
\pmprivacy{1}
\pmauthor{djao}{24}
\pmtype{Definition}
\pmcomment{trigger rebuild}
\pmclassification{msc}{54D30}
\pmclassification{msc}{81-00}
\pmclassification{msc}{83-00}
\pmclassification{msc}{82-00}
\pmclassification{msc}{46L05}
\pmclassification{msc}{22A22}
\pmrelated{QuasiCompact}
\pmrelated{LocallyCompact}
\pmrelated{HeineBorelTheorem}
\pmrelated{TychonoffsTheorem}
\pmrelated{Compactification}
\pmrelated{SequentiallyCompact}
\pmrelated{Lindelof}
\pmrelated{NoetherianTopologicalSpace}
\pmdefines{compact set}
\pmdefines{compact subset}

\usepackage{amssymb}
\usepackage{amsmath}
\usepackage{amsfonts}
%\usepackage{graphicx}
%%%%%\usepackage{xypic}
\begin{document}
\PMlinkescapeword{term}
A topological space $X$ is {\em compact} if, for every collection $\{U_i\}_{i \in I}$ of open sets in $X$ whose union is $X$, there exists a finite subcollection $\{U_{i_j}\}_{j=1}^n$ whose union is also $X$.

A subset $Y$ of a topological space $X$ is said to be compact if $Y$ with its subspace topology is a compact topological space.

\textbf{Note:} Some authors require that a compact topological space be Hausdorff as well, and use the term quasi-compact to refer to a non-Hausdorff compact space. The modern convention seems to be to use compact in the sense given here, but the old definition is still occasionally encountered (particularly in the French school).
%%%%%
%%%%%
%%%%%
%%%%%
\end{document}
