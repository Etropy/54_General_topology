\documentclass[12pt]{article}
\usepackage{pmmeta}
\pmcanonicalname{MultifuncoidHasAtomicArguments}
\pmcreated{2014-12-14 21:09:54}
\pmmodified{2014-12-14 21:09:54}
\pmowner{porton}{9363}
\pmmodifier{porton}{9363}
\pmtitle{Multifuncoid has atomic arguments}
\pmrecord{2}{87299}
\pmprivacy{1}
\pmauthor{porton}{9363}
\pmtype{Conjecture}
\pmclassification{msc}{54J05}
\pmclassification{msc}{54A05}
\pmclassification{msc}{54D99}
\pmclassification{msc}{54E05}
\pmclassification{msc}{54E17}
\pmclassification{msc}{54E99}

% this is the default PlanetMath preamble.  as your knowledge
% of TeX increases, you will probably want to edit this, but
% it should be fine as is for beginners.

% almost certainly you want these
\usepackage{amssymb}
\usepackage{amsmath}
\usepackage{amsfonts}

% need this for including graphics (\includegraphics)
\usepackage{graphicx}
% for neatly defining theorems and propositions
\usepackage{amsthm}

% making logically defined graphics
%\usepackage{xypic}
% used for TeXing text within eps files
%\usepackage{psfrag}

% there are many more packages, add them here as you need them

% define commands here

\begin{document}
A counter-example against this conjecture have been found.
See \href{http://www.mathematics21.org/algebraic-general-topology.html}{Algebraic General Topology}.

Prerequisites: \href{http://www.mathematics21.org/algebraic-general-topology.html}{Algebraic General Topology}.

{\bf Conjecture.} $L \in \mathrel{\left[ f \right]} \Rightarrow \mathrel{\left[ f \right]} \cap \prod_{i \in \operatorname{dom} \mathfrak{A}} \operatorname{atoms} L_i \neq \emptyset$ for every pre-multifuncoid $f$ of the form whose elements are atomic posets.

A weaker conjecture: It is true for forms whose elements are powersets.

The following is an attempted (partial) proof:

If $\operatorname{arity} f = 0$ our theorem is trivial, so let $\operatorname{arity} f \neq 0$. Let $\sqsubseteq$ is a well-ordering of $\operatorname{arity} f$ with greatest element $m$.

Let $\Phi$ is a function which maps non-least elements of posets into atoms under these elements and least elements into themselves. (Note that $\Phi$ is defined on least elements only for completeness, $\Phi$ is never taken on a least element in the proof below.) {\color{brown} [TODO: Fix the ''universal set'' paradox here.]}

Define a transfinite sequence $a$ by transfinite induction with the formula $ a_c = \Phi \left\langle f \right\rangle_c  \left( a|_{X \left( c \right) \setminus \left\{ c \right\}} \cup L|_{\left( \operatorname{arity} f \right) \setminus X \left( c \right)} \right)$.

Let $b_c = a|_{X \left( c \right) \setminus \left\{ c \right\}} \cup L|_{\left( \operatorname{arity} f \right) \setminus X \left( c \right)}$. Then $a_c = \Phi \left\langle f \right\rangle_c b_c$.

Let us prove by transfinite induction $a_c \in \operatorname{atoms} L_c .$ $a_c = \Phi \left\langle f \right\rangle_c L|_{\left( \operatorname{arity} f \right) \setminus \left\{ c \right\}} \sqsubseteq \left\langle f \right\rangle_c L|_{\left( \operatorname{arity} f \right) \setminus \left\{ c \right\}}$. Thus $a_c \sqsubseteq L_c$. [TODO: Is it true for pre-multifuncoids?]

The only thing remained to prove is that $\left\langle f \right\rangle_c b_c \neq 0$

that is $\langle f \rangle _ c  ( a|_{ X ( c ) \setminus \{ c \} } \cup L|_{( \operatorname{arity} f ) \setminus X ( c )} ) \neq 0$ that is $y \not\asymp \left\langle f \right\rangle_c b_c$.
\end{document}
