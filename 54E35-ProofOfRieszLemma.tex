\documentclass[12pt]{article}
\usepackage{pmmeta}
\pmcanonicalname{ProofOfRieszLemma}
\pmcreated{2013-03-22 14:56:14}
\pmmodified{2013-03-22 14:56:14}
\pmowner{gumau}{3545}
\pmmodifier{gumau}{3545}
\pmtitle{proof of Riesz' Lemma}
\pmrecord{6}{36627}
\pmprivacy{1}
\pmauthor{gumau}{3545}
\pmtype{Proof}
\pmcomment{trigger rebuild}
\pmclassification{msc}{54E35}
\pmclassification{msc}{15A03}

% this is the default PlanetMath preamble.  as your knowledge
% of TeX increases, you will probably want to edit this, but
% it should be fine as is for beginners.

% almost certainly you want these
\usepackage{amssymb}
\usepackage{amsmath}
\usepackage{amsfonts}

% used for TeXing text within eps files
%\usepackage{psfrag}
% need this for including graphics (\includegraphics)
%\usepackage{graphicx}
% for neatly defining theorems and propositions
%\usepackage{amsthm}
% making logically defined graphics
%%%\usepackage{xypic}

% there are many more packages, add them here as you need them

% define commands here
\begin{document}
Let's consider $x \in E-S$ and let $r = d(x, S)$. Recall that $r$ is the distance between $x$ and $S$: $d(x,S) = \textrm{inf}\{d(x, s) \textrm{ such that } s \in S\}$. Now, $r > 0$ because $S$ is closed. Next, we consider $b \in S$ such that 
$$\|x - b\| < \frac{r}{\alpha}$$
This vector $b$ exists: as $0 < \alpha < 1$ then $$\frac{r}{\alpha} > r$$
But then the definition of infimum implies there is $b \in S$ such that $$\|x - b\| < \frac{r}{\alpha}$$

Now, define $$x_{\alpha} = \frac{x - b}{\|x - b\|}$$
Trivially, $$ \|x_{\alpha}\| = 1 $$
Notice that $x_{\alpha} \in E-S$, because if $x_{\alpha} \in S$ then $x - b \in S$, and so $(x - b) + b = x \in S$, an absurd.
Plus, for every $s \in S$ we have 
$$ \|s - x_{\alpha}\| = \|s - \frac{x - b}{\|x - b\|}\| = \frac{1}{\|x - b\|} \cdot \| \|x - b\| \cdot s + b - x \| \geq \frac{r}{\|x - b\|}$$ 
because $$\|x -b\| \cdot s + b \in S$$
But $$\frac{r}{\|x - b\|} > \frac{\alpha}{r} \cdot r = \alpha $$ QED.
%%%%%
%%%%%
\end{document}
