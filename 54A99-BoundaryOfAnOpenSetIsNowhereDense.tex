\documentclass[12pt]{article}
\usepackage{pmmeta}
\pmcanonicalname{BoundaryOfAnOpenSetIsNowhereDense}
\pmcreated{2013-03-22 17:55:41}
\pmmodified{2013-03-22 17:55:41}
\pmowner{CWoo}{3771}
\pmmodifier{CWoo}{3771}
\pmtitle{boundary of an open set is nowhere dense}
\pmrecord{8}{40422}
\pmprivacy{1}
\pmauthor{CWoo}{3771}
\pmtype{Derivation}
\pmcomment{trigger rebuild}
\pmclassification{msc}{54A99}

\endmetadata

\usepackage{amssymb,amscd}
\usepackage{amsmath}
\usepackage{amsfonts}
\usepackage{mathrsfs}

% used for TeXing text within eps files
%\usepackage{psfrag}
% need this for including graphics (\includegraphics)
%\usepackage{graphicx}
% for neatly defining theorems and propositions
\usepackage{amsthm}
% making logically defined graphics
%%\usepackage{xypic}
\usepackage{pst-plot}

% define commands here
\newcommand*{\abs}[1]{\left\lvert #1\right\rvert}
\newtheorem{prop}{Proposition}
\newtheorem{thm}{Theorem}
\newtheorem{ex}{Example}
\newcommand{\real}{\mathbb{R}}
\newcommand{\pdiff}[2]{\frac{\partial #1}{\partial #2}}
\newcommand{\mpdiff}[3]{\frac{\partial^#1 #2}{\partial #3^#1}}
\def\int{\operatorname{int}}
\begin{document}
This entry provides another example of a nowhere dense set.

\begin{prop}  If $A$ is an open set in a topological space $X$, then $\partial A$, the boundary of $A$ is nowhere dense.
\end{prop}

\begin{proof}  Let $B=\partial A$.  Since $B = \overline{A}\cap \overline{A^\complement}$, it is closed, so all we need to show is that $B$ has empty interior $\int(B)=\varnothing$.  First notice that $B= \overline{A}\cap A^\complement$, since $A$ is open.  Now, we invoke one of the interior axioms, namely $\int(U\cap V)=\int(U)\cap \int(V)$.  So, by direct computation, we have 
$$\int(B)=\int(\overline{A})\cap \int(A^\complement) = \int(\overline{A})\cap \overline{A}^\complement \subseteq \overline{A}\cap \overline{A}^\complement =\varnothing.$$
The second equality and the inclusion follow from the general properties of the interior operation, the proofs of which can be found \PMlinkname{here}{DerivationOfPropertiesOnInteriorOperation}.
\end{proof}

\textbf{Remark}.  The fact that $A$ is open is essential.  Otherwise, the proposition fails in general.  For example, the rationals $\mathbb{Q}$, as a subset of the reals $\mathbb{R}$ under the usual order topology, is not open, and its boundary is not nowhere dense, as $\overline{\mathbb{Q}}\cap \overline{\mathbb{Q}^\complement} = \mathbb{R}\cap \mathbb{R}=\mathbb{R}$, whose interior is $\mathbb{R}$ itself, and thus not empty.
%%%%%
%%%%%
\end{document}
