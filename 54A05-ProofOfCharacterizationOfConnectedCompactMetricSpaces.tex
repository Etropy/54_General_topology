\documentclass[12pt]{article}
\usepackage{pmmeta}
\pmcanonicalname{ProofOfCharacterizationOfConnectedCompactMetricSpaces}
\pmcreated{2013-03-22 14:17:06}
\pmmodified{2013-03-22 14:17:06}
\pmowner{paolini}{1187}
\pmmodifier{paolini}{1187}
\pmtitle{proof of characterization of connected compact metric spaces.}
\pmrecord{4}{35739}
\pmprivacy{1}
\pmauthor{paolini}{1187}
\pmtype{Proof}
\pmcomment{trigger rebuild}
\pmclassification{msc}{54A05}

\endmetadata

% this is the default PlanetMath preamble.  as your knowledge
% of TeX increases, you will probably want to edit this, but
% it should be fine as is for beginners.

% almost certainly you want these
\usepackage{amssymb}
\usepackage{amsmath}
\usepackage{amsfonts}

% used for TeXing text within eps files
%\usepackage{psfrag}
% need this for including graphics (\includegraphics)
%\usepackage{graphicx}
% for neatly defining theorems and propositions
%\usepackage{amsthm}
% making logically defined graphics
%%%\usepackage{xypic}

% there are many more packages, add them here as you need them

% define commands here
\begin{document}
First we prove the right-hand arrow: if $X$ is connected then the property stated in the Theorem holds. This implication is true in every metric space $X$, without additional conditions.

Let us denote with $A_\varepsilon$ the set of all points $z\in X$ which can be joined to $x$ with a sequence of points $p_1,\ldots,p_n$ with $p_1=x$, $p_n=z$ and $d(p_i,p_{i+1})<\varepsilon$. If $z\in A_\varepsilon$ then also $B_\varepsilon(z)\subset A_\varepsilon$ since given $w\in B_\varepsilon(z)$ we can simply add the point $p_{n+1}=w$ to the sequence $p_1,\ldots,p_n$.
This immediately shows that $A_\varepsilon$ is an open subset of $X$. On the other hand we can show that $A_\varepsilon$ is also closed. In fact suppose that $x_n\in A_\varepsilon$ and $x_n \to \bar x\in X$. Then there exists $k$ such that $\bar x\in B_\varepsilon(x_k)$ and hence $\bar x\in A_\varepsilon$ by the property stated above. Since both $A_\varepsilon$ and its complementary set are open then, 
being $X$ connected, we conclude that $A_\varepsilon$ is either empty or its complementary set is empty. Clearly $x\in A_\varepsilon$ so we conclude that $A_\varepsilon=X$. Since this is true for all $\varepsilon>0$ the first implication is proven.

Let us prove the reverse implication. Suppose by contradiction that $X$ is not connected. This means that two non-empty open sets $A,B$ exist such that 
$A\cup B=X$ and $A\cap B = \emptyset$. Since $A$ is the complementary set of $B$ and vice-versa, we know that $A$ and $B$ are closed too. Being $X$ compact we conclude that both $A$ and $B$ are compact sets.
We now claim that 
\[
   \delta:= \inf_{a\in A, b\in B} d(a,b) > 0. 
\]
Suppose by contradiction that $\delta=0$.
In this case by definition of infimum, there exist two sequences 
$a_k\in A$ and $b_k\in B$ such that $d(a_k,b_k)\to 0$. Since $A$ and $B$ are compact, up to a subsequence we may and shall suppose that $a_k \to a \in A$ and $b_k \to b \in B$. By the continuity of the distance function we conclude that $d(a,b)=0$ i.e.\ $a=b$ which is in contradiction with the condition $A\cap B=\emptyset$. So the claim is proven.

As a consequence, given $\varepsilon<\delta$ it is not possible to join a point of $A$ with a point of $B$. In fact in the sequence $p_1,\ldots,p_n$ there should exists two consecutive points $p_i$ and $p_{i+1}$ with $p_i\in A$ and $p_{i+1}\in B$. By the previous observation we would conclude that $d(p_i,p_{i+1})\ge \delta > \varepsilon$.
%%%%%
%%%%%
\end{document}
