\documentclass[12pt]{article}
\usepackage{pmmeta}
\pmcanonicalname{PseudometricTopology}
\pmcreated{2013-03-22 14:40:47}
\pmmodified{2013-03-22 14:40:47}
\pmowner{matte}{1858}
\pmmodifier{matte}{1858}
\pmtitle{pseudometric topology}
\pmrecord{7}{36284}
\pmprivacy{1}
\pmauthor{matte}{1858}
\pmtype{Definition}
\pmcomment{trigger rebuild}
\pmclassification{msc}{54E35}
\pmdefines{pseudometrizable}
\pmdefines{pseudometric topology}
\pmdefines{pseudo-metric}
\pmdefines{pseudometrizable topological space}
\pmdefines{pseudo-metrizable topological space}

\endmetadata

% this is the default PlanetMath preamble.  as your knowledge
% of TeX increases, you will probably want to edit this, but
% it should be fine as is for beginners.

% almost certainly you want these
\usepackage{amssymb}
\usepackage{amsmath}
\usepackage{amsfonts}
\usepackage{amsthm}

\usepackage{mathrsfs}

% used for TeXing text within eps files
%\usepackage{psfrag}
% need this for including graphics (\includegraphics)
%\usepackage{graphicx}
% for neatly defining theorems and propositions
%
% making logically defined graphics
%%%\usepackage{xypic}

% there are many more packages, add them here as you need them

% define commands here

\newcommand{\sR}[0]{\mathbb{R}}
\newcommand{\sC}[0]{\mathbb{C}}
\newcommand{\sN}[0]{\mathbb{N}}
\newcommand{\sZ}[0]{\mathbb{Z}}

 \usepackage{bbm}
 \newcommand{\Z}{\mathbbmss{Z}}
 \newcommand{\C}{\mathbbmss{C}}
 \newcommand{\R}{\mathbbmss{R}}
 \newcommand{\Q}{\mathbbmss{Q}}



\newcommand*{\norm}[1]{\lVert #1 \rVert}
\newcommand*{\abs}[1]{| #1 |}



\newtheorem{thm}{Theorem}
\newtheorem{defn}{Definition}
\newtheorem{prop}{Proposition}
\newtheorem{lemma}{Lemma}
\newtheorem{cor}{Corollary}
\begin{document}
Let $(X,d)$ be a pseudometric space. As in a metric space, we define
$$
  B_\varepsilon(x)=\{ y\in X\mid d(x,y)<\varepsilon \}.
$$
for $x\in X$, $\varepsilon>0$.
  
In the below, we show that the collection of sets 
$$ 
\mathscr{B}= \{ B_\varepsilon(x)\mid \varepsilon>0, x\in X\}
$$
form a base for a topology for $X$. We call this topology
the \PMlinkescapetext{\emph{pseudometric topology}} on $X$ 
induced by $d$. Also,
a topological space $X$ is a \emph{pseudometrizable topological space}
if there exists a pseudometric $d$ on $X$ whose
pseudometric topology coincides with the given topology 
for $X$ \cite{kelley, willard}. 


\begin{prop} 
$\mathscr{B}$ is a base for a topology.
\end{prop}


\begin{proof} We shall use the  \PMlinkid{this result}{5845}
to prove that $\mathscr{B}$ is a base.

First, as $d(x,x)=0$ for all $x\in X$, it follows 
   that $\mathscr{B}$ is a cover. 
Second, suppose $B_1,B_2\in \mathscr{B}$ and $z\in B_1\cap B_2$. 
We claim that there exists a $B_3\in \mathscr{B}$ such that 
\begin{eqnarray}
\label{ii} 
  z&\in& B_3\subseteq B_1\cap B_2.
\end{eqnarray}
By definition, $B_1 = B_{\varepsilon_1}(x_1)$ 
   and $B_2 = B_{\varepsilon_2}(x_2)$ for some $x_1,x_2\in X$
and $\varepsilon_1,\varepsilon_2>0$. Then
$$
  d(x_1, z)<\varepsilon_1, \quad   d(x_2, z)<\varepsilon_2.
$$
Now we can define $\delta = \min\{ \varepsilon_1-d(x_1, z), \varepsilon_2-d(x_2, z)\}>0$, and put
$$
  B_3 = B_\delta(z).
$$
If $y\in B_3$, then for $k=1,2$, we have by the triangle inequality
\begin{eqnarray*}
d(x_k,y) &\le & d(x_k, z) + d(z,y) \\
         &< & d(x_k, z) + \delta \\
         &\le & \varepsilon_k,
\end{eqnarray*}
so $B_3\subseteq B_k$ and condition \ref{ii} holds. 
\end{proof}

\subsubsection*{Remark}
In the proof, we have not used the fact that $d$ is 
symmetric. Therefore, we have, in fact, also shown that any 
quasimetric induces a topology.
 
\begin{thebibliography}{9}
\bibitem{kelley} J.L. Kelley, \emph{General Topology},
D. van Nostrand Company, Inc., 1955.
\bibitem{willard} S. Willard, \emph{General Topology},
Addison-Wesley, Publishing Company, 1970.
\end{thebibliography}
%%%%%
%%%%%
\end{document}
