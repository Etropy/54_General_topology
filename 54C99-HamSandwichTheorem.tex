\documentclass[12pt]{article}
\usepackage{pmmeta}
\pmcanonicalname{HamSandwichTheorem}
\pmcreated{2013-03-22 13:59:43}
\pmmodified{2013-03-22 13:59:43}
\pmowner{mathcam}{2727}
\pmmodifier{mathcam}{2727}
\pmtitle{ham sandwich theorem}
\pmrecord{6}{34772}
\pmprivacy{1}
\pmauthor{mathcam}{2727}
\pmtype{Theorem}
\pmcomment{trigger rebuild}
\pmclassification{msc}{54C99}
\pmrelated{BorsukUlamTheorem}

\usepackage{amssymb}
\usepackage{amsmath}
\usepackage{amsfonts}
% used for TeXing text within eps files
%\usepackage{psfrag}
% need this for including graphics (\includegraphics)
%\usepackage{graphicx}
% for neatly defining theorems and propositions
%\usepackage{amsthm}
% making logically defined graphics
%%%\usepackage{xypic}
\begin{document}
Let $A_1,\ldots,A_m$ be measurable bounded subsets of $\mathbb{R}^m$. Then there exists an $(m-1)$-dimensional hyperplane which \PMlinkescapetext{divides} each $A_i$ into two subsets of equal measure.

This theorem has such a colorful \PMlinkescapetext{name} because in the case $m=3$ it can be viewed as cutting a ham sandwich in half. For example, $A_1$ and $A_3$ could be two pieces of bread and $A_2$ a piece of ham. According to this theorem it is possible to make one \PMlinkescapetext{cut} to simultaneously \PMlinkescapetext{cut} all three objects exactly in half.
%%%%%
%%%%%
\end{document}
