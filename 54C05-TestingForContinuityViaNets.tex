\documentclass[12pt]{article}
\usepackage{pmmeta}
\pmcanonicalname{TestingForContinuityViaNets}
\pmcreated{2013-03-22 19:08:58}
\pmmodified{2013-03-22 19:08:58}
\pmowner{CWoo}{3771}
\pmmodifier{CWoo}{3771}
\pmtitle{testing for continuity via nets}
\pmrecord{7}{42053}
\pmprivacy{1}
\pmauthor{CWoo}{3771}
\pmtype{Result}
\pmcomment{trigger rebuild}
\pmclassification{msc}{54C05}
\pmclassification{msc}{26A15}
\pmrelated{Net}
\pmrelated{FirstCountable}

\endmetadata

\usepackage{amssymb,amscd}
\usepackage{amsmath}
\usepackage{amsfonts}
\usepackage{mathrsfs}

% used for TeXing text within eps files
%\usepackage{psfrag}
% need this for including graphics (\includegraphics)
%\usepackage{graphicx}
% for neatly defining theorems and propositions
\usepackage{amsthm}
% making logically defined graphics
%%\usepackage{xypic}
\usepackage{pst-plot}

% define commands here
\newcommand*{\abs}[1]{\left\lvert #1\right\rvert}
\newtheorem{prop}{Proposition}
\newtheorem{thm}{Theorem}
\newtheorem{ex}{Example}
\newcommand{\real}{\mathbb{R}}
\newcommand{\pdiff}[2]{\frac{\partial #1}{\partial #2}}
\newcommand{\mpdiff}[3]{\frac{\partial^#1 #2}{\partial #3^#1}}
\begin{document}
\begin{prop}  Let $X,Y$ be topological spaces and $f:X\to Y$.  Then the following are equivalent: 
\begin{enumerate}
\item $f$ is continuous;
\item If $(x_i)$ is a net in $X$ converging to $x$, then $(f(x_i))$ is a net in $Y$ converging to $f(x)$.
\item Whenever two nets $(x_i)$ and $(y_j)$ in $X$ converge to the same point, then $(f(x_i))$ and $(f(y_j))$
converge to the same point in $Y$.
\end{enumerate}
\end{prop}
\begin{proof}
$(1)\Leftrightarrow (2)$.  Let $A$ be the (directed) index set for $i$.  Suppose $f(x)\in U$ is open in $Y$.  Then $x\in f^{-1}(U)$ is open in $X$ since $f$ is continuous.  By assumption, $(x_i)$ is a net, so there is $b \in A$ such that $x_j \in f^{-1}(U)$ for all $j\ge b$.  This means that $f(x_j)\in U$ for all $i\ge b$, so $(f(x_i))$ is a net too.

Conversely, suppose $f$ is not continuous, say, at a point $x\in X$.  Then there is an open set $V$ containing $f(x)$ such that $f^{-1}(V)$ does not contain any open set containing $x$.  Let $A$ be the set of all open sets containing $x$.  Then under reverse inclusion, $A$ is a directed set (if $U_1,U_2\in A$, then $U_1\cap U_2\in A$).  Define a relation $R\subseteq A\times X$ as follows: $$(U,x)\in R \qquad\mbox{iff}\qquad x\in U-f^{-1}(V).$$  Then for each $U\in A$, there is an $x\in X$ such that $(U,x)\in R$, since $U\not\subseteq f^{-1}(V)$.  By the axiom of choice, we get a function $d\subseteq R$ from $A$ to $X$.  Write $d(U):=x_U$.  Since $A$ is directed, $(x_U)$ is a net.  In addition, $(x_U)$ converges to $x$ (just pick any $U\in A$, then for any $W\ge U$, we have $x \in W$ by the definition of $A$).  However, $(f(x_U))$ does not converge to $f(x)$, since $x_U\notin f^{-1}(V)$ for any $U\in A$.

$(2)\Leftrightarrow (3)$.  Suppose nets $(x_i)$ and $(y_j)$ both converge to $z\in X$.  Then, by assumption, $(f(x_i))$ and $(f(y_j))$ are nets converging to $f(z)\in Y$.

Conversely, suppose $(x_i)$ converges to $x$, and $i$ is indexed by a directed set $A$.  Define a net $(y_i)$ such that $y_i=x$ for all $i\in A$.  Then $(y_i)=(x)$ clearly converges to $x$.  Hence both $(f(x_i))$ and $(f(y_i))$ converge to the same point in $Y$.  But $(f(y_i))=(f(x))$ converges to $f(x)$, we see that $(f(x_i))$ converges to $f(x)$ as well.
\end{proof}

\textbf{Remark}.  In particular, if $X,Y$ are first countable, we may replace nets by sequences in the proposition.  In other words, $f$ is continuous iff it preserves converging sequences.
%%%%%
%%%%%
\end{document}
