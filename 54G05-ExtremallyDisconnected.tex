\documentclass[12pt]{article}
\usepackage{pmmeta}
\pmcanonicalname{ExtremallyDisconnected}
\pmcreated{2013-03-22 12:42:00}
\pmmodified{2013-03-22 12:42:00}
\pmowner{PrimeFan}{13766}
\pmmodifier{PrimeFan}{13766}
\pmtitle{extremally disconnected}
\pmrecord{8}{32982}
\pmprivacy{1}
\pmauthor{PrimeFan}{13766}
\pmtype{Definition}
\pmcomment{trigger rebuild}
\pmclassification{msc}{54G05}
\pmsynonym{extremely disconnected}{ExtremallyDisconnected}
\pmrelated{ConnectedSpace}

% this is the default PlanetMath preamble.  as your knowledge
% of TeX increases, you will probably want to edit this, but
% it should be fine as is for beginners.

% almost certainly you want these
\usepackage{amssymb}
\usepackage{amsmath}
\usepackage{amsfonts}

% used for TeXing text within eps files
%\usepackage{psfrag}
% need this for including graphics (\includegraphics)
%\usepackage{graphicx}
% for neatly defining theorems and propositions
%\usepackage{amsthm}
% making logically defined graphics
%%%\usepackage{xypic}

% there are many more packages, add them here as you need them

% define commands here
\begin{document}
A topological space $X$ is said to be {\em extremally disconnected} if every open set in $X$ has an open closure.

It can be shown that $X$ is extremally disconnected iff any two disjoint open sets in $X$ have disjoint closures. Every extremally disconnected space is totally disconnected.

\paragraph{Notes}
Some authors like \cite{willard} and \cite{kelley} use the above
definition as is, while others (e.g.~\cite{steen, bourbaki}) require
that an extremally disconnected space should (in addition to the above
condition) also be a Hausdorff space. 

\begin{thebibliography}{9}
\bibitem{willard} S.~Willard, \emph{General Topology},
Addison-Wesley, Publishing Company, 1970.
\bibitem{kelley} J.~L.~Kelley, \emph{General Topology}, 
D.~van~Nostrand Company, Inc., 1955.
\bibitem{steen} L.~A.~Steen, J.~A.~Seebach, Jr.,\
\emph{Counterexamples in topology},
Holt, Rinehart and Winston, Inc., 1970.
\bibitem{bourbaki} N.~Bourbaki, \emph{General Topology, Part 1},
Addison-Wesley Publishing Company, 1966.
\end{thebibliography}
%%%%%
%%%%%
\end{document}
