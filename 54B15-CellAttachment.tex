\documentclass[12pt]{article}
\usepackage{pmmeta}
\pmcanonicalname{CellAttachment}
\pmcreated{2013-03-22 13:25:53}
\pmmodified{2013-03-22 13:25:53}
\pmowner{yark}{2760}
\pmmodifier{yark}{2760}
\pmtitle{cell attachment}
\pmrecord{13}{33991}
\pmprivacy{1}
\pmauthor{yark}{2760}
\pmtype{Definition}
\pmcomment{trigger rebuild}
\pmclassification{msc}{54B15}
\pmsynonym{cell adjunction}{CellAttachment}
\pmrelated{CWComplex}
\pmdefines{cell}
\pmdefines{open cell}
\pmdefines{closed cell}
\pmdefines{attaching map}

\endmetadata

\usepackage{amsmath}

\newtheorem{rmk}{Remark}

\newcommand{\funcsig}[2]{#1\rightarrow #2}
\newcommand{\funcdef}[3]{#1\colon\funcsig{#2}{#3}}
\newcommand{\bdry}{\partial}
\newcommand{\set}[1]{{\left\{#1\right\}}}

% open cells (not very nice...)
\newcommand{\oce}{\smash{\overset{\circ}e}}
\newcommand{\ocD}{D^\circ}

\begin{document}
\PMlinkescapeword{closed}
\PMlinkescapeword{open}

Let $X$ be a topological space,
and let $Y$ be the adjunction
$Y := X\cup_\varphi D^k$,
where $D^k$ is a closed \PMlinkname{$k$-ball}{StandardNBall}
and $\funcdef{\varphi}{S^{k-1}}{X}$ is a continuous map,
with $S^{k-1}$ is the $(k-1)$-sphere considered as the boundary of $D^k$.
Then, we say that $Y$ is obtained from $X$
by the {\em attachment of a $k$-cell,} by the {\em attaching map} $\varphi.$
The image $e^k$ of $D^k$ in $Y$ is called a {\em closed $k$-cell},
and the image $\oce^k$ of the interior
\[
  \ocD := D^k\setminus S^{k-1}
\]
of $D^k$ is the corresponding {\em open $k$-cell}.

Note that for $k=0$ the above definition reduces to
the statement that $Y$ is the disjoint union of $X$ with a one-point space.

More generally, we say that $Y$ is obtained from $X$ by {\em cell attachment\/}
if $Y$ is homeomorphic to an adjunction $X\cup_\set{\varphi_i} D^{k_i}$,
where the maps $\set{\varphi_i}$ into $X$
are defined on the boundary spheres of closed balls $\set{D^{k_i}}$.

%\begin{rmk}
%A recognition principle for attached cells is as follows:
%Let $Y$ be a Hausdorff topological space and 
%$e$ a closed subspace such that for some $k\ge 1$ there exists a map
%$\funcdef{\Phi}{D^k}{Y}$ satisfying
%\begin{enumerate}
%\item
%$\Phi(D^k) = e$ and
%\item
%the restriction of $\Phi$ to $\ocD^k := D^k\setminus \bdry D^k$
%is an embedding.
%\end{enumerate}
%Then, $Y$ is obtained from $X := Y\setminus\Phi(\ocD^k)$
%by the attachment of the $k$-cell $e$.
%$k$ is called the {\em dimension} of $e$,
%and is well-defined by virtue of the invariance of domain theorem.
%
%Attached $0$-cells are recognized as being isolated points of $X$.
%\end{rmk}
%%%%%
%%%%%
\end{document}
