\documentclass[12pt]{article}
\usepackage{pmmeta}
\pmcanonicalname{Torus}
\pmcreated{2013-03-22 12:55:17}
\pmmodified{2013-03-22 12:55:17}
\pmowner{Daume}{40}
\pmmodifier{Daume}{40}
\pmtitle{torus}
\pmrecord{15}{33274}
\pmprivacy{1}
\pmauthor{Daume}{40}
\pmtype{Definition}
\pmcomment{trigger rebuild}
\pmclassification{msc}{54B15}
\pmclassification{msc}{51H05}
\pmrelated{MobiusStrip}
\pmrelated{NTorus}
\pmrelated{SurfaceOfRevolution2}
\pmdefines{major radius}
\pmdefines{minor radius}

\endmetadata

% this is the default PlanetMath preamble.  as your knowledge
% of TeX increases, you will probably want to edit this, but
% it should be fine as is for beginners.

% almost certainly you want these
\usepackage{amssymb}
\usepackage{amsmath}
\usepackage{amsfonts}

% used for TeXing text within eps files
%\usepackage{psfrag}
% need this for including graphics (\includegraphics)
\usepackage{graphicx}
% for neatly defining theorems and propositions
%\usepackage{amsthm}
% making logically defined graphics
%%%\usepackage{xypic} 

% there are many more packages, add them here as you need them

% define commands here
\begin{document}
Visually, the torus looks like a doughnut. Informally, we take a rectangle, identify two edges to form a cylinder, and then identify the two ends of the cylinder to form the torus. Doing this gives us a surface of genus one. It can also be described as the Cartesian product of two circles, that is, $S^1 \times S^1$. The torus can be parameterized in Cartesian coordinates by:
$$x = \cos(s) \cdot(R + r \cdot \cos(t))$$
$$y = \sin(s) \cdot (R + r \cdot \cos(t))$$
$$z = r \cdot \sin(t)$$
with $R$ the \emph{major radius} and $r$ the \emph{minor radius} are constant, and $s,t \in [0,2\pi)$.

\begin{center}
\includegraphics[scale=0.8]{torus} \\
\tiny{Figure 1: A torus generated with Mathematica 4.1}
\end{center}

To create the torus mathematically, we start with the closed subset $X = [0,1] \times [0,1] \subseteq \mathbb{R}^2$. Let $X^*$ be the set with elements:
$$\{ x \times 0, x \times 1 \mid 0 < x < 1 \}$$
$$\{ 0 \times y, 1 \times y \mid 0 < y < 1 \}$$
and also the four-point set
$$\{ 0 \times 0, 1 \times 0, 0 \times 1, 1 \times 1 \}.$$

This can be schematically represented in the following diagram.
\begin{center}
\includegraphics[scale=0.5]{torus-2} \\
\tiny{Diagram 1: The identifications made on $I^2$ to make a torus. \\ Opposite sides are identified with equal orientations, and the four corners \\
are identified to one point.}
\end{center}

Note that $X^*$ is a partition of $X$, where we have identified opposite sides of the square together, and all four corners together. We can then form the quotient topology induced by the quotient map $p\colon X \longrightarrow X^*$ by sending each element $x \in X$ to the corresponding element of $X^*$ containing $x$. \\
%%%%%
%%%%%
\end{document}
