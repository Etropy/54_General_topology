\documentclass[12pt]{article}
\usepackage{pmmeta}
\pmcanonicalname{LongLine}
\pmcreated{2013-03-22 13:29:40}
\pmmodified{2013-03-22 13:29:40}
\pmowner{Dr_Absentius}{537}
\pmmodifier{Dr_Absentius}{537}
\pmtitle{long line}
\pmrecord{17}{34069}
\pmprivacy{1}
\pmauthor{Dr_Absentius}{537}
\pmtype{Definition}
\pmcomment{trigger rebuild}
\pmclassification{msc}{54G20}

\usepackage{amssymb}
\usepackage{amsmath}
\usepackage{amsfonts}

%---------------------  Greek letters, etc ------------------------- 

\newcommand{\CA}{\mathcal{A}}
\newcommand{\CC}{\mathcal{C}}
\newcommand{\CM}{\mathcal{M}}
\newcommand{\CP}{\mathcal{P}}
\newcommand{\CS}{\mathcal{S}}
\newcommand{\BC}{\mathbb{C}}
\newcommand{\BN}{\mathbb{N}}
\newcommand{\BR}{\mathbb{R}}
\newcommand{\BZ}{\mathbb{Z}}
\newcommand{\FF}{\mathfrak{F}}
\newcommand{\FL}{\mathfrak{L}}
\newcommand{\FM}{\mathfrak{M}}
\newcommand{\Ga}{\alpha}
\newcommand{\Gb}{\beta}
\newcommand{\Gg}{\gamma}
\newcommand{\GG}{\Gamma}
\newcommand{\Gd}{\delta}
\newcommand{\GD}{\Delta}
\newcommand{\Ge}{\varepsilon}
\newcommand{\Gz}{\zeta}
\newcommand{\Gh}{\eta}
\newcommand{\Gq}{\theta}
\newcommand{\GQ}{\Theta}
\newcommand{\Gi}{\iota}
\newcommand{\Gk}{\kappa}
\newcommand{\Gl}{\lambda}
\newcommand{\GL}{\Lamda}
\newcommand{\Gm}{\mu}
\newcommand{\Gn}{\nu}
\newcommand{\Gx}{\xi}
\newcommand{\GX}{\Xi}
\newcommand{\Gp}{\pi}
\newcommand{\GP}{\Pi}
\newcommand{\Gr}{\rho}
\newcommand{\Gs}{\sigma}
\newcommand{\GS}{\Sigma}
\newcommand{\Gt}{\tau}
\newcommand{\Gu}{\upsilon}
\newcommand{\GU}{\Upsilon}
\newcommand{\Gf}{\varphi}
\newcommand{\GF}{\Phi}
\newcommand{\Gc}{\chi}
\newcommand{\Gy}{\psi}
\newcommand{\GY}{\Psi}
\newcommand{\Gw}{\omega}
\newcommand{\GW}{\Omega}
\newcommand{\Gee}{\epsilon}
\newcommand{\Gpp}{\varpi}
\newcommand{\Grr}{\varrho}
\newcommand{\Gff}{\phi}
\newcommand{\Gss}{\varsigma}

\def\co{\colon\thinspace}
\begin{document}
\PMlinkescapeword{induced}
The \emph{long line} is a non-paracompact Hausdorff $1$-dimensional manifold
 constructed as
follows. Let $\GW$ be the first uncountable ordinal (viewed as an ordinal space) and consider the set 
$$L:=\GW\times [0,1)$$
endowed with the order topology induced by the 
lexicographical order, that is the order defined by 
$$(\Ga_1,t_1) < (\Ga_2,t_2) \iff \Ga_1<\Ga_2 \quad\text{or}\quad
(\Ga_1=\Ga_2 \quad\text{and}\quad t_1<t_2)\,.$$
Intuitively $L$ is obtained by ``filling the gaps'' between consecutive
ordinals in $\GW$ with intervals, much the same way that 
nonnegative  reals are 
obtained by filling the gaps between consecutive natural numbers with intervals. 

Some of the  properties of the long line:
\begin{itemize}
\item $L$ is a chain.

\item $L$ is not compact; in fact $L$ is not Lindel\"of.

Indeed $\left\{\,[\,0,\Ga):\Ga<\GW\right\}$ is an open cover of $L$ that has no
countable subcovering. To see this notice that 
$$\bigcup\left\{\, [\,0,\Ga_x):x\in X\right\}=\left[\,0,\sup\{\Ga_x:x\in X\}\right)\,$$
 and since the supremum of a countable
collection of countable ordinals is a countable ordinal such a union can
 never be $[\,0,\GW)$.

 \item However, $L$ is sequentially compact.

Indeed every sequence has a convergent subsequence. To see this notice that
given a sequence $a:=(a_n)$ of elements of $L$ there is an ordinal $\Ga$ such
that all the terms of  $a$ are in the subset $[\,0,\Ga\,]$. Such a subset is
compact since it is homeomorphic to $[\,0,1\,]$.

\item $L$ therefore is not metrizable.

\item $L$ is a $1$--dimensional locally Euclidean

\item $L$ therefore is not paracompact.

\item $L$ is first countable.

\item $L$ is not separable.

\item All homotopy groups of $L$ are trivial.

\item However, $L$ is not contractible.
 
\end{itemize}

\section*{Variants}
There are several variations of the above construction.

\begin{itemize}
\item Instead of $[\,0,\GW)$ one can use $(0,\GW)$ or $[\,0,\GW\,]$. The latter (obtained by adding a single point to $L$) is compact.

\item One can consider the ``double'' of the above construction. That is the
  space obtained by gluing two copies of $L$ along $0$. The resulting open
  manifold is not homeomorphic to $L\setminus \{0\}$.
\end{itemize}
%%%%%
%%%%%
\end{document}
