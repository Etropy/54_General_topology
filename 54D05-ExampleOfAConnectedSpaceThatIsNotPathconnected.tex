\documentclass[12pt]{article}
\usepackage{pmmeta}
\pmcanonicalname{ExampleOfAConnectedSpaceThatIsNotPathconnected}
\pmcreated{2013-03-22 12:46:33}
\pmmodified{2013-03-22 12:46:33}
\pmowner{yark}{2760}
\pmmodifier{yark}{2760}
\pmtitle{example of a connected space that is not path-connected}
\pmrecord{16}{33087}
\pmprivacy{1}
\pmauthor{yark}{2760}
\pmtype{Example}
\pmcomment{trigger rebuild}
\pmclassification{msc}{54D05}
\pmrelated{ConnectedSpace}
\pmrelated{PathConnected}
\pmdefines{topologist's sine curve}

\usepackage{amssymb}
\usepackage{amsmath}
\usepackage{amsfonts}

%\usepackage{graphicx}

\newcommand{\Prob}[2]{\mathbb{P}_{#1}\left\{#2\right\}}
\newcommand{\Expect}{\mathbb{E}}
\newcommand{\norm}[1]{\left\|#1\right\|}

% Some sets
\newcommand{\Nats}{\mathbb{N}}
\newcommand{\Ints}{\mathbb{Z}}
\newcommand{\Reals}{\mathbb{R}}
\newcommand{\Complex}{\mathbb{C}}
\begin{document}
\PMlinkescapeword{contain}
\PMlinkescapeword{contains}
\PMlinkescapeword{continuous}
\PMlinkescapeword{example}
\PMlinkescapeword{induced}
\PMlinkescapeword{open}

This standard example shows that
a connected topological space need not be path-connected
(the converse is true, however).

Consider the topological spaces
\begin{eqnarray*}
X_1 &= \left\{(0,y)\mid y\in[-1,1]\right\}\\
X_2 &= \left\{(x,\sin\frac{1}{x})\mid x>0\right\}\\
X &= X_1 \cup X_2\\
\end{eqnarray*}
with the topology induced from $\Reals^2$.

$X_2$ is often called the ``\emph{topologist's sine curve}'', and $X$ is its closure.

$X$ is not path-connected.
Indeed, assume to the contrary that there exists a \PMlinkname{path}{PathConnected}
$\gamma\colon[0,1]\to X$ with
$\gamma(0)=(\frac{1}{\pi},0)$ and $\gamma(1)=(0,0)$.
Let
\[
  c = \inf \left\{ t\in[0,1] \mid \gamma(t)\in X_1 \right\}.
\]
Then $\gamma([0,c])$ contains at most one point of $X_1$,
while $\overline{\gamma([0,c])}$ contains all of $X_1$.
So $\gamma([0,c])$ is not closed, and therefore not compact.
But $\gamma$ is continuous and $[0,c]$ is compact,
so $\gamma([0,c])$ must be compact
(as a continuous image of a compact set is compact),
which is a contradiction.

But $X$ is connected.
Since both ``parts'' of the topologist's sine curve are themselves connected,
neither can be partitioned into two open sets.
And any open set which contains points of the line segment $X_1$
must contain points of $X_2$.
So $X$ is not the disjoint union of two nonempty open sets,
and is therefore connected.
%%%%%
%%%%%
\end{document}
