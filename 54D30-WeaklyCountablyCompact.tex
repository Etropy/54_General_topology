\documentclass[12pt]{article}
\usepackage{pmmeta}
\pmcanonicalname{WeaklyCountablyCompact}
\pmcreated{2013-03-22 12:06:46}
\pmmodified{2013-03-22 12:06:46}
\pmowner{yark}{2760}
\pmmodifier{yark}{2760}
\pmtitle{weakly countably compact}
\pmrecord{9}{31234}
\pmprivacy{1}
\pmauthor{yark}{2760}
\pmtype{Definition}
\pmcomment{trigger rebuild}
\pmclassification{msc}{54D30}
\pmsynonym{limit point compact}{WeaklyCountablyCompact}
\pmsynonym{limit-point compact}{WeaklyCountablyCompact}
%\pmkeywords{topology}
\pmrelated{Compact}
\pmrelated{CountablyCompact}
\pmrelated{SequentiallyCompact}
\pmrelated{PseudocompactSpace}
\pmdefines{limit point compactness}
\pmdefines{weak countable compactness}


\begin{document}
A topological space $X$ is said to be \emph{weakly countably compact}
(or \emph{limit point compact})
if every infinite subset of $X$ has a limit point.

Every countably compact space is weakly countably compact.
The converse is true in \PMlinkname{$\mathrm{T}_1$ spaces}{T1Space}.

A metric space is weakly countably compact if and only if it is compact.

An easy example of a space $X$
that is not weakly countably compact
is any infinite set with the discrete topology.
A more interesting example is the countable complement topology
on an uncountable set.

%%%%%
%%%%%
%%%%%
\end{document}
