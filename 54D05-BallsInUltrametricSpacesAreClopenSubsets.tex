\documentclass[12pt]{article}
\usepackage{pmmeta}
\pmcanonicalname{BallsInUltrametricSpacesAreClopenSubsets}
\pmcreated{2013-03-22 18:20:15}
\pmmodified{2013-03-22 18:20:15}
\pmowner{MFH}{21412}
\pmmodifier{MFH}{21412}
\pmtitle{balls in ultrametric spaces are clopen subsets}
\pmrecord{5}{40970}
\pmprivacy{1}
\pmauthor{MFH}{21412}
\pmtype{Example}
\pmcomment{trigger rebuild}
\pmclassification{msc}{54D05}


\begin{document}
In an ultrametric space, both open and closed balls are clopen subsets.

It is indeed straightforward (exercise!) to show that the set of all open balls of radius $r$, 
centered in any of the points of a closed ball of radius $r$, forms a partition of the latter.

Thus, in particular, any point of a closed ball is an interior point,
and the same holds for the complement of an open ball.

% Indeed, if $x$ is any point of a closed ball of radius $r$, 
% then the open ball of radius $r$, centered in $x$, is contained in the closed ball.
% PROOF:
% Let x be in the closed ball with center c and radius r.
% Let y be in the open ball with center x and radius r.
% Then |cy| <= max(|cx|,|xy|) = |cx| = r, i.e. y is in the closed ball.
% Thus, the whole open B(x,r) is in the closed ball B'(c,r)

% Let |xy| >= r, assume there is z in B(x,r) and B(y,r).
% Then |xz|<r,|yz|<r, impossible since r <= |xy| <= max(|xz|,|yz|) < r.
% Thus, open balls with centers at distance >= r do not intersect.
% This shows that each point of the complement of the open ball 
% is an interior point of that complement (which in fact contains 
% the whole open ball of radius r around the given point).

% Assume z is in B(x,r) and in B(y,r).
% Then |xy| <= max(xz,yz) < r. Thus x is in B(y,r) and y is in B(x,r).
% Now any other point t which is in B(x,r) verifies yt <= max(yx,tx) < r
% and is in B(y,r). So B(x,r) c B(y,r) and reciprocally, i.e. equality.
% Thus 2 open balls of radius r either are the same (if their centers
% are at distance < r), or are disjoint (if the centers are at dist. >= r).
%%%%%
%%%%%
\end{document}
