\documentclass[12pt]{article}
\usepackage{pmmeta}
\pmcanonicalname{TheUnionOfALocallyFiniteCollectionOfClosedSetsIsClosed}
\pmcreated{2013-03-22 16:14:45}
\pmmodified{2013-03-22 16:14:45}
\pmowner{yark}{2760}
\pmmodifier{yark}{2760}
\pmtitle{the union of a locally finite collection of closed sets is closed}
\pmrecord{10}{38349}
\pmprivacy{1}
\pmauthor{yark}{2760}
\pmtype{Theorem}
\pmcomment{trigger rebuild}
\pmclassification{msc}{54A99}

\usepackage{amssymb}
\usepackage{amsmath}
\usepackage{amsfonts}

\usepackage{amsthm}
\newtheorem*{thm*}{Theorem}

\begin{document}
\PMlinkescapeword{meet}
\PMlinkescapeword{meets}
\PMlinkescapeword{open}

The union of a collection of closed subsets of a topological space need not,
of course, be closed. However, we do have the following result:

\begin{thm*}
The union of a locally finite collection of closed subsets
of a topological space is itself closed.
\end{thm*}

\begin{proof}
Let $\cal S$ be a locally finite collection of closed subsets
of a topological space $X$, and put $Y=\cup\cal S$.
Let $x\in X\setminus Y$.
By local finiteness there is an open neighbourhood $U$ of $x$
that meets only finitely many members of $\cal S$,
say $A_1,\dots,A_n$.
So $U\setminus Y=U\setminus\bigcup_{i=1}^n A_i$, which is open.
Thus $U\setminus Y$ is an open neighbourhood of $x$ that does not meet $Y$.
It follows that $Y$ is closed.
\end{proof}

One use for this result can be found in the entry on gluing together continuous functions.
%%%%%
%%%%%
\end{document}
