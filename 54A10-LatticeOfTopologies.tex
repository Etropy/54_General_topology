\documentclass[12pt]{article}
\usepackage{pmmeta}
\pmcanonicalname{LatticeOfTopologies}
\pmcreated{2013-03-22 16:54:42}
\pmmodified{2013-03-22 16:54:42}
\pmowner{CWoo}{3771}
\pmmodifier{CWoo}{3771}
\pmtitle{lattice of topologies}
\pmrecord{8}{39172}
\pmprivacy{1}
\pmauthor{CWoo}{3771}
\pmtype{Definition}
\pmcomment{trigger rebuild}
\pmclassification{msc}{54A10}
\pmrelated{Coarser}
\pmdefines{common refinement}

\endmetadata

\usepackage{amssymb,amscd}
\usepackage{amsmath}
\usepackage{amsfonts}

% used for TeXing text within eps files
%\usepackage{psfrag}
% need this for including graphics (\includegraphics)
%\usepackage{graphicx}
% for neatly defining theorems and propositions
\usepackage{amsthm}
% making logically defined graphics
%%\usepackage{xypic}
\usepackage{pst-plot}
\usepackage{psfrag}

% define commands here
\newtheorem{prop}{Proposition}
\newtheorem{thm}{Theorem}
\newtheorem{ex}{Example}
\newcommand{\real}{\mathbb{R}}
\begin{document}
Let $X$ be a set.  Let $L$ be the set of all topologies on $X$.  We may order $L$ by inclusion.  When $\mathcal{T}_1\subseteq \mathcal{T}_2$, we say that $\mathcal{T}_2$ is \PMlinkname{finer}{Finer} than $\mathcal{T}_1$, or that $\mathcal{T}_2$ refines $\mathcal{T}_1$.

\begin{thm}  $L$, ordered by inclusion, is a complete lattice. \end{thm}
\begin{proof}
Clearly $L$ is a partially ordered set when ordered by $\subseteq$.  Furthermore, given any family of topologies $\mathcal{T}_i$ on $X$, their intersection $\bigcap \mathcal{T}_i$ also defines a topology on $X$.  Finally, let $\mathcal{B}_i$'s be the corresponding subbases for the $\mathcal{T}_i$'s and let $\mathcal{B}=\bigcup \mathcal{B}_i$.  Then $\mathcal{T}$ generated by $\mathcal{B}$ is easily seen to be the supremum of the $\mathcal{T}_i$'s.
\end{proof}

Let $L$ be the lattice of topologies on $X$.  Given $\mathcal{T}_i\in L$, $\mathcal{T}:=\bigvee \mathcal{T}_i$ is called the \emph{common refinement} of $\mathcal{T}_i$.  By the proof above, this is the coarsest topology that is \PMlinkescapetext{finer} than each $\mathcal{T}_i$.

If $X$ is non-empty with more than one element, $L$ is also an atomic lattice.  Each atom is a topology generated by one non-trivial subset of $X$ (non-trivial being non-empty and not $X$).  The atom has the form $\lbrace \varnothing, A, X\rbrace$, where $\varnothing \subset A\subset X$.

\textbf{Remark}.  In general, a \emph{lattice of topologies} on a set $X$ is a sublattice of \emph{the} lattice of topologies $L$ (mentioned above) on $X$.
%%%%%
%%%%%
\end{document}
