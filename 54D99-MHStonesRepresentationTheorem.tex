\documentclass[12pt]{article}
\usepackage{pmmeta}
\pmcanonicalname{MHStonesRepresentationTheorem}
\pmcreated{2013-03-22 13:25:34}
\pmmodified{2013-03-22 13:25:34}
\pmowner{rspuzio}{6075}
\pmmodifier{rspuzio}{6075}
\pmtitle{M. H. Stone's representation theorem}
\pmrecord{19}{33983}
\pmprivacy{1}
\pmauthor{rspuzio}{6075}
\pmtype{Theorem}
\pmcomment{trigger rebuild}
\pmclassification{msc}{54D99}
\pmclassification{msc}{06E99}
\pmclassification{msc}{03G05}
\pmsynonym{Stone representation theorem}{MHStonesRepresentationTheorem}
\pmsynonym{Stone's representation theorem}{MHStonesRepresentationTheorem}
\pmrelated{RepresentingABooleanLatticeByFieldOfSets}
\pmrelated{DualSpaceOfABooleanAlgebra}

%\documentclass{amsart}
\usepackage{amsmath}
%\usepackage[all,poly,knot,dvips]{xy}
%\usepackage{pstricks,pst-poly,pst-node,pstcol}


\usepackage{amssymb,latexsym}

\usepackage{amsthm,latexsym}
\usepackage{eucal,latexsym}

% THEOREM Environments --------------------------------------------------

\newtheorem{thm}{Theorem}
 \newtheorem*{mainthm}{Main~Theorem}
 \newtheorem{cor}[thm]{Corollary}
 \newtheorem{lem}[thm]{Lemma}
 \newtheorem{prop}[thm]{Proposition}
 \newtheorem{claim}[thm]{Claim}
 \theoremstyle{definition}
 \newtheorem{defn}[thm]{Definition}
 \theoremstyle{remark}
 \newtheorem{rem}[thm]{Remark}
 \numberwithin{equation}{subsection}


%---------------------  Greek letters, etc ------------------------- 

\newcommand{\CA}{\mathcal{A}}
\newcommand{\CC}{\mathcal{C}}
\newcommand{\CM}{\mathcal{M}}
\newcommand{\CP}{\mathcal{P}}
\newcommand{\CS}{\mathcal{S}}
\newcommand{\BC}{\mathbb{C}}
\newcommand{\BN}{\mathbb{N}}
\newcommand{\BR}{\mathbb{R}}
\newcommand{\BZ}{\mathbb{Z}}
\newcommand{\FF}{\mathfrak{F}}
\newcommand{\FL}{\mathfrak{L}}
\newcommand{\FM}{\mathfrak{M}}
\newcommand{\Ga}{\alpha}
\newcommand{\Gb}{\beta}
\newcommand{\Gg}{\gamma}
\newcommand{\GG}{\Gamma}
\newcommand{\Gd}{\delta}
\newcommand{\GD}{\Delta}
\newcommand{\Ge}{\varepsilon}
\newcommand{\Gz}{\zeta}
\newcommand{\Gh}{\eta}
\newcommand{\Gq}{\theta}
\newcommand{\GQ}{\Theta}
\newcommand{\Gi}{\iota}
\newcommand{\Gk}{\kappa}
\newcommand{\Gl}{\lambda}
\newcommand{\GL}{\Lamda}
\newcommand{\Gm}{\mu}
\newcommand{\Gn}{\nu}
\newcommand{\Gx}{\xi}
\newcommand{\GX}{\Xi}
\newcommand{\Gp}{\pi}
\newcommand{\GP}{\Pi}
\newcommand{\Gr}{\rho}
\newcommand{\Gs}{\sigma}
\newcommand{\GS}{\Sigma}
\newcommand{\Gt}{\tau}
\newcommand{\Gu}{\upsilon}
\newcommand{\GU}{\Upsilon}
\newcommand{\Gf}{\varphi}
\newcommand{\GF}{\Phi}
\newcommand{\Gc}{\chi}
\newcommand{\Gy}{\psi}
\newcommand{\GY}{\Psi}
\newcommand{\Gw}{\omega}
\newcommand{\GW}{\Omega}
\newcommand{\Gee}{\epsilon}
\newcommand{\Gpp}{\varpi}
\newcommand{\Grr}{\varrho}
\newcommand{\Gff}{\phi}
\newcommand{\Gss}{\varsigma}

\def\co{\colon\thinspace}
\begin{document}
\begin{thm}
Given a Boolean algebra $B$ there exists a totally disconnected compact
Hausdorff space $X$ such that $B$ is isomorphic to the Boolean algebra 
of clopen subsets of $X$.
\end{thm}

\begin{proof}
Let $X=B^*$, the \PMlinkname{dual space}{DualSpaceOfABooleanAlgebra} of $B$, which is composed of all maximal ideals of $B$.  According to this \PMlinkname{entry}{DualSpaceOfABooleanAlgebra}, $X$ is a Boolean space (totally disconnected compact Hausdorff) whose topology is generated by the basis $$\mathcal{B}:=\lbrace M(a)\mid a\in B\rbrace,$$ where $M(a)=\lbrace M\in B^* \mid a\notin M\rbrace$.

Next, we show a general fact about the dual space $B^*$:
\begin{lem} $\mathcal{B}$ is the set of \emph{all} clopen sets in $X$. \end{lem}
\begin{proof}  Clearly, every element of $\mathcal{B}$ is clopen, by definition.  Conversely, suppose $U$ is clopen.  Then $U=\bigcup \lbrace M(a_i)\mid i\in I\rbrace$ for some index set $I$, since $U$ is open.  But $U$ is closed, so $B^*-U=\bigcup \lbrace M(a_j) \mid j\in J\rbrace$ for some index set $J$.  Hence $B^*=\bigcup \lbrace M(a_k) \mid k\in I\cup J \rbrace$.  Since $B^*$ is compact, there is a finite subset $K$ of $I\cup J$ such that $B^* =\bigcup \lbrace M(a_k)\mid k\in K\rbrace$.  Let $V= \bigcup \lbrace M(a_i)\mid i\in K\cap I\rbrace$.  Then $V\subseteq U$.  But $B^*-V \subseteq B^*-U$ also.  So $U=V$.  Let $y=\bigvee \lbrace a_i \mid i\in K\cap I\rbrace$, which exists because $K\cap I$ is finite.  As a result, $$U=V= \bigcup \lbrace M(a_i)\mid i\in K\cap I\rbrace = M(\bigvee \lbrace a_i\mid i\in K\cap I\rbrace)=M(y)\in \mathcal{B}.$$
\end{proof}

Finally, based on the result of \PMlinkname{this entry}{RepresentingABooleanLatticeByFieldOfSets}, $B$ is isomorphic to the field of sets $$F:=\lbrace F(a)\mid a\in B\rbrace,$$ where $F(a)=\lbrace P\mid P\mbox{ prime in }B,\mbox{ and }a\notin P\rbrace$.  Realizing that prime ideals and maximal ideals coincide in any Boolean algebra, the set $F$ is precisely $\mathcal{B}$.
\end{proof}

\textbf{Remark}.  There is also a dual version of the Stone representation theorem, which says that every Boolean space is homeomorphic to the dual space of some Boolean algebra.
%%%%%
%%%%%
\end{document}
