\documentclass[12pt]{article}
\usepackage{pmmeta}
\pmcanonicalname{LimitPointsAndClosureForConnectedSets}
\pmcreated{2013-03-22 15:17:56}
\pmmodified{2013-03-22 15:17:56}
\pmowner{matte}{1858}
\pmmodifier{matte}{1858}
\pmtitle{limit points and closure for connected sets}
\pmrecord{7}{37097}
\pmprivacy{1}
\pmauthor{matte}{1858}
\pmtype{Theorem}
\pmcomment{trigger rebuild}
\pmclassification{msc}{54D05}

\endmetadata

% this is the default PlanetMath preamble.  as your knowledge
% of TeX increases, you will probably want to edit this, but
% it should be fine as is for beginners.

% almost certainly you want these
\usepackage{amssymb}
\usepackage{amsmath}
\usepackage{amsfonts}
\usepackage{amsthm}

\usepackage{mathrsfs}

% used for TeXing text within eps files
%\usepackage{psfrag}
% need this for including graphics (\includegraphics)
%\usepackage{graphicx}
% for neatly defining theorems and propositions
%
% making logically defined graphics
%%%\usepackage{xypic}

% there are many more packages, add them here as you need them

% define commands here

\newcommand{\sR}[0]{\mathbb{R}}
\newcommand{\sC}[0]{\mathbb{C}}
\newcommand{\sN}[0]{\mathbb{N}}
\newcommand{\sZ}[0]{\mathbb{Z}}

 \usepackage{bbm}
 \newcommand{\Z}{\mathbbmss{Z}}
 \newcommand{\C}{\mathbbmss{C}}
 \newcommand{\F}{\mathbbmss{F}}
 \newcommand{\R}{\mathbbmss{R}}
 \newcommand{\Q}{\mathbbmss{Q}}



\newcommand*{\norm}[1]{\lVert #1 \rVert}
\newcommand*{\abs}[1]{| #1 |}



\newtheorem{thm}{Theorem}
\newtheorem{defn}{Definition}
\newtheorem{prop}{Proposition}
\newtheorem{lemma}{Lemma}
\newtheorem{cor}{Corollary}
\begin{document}
The below theorem shows that adding limit points to a connected 
set preserves  connectedness. 

\begin{thm} Suppose $A$ is a connected set in a topological space. 
If $A\subseteq B \subseteq \overline{A}$, then $B$ is connected.
     In particular, $\overline{A}$ is connected.
\end{thm}

Thus, on{e} wa{y} to prove that a space $X$ is connected is to find a dense
subspace in $X$ which is connected.

Two touching closed balls in $\R^2$ shows that this theorem does not hold
for the interior. Along the same lines, taking the closure does not
preserve separatedness.

\begin{proof} Let $X$ be the ambient topological space. 
By assumption, if $U,V\subseteq A$ are open and $U\cup V=A$, then 
$U\cap V\neq \emptyset$. 
To prove that $B$ is connected, let $U,V$ be open sets in 
$B$ such that $U\cup V = B$ and for a
contradition, suppose that $U\cap V=\emptyset$. 
Then there are open sets $R,S\subseteq X$ such that 
$$
   U= R\cap B, \quad V= S\cap B.
$$
It follows that $(R\cup S) \cap B=B$
  and $(R\cap S)\cap B=\emptyset$. 
Next, let $\tilde{U}, \tilde{V}$ be open sets in $A$ defined as
$$
   \tilde{U}= R\cap A, \quad \tilde{V}= S\cap A.
$$
Now 
$$
  A= B\cap A \subseteq (R\cup S)\cap A \subseteq A
$$
and as $ (R\cup S)\cap A = \tilde{U}\cup \tilde{V}$, it follows that
$\emptyset \neq \tilde{U}\cap \tilde{V} = (R\cap S) \cap A$. 
Then, by the properties of the closure operator, 
$$
  \emptyset \neq \overline{(R\cap S)\cap A} \supseteq (R\cap S)\cap \overline{A} \supseteq (R\cap S)\cap B = \emptyset.
$$
\end{proof}
%%%%%
%%%%%
\end{document}
