\documentclass[12pt]{article}
\usepackage{pmmeta}
\pmcanonicalname{HausdorffMetricInheritsCompleteness}
\pmcreated{2013-03-22 14:08:51}
\pmmodified{2013-03-22 14:08:51}
\pmowner{mps}{409}
\pmmodifier{mps}{409}
\pmtitle{Hausdorff metric inherits completeness}
\pmrecord{8}{35563}
\pmprivacy{1}
\pmauthor{mps}{409}
\pmtype{Theorem}
\pmcomment{trigger rebuild}
\pmclassification{msc}{54E35}
%\pmkeywords{Hausdorff}
%\pmkeywords{complete}
\pmrelated{Complete}

% this is the default PlanetMath preamble.  as your knowledge
% of TeX increases, you will probably want to edit this, but
% it should be fine as is for beginners.

% almost certainly you want these
\usepackage{amssymb}
\usepackage{amsmath}
\usepackage{amsfonts}

% used for TeXing text within eps files
%\usepackage{psfrag}
% need this for including graphics (\includegraphics)
%\usepackage{graphicx}
% for neatly defining theorems and propositions
\usepackage{amsthm}
% making logically defined graphics
%%%\usepackage{xypic}

% there are many more packages, add them here as you need them

% define commands here
\newtheorem{Theorem}{Theorem}
\begin{document}
\PMlinkescapeword{induced}
\begin{Theorem}
If $(X,d)$ is a complete metric space, then the Hausdorff metric induced by $d$ is also complete.
%If $(X,d)$ is complete, then so is $(\mathcal{F},d_H)$, where $\mathcal{F}$
%is the family of nonempty closed and bounded subsets of $X$ and $d_H$ is the
%Hausdorff metric.
\end{Theorem}

\begin{proof}
Suppose $(A_n)$ is a Cauchy sequence with respect to the Hausdorff metric.  By selecting a subsequence if necessary, we may assume that $A_n$ and $A_{n+1}$ are within $2^{-n}$ of each other, that is, that $A_n\subset K(A_{n+1},2^{-n})$ and $A_{n+1}\subset K(A_n, 2^{-n})$.
%Let us denote the Hausdorff metric by $d_H$.
%Suppose $(A_n)$ be a Cauchy sequence of nonempty, closed and bounded subsets of %$X$. 
%By selecting a subsequence if
%necessary, we may assume that $d_H(A_n,A_{n+1})<2^{-n}$ for all $n$.
Now for any natural number $N$, there is a sequence $(x_n)_{n\ge N}$ in $X$
such that $x_n\in A_n$ and $d(x_n,x_{n+1})<2^{-n}$.  Any such sequence is Cauchy 
with respect to $d$ and thus converges to some $x\in X$.  By applying the triangle inequality, we see that for any $n\ge N$, $d(x_n,x)<2^{-n+1}$.
%We claim that for any natural number $N$, there is a sequence
%$(d_n)_{n\ge N}$ in $X$ such that $x_n\in A_n$ and $d(x_n,x_{n+1})<2^{-n}$.
%We do this by applying the fact that $A_n\subset K(A_{n+1},2^{-n})$ for
%each $n$.  But any such sequence is Cauchy in $X$ and thus converges to some
%$x\in X$.  By applying the triangle inequality, we see that for any $n\ge N$,
%$d(x_n,x)<2^{-n+1}$.

Define $A$ to be the set of all $x$ such that $x$ is the limit of a sequence
$(x_n)_{n\ge 0}$ with $x_n\in A_n$ and $d(x_n,x_{n+1})<2^{-n}$.  
Then $A$ is nonempty.  
%The above
%claim implies that $A$ is nonempty.  
Furthermore, for any $n$, if $x\in A$,
then there is some $x_n\in A_n$ such that $d(x_n,x)<2^{-n+1}$, and so
$A\subset K(A_n,2^{-n+1})$.  Consequently, the set $\overline{A}$ is
nonempty, closed and bounded.

Suppose $\epsilon>0$.  Thus $\epsilon>2^{-N}>0$ for some $N$.  Let $n\ge N+1$.
Then by applying the claim in the first paragraph, we have that for any
$x_n\in A_n$, there is some $x\in X$ with $d(x_n,x)<2^{-n+1}$.  Hence
$A_n\subset K(\overline{A},2^{-n+1})$.  
Hence the sequence $(A_n)$ converges to $A$ in the Hausdorff metric.
\end{proof}

This proof is based on a sketch given in an exercise in~\cite{Mu}.
An exercise for the reader: is the set $A$ constructed above closed?

\begin{thebibliography}{9}
\bibitem{Mu}J. Munkres, \emph{Topology} (2nd edition), Prentice Hall, 1999.
\end{thebibliography}
%%%%%
%%%%%
\end{document}
