\documentclass[12pt]{article}
\usepackage{pmmeta}
\pmcanonicalname{AdherentPoint}
\pmcreated{2013-03-22 14:38:18}
\pmmodified{2013-03-22 14:38:18}
\pmowner{mathcam}{2727}
\pmmodifier{mathcam}{2727}
\pmtitle{adherent point}
\pmrecord{7}{36224}
\pmprivacy{1}
\pmauthor{mathcam}{2727}
\pmtype{Definition}
\pmcomment{trigger rebuild}
\pmclassification{msc}{54A99}

% this is the default PlanetMath preamble.  as your knowledge
% of TeX increases, you will probably want to edit this, but
% it should be fine as is for beginners.

% almost certainly you want these
\usepackage{amssymb}
\usepackage{amsmath}
\usepackage{amsfonts}
\usepackage{amsthm}

% used for TeXing text within eps files
%\usepackage{psfrag}
% need this for including graphics (\includegraphics)
%\usepackage{graphicx}
% for neatly defining theorems and propositions
%\usepackage{amsthm}
% making logically defined graphics
%%%\usepackage{xypic}

% there are many more packages, add them here as you need them

% define commands here

\newcommand{\mc}{\mathcal}
\newcommand{\mb}{\mathbb}
\newcommand{\mf}{\mathfrak}
\newcommand{\ol}{\overline}
\newcommand{\ra}{\rightarrow}
\newcommand{\la}{\leftarrow}
\newcommand{\La}{\Leftarrow}
\newcommand{\Ra}{\Rightarrow}
\newcommand{\nor}{\vartriangleleft}
\newcommand{\Gal}{\text{Gal}}
\newcommand{\GL}{\text{GL}}
\newcommand{\Z}{\mb{Z}}
\newcommand{\R}{\mb{R}}
\newcommand{\Q}{\mb{Q}}
\newcommand{\C}{\mb{C}}
\newcommand{\<}{\langle}
\renewcommand{\>}{\rangle}
\begin{document}
Let $X$ be a topological space and $A\subset X$ be a subset.  A point $x\in X$ is an \emph{adherent point} for $A$ if every open set containing $x$ contains at least one point of $A$.  A point $x$ is an adherent point for $A$ if and only if $x$ is in the closure of $A$.

Note that this definition is slightly more general than that of a limit point, in that for a limit point it is required that every open set containing $x$ contains at least one point of $A$ different from $x$.

\begin{thebibliography}{9}
\bibitem{steen} L.A. Steen, J.A.Seebach, Jr.,
\emph{Counterexamples in topology},
Holt, Rinehart and Winston, Inc., 1970.
\end{thebibliography}
%%%%%
%%%%%
\end{document}
