\documentclass[12pt]{article}
\usepackage{pmmeta}
\pmcanonicalname{SchauderLemma}
\pmcreated{2013-03-22 19:09:44}
\pmmodified{2013-03-22 19:09:44}
\pmowner{karstenb}{16623}
\pmmodifier{karstenb}{16623}
\pmtitle{Schauder lemma}
\pmrecord{4}{42068}
\pmprivacy{1}
\pmauthor{karstenb}{16623}
\pmtype{Theorem}
\pmcomment{trigger rebuild}
\pmclassification{msc}{54E50}
\pmclassification{msc}{46A30}
%\pmkeywords{almost open set}
%\pmkeywords{open mapping}
\pmrelated{OpenMappingTheorem}
\pmdefines{almost open set}

\usepackage{amssymb}
\usepackage{amsmath}
\usepackage{amsfonts}
\usepackage{amsthm}
\usepackage{mathrsfs}
\usepackage[sort&compress]{natbib}

%\usepackage{psfrag}
%\usepackage{graphicx}
%%%\usepackage{xypic}

%theorems
\theoremstyle{definition}
\newtheorem{Def}{Definition}

\theoremstyle{plain}
\newtheorem{Lem}{Theorem}
\newtheorem{Lem2}{Lemma}
\newtheorem{Cor}{Corollary}
\newtheorem{Rem}{Remark}




\begin{document}
The following theorem is in the functional analysis literature generally referred to as the \emph{Schauder lemma}. It is a version of the open mapping theorem in Fr\'echet spaces and is often used to verify the open-ness of linear, continuous maps.

\textbf{Theorem.} Let $E, F$ be Fr\'echet spaces. Denote by $\mathcal{U}_0(E), \mathcal{U}_0(F)$ the zero neighborhood filter of $E$ and $F$ respectively. Let $T \colon E \to F$ be a linear and continuous map which is \emph{almost open}, i.e. 
\begin{align*}
\forall_{U \in \mathscr{U}_0(E)} \overline{T(U)}^{F} &\in \mathscr{U}_0(F)
\end{align*}

Then $T$ is open.
%%%%%
%%%%%
\end{document}
