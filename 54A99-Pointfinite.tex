\documentclass[12pt]{article}
\usepackage{pmmeta}
\pmcanonicalname{Pointfinite}
\pmcreated{2013-03-22 16:17:02}
\pmmodified{2013-03-22 16:17:02}
\pmowner{yark}{2760}
\pmmodifier{yark}{2760}
\pmtitle{point-finite}
\pmrecord{5}{38398}
\pmprivacy{1}
\pmauthor{yark}{2760}
\pmtype{Definition}
\pmcomment{trigger rebuild}
\pmclassification{msc}{54A99}
\pmsynonym{point finite}{Pointfinite}
\pmrelated{LocallyFinite}
\pmrelated{Metacompact}
\pmdefines{point finite collection}
\pmdefines{point-finite collection}
\pmdefines{point finiteness}
\pmdefines{point-finiteness}

\endmetadata

\usepackage{amssymb}
\usepackage{amsmath}
\usepackage{amsfonts}

\begin{document}
A collection $\mathcal{U}$ of subsets of a topological space $X$ is said to be \emph{point-finite} if every point of $X$ lies in only finitely many members of $\mathcal{U}$.

Compare this to the stronger property of being locally finite.

Point-finiteness is used in the definition of metacompactness.
%%%%%
%%%%%
\end{document}
