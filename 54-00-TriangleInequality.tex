\documentclass[12pt]{article}
\usepackage{pmmeta}
\pmcanonicalname{TriangleInequality}
\pmcreated{2013-03-22 12:14:49}
\pmmodified{2013-03-22 12:14:49}
\pmowner{drini}{3}
\pmmodifier{drini}{3}
\pmtitle{triangle inequality}
\pmrecord{12}{31629}
\pmprivacy{1}
\pmauthor{drini}{3}
\pmtype{Definition}
\pmcomment{trigger rebuild}
\pmclassification{msc}{54-00}
\pmclassification{msc}{54E35}
\pmrelated{ProofOfLimitRuleOfProduct}
\pmrelated{TriangleInequalityOfComplexNumbers}
\pmdefines{reverse triangle inequality}

\endmetadata

%\usepackage{graphicx}
%%%%\usepackage{xypic} 
\usepackage{bbm}
\newcommand{\Z}{\mathbbmss{Z}}
\newcommand{\C}{\mathbbmss{C}}
\newcommand{\R}{\mathbbmss{R}}
\newcommand{\Q}{\mathbbmss{Q}}
\newcommand{\mathbb}[1]{\mathbbmss{#1}}
\newcommand{\eqref}[1]{\textrm{(\ref{#1})}}
\begin{document}
Let $(X,d)$ be a metric space. The \emph{triangle inequality} states
that for any three points $x,y,z\in X$ we have
\[
  d(x,y) \le d(x,z) + d(z,y).
\]

The name comes from the special case of $\R^n$ with the standard
topology, and geometrically meaning that in any triangle, the sum of
the lengths of two sides is greater (or equal) than the third.

Actually, the triangle inequality is one of the properties that define
a metric, so it holds in any metric space. Two important cases are
$\R$ with $d(x,y)=|x-y|$ and $\C$ with $d(x,y)=\Vert x-y\Vert$ (here
we are using complex modulus, not absolute value).

There is a second triangle inequality, sometimes called the
\emph{reverse triangle inequality}, which also holds in any metric
space and is derived from the definition of metric:
\[
  d(x,y) \ge |d(x,z) - d(z,y)|.
\]

In Euclidean geometry, this inequality is expressed by saying that each
side of a triangle is greater than the difference of the other two.

The reverse triangle inequality can be proved from the first triangle 
inequality, as we now show.

Let $x, y, z \in X$ be given. For any $a,b,c \in X$, from the first
triangle inequality we have:
\[
  d(a,b) \le d(a,c) + d(c,b)
\]
and thus (using $d(b,c) = d(c,b)$ for any $b, c \in X$):
\begin{equation}\label{eq:ste1}
d(a,c) \ge d(a,b) - d(b,c)
\end{equation}
and writing (\ref{eq:ste1}) with $a=x, b=z, c=y$:
\begin{equation}\label{eq:ste2}
d(x,y) \ge d(x,z) - d(z,y)
\end{equation}
while writing (\ref{eq:ste1}) with $a=y, b=z, c=x$ we get:
\[
  d(y,x) \ge d(y,z) - d(z,x)
\]
or
\begin{equation}\label{eq:ste3}
d(x,y) \ge d(z,y) - d(x,z);
\end{equation}
from (\ref{eq:ste2}) and (\ref{eq:ste3}), using the properties of the
absolute value, it follows finally:
\[
  d(x,y) \ge \left|d(x,z) - d(z,y)\right|
\]
which is the second triangle inequality.
%%%%%
%%%%%
%%%%%
\end{document}
