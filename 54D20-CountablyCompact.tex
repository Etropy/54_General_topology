\documentclass[12pt]{article}
\usepackage{pmmeta}
\pmcanonicalname{CountablyCompact}
\pmcreated{2013-03-22 12:06:43}
\pmmodified{2013-03-22 12:06:43}
\pmowner{Evandar}{27}
\pmmodifier{Evandar}{27}
\pmtitle{countably compact}
\pmrecord{8}{31233}
\pmprivacy{1}
\pmauthor{Evandar}{27}
\pmtype{Definition}
\pmcomment{trigger rebuild}
\pmclassification{msc}{54D20}
\pmsynonym{countable compactness}{CountablyCompact}
%\pmkeywords{topology}
\pmrelated{Compact}
\pmrelated{Lindelof}
\pmrelated{LimitPointCompact}

\usepackage{amssymb}
\usepackage{amsmath}
\usepackage{amsfonts}
\usepackage{graphicx}
%%%\usepackage{xypic}
\begin{document}
A topological space $X$ is said to be \emph{countably compact} if every countable open cover has a finite subcover.

Countable compactness is equivalent to limit point compactness if $A$ is $T_1$ spaces, and is equivalent to \PMlinkname{compactness}{Compact} if $X$ is a metric space.

\PMlinkescapeword{compact}
%%%%%
%%%%%
%%%%%
\end{document}
