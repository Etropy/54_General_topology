\documentclass[12pt]{article}
\usepackage{pmmeta}
\pmcanonicalname{AnotherProofOfDinisTheorem}
\pmcreated{2013-03-22 14:04:37}
\pmmodified{2013-03-22 14:04:37}
\pmowner{gumau}{3545}
\pmmodifier{gumau}{3545}
\pmtitle{another proof of Dini's theorem}
\pmrecord{11}{35437}
\pmprivacy{1}
\pmauthor{gumau}{3545}
\pmtype{Proof}
\pmcomment{trigger rebuild}
\pmclassification{msc}{54A20}

% this is the default PlanetMath preamble.  as your knowledge
% of TeX increases, you will probably want to edit this, but
% it should be fine as is for beginners.

% almost certainly you want these
\usepackage{amssymb}
\usepackage{amsmath}
\usepackage{amsfonts}

% used for TeXing text within eps files
%\usepackage{psfrag}
% need this for including graphics (\includegraphics)
%\usepackage{graphicx}
% for neatly defining theorems and propositions
%\usepackage{amsthm}
% making logically defined graphics
%%%\usepackage{xypic}

% there are many more packages, add them here as you need them

% define commands here
\begin{document}
This is the version of the \textbf{Dini's theorem} I will prove:
Let $K$ be a compact metric space and $\left( {f_n }
\right)_{n\in N} \subset C(K)$ which converges pointwise to $f\in
C(K)$.

Besides, $f_n (x)\ge f_{n+1} (x)\quad \forall x\in K,\forall n$.

Then $\left( {f_n } \right)_{n\in N} $ converges uniformly in $K$.

\textbf{Proof}

Suppose that the sequence does not converge uniformly. Then, by definition,
\[
\exists \varepsilon >0\;\text{such that }\forall m\in N\;\exists \;n_m
>m,\;x_m \in K\;\text{such that }\quad \left| {f_{n_m } (x_m )-f(x_m )}
\right|\ge \varepsilon .
\]
So,
\[
\begin{array}{l}
 \text{For } \;m=1\;\exists \;n_1 >1\;,\;x_1 \in K\;\text{such that }\;\left| {f_{n_1 } (x_1
)-f(x_1 )} \right|\ge \varepsilon \\
 \exists \;n_2 >n_1 \;,\;x_2 \in K\;\text{such that }\;\left| {f_{n_2 } (x_2
)-f(x_2 )} \right|\ge \varepsilon \\
 \vdots \\
 \exists \;n_m >n_{m-1} \;,\;x_m \in K\;\text{such that }\;\left| {f_{n_m } (x_m
)-f(x_m )} \right|\ge \varepsilon \\
 \end{array}
\]
Then we have a sequence $\left( {x_m } \right)_m \subset K$ and $\left( {f_{n_{m}} } \right)_m \subset \left( {f_n } \right)_n$
is a subsequence of the original sequence of functions. $K$ is compact, so there is a subsequence of $\left( {x_m } \right)_m$
which converges in $K$, that is, $\left( {x_{m_{j}} } \right)_j $ such that
\[
x_{m_{j}}\longrightarrow x \in K
\]
I will prove that $f$ is not continuous in $x$ (A contradiction with one of the hypothesis).

To do this, I will show that $f\left( {x_{m_{j}} } \right)_j $ does not converge to $f(x)$, using above's $\varepsilon$.


Let $j_0 \;\text{such that }\;j\ge j_0 \Rightarrow \left| {f_{n_{m_j } } (x)-f(x)}
\right|<\raise0.7ex\hbox{$\varepsilon $} \!\mathord{\left/ {\vphantom
{\varepsilon 4}}\right.\kern-\nulldelimiterspace}\!\lower0.7ex\hbox{$4$}$,
which exists due to the punctual convergence of the sequence. Then,
particularly, $\left| {f_{n_{m_{j_o } } } (x)-f(x)}
\right|<\raise0.7ex\hbox{$\varepsilon $} \!\mathord{\left/ {\vphantom
{\varepsilon 4}}\right.\kern-\nulldelimiterspace}\!\lower0.7ex\hbox{$4$}$.

Note that
\[
\left| {f_{n_{m_j } } (x_{m_j } )-f(x_{m_j } )} \right|=f_{n_{m_j } }
(x_{m_j } )-f(x_{m_j } )
\]
because (using the hypothesis $f_n (y)\ge f_{n+1} (y)\quad \forall y\in K,\forall n$) it's easy to see that 
\[
f_n (y)\ge f(y)\quad \forall y\in K,\forall n
\]

Then, $f_{n_{m_j } } (x_{m_j } )-f(x_{m_j } )\ge \varepsilon \;\forall
j$. And also the hypothesis implies
\[
f_{n_{m_j } } (y)\ge f_{n_{m_{j+1} } } (y)\;\;\forall y\in K,\forall j
\]
So, $j\ge j_0 \Rightarrow f_{n_{m_{j_0 } } } (x_{m_j } )\ge f_{n_{m_j } } (x_{m_j } ),$ which implies
\[
\left| {f_{n_{m_{j_0 } } } (x_{m_j } )-f(x_{m_j } )} \right|\ge \varepsilon
\]

Now,
\[
\left| {f_{n_{m_{j_0 } } } (x_{m_j } )-f(x)} \right|+\left| {f(x_{m_j }
)-f(x)} \right|\ge \left| {f_{n_{m_{j_0 } } } (x_{m_j } )-f(x_{m_j } )}
\right|\ge \varepsilon \;\;\forall j\ge j_0
\]
and so
\[
\left| {f(x_{m_j } )-f(x)} \right|\ge \varepsilon -\left| {f_{n_{m_{j_0 } }
} (x_{m_j } )-f(x)} \right|\;\;\forall j\ge j_0 .
\]

On the other hand,
\[
\left| {f_{n_{m_{j_0 } } } (x_{m_j } )-f(x)} \right|\le \left|
{f_{n_{m_{j_0 } } } (x_{m_j } )-f_{n_{m_{j_0 } } } (x)} \right|+\left|
{f_{n_{m_{j_0 } } } (x)-f(x)} \right|
\]

And as $f_{n_{m_{j_0 } } } $ is continuous, there is a $j_1 $
such that
\[
j\ge j_1 \Rightarrow \left| {f_{n_{m_{j_0 } } } (x_{m_j }
)-f_{n_{m_{j_0 } } } (x)} \right|<\raise0.7ex\hbox{$\varepsilon $}
\!\mathord{\left/ {\vphantom {\varepsilon
4}}\right.\kern-\nulldelimiterspace}\!\lower0.7ex\hbox{$4$}
\]

Then,
\[
j\ge j_1\Rightarrow \left|
{f_{n_{m_{j_0 } } } (x_{m_j } )-f(x)} \right|\le \left| {f_{n_{m_{j_0 } } }
(x_{m_j } )-f_{n_{m_{j_0 } } } (x)} \right|+\left| {f_{n_{m_{j_0 } } }
(x)-f(x)} \right|<\raise0.7ex\hbox{$\varepsilon $} \!\mathord{\left/
{\vphantom {\varepsilon
2}}\right.\kern-\nulldelimiterspace}\!\lower0.7ex\hbox{$2$},\]
which implies
\[
\left| {f(x_{m_j } )-f(x)} \right|\ge \varepsilon -\left| {f_{n_{m_{j_0 } }
} (x_{m_j } )-f(x)} \right|\ge \raise0.7ex\hbox{$\varepsilon $}
\!\mathord{\left/ {\vphantom {\varepsilon
2}}\right.\kern-\nulldelimiterspace}\!\lower0.7ex\hbox{$2$}\;\;\forall j\ge
\max \left( {j_0 ,j_1 } \right).
\] Then, particularly, $f(x_{m_j } )_j $ does not converge to $f(x)$. QED.
%%%%%
%%%%%
\end{document}
