\documentclass[12pt]{article}
\usepackage{pmmeta}
\pmcanonicalname{LindelofTheorem}
\pmcreated{2014-11-06 13:45:28}
\pmmodified{2014-11-06 13:45:28}
\pmowner{drini}{3}
\pmmodifier{pahio}{2872}
\pmtitle{Lindel\"of theorem}
\pmrecord{7}{32118}
\pmprivacy{1}
\pmauthor{drini}{2872}
\pmtype{Theorem}
\pmcomment{trigger rebuild}
\pmclassification{msc}{54D99}
\pmrelated{SecondCountable}
\pmrelated{Lindelof}

%\usepackage{graphicx}
%%%\usepackage{xypic} 
\usepackage{bbm}
\newcommand{\Z}{\mathbbmss{Z}}
\newcommand{\C}{\mathbbmss{C}}
\newcommand{\R}{\mathbbmss{R}}
\newcommand{\Q}{\mathbbmss{Q}}
\newcommand{\mathbb}[1]{\mathbbmss{#1}}
\begin{document}
If a topological space $(X,\tau)$ satisfies the second axiom of countability, and if $A$ is any subset of $X$, then any open cover for $A$ has a countable subcover. 

In particular, we have that $(X,\tau)$ is a \PMlinkname{Lindel\"of space}{lindelofspace}.
%%%%%
%%%%%
\end{document}
