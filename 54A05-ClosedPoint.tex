\documentclass[12pt]{article}
\usepackage{pmmeta}
\pmcanonicalname{ClosedPoint}
\pmcreated{2013-03-22 16:22:24}
\pmmodified{2013-03-22 16:22:24}
\pmowner{jocaps}{12118}
\pmmodifier{jocaps}{12118}
\pmtitle{closed point}
\pmrecord{9}{38512}
\pmprivacy{1}
\pmauthor{jocaps}{12118}
\pmtype{Definition}
\pmcomment{trigger rebuild}
\pmclassification{msc}{54A05}
%\pmkeywords{generization}
%\pmkeywords{specialization}
%\pmkeywords{generic points}

% this is the default PlanetMath preamble.  as your knowledge
% of TeX increases, you will probably want to edit this, but
% it should be fine as is for beginners.

% almost certainly you want these
\usepackage{amssymb}
\usepackage{amsmath}
\usepackage{amsfonts}

% used for TeXing text within eps files
%\usepackage{psfrag}
% need this for including graphics (\includegraphics)
%\usepackage{graphicx}
% for neatly defining theorems and propositions
%\usepackage{amsthm}
% making logically defined graphics
%%%\usepackage{xypic}

% there are many more packages, add them here as you need them

% define commands here

\begin{document}
Let $X$ be a topological space and suppose that $x\in X$. If $\{x\}=\overline{\{x\}}$ then we say that $x$ is a 
\emph{closed point}.  In other words, $x$ is closed if $\lbrace x\rbrace$ is a closed set.

For example, the real line $\mathbb{R}$ equipped with the usual metric topology, every point is a closed point.

More generally, if a topological space is \PMlinkname{$T_1$}{T1}, then every point in it is closed.  If we remove the condition of being $T_1$, then the property fails, as in the case of the Sierpinski space $X=\lbrace x,y\rbrace$, whose open sets are $\varnothing$, $X$, and $\lbrace x \rbrace$.  The closure of $\lbrace x\rbrace$ is $X$, not $\lbrace x\rbrace$.
%%%%%
%%%%%
\end{document}
