\documentclass[12pt]{article}
\usepackage{pmmeta}
\pmcanonicalname{KuratowskiClosurecomplementTheorem}
\pmcreated{2013-03-22 17:59:28}
\pmmodified{2013-03-22 17:59:28}
\pmowner{CWoo}{3771}
\pmmodifier{CWoo}{3771}
\pmtitle{Kuratowski closure-complement theorem}
\pmrecord{9}{40502}
\pmprivacy{1}
\pmauthor{CWoo}{3771}
\pmtype{Theorem}
\pmcomment{trigger rebuild}
\pmclassification{msc}{54A99}
\pmclassification{msc}{54A05}

\endmetadata

\usepackage{amssymb,amscd}
\usepackage{amsmath}
\usepackage{amsfonts}
\usepackage{mathrsfs}

% used for TeXing text within eps files
%\usepackage{psfrag}
% need this for including graphics (\includegraphics)
%\usepackage{graphicx}
% for neatly defining theorems and propositions
\usepackage{amsthm}
% making logically defined graphics
%%\usepackage{xypic}
\usepackage{pst-plot}

% define commands here
\newcommand*{\abs}[1]{\left\lvert #1\right\rvert}
\newtheorem{prop}{Proposition}
\newtheorem{thm}{Theorem}
\newtheorem{ex}{Example}
\newcommand{\real}{\mathbb{R}}
\newcommand{\pdiff}[2]{\frac{\partial #1}{\partial #2}}
\newcommand{\mpdiff}[3]{\frac{\partial^#1 #2}{\partial #3^#1}}
\begin{document}
\textbf{Problem}.  Let $X$ be a topological space and $A$ a subset of $X$.  How many (distinct) sets can be obtained by iteratively applying the closure and complement operations to $A$?

Kuratowski studied this problem, and showed that at most $14$ sets that can be generated from a given set in an arbitrary topological space.  This is known as the \emph{Kuratowski closure-complement theorem}.

Let us examine this problem more closely.  For convenience, let us denote $^-:X\to X$ be the closure operator: $$A\mapsto A^-,$$ and $^c:X\to X$ the complementation operator: $$A \mapsto A^c.$$  A set that can be obtained from $A$ by iteratively applying $^-$ and $^c$ has the form $A^\sigma$, where $\sigma$ is an operator on $X$ that is the composition of finitely many $^-$ and $^c$.  In other words, $\sigma$ is a word on the alphabet $\lbrace ^-, ^c\rbrace$.

First, notice that $A^{--}=A^{-}$ and $A^{cc}=A$.   This means that $\sigma$ can be reduced (or simplified) to a form such that $^-$ and $^c$ occurs alternately.

In addition, we have the following:
\begin{prop}  $A^{-c-c-c-}=A^{-c-}$. \end{prop}
\begin{proof}  For any set $A$ in a topological space $X$, $A^{-}$ is closed, so that $A^{-c-}$ is regular closed.  This means that $A^{-c-}=A^{-c-c-c-}$.
\end{proof}
This means that $\sigma$ can be reduced to one of the following cases: $$1, ^-, ^{-c}, ^{-c-}, ^{-c-c}, ^{-c-c-}, ^{-c-c-c}, ^c, ^{c-}, ^{c-c}, ^{c-c-}, ^{c-c-c}, ^{c-c-c-}, ^{c-c-c-c},$$ where $1=\, ^{cc}$ is the identity operator.  As there are a total of 14 combinations, proving the closure-complement theorem is to exhibit an example.  To do this, pick $X=\mathbb{R}$, the real line.  Let $A=(0,1)\cup \lbrace 2\rbrace \cup ((3,4)\cap \mathbb{Q}) \cup ((5,7)-\lbrace 6\rbrace)$.  In other words, $A$ is the union of a real interval, a point, a rational interval, and a real interval with a point deleted.  Then
\begin{enumerate}
\item $A^-=[0,1]\cup \lbrace 2\rbrace \cup [3,4]\cup [5,7]$,
\item $A^{-c}=(-\infty,0)\cup (1,2)\cup (2,3)\cup (4,5)\cup (7,\infty)$,
\item $A^{-c-}=(-\infty,0]\cup [1,3]\cup [4,5]\cup [7,\infty)$,
\item $A^{-c-c}=(0,1)\cup (3,4)\cup (5,7)$,
\item $A^{-c-c-}=[0,1]\cup [3,4]\cup [5,7]$,
\item $A^{-c-c-c}=(-\infty,0)\cup (1,3)\cup (4,5)\cup (7,\infty)$,
\item $A^c=(-\infty,0]\cup [1,2)\cup (2,3]\cup ((3,4)-\mathbb{Q}) \cup [4,5]\cup \lbrace 6\rbrace \cup [7,\infty)$,
\item $A^{c-}=(-\infty,0]\cup [1,5]\cup \lbrace 6\rbrace \cup [7,\infty)$,
\item $A^{c-c}=(0,1)\cup (5,6)\cup (6,7)$,
\item $A^{c-c-}=[0,1]\cup [5,7]$,
\item $A^{c-c-c}=(-\infty,0)\cup (1,5)\cup (7,\infty)$,
\item $A^{c-c-c-}=(-\infty,0]\cup [1,5]\cup [7,\infty)$,
\item $A^{c-c-c-c}=(0,1)\cup (5,7)$,
\end{enumerate}
together with $A$, are $14$ pairwise distinct sets that can be generated by $^-$ and $^c$.
%%%%%
%%%%%
\end{document}
