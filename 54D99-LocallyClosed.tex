\documentclass[12pt]{article}
\usepackage{pmmeta}
\pmcanonicalname{LocallyClosed}
\pmcreated{2013-03-22 17:36:12}
\pmmodified{2013-03-22 17:36:12}
\pmowner{asteroid}{17536}
\pmmodifier{asteroid}{17536}
\pmtitle{locally closed}
\pmrecord{5}{40018}
\pmprivacy{1}
\pmauthor{asteroid}{17536}
\pmtype{Definition}
\pmcomment{trigger rebuild}
\pmclassification{msc}{54D99}

% this is the default PlanetMath preamble.  as your knowledge
% of TeX increases, you will probably want to edit this, but
% it should be fine as is for beginners.

% almost certainly you want these
\usepackage{amssymb}
\usepackage{amsmath}
\usepackage{amsfonts}

% used for TeXing text within eps files
%\usepackage{psfrag}
% need this for including graphics (\includegraphics)
%\usepackage{graphicx}
% for neatly defining theorems and propositions
%\usepackage{amsthm}
% making logically defined graphics
%%%\usepackage{xypic}

% there are many more packages, add them here as you need them

% define commands here

\begin{document}
{\bf \PMlinkescapetext{Definition} -} A subset $Y$ of a topological space $X$ is said to be {\bf locally closed} if it is the intersection of an open and a closed subset.

The following result provides some \PMlinkescapetext{equivalent} definitions:

{\bf \PMlinkescapetext{Proposition} -} The following are equivalent:
\begin{enumerate}
\item $Y$ is locally closed in $X$.
\item Each point in $Y$ has an open neighborhood $U \subseteq X$ such that $U \cap Y$ is closed in $U$ (with the subspace topology).
\item $Y$ is open in its closure $\overline{Y}$ (with the subspace topology).
\end{enumerate}
%%%%%
%%%%%
\end{document}
