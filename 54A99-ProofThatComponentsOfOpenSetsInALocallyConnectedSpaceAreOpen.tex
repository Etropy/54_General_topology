\documentclass[12pt]{article}
\usepackage{pmmeta}
\pmcanonicalname{ProofThatComponentsOfOpenSetsInALocallyConnectedSpaceAreOpen}
\pmcreated{2013-03-22 17:06:07}
\pmmodified{2013-03-22 17:06:07}
\pmowner{Mathprof}{13753}
\pmmodifier{Mathprof}{13753}
\pmtitle{proof that components of open sets in a locally connected space are open}
\pmrecord{6}{39399}
\pmprivacy{1}
\pmauthor{Mathprof}{13753}
\pmtype{Theorem}
\pmcomment{trigger rebuild}
\pmclassification{msc}{54A99}

% this is the default PlanetMath preamble.  as your knowledge
% of TeX increases, you will probably want to edit this, but
% it should be fine as is for beginners.

% almost certainly you want these
\usepackage{amssymb}
\usepackage{amsmath}
\usepackage{amsfonts}

% used for TeXing text within eps files
%\usepackage{psfrag}
% need this for including graphics (\includegraphics)
%\usepackage{graphicx}
% for neatly defining theorems and propositions
\usepackage{amsthm}
% making logically defined graphics
%%%\usepackage{xypic}

% there are many more packages, add them here as you need them

% define commands here
\newtheorem*{thm}{Theorem}

\begin{document}
\begin{thm} A topological space $X$ is locally connected if and only if each component of an open set 
is open.
\end{thm}
\begin{proof}
First, suppose that $X$ is locally connected and that $U$ is an open set of $X$.
Let $p \in C$, where $C$ is a component of $U$.
Since $X$ is locally connected there is an open connected set, say $V$ with
$p \in V \subset U$. Since $C$ is a component of $U$ it must be that $V \subset C$.
Hence, $C$ is open.
For the converse, suppose that each component of each  open set is open. Let $p \in X$.
Let $U$ be an open set containing $p$. Let $C$ be the component of $U$ which
contains $p$. Then $C$ is open and connected, so $X$ is locally connected.

\end{proof}

As a corollary, we have that the components of a locally connected space are both
open and closed.

%%%%%
%%%%%
\end{document}
