\documentclass[12pt]{article}
\usepackage{pmmeta}
\pmcanonicalname{SierpinskiSpace}
\pmcreated{2013-03-22 12:06:26}
\pmmodified{2013-03-22 12:06:26}
\pmowner{CWoo}{3771}
\pmmodifier{CWoo}{3771}
\pmtitle{Sierpinski space}
\pmrecord{9}{31222}
\pmprivacy{1}
\pmauthor{CWoo}{3771}
\pmtype{Definition}
\pmcomment{trigger rebuild}
\pmclassification{msc}{54G20}
\pmsynonym{Sierpi\'nski space}{SierpinskiSpace}
%\pmkeywords{topology}
\pmrelated{T1Space}
\pmrelated{T2Space}
\pmrelated{SeparationAxioms}

\usepackage{amssymb}
\usepackage{amsmath}
\usepackage{amsfonts}
\usepackage{graphicx}
%%%\usepackage{xypic}
\begin{document}
\emph{Sierpinski space} is the topological space $X=\lbrace x,y\rbrace$ with the topology given by $\lbrace X, \{ x\} ,\emptyset \rbrace$.

Sierpinski space is \PMlinkname{$T_0$}{T0} but not \PMlinkname{$T_1$}{T1}.  It is $T_0$ because $\lbrace x\rbrace$ is the open set containing $x$ but not $y$.  It is not $T_1$ because every open set $U$ containing $y$ (namely $X$) contains $x$ (in other words, there is no open set containing $y$ but not containing $x$).

\textbf{Remark}.  From the Sierpinski space, one can construct many non-$T_1$ $T_0$ spaces, simply by taking any set $X$ with at least two elements, and take any non-empty proper subset $U\subset X$, and set the topology $\mathcal{T}$ on $X$ by $\mathcal{T}=P(U)\cup \lbrace X\rbrace$.
%%%%%
%%%%%
%%%%%
\end{document}
