\documentclass[12pt]{article}
\usepackage{pmmeta}
\pmcanonicalname{InjectiveMapBetweenRealNumbersIsAHomeomorphism}
\pmcreated{2013-03-22 18:53:58}
\pmmodified{2013-03-22 18:53:58}
\pmowner{joking}{16130}
\pmmodifier{joking}{16130}
\pmtitle{injective map between real numbers is a homeomorphism}
\pmrecord{4}{41747}
\pmprivacy{1}
\pmauthor{joking}{16130}
\pmtype{Theorem}
\pmcomment{trigger rebuild}
\pmclassification{msc}{54C05}

% this is the default PlanetMath preamble.  as your knowledge
% of TeX increases, you will probably want to edit this, but
% it should be fine as is for beginners.

% almost certainly you want these
\usepackage{amssymb}
\usepackage{amsmath}
\usepackage{amsfonts}

% used for TeXing text within eps files
%\usepackage{psfrag}
% need this for including graphics (\includegraphics)
%\usepackage{graphicx}
% for neatly defining theorems and propositions
%\usepackage{amsthm}
% making logically defined graphics
%%%\usepackage{xypic}

% there are many more packages, add them here as you need them

% define commands here

\begin{document}
\textbf{Lemma.} Assume that $I$ is an open interval and $f:I\to\mathbb{R}$ is an injective, continuous map. Then $f(I)\subseteq\mathbb{R}$ is an open subset.

\textit{Proof.} Since $f$ is injective, then of course $f$ is monotonic. Without loss of generality, we may assume that $f$ is increasing. Let $y=f(x)\in f(I)$. Since $I$ is open, then there are $\alpha,\beta\in I$ such that $\alpha<x<\beta$. Therefore $f(\alpha)<y<f(\beta)$ and (because continuous functions are Darboux functions) for any $y'\in \big(f(\alpha),f(\beta)\big)$ there exists $x'\in I$ such that $f(x')=y'$. This shows that $\big(f(\alpha),f(\beta)\big)$ is an open neighbourhood of $y$ contained in $f(I)$ and therefore (since $y$ was arbitrary) $f(I)$ is open. $\square$

\textbf{Proposition.} Assume that $I$ is an open interval and $f:I\to\mathbb{R}$ is an injective, continuous map. Then $f$ is a homeomorphism onto image.

\textit{Proof.} Of course, it is enough to show that $f$ is an open map. But if $U\subseteq I$ is open, then there are disjoint, open intervals $I_{\alpha}$ such that $$U=\bigcup_{\alpha} I_{\alpha}.$$ Therefore we obtain continuous, injective maps $f_{\alpha}:I_{\alpha}\to\mathbb{R}$ which are restrictions of $f$ to $I_{\alpha}$. By lemma we have that $f_{\alpha}(I_{\alpha})$ is open and therefore
$$f(U)=f\bigg(\bigcup_{\alpha}I_{\alpha}\bigg)=\bigcup_{\alpha}f(I_{\alpha})=\bigcup_{\alpha}f_{\alpha}(I_{\alpha})$$
is open. This shows that $f$ is a homeomorphism onto image. $\square$
%%%%%
%%%%%
\end{document}
