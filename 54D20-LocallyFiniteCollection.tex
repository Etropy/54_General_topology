\documentclass[12pt]{article}
\usepackage{pmmeta}
\pmcanonicalname{LocallyFiniteCollection}
\pmcreated{2013-03-22 12:12:51}
\pmmodified{2013-03-22 12:12:51}
\pmowner{yark}{2760}
\pmmodifier{yark}{2760}
\pmtitle{locally finite collection}
\pmrecord{11}{31542}
\pmprivacy{1}
\pmauthor{yark}{2760}
\pmtype{Definition}
\pmcomment{trigger rebuild}
\pmclassification{msc}{54D20}
%\pmkeywords{topology}
\pmrelated{PointFinite}
\pmdefines{locally finite}
\pmdefines{locally countable collection}
\pmdefines{locally countable}

\usepackage{amssymb}
\usepackage{amsmath}
\usepackage{amsfonts}

\begin{document}
Let $\mathcal{C}$ be a collection of subsets of a topological space $X$.

$\mathcal{C}$ is said to be \emph{locally finite}
if for all $x\in X$ there is a neighbourhood $U$ of $x$
such that $U \cap A = \varnothing$ for all but finitely many $A \in \mathcal{C}$.

Similarly, $\mathcal{C}$ is said to be \emph{locally countable}
if for all $x\in X$ there is a neighbourhood $U$ of $x$
such that $U \cap A = \varnothing$ for all but countably many $A \in \mathcal{C}$.
%%%%%
%%%%%
%%%%%
\end{document}
