\documentclass[12pt]{article}
\usepackage{pmmeta}
\pmcanonicalname{EquivalentFormulationOfTheTubeLemma}
\pmcreated{2013-03-22 19:15:18}
\pmmodified{2013-03-22 19:15:18}
\pmowner{joking}{16130}
\pmmodifier{joking}{16130}
\pmtitle{equivalent formulation of the tube lemma}
\pmrecord{4}{42181}
\pmprivacy{1}
\pmauthor{joking}{16130}
\pmtype{Theorem}
\pmcomment{trigger rebuild}
\pmclassification{msc}{54D30}

\endmetadata

% this is the default PlanetMath preamble.  as your knowledge
% of TeX increases, you will probably want to edit this, but
% it should be fine as is for beginners.

% almost certainly you want these
\usepackage{amssymb}
\usepackage{amsmath}
\usepackage{amsfonts}

% used for TeXing text within eps files
%\usepackage{psfrag}
% need this for including graphics (\includegraphics)
%\usepackage{graphicx}
% for neatly defining theorems and propositions
%\usepackage{amsthm}
% making logically defined graphics
%%%\usepackage{xypic}

% there are many more packages, add them here as you need them

% define commands here

\begin{document}
Let us recall the thesis of the tube lemma. Assume, that $X$ and $Y$ are topological spaces.

\textbf{(TL)} If $U\subseteq X\times Y$ is open (in product topology) and if $x\in X$ is such that $x\times Y\subseteq U$, then there exists an open neighbourhood $V\subseteq X$ of $x$ such that $V\times Y\subseteq U$.

We wish to give a relation between (TL) and the the following thesis, concering closed projections:

\textbf{(CP)} The projection $\pi:X\times Y\to X$ given by $\pi(x,y)=x$ is a closed map.

The following theorem relates these two statements:

\textbf{Theorem.} (TL) is equivalent to (CP).

\textit{Proof.} ,,$\Rightarrow$'' Let $F\subseteq X\times Y$ be a closed set and let $U=(X\times Y)\backslash F$ be its open complement. We will show, that $\pi(F)$ is closed, by showing that $V=X\backslash\pi(F)$ is open. So assume, that $x\in V$. Obviously 
$$\big(\pi^{-1}(x)=x\times Y\big)\cap F=\emptyset.$$
Therefore $x\times Y\subseteq U$ and by (TL) there exists open neighbourhood $V'\subseteq X$ of $x$ such that $V'\times Y\subseteq U$. It easily follows, that $V'\subseteq V$ and it is open, so (since $x$ was chosen arbitrary) $V$ is open.

,,$\Leftarrow$'' Let $U\subseteq X\times Y$ be an open subset such that $x\times Y\subseteq U$ for some $x\in X$. Let $F=(X\times Y)\backslash U$. Then $F$ is closed and by (CP) we have that $\pi(F)\subseteq X$ is closed. Also $x\not\in\pi(F)$ and thus $V=X\backslash\pi(F)$ is an open neighbourhood of $x$. It can be easily checked, that $V\times Y\subseteq U$, which completes the proof. $\square$

\textbf{Remark.} The theorem doesn't state that any of statements is true. It is well known (see tha parent object), that if both $X$ and $Y$ are Hausdorff with $Y$ compact, then both are true. On the other hand, for example for $X=Y=\mathbb{R}$, where $\mathbb{R}$ denotes reals with standard topology, they are both false. For example consider 
$$F=\{(x,y)\in\mathbb{R}^2\ |\ xy=1\}.$$
Of course $F$ is closed, but $\pi(F)=\mathbb{R}\backslash\{0\}$ is not closed, so the (CP) is false.
%%%%%
%%%%%
\end{document}
