\documentclass[12pt]{article}
\usepackage{pmmeta}
\pmcanonicalname{ExamplesOfNowhereDenseSets}
\pmcreated{2013-03-22 17:07:05}
\pmmodified{2013-03-22 17:07:05}
\pmowner{Wkbj79}{1863}
\pmmodifier{Wkbj79}{1863}
\pmtitle{examples of nowhere dense sets}
\pmrecord{4}{39419}
\pmprivacy{1}
\pmauthor{Wkbj79}{1863}
\pmtype{Example}
\pmcomment{trigger rebuild}
\pmclassification{msc}{54A99}
\pmrelated{ExampleOfAMeagerSet}

\endmetadata

\usepackage{amssymb}
\usepackage{amsmath}
\usepackage{amsfonts}

\usepackage{psfrag}
\usepackage{graphicx}
\usepackage{amsthm}
%%\usepackage{xypic}

\begin{document}
Note that $\mathbb{Z}$ is nowhere dense in $\mathbb{R}$ under the usual topology:  $\operatorname{int} \overline{\mathbb{Z}}=\operatorname{int} \mathbb{Z}=\emptyset$.  Similarly, $\frac{1}{n} \mathbb{Z}$ is nowhere dense for every $n \in \mathbb{Z}$ with $n>0$.

This result provides an alternative way to prove that $\mathbb{Q}$ is meager in $\mathbb{R}$ under the usual topology, since $\displaystyle \mathbb{Q}=\bigcup_{n \in \mathbb{Z} \text{ and } n>0} \textstyle{\frac{1}{n}} \mathbb{Z}$.
%%%%%
%%%%%
\end{document}
