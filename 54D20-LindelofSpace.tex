\documentclass[12pt]{article}
\usepackage{pmmeta}
\pmcanonicalname{LindelofSpace}
\pmcreated{2013-03-22 12:06:34}
\pmmodified{2013-03-22 12:06:34}
\pmowner{yark}{2760}
\pmmodifier{yark}{2760}
\pmtitle{Lindel\"of space}
\pmrecord{11}{31226}
\pmprivacy{1}
\pmauthor{yark}{2760}
\pmtype{Definition}
\pmcomment{trigger rebuild}
\pmclassification{msc}{54D20}
%\pmkeywords{topology}
\pmrelated{SecondCountable}
\pmrelated{Separable}
\pmrelated{Compact}
\pmrelated{LindelofTheorem}
\pmrelated{CompactMetricSpacesAreSecondCountable}
\pmrelated{ErnstLindelof}
\pmdefines{Lindel\"of}
\pmdefines{Lindel\"of property}

\endmetadata


\begin{document}
\section*{Definition}

A topological space is said to be \emph{Lindel\"of} if every open cover has a countable subcover.

\section*{Notes}

A second-countable space is Lindel\"of.
A compact space is Lindel\"of.

A \PMlinkname{regular}{T3Space} Lindel\"of space is \PMlinkid{normal}{1530}.

\PMlinkname{$F_\sigma$ sets}{F_sigmaSet} in Lindel\"of spaces are Lindel\"of.
Continuous images of Lindel\"of spaces are Lindel\"of.

A Lindel\"of space is compact if and only if it is countably compact.
%%%%%
%%%%%
%%%%%
\end{document}
