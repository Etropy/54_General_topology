\documentclass[12pt]{article}
\usepackage{pmmeta}
\pmcanonicalname{SeparableSpace}
\pmcreated{2013-03-22 12:05:45}
\pmmodified{2013-03-22 12:05:45}
\pmowner{yark}{2760}
\pmmodifier{yark}{2760}
\pmtitle{separable space}
\pmrecord{13}{31193}
\pmprivacy{1}
\pmauthor{yark}{2760}
\pmtype{Definition}
\pmcomment{trigger rebuild}
\pmclassification{msc}{54D65}
\pmsynonym{separable topological space}{SeparableSpace}
%\pmkeywords{topology}
\pmrelated{SecondCountable}
\pmrelated{Lindelof}
\pmrelated{EverySecondCountableSpaceIsSeparable}
\pmrelated{HewittMarczewskiPondiczeryTheorem}
\pmdefines{separable}

\endmetadata


\begin{document}
\section*{Definition}

A topological space is said to be \emph{separable}
if it has a countable dense subset.

\section*{Properties}

All second-countable spaces are separable.
A metric space is separable if and only if it is second-countable.

A continuous image of a separable space is separable.

An open subset of a separable space is separable (in the subspace topology).

A \PMlinkname{product}{ProductTopology} of $2^{\aleph_0}$ or fewer separable spaces
is separable. This is a special case of the Hewitt-Marczewski-Pondiczery Theorem.

A Hilbert space is separable if and only if it has a countable orthonormal basis.
%%%%%
%%%%%
%%%%%
\end{document}
