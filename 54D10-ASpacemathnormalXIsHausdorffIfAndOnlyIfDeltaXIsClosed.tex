\documentclass[12pt]{article}
\usepackage{pmmeta}
\pmcanonicalname{ASpacemathnormalXIsHausdorffIfAndOnlyIfDeltaXIsClosed}
\pmcreated{2013-03-22 14:20:47}
\pmmodified{2013-03-22 14:20:47}
\pmowner{mathcam}{2727}
\pmmodifier{mathcam}{2727}
\pmtitle{a space $\mathnormal{X}$ is Hausdorff if and only if $\Delta(X)$ is closed}
\pmrecord{9}{35820}
\pmprivacy{1}
\pmauthor{mathcam}{2727}
\pmtype{Proof}
\pmcomment{trigger rebuild}
\pmclassification{msc}{54D10}
%\pmkeywords{Hausdorff}
%\pmkeywords{T2}
%\pmkeywords{diagonal}
\pmrelated{DiagonalEmbedding}
\pmrelated{T2Space}
\pmrelated{ProductTopology}
\pmrelated{SeparatedScheme}

\endmetadata

% this is the default PlanetMath preamble.  as your knowledge
% of TeX increases, you will probably want to edit this, but
% it should be fine as is for beginners.

% almost certainly you want these
\usepackage{amssymb}
\usepackage{amsmath}
\usepackage{amsfonts}

% used for TeXing text within eps files
%\usepackage{psfrag}
% need this for including graphics (\includegraphics)
%\usepackage{graphicx}
% for neatly defining theorems and propositions
\usepackage{amsthm}
% making logically defined graphics
%%%\usepackage{xypic}

% there are many more packages, add them here as you need them

% define commands here
\def\co{\colon\thinspace}
\theoremstyle{definition}
\newtheorem*{unt}{Theorem}
\begin{document}
\PMlinkescapeword{coordinates}

\begin{unt}
A space $X$ is Hausdorff if and only if $$ \{ (x,x)\in X\times X \mid x\in X\}$$ is closed in $X\times X$ under the product topology.
\end{unt}

\begin{proof}

First, some preliminaries:  Recall that the diagonal map $\Delta\co X\to X\times X$ is defined as $x\stackrel{\Delta}{\longmapsto}(x,x)$.  Also recall that in a topology generated by a basis (like the product topology), a set $Y$ is open if and only if, for every point $y\in Y$, there's a basis element $B$ with $y\in B\subset Y$.  Basis elements for $X\times X$ have the form $U\times V$ where $U,V$ are open sets in $X$.

Now, suppose that $X$ is Hausdorff.  We'd like to show its image under $\Delta$ is closed.  We can do that by showing that its complement $\Delta(X)^c$ is open.  $\Delta(X)$ consists of points with equal coordinates, so $\Delta(X)^c$ consists of points $(x,y)$ with $x$ and $y$ distinct.

For any $(x,y)\in \Delta(X)^c$, the Hausdorff condition gives us disjoint open $U,V\subset X$ with $x\in U, y\in V$.  Then $U\times V$ is a basis element containing $(x,y)$.  $U$ and $V$ have no points in common, so $U\times V$ contains nothing in the image of the diagonal map: $U\times V$ is contained in $\Delta(X)^c$.  So $\Delta(X)^c$ is open, making $\Delta(X)$ closed.

Now let's suppose $\Delta(X)$ is closed.  Then $\Delta(X)^c$ is open.  Given any $(x,y)\in\Delta(X)^c$, there's a basis element $U\times V$ with $(x,y)\in U\times V\subset\Delta(X)^c$.  $U\times V$ lying in $\Delta(X)^c$ implies that $U$ and $V$ are disjoint.

If we have $x\neq y$ in $X$, then $(x,y)$ is in $\Delta(X)^c$.  The basis element containing $(x,y)$ gives us open, disjoint $U,V$ with $x\in U, y\in V$.  $X$ is Hausdorff, just like we wanted.
\end{proof}
%%%%%
%%%%%
\end{document}
