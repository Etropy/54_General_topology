\documentclass[12pt]{article}
\usepackage{pmmeta}
\pmcanonicalname{CrosscapSlide}
\pmcreated{2013-03-22 15:42:23}
\pmmodified{2013-03-22 15:42:23}
\pmowner{juanman}{12619}
\pmmodifier{juanman}{12619}
\pmtitle{crosscap slide}
\pmrecord{5}{37651}
\pmprivacy{1}
\pmauthor{juanman}{12619}
\pmtype{Definition}
\pmcomment{trigger rebuild}
\pmclassification{msc}{54C10}
\pmsynonym{y-homeomorphism}{CrosscapSlide}
%\pmkeywords{homeomorphism}
%\pmkeywords{non orientable surface}
\pmrelated{homeomorphism}
\pmrelated{YHomeomorphism2}
\pmrelated{Homeomorphism}

% this is the default PlanetMath preamble.  as your knowledge
% of TeX increases, you will probably want to edit this, but
% it should be fine as is for beginners.

% almost certainly you want these
\usepackage{amssymb}
\usepackage{amsmath}
\usepackage{amsfonts}

% used for TeXing text within eps files
%\usepackage{psfrag}
% need this for including graphics (\includegraphics)
%\usepackage{graphicx}
% for neatly defining theorems and propositions
%\usepackage{amsthm}
% making logically defined graphics
%%%\usepackage{xypic}

% there are many more packages, add them here as you need them

% define commands here
\begin{document}
The {\bf crosscap slide} is an autohomeomorphism which can be defined only for 
{\bf non orientable surfaces} whose genus is greater than one. 

It is also known as the  y-homeomorphism. This name was coined by Lickorish and the name crosscap slide was introduced by Korkmaz.

\begin{enumerate}
\item D.R.J. Chillingworth. {\it A finite set of generators for the 
ho\-meo\-to\-py group of a non-orientable surface}, Proc. Camb. Phil. Soc. 
65(1969), 409-430.
\item M. Korkmaz. {\it Mapping Class Groups of Non-orientable Surfaces}, Geometriae Dedicata 89 (2002), 109-133.
\item W.B.R. Lickorish. {\it Homeomorphisms of non-orientable two-manifolds}, 
Math. Proc. Camb. Phil. Soc. 59 (1963), 307-317.
\end{enumerate}
%%%%%
%%%%%
\end{document}
