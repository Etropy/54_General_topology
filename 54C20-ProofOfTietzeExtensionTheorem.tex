\documentclass[12pt]{article}
\usepackage{pmmeta}
\pmcanonicalname{ProofOfTietzeExtensionTheorem}
\pmcreated{2013-03-22 14:08:58}
\pmmodified{2013-03-22 14:08:58}
\pmowner{bbukh}{348}
\pmmodifier{bbukh}{348}
\pmtitle{proof of Tietze extension theorem}
\pmrecord{10}{35566}
\pmprivacy{1}
\pmauthor{bbukh}{348}
\pmtype{Proof}
\pmcomment{trigger rebuild}
\pmclassification{msc}{54C20}

\endmetadata

\usepackage{amssymb}
\usepackage{amsmath}
\usepackage{amsfonts}
\usepackage{amsthm}
\newtheorem{theorem}{Theorem}
\newtheorem{lemma}{Lemma}

\newcommand*{\abs}[1]{\left\lvert #1\right\rvert}
\begin{document}
To prove the Tietze Extension Theorem, we first need a lemma.

\begin{lemma}
If $X$ is a normal topological space and $A$ is closed in $X$, then for any continuous function $f\colon A\to\mathbb{R}$ such that $|f(x)|\leq 1$, there is a continuous function $g\colon X\to\mathbb{R}$ such that $\abs{g(x)}\leq\frac{1}{3}$ for $x\in X$, and $\abs{f(x)-g(x)}\leq\frac{2}{3}$ for $x\in A$.
\end{lemma}
\renewcommand{\proofname}{Proof} % This is to circumvent the bug in LaTeX2HTML
\begin{proof}
The sets $f^{-1}\bigl((-\infty,-\frac{1}{3}]\bigr)$ and $f^{-1}\bigl([\frac{1}{3},\infty)\bigr)$ are disjoint and closed in $A$. Since $A$ is closed, they are also closed in $X$. Since $X$ is normal, then by Urysohn's lemma and the fact that $[0,1]$ is homeomorphic to $[-\frac{1}{3},\frac{1}{3}]$, there is a continuous function $g\colon X\to[-\frac{1}{3},\frac{1}{3}]$ such that $g\Bigl(f^{-1}\bigl((-\infty,-\frac{1}{3}c]\bigr)\Bigr)=-\frac{1}{3}$ and $g\Bigl(f^{-1}\bigl([\frac{1}{3},\infty)\bigr)\Bigr)=\frac{1}{3}$. Thus $\abs{g(x)}\leq \frac{1}{3}$ for $x\in X$. Now if $-\leq f(x)\leq -\frac{1}{3}$, then $g(x)=-\frac{1}{3}$ and thus $\abs{f(x)-g(x)}\leq\frac{2}{3}$. Similarly if $\frac{1}{3}\leq f(x)\leq 1$, then $g(x)=\frac{1}{3}$ and thus $|f(x)-g(x)|\leq\frac{2}{3}$. Finally, for $\abs{f(x)}\leq \frac{1}{3}$ we have that $\abs{g(x)}\leq \frac{1}{3}$, and so $\abs{f(x)-g(x)}\leq\frac{2}{3}$. Hence $\abs{f(x)-g(x)}\leq\frac{2}{3}$ holds for all $x\in A$.
\end{proof}

This puts us in a position to prove the main theorem.

%\begin[Proof of the Main Theorem]{proof} % LaTeX2HTML does not support this
\renewcommand{\proofname}{Proof of the Tietze extension theorem}
\begin{proof}
First suppose that for any continuous function on a closed subset there is a continuous extension. Let $C$ and $D$ be disjoint and closed in $X$. Define $f\colon C\cup D\to\mathbb{R}$ by $f(x)=0$ for $x\in C$ and $f(x)=1$ for $x\in D$. Now $f$ is continuous and we can extend it to a continuous function $F\colon X\to\mathbb{R}$. By Urysohn's lemma, $X$ is normal because $F$ is a continuous function such that $F(x)=0$ for $x\in C$ and $F(x)=1$ for $x\in D$.

Conversely, let $X$ be normal and $A$ be closed in $X$. By the lemma, there is a continuous function $g_0\colon X\rightarrow\mathbb{R}$ such that $\abs{g_0(x)}\leq\frac{1}{3}$ for $x\in X$ and $\abs{f(x)-g_0(x)}\leq\frac{2}{3}$ for $x\in A$. Since $(f-g_0)\colon A\rightarrow\mathbb{R}$ is continuous, the lemma tells us there is a continuous function $g_1\colon X\rightarrow\mathbb{R}$ such that $\abs{g_1(x)}\leq\frac{1}{3}(\frac{2}{3})$ for $x\in X$ and $\abs{f(x)-g_0(x)-g_1(x)}\leq\frac{2}{3}(\frac{2}{3})$ for $x\in A$. By repeated application of the lemma we can construct a sequence of continuous functions $g_0,g_1,g_2,\dotsc$ such that $\abs{g_n(x)}\leq \frac{1}{3}(\frac{2}{3})^n$ for all $x\in X$, and $\abs{f(x)-g_0(x)-g_1(x)-g_2(x)-\dotsb}\leq (\frac{2}{3})^n$ for $x\in A$.
%Now assume we have found continuous $g_{n-1}\colon X\rightarrow\mathbb{R}$ for %$n>1$ such that $|g_{n-1}(x)|\leq\frac{1}{3}(\frac{2}{3})^{n-1}c$ for $x\in X$ %and $|f(x)-\sum_{k=0}^{n-1}g_k(x)|\leq(\frac{2}{3})^nc$ for $x\in A$. By the %lemma, there is a continuous function $g_n\colon X\rightarrow\mathbb{R}$ such %that $|g_n(x)|\leq\frac{1}{3}(\frac{2}{3})^nc$ for $x\in X$ and %$|f(x)-\sum_{k=0}^ng_k(x)|\leq(\frac{2}{3})^{n+1}c$ for all $x\in A$. Thus we have a %sequence of functions $g_k$, $k\geq 0$.

Define $F(x)=\sum_{n=0}^\infty g_n(x)$. Since $\abs{g_n(x)}\leq\frac{1}{3}(\frac{2}{3})^n$ and $\sum_{n=0}^\infty\frac{1}{3}(\frac{2}{3})^n$ converges as a geometric series, then $\sum_{n=0}^\infty g_n(x)$ converges absolutely and uniformly, so $F$ is a continuous function defined everywhere. Moreover $\sum_{n=0}^\infty\frac{1}{3}(\frac{2}{3})^n=1$ implies that $\abs{F(x)}\leq 1$.

%Now let $x\in X$ and let $s_k(x)=\sum_{n=0}^kg_n(x)$ be the $k$th partial sum at
%$x$. If $k>j$, then
%\begin{align*}
%\abs{s_k(x)-s_j(x)} &=\abs{\sum_{n=0}^k g_n(x)-\sum_{n=0}^j g_n(x)}\\
%&=\abs{\sum_{n=j+1}^k g_n(x)}\\
%&\leq \sum_{n=j+1}^k \abs{g_n(x)}\\
%&\leq \sum_{n=j+1}^k \tfrac{1}{3}(2/3)^{n}c\\
%&\leq \sum
%\end{align*}
%%\begin{displaymath}
%%\begin{array}{rcl}
%%|s_k(x)-s_j(x)| & = & |\sum_{n=0}^k g_n(x)-\sum_{n=0}^j g_n(x)|=|\sum_{n=j+1}^k %%g_n(x)|\\
%% & \leq & \sum_{n=j+1}^k |g_n(x)|\\
%% & \leq & \sum_{n=j+1}^k %%\frac{1}{3}(\frac{2}{3})^nc=\frac{1}{3}(\frac{2}{3})^{j+1}c\sum_{n=0}^{k-j-1}(\frac{2}{3})^n=((\frac{2}{3})^{j+1}-(\frac{2}{3})^{k+1})c
%%\end{array}
%%\end{displaymath}
%Letting $k$ go to $\infty$ with $j$ fixed, we have %$|F(x)-s_j(x)|\leq(\frac{2}{3})^{j+1}c$. Now since the right goes
%to zero as $j$ goes to $\infty$ independently of $x$, then the partial sums %converge uniformly to $F(x)$ and as the partial sums are continuous, then $F(x)$ %is continuous.

Now for $x\in A$, we have that $\abs{f(x)-\sum_{n=0}^kg_n(x)}\leq (\frac{2}{3})^{k+1}$ and as $k$ goes to infinity, the right side goes to zero and so the sum goes to $F(x)$. Thus $\abs{f(x)-F(x)}=0$ Therefore $F$ extends $f$.\end{proof}

\emph{Remarks:}
If $f$ was a function satisfying $\abs{f(x)}<1$, then the theorem can be strengthened as follows. Find an extension $F$ of $f$ as above. The set $B=F^{-1}(\{-1\}\cup\{1\})$ is closed and disjoint from $A$ because $\abs{F(x)}=\abs{f(x)}<1$ for $x\in A$. By Urysohn's lemma there is a continuous function $\phi$ such that $\phi(A)=\{1\}$ and $\phi(B)=\{0\}$. Hence $F(x)\phi(x)$ is a continuous extension of $f(x)$, and has the property that $\abs{F(x)\phi(x)}<1$.

If $f$ is unbounded, then Tietze extension theorem holds as well. To see that consider $t(x)=\tan^{-1}(x)/(\pi/2)$. The function $t\circ f$ has the property that $(t\circ f)(x)<1$ for $x\in A$, and so it can be extended to a continuous function $h\colon X\to\mathbb{R}$ which has the property $\abs{h(x)}<1$. Hence $t^{-1}\circ h$ is a continuous extension of $f$.
%%%%%
%%%%%
\end{document}
