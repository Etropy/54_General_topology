\documentclass[12pt]{article}
\usepackage{pmmeta}
\pmcanonicalname{InteriorAxioms}
\pmcreated{2013-03-22 16:30:37}
\pmmodified{2013-03-22 16:30:37}
\pmowner{rspuzio}{6075}
\pmmodifier{rspuzio}{6075}
\pmtitle{interior axioms}
\pmrecord{8}{38687}
\pmprivacy{1}
\pmauthor{rspuzio}{6075}
\pmtype{Definition}
\pmcomment{trigger rebuild}
\pmclassification{msc}{54A05}
\pmrelated{GaloisConnection}
\pmdefines{interior operator}

% this is the default PlanetMath preamble.  as your knowledge
% of TeX increases, you will probably want to edit this, but
% it should be fine as is for beginners.

% almost certainly you want these
\usepackage{amssymb}
\usepackage{amsmath}
\usepackage{amsfonts}

% used for TeXing text within eps files
%\usepackage{psfrag}
% need this for including graphics (\includegraphics)
%\usepackage{graphicx}
% for neatly defining theorems and propositions
\usepackage{amsthm}
% making logically defined graphics
%%%\usepackage{xypic}

% there are many more packages, add them here as you need them

% define commands here
\newtheorem{axiom}{Axiom}
\newtheorem{theorem}{Theorem}
\begin{document}
Let $S$ be a set.  Then an \emph{interior operator} is a function
$\,^\circ \colon \mathcal{P}(S) \to \mathcal{P}(S)$ which satisfies the 
following properties:
\begin{axiom}
$S^\circ = S$
\end{axiom}
\begin{axiom}
For all $X \subset S$, one has $X^\circ \subseteq S$.
\end{axiom}
\begin{axiom}
For all $X \subset S$, one has $(X^\circ)^\circ = X^\circ$.
\end{axiom}
\begin{axiom}
For all $X, Y \subset S$, one has $(X \cap Y)^\circ = 
X^\circ \cap Y^\circ$.
\end{axiom}

If $S$ is a topological space, then the operator which assigns to
each set its interior satisfies these axioms.  Conversely, given an
interior operator $\,^\circ$ on a set $S$, the set $\{X^\circ \mid
X \subset S\}$ defines a topology on $S$ in which $X^\circ$ is the
interior of $X$ for any subset $X$ of $S$.  Thus, specifying an
interior operator on a set is equivalent to specifying a topology
on that set.

The concepts of interior operator and closure operator are closely
related.  
Given an interior operator $\,^\circ$, one can 
define a closure operator $\,^c$ by the condition
 \[ X^c = ({(X')^\circ})\vphantom{X}' \]
and, given a closure operator $\,^c$, one can 
define an interior operator $\,^\circ$ by the condition
\ \[ X^\circ = ({(X')^c})\vphantom{X}' .\]
%%%%%
%%%%%
\end{document}
