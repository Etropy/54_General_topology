\documentclass[12pt]{article}
\usepackage{pmmeta}
\pmcanonicalname{ProofOfBaireSpaceIsUniversalForPolishSpaces}
\pmcreated{2013-03-22 18:46:57}
\pmmodified{2013-03-22 18:46:57}
\pmowner{gel}{22282}
\pmmodifier{gel}{22282}
\pmtitle{proof of Baire space is universal for Polish spaces}
\pmrecord{4}{41577}
\pmprivacy{1}
\pmauthor{gel}{22282}
\pmtype{Proof}
\pmcomment{trigger rebuild}
\pmclassification{msc}{54E50}
%\pmkeywords{Baire space}
%\pmkeywords{Polish space}

\endmetadata

% almost certainly you want these
\usepackage{amssymb}
\usepackage{amsmath}
\usepackage{amsfonts}

% used for TeXing text within eps files
%\usepackage{psfrag}
% need this for including graphics (\includegraphics)
%\usepackage{graphicx}
% for neatly defining theorems and propositions
\usepackage{amsthm}
% making logically defined graphics
%%%\usepackage{xypic}

% there are many more packages, add them here as you need them

% define commands here
\newtheorem*{theorem*}{Theorem}
\newtheorem*{lemma*}{Lemma}
\newtheorem*{corollary*}{Corollary}
\newtheorem*{definition*}{Definition}
\newtheorem{theorem}{Theorem}
\newtheorem{lemma}{Lemma}
\newtheorem{corollary}{Corollary}
\newtheorem{definition}{Definition}

\begin{document}
\PMlinkescapeword{map}
\PMlinkescapeword{integers}
\PMlinkescapeword{induction}
\PMlinkescapeword{sequence}
\PMlinkescapeword{empty sets}
\PMlinkescapeword{finite}
\PMlinkescapeword{infinite}
\PMlinkescapeword{function}
\PMlinkescapeword{limit}
\PMlinkescapeword{contain}
\PMlinkescapeword{satisfy}
\PMlinkescapeword{equation}
\PMlinkescapeword{term}
\PMlinkescapeword{closed}

Let $X$ be a nonempty Polish space. We construct a continuous onto map $f\colon \mathcal{N}\rightarrow X$, where $\mathcal{N}$ is Baire space.

Let $d$ be a complete metric on $X$.
We choose nonempty closed subsets $C(n_1,\ldots,n_k)\subseteq X$ for all integers $k\ge 0$ and $n_1,\ldots,n_k\in\mathbb{N}$ satisfying the following.
\begin{enumerate}
\item $C()=X$.
\item $C(n_1,\ldots,n_k)$ has diameter no more than $2^{-k}$.
\item For any $k\ge 0$ and $n_1,\ldots n_k\in\mathbb{N}$ then
\begin{equation}\label{eq:1}
C(n_1,\ldots,n_k)=\bigcup_{m=1}^\infty C(n_1,\ldots,n_k,m).
\end{equation}
\end{enumerate}
This can be done by induction. Suppose that the set $S=C(n_1,\ldots,n_k)$ has already been chosen for some $k\ge 0$. As $X$ is separable, $S$ can be covered by a sequence of closed sets $S_1,S_2,\ldots$. Replacing $S_j$ by $S_j\cap S$, we suppose that $S_j\subseteq S$. Then, remove any empty sets from the sequence. If the resulting sequence is finite, then it can be extended to an infinite sequence by repeating the last term.
We can then set $C(n_1,\ldots,n_k,n_{k+1})=S_{n_{k+1}}$.

We now define the function $f\colon\mathcal{N}\rightarrow X$. For any $n\in\mathbb{N}$ choose a sequence $x_k\in C(n_1,\ldots,n_k)$. Since this has diameter no more than $2^{-k}$ it follows that $d(x_j,x_k)\le 2^{-k}$ for $j\ge k$. So, the sequence is \PMlinkname{Cauchy}{CauchySequence} and has a limit $x$. As the sets $C(n_1,\ldots,n_k)$ are closed, they contain $x$ and,
\begin{equation}\label{eq:2}
\bigcap_{k=1}^\infty C(n_1,\ldots,n_k)\not=\emptyset.
\end{equation}
In fact, this has diameter zero, and must contain a single element, which we define to be $f(n)$.

This defines the function $f\colon\mathcal{N}\rightarrow X$. We show that it is continuous. If $m,n\in\mathcal{N}$ satisfy $m_j=n_j$ for $j\le k$ then $f(m),f(n)$ are in $C(m_1,\ldots,m_k)$ which, having diameter no more than $2^{-k}$, gives $d(f(m),f(n))\le 2^{-k}$. So, $f$ is indeed continuous.

Finally, choose any $x\in X$. Then $x\in C()$ and equation (\ref{eq:1}) allows us to choose $n_1,n_2,\ldots$ such that $x\in C(n_1,\ldots,n_k)$ for all $k\ge 0$. If $n=(n_1,n_2,\ldots)$ then $x$ and $f(n)$ are both in the set in equation (\ref{eq:2}) which, since it is a singleton, gives $f(n)=x$. Hence, $f$ is onto.

%%%%%
%%%%%
\end{document}
