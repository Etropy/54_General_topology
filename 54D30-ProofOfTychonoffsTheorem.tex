\documentclass[12pt]{article}
\usepackage{pmmeta}
\pmcanonicalname{ProofOfTychonoffsTheorem}
\pmcreated{2013-03-22 17:25:24}
\pmmodified{2013-03-22 17:25:24}
\pmowner{asteroid}{17536}
\pmmodifier{asteroid}{17536}
\pmtitle{proof of Tychonoff's theorem}
\pmrecord{8}{39797}
\pmprivacy{1}
\pmauthor{asteroid}{17536}
\pmtype{Proof}
\pmcomment{trigger rebuild}
\pmclassification{msc}{54D30}

% this is the default PlanetMath preamble.  as your knowledge
% of TeX increases, you will probably want to edit this, but
% it should be fine as is for beginners.

% almost certainly you want these
\usepackage{amssymb}
\usepackage{amsmath}
\usepackage{amsfonts}

% used for TeXing text within eps files
%\usepackage{psfrag}
% need this for including graphics (\includegraphics)
%\usepackage{graphicx}
% for neatly defining theorems and propositions
%\usepackage{amsthm}
% making logically defined graphics
%%%\usepackage{xypic}

% there are many more packages, add them here as you need them

% define commands here

\begin{document}
This is a proof in \PMlinkescapetext{terms} of nets. Recall the following facts:

{\bf \PMlinkescapetext{Lemma} 1 -} A net $(x_{\alpha})_{\alpha \in \mathcal{A}}$ in $\prod_{i \in I}X_i$ converges to
 $x \in \prod_{i \in I}X_i$ if and only if each coordinate $(x_{\alpha}^i)_{\alpha \in \mathcal{A}}$ converges to $x^i \in X_i$

{\bf \PMlinkescapetext{Lemma} 2 -} A topological space $X$ is compact if and only if every net in $X$ has a convergent subnet.

{\bf \PMlinkescapetext{Lemma} 3 -} Every net has a universal subnet.

{\bf \PMlinkescapetext{Lemma} 4 -} A \PMlinkname{universal net}{Ultranet} $(x_{\alpha})_{\alpha \in \mathcal{A}}$ in a compact space $X$ is convergent. (see this \PMlinkname{entry}{UniversalNetsInCompactSpacesAreConvergent})

We now prove Tychonoff's theorem.

{\bf Proof (Tychonoff's theorem) :} Let $(x_{\alpha})_{\alpha \in \mathcal{A}}$ be a net in $\prod_{i \in I}X_i$.

 Using Lemma 3 we can find a \PMlinkescapetext{universal} subnet $(y_{\beta})_{\beta \in \mathcal{B}}$ of $(x_{\alpha})_{\alpha \in \mathcal{A}}$.

It is easily seen that each coordinate net $(y_{\beta}^i)_{\beta \in \mathcal{B}}$ is a \PMlinkescapetext{universal} net in $X_i$.

Using Lemma 4 we see that each coordinate net converges, because $X_i$ is compact.

Using Lemma 1 we see that the whole net $(y_{\beta})_{\beta \in \mathcal{B}}$ converges in $\prod_{i \in I}X_i$.

We conclude that every net in $\prod_{i \in I}X_i$ has a convergent subnet, so, by Lemma 2, $\prod_{i \in I}X_i$ must be compact. $\square$
%%%%%
%%%%%
\end{document}
