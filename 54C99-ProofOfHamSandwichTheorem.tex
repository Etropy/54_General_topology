\documentclass[12pt]{article}
\usepackage{pmmeta}
\pmcanonicalname{ProofOfHamSandwichTheorem}
\pmcreated{2013-03-22 16:40:29}
\pmmodified{2013-03-22 16:40:29}
\pmowner{Statusx}{15142}
\pmmodifier{Statusx}{15142}
\pmtitle{proof of ham sandwich theorem}
\pmrecord{5}{38882}
\pmprivacy{1}
\pmauthor{Statusx}{15142}
\pmtype{Proof}
\pmcomment{trigger rebuild}
\pmclassification{msc}{54C99}

\endmetadata

% this is the default PlanetMath preamble.  as your knowledge
% of TeX increases, you will probably want to edit this, but
% it should be fine as is for beginners.

% almost certainly you want these
\usepackage{amssymb}
\usepackage{amsmath}
\usepackage{amsfonts}

% used for TeXing text within eps files
%\usepackage{psfrag}
% need this for including graphics (\includegraphics)
%\usepackage{graphicx}
% for neatly defining theorems and propositions
%\usepackage{amsthm}
% making logically defined graphics
%%%\usepackage{xypic}

% there are many more packages, add them here as you need them

% define commands here

\begin{document}
This proof uses the Borsuk-Ulam theorem, which states that any continuous function from $S^n$ to $\mathbb{R}^n$ maps some pair of antipodal points to the same point.

Let $A$ be a measurable bounded subset of $\mathbb{R}^n$.  Given any unit vector $\hat n \in S^{n-1}$ and $s \in \mathbb{R}$, there is a unique $n-1$ dimensional hyperplane normal to $\hat n$ and containing $s \hat n$.  

Define $f:S^{n-1} \times \mathbb{R} \rightarrow [0,\infty)$ by sending  $(\hat n,s)$ to the measure of the subset of $A$ lying on the side of the plane corresponding to $(\hat n,s)$ in the direction in which $\hat n$ points.  Note that $(\hat n,s)$ and $(-\hat n,-s)$ correspond to the same plane, but to different sides of the plane, so that $f(\hat n,s)+f(-\hat n,-s)=m(A)$.

Since $A$ is bounded, there is an $r>0$ such that $A$ is contained in $\overline{B_r}$, the closed ball of  radius $r$ centered at the origin.  For sufficiently small changes in $(\hat n,s)$, the measure of the portion of $\overline{B_r}$ between the different corresponding planes can be made arbitrarily small, and this bounds the change in $f(\hat n,s)$, so that $f$ is a continuous function.

Finally, it's easy to see that, for fixed $\hat n$, $f(\hat n,s)$ is monotonically decreasing in $s$, with $f(\hat n,-s)=m(A)$ and $f(\hat n,s)=0$  for $s$ sufficiently large.

Given these properties of $f$, we see by the intermediate value theorem that, for fixed $\hat n$, there is an interval $[a,b]$ such that the set of $s$ with $f(\hat n,s)=m(A)/2$ is $[a,b]$.  If we define $g(\hat n)$ to be the midpoint of this interval, then, since $f$ is continuous, we see $g$ is a continuous function from $S^{n-1}$ to $\mathbb{R}$.  Also, since $f(\hat n,s)+f(-\hat n,-s)=m(A)$, if $[a,b]$ is the interval corresponding to $\hat n$, then $[-b,-a]$ is the interval corresponding to $-\hat n$, and so $g(\hat n)=-g(-\hat n)$.

Now let $A_1,A_2,...,A_n$ be measurable bounded subsets of $\mathbb{R}^n$, and let $f_i,g_i$ be the maps constructed above for $A_i$.  Then we can define $h:S^{n-1} \rightarrow R^{n-1}$ by:

\[ h(\hat n) = (f_1(\hat n,g_n(\hat n)),f_2(\hat n,g_n(\hat n)),...f_{n-1}(\hat n,g_n(\hat n))) \]

This is continuous, since each coordinate function is the composition of continuous functions.  Thus we can apply the Borsuk-Ulam theorem to see there is some $\hat n \in S^{n-1}$ with $h(\hat n)=h(-\hat n)$, ie, with:

\[ f_i(\hat n,g_n(\hat n))=f_i(-\hat n,g_n(-\hat n))=f_i(-\hat n,-g_n(\hat n)) \]

where we've used the property of $g$ mentioned above.  But this just means that for each $A_i$ with $1 \leq i \leq n-1$, the measure of the subset of $A_i$ lying on one side of the plane corresponding to $(\hat n,g_n(\hat n))$, which is $f_i(\hat n,g_n(\hat n))$, is the same as the measure of the subset of $A_i$ lying on the other side of the plane, which is $f_i(-\hat n,-g_n(\hat n))$.  In other words, the plane corresponding to $(\hat n,g_n(\hat n))$ bisects each $A_i$ with $1 \leq i \leq n-1$.  Finally, by the definition of $g_n$, this plane also bisects $A_n$, and so it bisects each of the $A_i$ as claimed.
%%%%%
%%%%%
\end{document}
