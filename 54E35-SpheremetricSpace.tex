\documentclass[12pt]{article}
\usepackage{pmmeta}
\pmcanonicalname{SpheremetricSpace}
\pmcreated{2013-03-22 14:47:38}
\pmmodified{2013-03-22 14:47:38}
\pmowner{rspuzio}{6075}
\pmmodifier{rspuzio}{6075}
\pmtitle{sphere (metric space)}
\pmrecord{6}{36446}
\pmprivacy{1}
\pmauthor{rspuzio}{6075}
\pmtype{Definition}
\pmcomment{trigger rebuild}
\pmclassification{msc}{54E35}
\pmsynonym{sphere}{SpheremetricSpace}

% this is the default PlanetMath preamble.  as your knowledge
% of TeX increases, you will probably want to edit this, but
% it should be fine as is for beginners.

% almost certainly you want these
\usepackage{amssymb}
\usepackage{amsmath}
\usepackage{amsfonts}

% used for TeXing text within eps files
%\usepackage{psfrag}
% need this for including graphics (\includegraphics)
%\usepackage{graphicx}
% for neatly defining theorems and propositions
%\usepackage{amsthm}
% making logically defined graphics
%%%\usepackage{xypic}

% there are many more packages, add them here as you need them

% define commands here
\begin{document}
The set $\{ x \mid d(x,c) = r \}$ is called the \emph{sphere} of radius $r$ with centre $c$.  This generalizes the notion of spheres to metric spaces.  

Note that the sphere in a metric space need not look like a sphere in Euclidean space.  For instance, if we impose the metric $d(x,y) = max \{|x_1-y_1|, |x_2-y_2|, |x_3-y_3|\}$ on $\mathbb{R}^3$ instead of the Euclidean metric, spheres according to this metric are actually cubes!  Even more bizarre situations can occur in general --- a sphere might be disconnected, or it may be discrete, or it may even be an empty set.
%%%%%
%%%%%
\end{document}
