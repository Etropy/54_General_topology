\documentclass[12pt]{article}
\usepackage{pmmeta}
\pmcanonicalname{ClosedSubsetsOfACompactSetAreCompact}
\pmcreated{2013-03-22 13:55:56}
\pmmodified{2013-03-22 13:55:56}
\pmowner{Wkbj79}{1863}
\pmmodifier{Wkbj79}{1863}
\pmtitle{closed subsets of a compact set are compact}
\pmrecord{16}{34691}
\pmprivacy{1}
\pmauthor{Wkbj79}{1863}
\pmtype{Theorem}
\pmcomment{trigger rebuild}
\pmclassification{msc}{54D30}
\pmrelated{AClosedSetInACompactSpaceIsCompact}
\pmrelated{ACompactSetInAHausdorffSpaceIsClosed}

% this is the default PlanetMath preamble.  as your knowledge
% of TeX increases, you will probably want to edit this, but
% it should be fine as is for beginners.

% almost certainly you want these
\usepackage{amssymb}
\usepackage{amsmath}
\usepackage{amsfonts}
\usepackage{amsthm}

\usepackage{mathrsfs}

% used for TeXing text within eps files
%\usepackage{psfrag}
% need this for including graphics (\includegraphics)
%\usepackage{graphicx}
% for neatly defining theorems and propositions
%
% making logically defined graphics
%%%\usepackage{xypic}

% there are many more packages, add them here as you need them

% define commands here

\newcommand{\sR}[0]{\mathbb{R}}
\newcommand{\sC}[0]{\mathbb{C}}
\newcommand{\sN}[0]{\mathbb{N}}
\newcommand{\sZ}[0]{\mathbb{Z}}

 \usepackage{bbm}
 \newcommand{\Z}{\mathbbmss{Z}}
 \newcommand{\C}{\mathbbmss{C}}
 \newcommand{\F}{\mathbbmss{F}}
 \newcommand{\R}{\mathbbmss{R}}
 \newcommand{\Q}{\mathbbmss{Q}}



\newcommand*{\norm}[1]{\lVert #1 \rVert}
\newcommand*{\abs}[1]{| #1 |}



\newtheorem{thm}{Theorem}
\newtheorem{defn}{Definition}
\newtheorem{prop}{Proposition}
\newtheorem{lemma}{Lemma}
\newtheorem{cor}{Corollary}
\begin{document}
\begin{thm}
Suppose $X$ is a topological space.  If $K$ is a compact subset of $X$, $C$ is a closed set in $X$, and $C \subseteq K$, then $C$ is a compact set in $X$.
\end{thm}

The below proof follows \PMlinkname{e.g.}{Eg}~\cite{jameson}.  A proof based on the finite intersection property is given in \cite{singer}.

\begin{proof}
Let $I$ be an indexing set and $F=\{ V_\alpha \mid \alpha \in I\}$ be an arbitrary open cover for $C$. Since $X\setminus C$ is open, it follows that $F$ together with $X\setminus C$ is an open cover for $K$. Thus, $K$ can be covered by a finite number of sets, say, $V_1, \ldots, V_N$ from $F$ together with possibly $X\setminus C$. Since $C\subset K$, $V_1, \ldots, V_N$ cover $C$, and it follows that $C$ is compact.
\end{proof}

% Note: This proof was written from scratch with no reference to [4].
The following proof uses the \PMlinkname{finite intersection property}{ASpaceIsCompactIfAndOnlyIfTheSpaceHasTheFiniteIntersectionProperty}.

\begin{proof}
Let $I$ be an indexing set and $\{A_{\alpha}\}_{\alpha \in I}$ be a collection of $X$-closed sets contained in $C$ such that, for any finite $J \subseteq I$, $\displaystyle \bigcap_{\alpha \in J} A_{\alpha}$ is not empty.  Recall that, for every $\alpha \in I$, $A_{\alpha} \subseteq C\subseteq K$.  Thus, for every $\alpha \in I$, $A_{\alpha}= K\cap A_{\alpha}$.  Therefore, $\{A_{\alpha}\}_{\alpha \in I}$ are $K$-closed subsets of $K$ (see \PMlinkname{this page}{ClosedSetInASubspace}) such that, for any finite $J \subseteq I$, $\displaystyle \bigcap_{\alpha \in J} A_{\alpha}$ is not empty.  As $K$ is compact, $\displaystyle \bigcap_{\alpha \in I} A_{\alpha}$ is not empty (again, by \PMlinkname{this result}{ASpaceIsCompactIfAndOnlyIfTheSpaceHasTheFiniteIntersectionProperty}).
This proves the claim.
\end{proof}

\begin{thebibliography}{9}
 \bibitem{kelley}
 J.L.~Kelley, \emph{General Topology}, D.~van Nostrand Company, Inc., 1955.
\bibitem{lang}
 S.~Lang, \emph{Analysis II},
 Addison-Wesley Publishing Company Inc., 1969.
 \bibitem{jameson} G.J.~Jameson, \emph{Topology and Normed Spaces},
 Chapman and Hall, 1974.
 \bibitem{singer}
 I.M.~Singer, J.A.~Thorpe,
 \emph{Lecture Notes on Elementary Topology and Geometry},
 Springer-Verlag, 1967.
\end{thebibliography}
%%%%%
%%%%%
\end{document}
