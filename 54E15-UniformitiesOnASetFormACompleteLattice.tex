\documentclass[12pt]{article}
\usepackage{pmmeta}
\pmcanonicalname{UniformitiesOnASetFormACompleteLattice}
\pmcreated{2013-03-22 16:30:46}
\pmmodified{2013-03-22 16:30:46}
\pmowner{mps}{409}
\pmmodifier{mps}{409}
\pmtitle{uniformities on a set form a complete lattice}
\pmrecord{4}{38690}
\pmprivacy{1}
\pmauthor{mps}{409}
\pmtype{Derivation}
\pmcomment{trigger rebuild}
\pmclassification{msc}{54E15}
\pmclassification{msc}{06B23}
\pmdefines{discrete uniformity}
\pmdefines{initial uniformity}
\pmdefines{weak uniformity}

% this is the default PlanetMath preamble.  as your knowledge
% of TeX increases, you will probably want to edit this, but
% it should be fine as is for beginners.

% almost certainly you want these
\usepackage{amssymb}
\usepackage{amsmath}
\usepackage{amsfonts}

% used for TeXing text within eps files
%\usepackage{psfrag}
% need this for including graphics (\includegraphics)
%\usepackage{graphicx}
% for neatly defining theorems and propositions
\usepackage{amsthm}
% making logically defined graphics
%%%\usepackage{xypic}

% there are many more packages, add them here as you need them

% define commands here
\newtheorem*{theorem*}{Theorem}
\newtheorem*{corollary*}{Corollary}


\newcommand{\UU}{\mathcal{U}}
\newcommand{\BB}{\mathcal{B}}
\begin{document}
\begin{theorem*}
The collection of uniformities on a given set ordered by set inclusion forms a complete lattice.
\end{theorem*}

\begin{proof}
Let $X$ be a set.  Let $\mathfrak{U}(X)$ denote the collection of uniformities on $X$.  The coarsest uniformity on $X$ is $\{X\times X\}$, and the finest is the \emph{discrete uniformity}:
\[
\{S\subset X\times X\colon \Delta(X)\subseteq S\}.
\]
Hence $\mathfrak{U}(X)$ is bounded.  To show that $\mathfrak{U}(X)$ is complete, we must prove that it has the least upper bound property.  

Suppose $\{\UU_{\alpha}\}_{\alpha\in I}$ is a nonempty family of uniformities on $X$.  Let $\BB$ consist of all finite intersections of elements of the $\UU_{\alpha}$.  Let us check that $\BB$ is a fundamental system of entourages for a uniformity on $X$.

(B1) Let $S$, $T\in\BB$.  Each of $S$ and $T$ is a finite intersection of elements of the $\UU_{\alpha}$, so their intersection is as well.  Hence $S\cap T\in\BB$.

(B2) Every element of $\BB$ is a finite intersection of subsets of $X\times X$ containing $\Delta(X)$.  So every element of $\BB$ contains the diagonal.

(B3) Let $S\in\BB$.  Without loss of generality, $S=S_{\alpha}\cap S_{\beta}$, where $S_{\alpha}\in\UU_{\alpha}$ and $S_{\beta}\in\UU_{\beta}$.  Since $S_{\alpha}\in\UU_{\alpha}$, $S_{\alpha}^{-1}\in\UU_{\alpha}$.  Similarly, $S_{\beta}^{-1}\in\UU_{\beta}$.  Since the process of taking the inverse of a relation commutes with taking finite intersections, $(S_{\alpha}\cap S_{\beta})^{-1}\in\BB$.

(B4) Let $S\in\BB$.  Again suppose $S=S_{\alpha}\cap S_{\beta}$ with $S_{\alpha}\in\UU_{\alpha}$ and $S_{\beta}\in\UU_{\beta}$.  Then there exist $T_{\alpha}\in\UU_{\alpha}$ and $T_{\beta}\in\UU_{\beta}$ such that 
$T_{\alpha}\circ T_{\alpha}\subseteq S_{\alpha}$ and $T_{\beta}\circ T_{\beta}\subseteq S_{\beta}$.  The set $T=T_{\alpha}\cap T_{\beta}$ is in $\UU$, and since $T\circ T$ is a subset of both $S_{\alpha}$ and $S_{\beta}$, it is a subset of $S$.

The fundamental system $\BB$ generates a uniformity $\UU$.  By construction, $\UU$ is an upper bound of the $\UU_{\alpha}$.  But any upper bound of the $\UU_{\alpha}$ would have to contain all finite intersections of elements of the $\UU_{\alpha}$.  So $\UU=\sup_{\alpha\in I} \UU_{\alpha}$.
\end{proof}

This theorem is useful because it allows us to assert the existence of the coarsest uniform space satisfying a particular property.

\begin{corollary*}
Let $X$ be a set and let $\{Y_{\alpha}\}_{\alpha\in I}$ be a family of uniform spaces.  Then for any family of functions $\{f_{\alpha}\colon X\to Y_{\alpha}\}$, there is a coarsest uniformity on $X$ making all the $f_{\alpha}$ uniformly continous.
\end{corollary*}

The coarsest uniformity making a family of functions uniformly continuous is called the \emph{initial uniformity} or \emph{weak uniformity}.

\begin{thebibliography}{9}
\bibitem{TG}
Nicolas Bourbaki, {\it Elements of Mathematics: General Topology: Part 1}, Hermann, 1966.
\end{thebibliography}

%%%%%
%%%%%
\end{document}
