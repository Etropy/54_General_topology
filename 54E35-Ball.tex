\documentclass[12pt]{article}
\usepackage{pmmeta}
\pmcanonicalname{Ball}
\pmcreated{2013-03-22 12:07:47}
\pmmodified{2013-03-22 12:07:47}
\pmowner{CWoo}{3771}
\pmmodifier{CWoo}{3771}
\pmtitle{ball}
\pmrecord{22}{31296}
\pmprivacy{1}
\pmauthor{CWoo}{3771}
\pmtype{Definition}
\pmcomment{trigger rebuild}
\pmclassification{msc}{54E35}
\pmsynonym{open ball}{Ball}
\pmsynonym{closed ball}{Ball}
%\pmkeywords{topological space}
%\pmkeywords{metric space}
\pmrelated{Topology}
\pmrelated{Neighborhood}
\pmrelated{UnitDisc2}
\pmrelated{CompactMetricSpacesAreSecondCountable}
\pmrelated{T0Space}
\pmdefines{disc}

\endmetadata

\usepackage{amssymb}
\usepackage{amsmath}
\usepackage{amsfonts}
\usepackage{color,multido,pstricks,pst-plot}
\usepackage{graphicx}
%%%\usepackage{xypic}
\begin{document}
Let $X$ be a metric space, and $c\in X$. An open \emph{ball} around $c$ with radius $r>0$ is the set
$$B_r(c)=\{x\in X: d(c,x)<r\}$$
where $d(c,x)$ is the distance from $c$ to $x$.
Sometimes, when there is no danger of confusion, an open ball is simply called a ball.

The name is derived from the fact that, in the euclidean space $\mathbb{R}^3$ with the usual metric (distance between two points), a ball has the shape of a ``ball'' in the literal sense.  Also, under the usual metric, balls are open discs in the euclidean plane $\mathbb{R}^2$ (see the figure below), and open intervals in the line $\mathbb{R}$.

\begin{center}
\begin{pspicture}(-1,-0.5)(5,3)
\psaxes[Dx=5,Dy=5]{->}(0,0)(-1,-0.5)(5,3)
\pscircle[linestyle=dashed,fillcolor=lightgray,fillstyle=solid](2.5,1.75){1}
\end{pspicture}
\end{center}

So, on $\mathbb{R}$ (with the standard topology), the ball with radius 1 around $5$ is the open interval given by $\{x : |5-x|<1\}$, that is, $(4,6)$.

It should be noted that the definition of ball depends on the metric attached to the space. If we had considered $\mathbb{R}^2$ with the \emph{taxicab metric}, the ball with radius $1$ around zero would be the rhombus with vertices at $(-1,0),(0,-1),(1,0),(0,1)$ (see the figure below).

\begin{center}
\begin{pspicture}(-2,-2)(2,2)
\pspolygon[linestyle=dashed,fillcolor=lightgray,fillstyle=solid](-1,-1)(-1,1)(1,1)(1,-1)
\psaxes[Dx=5,Dy=5]{->}(0,0)(-2,-2)(2,2)
\end{pspicture}
\end{center}

Balls are open sets under the topology induced by the metric, and therefore are examples of neighborhoods.

We can also talk of closed balls (or discs):
$$\overline B_r(c)=\{x\in X: d(c,x)\leq r\}$$

Another common notation is $B(c,r)$.

\textbf{Remark}.  A ball is sometimes referred to as a \emph{disc}, although disc is usually reserved for a ball in a metric space having the structure of a two-dimensional vector space.  The boundary of a closed ball is called a \emph{sphere}.  In the case when the metric space is a two-dimensional vector space, a sphere is called a \emph{circle}.
%%%%%
%%%%%
%%%%%
\end{document}
