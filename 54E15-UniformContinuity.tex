\documentclass[12pt]{article}
\usepackage{pmmeta}
\pmcanonicalname{UniformContinuity}
\pmcreated{2013-03-22 16:43:15}
\pmmodified{2013-03-22 16:43:15}
\pmowner{CWoo}{3771}
\pmmodifier{CWoo}{3771}
\pmtitle{uniform continuity}
\pmrecord{7}{38941}
\pmprivacy{1}
\pmauthor{CWoo}{3771}
\pmtype{Definition}
\pmcomment{trigger rebuild}
\pmclassification{msc}{54E15}
\pmrelated{UniformlyContinuous}
\pmrelated{UniformContinuityOverLocallyCompactQuantumGroupoids}
\pmdefines{uniformly continuous}

\endmetadata

\usepackage{amssymb,amscd}
\usepackage{amsmath}
\usepackage{amsfonts}

% used for TeXing text within eps files
%\usepackage{psfrag}
% need this for including graphics (\includegraphics)
%\usepackage{graphicx}
% for neatly defining theorems and propositions
\usepackage{amsthm}
% making logically defined graphics
%%\usepackage{xypic}
\usepackage{pst-plot}
\usepackage{psfrag}

% define commands here
\newtheorem{prop}{Proposition}
\begin{document}
In this entry, we extend the usual definition of a uniformly continuous function between metric spaces to arbitrary uniform spaces.

Let $(X,\mathcal{U}),(Y,\mathcal{V})$ be uniform spaces (the second component is the uniformity on the first component).  A function $f:X\to Y$ is said to be \emph{uniformly continuous} if for any $V\in\mathcal{V}$ there is a $U\in \mathcal{U}$ such that for all $x\in X$, $U[x]\subseteq f^{-1}(V[f(x)])$.

Sometimes it is useful to use an alternative but equivalent version of uniform continuity of a function:
\begin{prop}
Suppose $f:X\to Y$ is a function and $g:X\times X \to Y\times Y$ is defined by $g(x_1,x_2)=(f(x_1), f(x_2))$.  Then $f$ is uniformly continuous iff for any $V\in \mathcal{V}$, there is a $U\in \mathcal{U}$ such that $U\subseteq g^{-1}(V)$.
\end{prop}
\begin{proof}  Suppose $f$ is uniformly continuous.  Pick any $V\in \mathcal{V}$.  Then $U\in \mathcal{U}$ exists with $U[x]\subseteq f^{-1}(V[f(x)])$ for all $x\in X$.  If $(a,b)\in U$, then $b\in U[a]\subseteq f^{-1}(V[f(a)])$, or $f(b)\subseteq V[f(a)]$, or $g(a,b)=(f(a),f(b))\in V$.  The converse is straightforward.
\end{proof}

\textbf{Remark}.  Note that we could have picked $U$ so the inclusion becomes an equality.

\begin{prop}.  If $f:X\to Y$ is uniformly continuous, then it is continuous under the uniform topologies of $X$ and $Y$.\end{prop}
\begin{proof}
Let $A$ be open in $Y$ and set $B=f^{-1}(A)$.  Pick any $x\in B$.  Then $y=f(x)$ has a uniform neighborhood $V[y]\subseteq A$.  By the uniform continuity of $f$, there is an entourage $U\in \mathcal{U}$ with $x\in U[x]\subseteq f^{-1}(V[y])\subseteq f^{-1}(A)=B$.
\end{proof}

\textbf{Remark}.  The converse is not true, even in metric spaces.
%%%%%
%%%%%
\end{document}
