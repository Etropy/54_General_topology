\documentclass[12pt]{article}
\usepackage{pmmeta}
\pmcanonicalname{InvariantFormsOnRepresentationsOfCompactGroups}
\pmcreated{2013-03-22 13:23:40}
\pmmodified{2013-03-22 13:23:40}
\pmowner{bwebste}{988}
\pmmodifier{bwebste}{988}
\pmtitle{invariant forms on representations of compact groups}
\pmrecord{11}{33933}
\pmprivacy{1}
\pmauthor{bwebste}{988}
\pmtype{Theorem}
\pmcomment{trigger rebuild}
\pmclassification{msc}{54-00}

% this is the default PlanetMath preamble.  as your knowledge
% of TeX increases, you will probably want to edit this, but
% it should be fine as is for beginners.

% almost certainly you want these
\usepackage{amssymb}
\usepackage{amsmath}
\usepackage{amsfonts}

% used for TeXing text within eps files
%\usepackage{psfrag}
% need this for including graphics (\includegraphics)
%\usepackage{graphicx}
% for neatly defining theorems and propositions
\usepackage{amsthm}
% making logically defined graphics
%%%\usepackage{xypic}

% there are many more packages, add them here as you need them

% define commands here
\newtheorem{thm}{Theorem}
\newtheorem{prop}{Proposition}

\newcommand{\ab}[1]{{#1}_{\mathrm{ab}}}
\newcommand{\Ad}{\mathrm{Ad}}
\newcommand{\ad}{\mathrm{ad}}
\newcommand{\Aut}{\mathrm{Aut}\,}
\newcommand{\Aff}[2]{\mathrm{Aff}_{#1} #2}
\newcommand{\aff}[2]{\mathfrak{aff}_{#1} #2}
\newcommand{\mcB}{\mathcal{B}}
\newcommand{\bb}[1]{\mathbb{#1}}
\newcommand{\bfrac}[2]{\left[\frac{#1}{#2}\\right]}
\newcommand{\bkh}{\backslash}
\newcommand{\Cyc}[2]{\mathcal{C}^{#1}_{#2}}
\newcommand{\Cbar}[2]{\overline{\C{#1}{#2}}}
%\newcommand{\CD}{\R[\Delta]}
\newcommand{\C}{\mathbb{C}}
\newcommand{\CF}[2]{\ensuremath{\mathfrak{C}(#1,#2)}}
\newcommand{\Cinf}{\EuScript{C}^{\infty}}
\newcommand{\cmp}{cyclic mod $p$\xspace}
\newcommand{\cp}{\mathrm{c.p.}}
\newcommand{\CS}{\EuScript{CS}}
\newcommand{\deck}{\EuScript{D}}
\newcommand{\defl}[1]{\mathfrak{def}_{#1}}
\newcommand{\Der}{\mathrm{Der}\,}
\newcommand{\eH}{[X_H]-[Y_H]}
\newcommand{\EL}{\mathcal{EL}}
\newcommand{\End}{\mathrm{End}}
\newcommand{\ES}[1]{\EuScript{#1}}
\newcommand{\Ext}{\mathrm{Ext}}
\newcommand{\Fix}{\mathrm{Fix}}
\newcommand{\fr}[1]{\mathfrak{#1}}
\newcommand{\Frat}{\mathrm{Frat}\,}
\newcommand{\Gal}[1]{\Gamma(#1 |\Q)}
\newcommand{\GL}[2]{\mathrm{GL}_{#1} #2}
\newcommand{\gl}[2]{\mathfrak{gl}_{#1} #2}
\newcommand{\GrR}[1]{a(#1 G)}
\newcommand{\Gr}{\mathrm{Gr}\,}
\newcommand{\mcH}{\mathcal{H}}
\renewcommand{\H}{\mathbb{H}}
\newcommand{\Hom}[2]{\mathrm{Hom}(#1,#2)}
\newcommand{\id}{\mathrm{id}}
\newcommand{\im}{\mathrm{im}}
\newcommand{\ind}[2]{\mathrm{ind}^{#1}_{#2}}
\newcommand{\indp}[2]{\mathfrak{ind}^{#1}_{#2}}
\newcommand{\inn}[1]{\langle #1\rangle}
\newcommand{\Iso}{\mathrm{Iso}}
\newcommand{\K}{\mathcal{K}}
\renewcommand{\ker}{\mathrm{ker}\,}
\renewcommand{\L}[1]{\mathfrak{L}(#1)}
\newcommand{\lap}[1]{\Delta_{#1}}
\newcommand{\lapM}{\Delta_M}
\newcommand{\Lie}{\mathrm{Lie}}
\newcommand{\lineq}{linearly equivalent\xspace}
\newcommand{\mc}[1]{\mathcal{#1}}
\newcommand{\mG}{m_G}
\newcommand{\mK}{m_{\K}}
\newcommand{\mindeg}[1]{\fr{md}(#1)}
\newcommand{\N}{\mathbb{N}}
\renewcommand{\O}{\mathcal{O}}
\newcommand{\Om}{\Omega}
\newcommand{\om}{\omega}
\newcommand{\Orb}{\mathrm{Orb}}
\newcommand{\pad}{\hat{\Z}_p}
\newcommand{\pder}[2]{\frac{\partial #1}{\partial #2}}
\newcommand{\pderw}[1]{\frac{\partial}{\partial #1}}
\newcommand{\pdersec}[2]{\frac{\partial^2 #1}{\partial {#2}^2}} 
\newcommand{\perm}[1]{\pi_{#1}}
\newcommand{\Q}{\mathbb{Q}}
\newcommand{\R}{\mathbb{R}}
\newcommand{\rad}{\mathrm{rad}\,}
\newcommand{\res}[2]{\mathrm{res}^{#1}_{#2}}
\newcommand{\resp}[2]{\mathfrak{res}^{#1}_{#2}}
\newcommand{\RG}{\EuScript{R}_G}
\newcommand{\rk}{\mathrm{rk}\,}
\newcommand{\V}[1]{\mathbf{#1}}
\newcommand{\vp}{\varphi}
\newcommand{\Stab}{\mathrm{Stab}}
\newcommand{\SL}[2]{\mathrm{SL}_{#1} #2}
\renewcommand{\sl}[2]{\fr{sl}_{#1} #2}
\newcommand{\SO}[2]{\mathrm{SO}_{#1} #2}

\newcommand{\Sp}[2]{\mathrm{Sp}_{#1} #2}
\renewcommand{\sp}[2]{\fr{sp}_{#1} #2}
\newcommand{\SU}[1]{\mathrm{SU}( #1)}
\newcommand{\su}[1]{\fr{su}_{#1}}
\newcommand{\Sym}{\mathrm{Sym}}
\newcommand{\sym}{\mathrm{sym}}
\newcommand{\Tg}{\mc{T}(\fr g)}
\newcommand{\tom}{\tilde{\omega}}
\newcommand{\ghtghp}{\fr g/\fr h\oplus(\fr g/\fr h^\perp)^*}
\newcommand{\ghps}{(\fr g/\fr h^\perp)^*}
\newcommand{\Tr}{\mathrm{Tr}}
\newcommand{\tr}{\mathrm{tr}}
%\renewcommand{\thechapter}{\Roman{chapter}}
%\renewcommand{\thesection}{\thechapter.\arabic{section}}
%\renewcommand{\thethm}{\thechapter.\arabic{thm}}
\newcommand{\Ug}{\mc{U}(\fr g)}
\newcommand{\Uh}{\mc{U}(\fr h)}
\renewcommand{\V}[1]{\mathbf{#1}}
\newcommand{\Z}{\mathbb{Z}}
\newcommand{\Zp}{\Z/p}
\begin{document}
Let $G$ be a real Lie group.  TFAE:

\begin{enumerate}
\item Every real representation of $G$ has an invariant positive definite form, and $G$ has at least one faithful representation.
\item One faithful representation of $G$ has an invariant positive definite form.
\item $G$ is compact.
\end{enumerate}

Also, any group satisfying these criteria is reductive, and its Lie algebra is the direct sum of simple algebras and an abelian algebra (such an algebra is often called reductive).

\begin{proof}
$(1)\Rightarrow (2)$: Obvious.

$(2)\Rightarrow (3)$: Let $\Omega$ be the invariant form on a faithful representation $V$.  Let then representation gives an embedding 
$\rho:G\to \mathrm{SO}(V,\Omega)$, the group of automorphisms of $V$ preserving $\Omega$.  Thus, $G$ is homeomorphic to a closed subgroup of $\mathrm{SO}(V,\Omega)$.  Since this group is compact, $G$ must be compact as well.

(Proof that $\mathrm{SO}(V,\Omega)$ is compact: By induction on $\dim V$.  Let $v\in V$ be an arbitrary vector.  Then there is a map, evaluation on $v$, from
$\mathrm{SO}(V,\Omega)\to S^{\dim V-1}\subset V$ (this is topologically a sphere, since $(V,\omega)$ is isometric to $\R^{\dim V}$ with the standard norm).  This is a a fiber bundle, and the fiber over any point is a copy of $\mathrm{SO}(v^{\perp},\Omega)$, which is compact by the inductive hypothesis.  Any fiber bundle over a compact base with compact fiber has compact total space.  Thus $\mathrm{SO}(V,\Omega)$ is compact).

$(3)\Rightarrow(1)$: Let $V$ be an arbitrary representation of $G$.  Choose an arbitrary positive definite form $\Omega$ on $V$.
Then define $$\tilde{\Omega}(v,w)=\int_{G}\Omega(gv,gw)dg,$$
where $dg$ is Haar measure (normalized so that $\int_Gdg=1$).  Since $K$ is compact, this gives a well defined form.  It is obviously bilinear, b$\mathrm{SO}(V,\Omega)$y the linearity of integration, and positive definite since $$\tilde{\Omega}(gv,gv)=
\int_G\Omega(gv,gv)dg\geq \inf_{g\in G}\Omega(gv,gv)>0.$$  Furthermore, $\tilde\Omega$ is invariant, since $$\tilde\Omega(hv,hw)=\int_G\Omega(ghv,ghw)dg=\int_G\Omega(ghv,ghw)d(gh)=
\tilde\Omega(v,w).$$

For representation $\rho:T\to \mathrm{GL}(V)$ of the maximal torus $T\subset K$, there exists a representation $\rho'$ of $K$, with $\rho$ a $T$-subrepresentation of $\rho'$.  Also, since every conjugacy class of $K$ intersects any maximal torus, a representation of $K$ is faithful if and only if it restricts to a faithful representation of $T$.  Since any torus has a faithful representation, $K$ must have one as well.

Given that these criteria hold, let $V$ be a representation of $G$, $\Omega$ is positive definite real form, and $W$ a subrepresentation.  Now consider $$W^{\perp}=\{v\in V|\Omega(v,w)=0 \,\forall w\in W\}.$$  By the positive definiteness of $\Omega$, $V=W\oplus W^{\perp}$.  By induction, $V$ is completely reducible.

Applying this to the adjoint representation of $G$ on $\fr g$, its Lie algebra,
we find that $\fr g$ in the direct sum of simple algebras $\fr g_1,\ldots,\fr g_n$, in the sense that $\fr g_i$ has no proper nontrivial ideals, meaning that $\fr g_i$ is simple in the usual sense or it is abelian.  
\end{proof}
%%%%%
%%%%%
\end{document}
