\documentclass[12pt]{article}
\usepackage{pmmeta}
\pmcanonicalname{UniformizableSpace}
\pmcreated{2013-03-22 16:49:05}
\pmmodified{2013-03-22 16:49:05}
\pmowner{CWoo}{3771}
\pmmodifier{CWoo}{3771}
\pmtitle{uniformizable space}
\pmrecord{5}{39055}
\pmprivacy{1}
\pmauthor{CWoo}{3771}
\pmtype{Definition}
\pmcomment{trigger rebuild}
\pmclassification{msc}{54E15}
\pmdefines{uniformizable}
\pmdefines{completely uniformizable}

\usepackage{amssymb,amscd}
\usepackage{amsmath}
\usepackage{amsfonts}

% used for TeXing text within eps files
%\usepackage{psfrag}
% need this for including graphics (\includegraphics)
%\usepackage{graphicx}
% for neatly defining theorems and propositions
\usepackage{amsthm}
% making logically defined graphics
%%\usepackage{xypic}
\usepackage{pst-plot}
\usepackage{psfrag}

% define commands here
\newtheorem{prop}{Proposition}
\newtheorem{thm}{Theorem}
\newtheorem{ex}{Example}
\newcommand{\real}{\mathbb{R}}
\begin{document}
Let $X$ be a topological space with $\mathcal{T}$ the topology defined on it.  $X$ is said to be \emph{uniformizable}
\begin{enumerate}
\item there is a uniformity $\mathcal{U}$ defined on $X$, and 
\item $\mathcal{T}=T_{\mathcal{U}}$, the uniform topology induced by $\mathcal{U}$.
\end{enumerate}

It can be shown that a topological space is uniformizable iff it is completely regular.

Clearly, every pseudometric space is uniformizable.  The converse is true if the space has a countable basis.  Pushing this idea further, one can show that a uniformizable space is metrizable iff it is separating (or Hausdorff) and has a countable basis.

Let $X$, $\mathcal{T}$, and $\mathcal{U}$ be defined as above.  Then $X$ is said to be \emph{completely uniformizable} if $\mathcal{U}$ is a complete uniformity.

Every paracompact space is completely uniformizable.  Every completely uniformizable space is completely regular, and hence uniformizable.
%%%%%
%%%%%
\end{document}
