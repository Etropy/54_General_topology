\documentclass[12pt]{article}
\usepackage{pmmeta}
\pmcanonicalname{ContinuousImagesOfPathConnectedSpacesArePathConnected}
\pmcreated{2013-03-22 15:52:38}
\pmmodified{2013-03-22 15:52:38}
\pmowner{mps}{409}
\pmmodifier{mps}{409}
\pmtitle{continuous images of path connected spaces are path connected}
\pmrecord{6}{37874}
\pmprivacy{1}
\pmauthor{mps}{409}
\pmtype{Result}
\pmcomment{trigger rebuild}
\pmclassification{msc}{54D05}

% this is the default PlanetMath preamble.  as your knowledge
% of TeX increases, you will probably want to edit this, but
% it should be fine as is for beginners.

% almost certainly you want these
\usepackage{amssymb}
\usepackage{amsmath}
\usepackage{amsfonts}

% used for TeXing text within eps files
%\usepackage{psfrag}
% need this for including graphics (\includegraphics)
%\usepackage{graphicx}
% for neatly defining theorems and propositions
\usepackage{amsthm}
% making logically defined graphics
%%%\usepackage{xypic}

% there are many more packages, add them here as you need them

% define commands here
\newtheorem*{proposition*}{Proposition}
\newcommand{\fm}[1]{{\it #1}}
\begin{document}
\begin{proposition*}
The continuous image of a path connected space is path connected.
\end{proposition*}

\begin{proof} 
Let \fm{X} be a path connected space, and suppose \fm{f} is a
continuous surjection whose domain is \fm{X}.  Let \fm{a} and \fm{b}
be points in the image of \fm{f}.  Each has at least one preimage in
\fm{X}, and by the path connectedness of \fm{X}, there is a path in
\fm{X} from a preimage of \fm{a} to a preimage of \fm{b}.  Applying
\fm{f} to this path yields a path in the image of \fm{f} from \fm{a}
to \fm{b}.
\end{proof}

\PMlinkescapeword{between}
\PMlinkescapeword{connected}
\PMlinkescapeword{connectedness}
\PMlinkescapeword{path}

%%%%%
%%%%%
\end{document}
