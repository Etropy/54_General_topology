\documentclass[12pt]{article}
\usepackage{pmmeta}
\pmcanonicalname{ProductTopologyAndSubspaceTopology}
\pmcreated{2013-03-22 15:35:33}
\pmmodified{2013-03-22 15:35:33}
\pmowner{matte}{1858}
\pmmodifier{matte}{1858}
\pmtitle{product topology and subspace topology}
\pmrecord{6}{37504}
\pmprivacy{1}
\pmauthor{matte}{1858}
\pmtype{Theorem}
\pmcomment{trigger rebuild}
\pmclassification{msc}{54B10}

\endmetadata

% this is the default PlanetMath preamble.  as your knowledge
% of TeX increases, you will probably want to edit this, but
% it should be fine as is for beginners.

% almost certainly you want these
\usepackage{amssymb}
\usepackage{amsmath}
\usepackage{amsfonts}
\usepackage{amsthm}

\usepackage{mathrsfs}

% used for TeXing text within eps files
%\usepackage{psfrag}
% need this for including graphics (\includegraphics)
%\usepackage{graphicx}
% for neatly defining theorems and propositions
%
% making logically defined graphics
%%%\usepackage{xypic}

% there are many more packages, add them here as you need them

% define commands here

\newcommand{\sR}[0]{\mathbb{R}}
\newcommand{\sC}[0]{\mathbb{C}}
\newcommand{\sN}[0]{\mathbb{N}}
\newcommand{\sZ}[0]{\mathbb{Z}}

 \usepackage{bbm}
 \newcommand{\Z}{\mathbbmss{Z}}
 \newcommand{\C}{\mathbbmss{C}}
 \newcommand{\F}{\mathbbmss{F}}
 \newcommand{\R}{\mathbbmss{R}}
 \newcommand{\Q}{\mathbbmss{Q}}



\newcommand*{\norm}[1]{\lVert #1 \rVert}
\newcommand*{\abs}[1]{| #1 |}



\newtheorem{thm}{Theorem}
\newtheorem{defn}{Definition}
\newtheorem{prop}{Proposition}
\newtheorem{lemma}{Lemma}
\newtheorem{cor}{Corollary}
\begin{document}
Let $X_\alpha$ with $\alpha\in A$ be a collection of topological spaces,
and let $Z_\alpha\subseteq X_\alpha$ be subsets. Let
$$
  X=\prod_{\alpha} X_\alpha
$$
and
$$
  Z = \prod_{\alpha} Z_\alpha.
$$
In other words, $z\in Z$ means that $z$ is a function 
$z\colon A\to \cup_\alpha Z_\alpha$
such that $z(\alpha)\in Z_\alpha$ for each $\alpha$. Thus, $z\in X$ and 
we have
$$
   Z\subseteq X
$$
as sets. 

\begin{thm} The product topology of $Z$ coincides with the subspace topology induced by $X$. 
\end{thm}

\begin{proof}
Let us denote by $\tau_X$ and $\tau_Z$ the product topologies for $X$ and $Z$, respectively. 
Also, let 
$$
  \pi_{X,\alpha}\colon X\to X_\alpha, \quad   \pi_{Z,\alpha}\colon Z\to Z_\alpha
$$
be the canonical projections defined for $X$ and $Z$.
The \PMlinkname{subbases}{Subbasis} for $X$ and $Z$ are given by
\begin{eqnarray*}
\beta_X &=& \{ \pi_{X,\alpha}^{-1}(U) : \alpha \in A, U\in \tau(X_\alpha) \}, \\
\beta_Z &=& \{ \pi_{Z,\alpha}^{-1}(U) : \alpha \in A, U\in \tau(Z_\alpha) \}, 
\end{eqnarray*}
where $\tau(X_\alpha)$ is the topology of $X_\alpha$ and $\tau(Z_\alpha)$ is the
subspace topology of $Z_\alpha\subseteq X_\alpha$. 
The claim follows as
$$
  \beta_Z = \{ B\cap Z : B\in \beta_X \}.
$$
\end{proof}
%%%%%
%%%%%
\end{document}
