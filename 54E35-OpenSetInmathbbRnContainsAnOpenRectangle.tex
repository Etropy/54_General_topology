\documentclass[12pt]{article}
\usepackage{pmmeta}
\pmcanonicalname{OpenSetInmathbbRnContainsAnOpenRectangle}
\pmcreated{2013-03-22 14:07:46}
\pmmodified{2013-03-22 14:07:46}
\pmowner{matte}{1858}
\pmmodifier{matte}{1858}
\pmtitle{open set in $\mathbb{R}^n$ contains an open rectangle}
\pmrecord{5}{35537}
\pmprivacy{1}
\pmauthor{matte}{1858}
\pmtype{Theorem}
\pmcomment{trigger rebuild}
\pmclassification{msc}{54E35}
\pmrelated{Interval}

% this is the default PlanetMath preamble.  as your knowledge
% of TeX increases, you will probably want to edit this, but
% it should be fine as is for beginners.

% almost certainly you want these
\usepackage{amssymb}
\usepackage{amsmath}
\usepackage{amsfonts}

% used for TeXing text within eps files
%\usepackage{psfrag}
% need this for including graphics (\includegraphics)
%\usepackage{graphicx}
% for neatly defining theorems and propositions
%\usepackage{amsthm}
% making logically defined graphics
%%%\usepackage{xypic}

% there are many more packages, add them here as you need them

% define commands here

\newcommand{\sR}[0]{\mathbb{R}}
\newcommand{\sC}[0]{\mathbb{C}}
\newcommand{\sN}[0]{\mathbb{N}}
\newcommand{\sZ}[0]{\mathbb{Z}}

% The below lines should work as the command
% \renewcommand{\bibname}{References}
% without creating havoc when rendering an entry in 
% the page-image mode.
\makeatletter
\@ifundefined{bibname}{}{\renewcommand{\bibname}{References}}
\makeatother

\newcommand*{\norm}[1]{\lVert #1 \rVert}
\newcommand*{\abs}[1]{| #1 |}
\begin{document}
{\bf Theorem} Suppose $\sR^n$ is
 equipped with the usual topology induced by the open balls of the
 Euclidean metric.
 Then, if $U$ is a non-empty open set in $\sR^n$, there
 exist real numbers $a_i, b_i$ for $i=1,\ldots, n$ such that
 $a_i<b_i$ and $[a_1,b_1]\times \cdots \times [a_n, b_n]$ is a subset of $U$.
 
 {\bf Proof.} 
 Since $U$ is non-empty, there exists some point $x$
 in $U$. Further, since $U$ is a topological space, $x$ is contained in
 some open set. Since the topology has a basis consisting of
 open balls, there exists a $y\in U$ and $\varepsilon >0$ such that $x$
 is contained in the open ball $B(y,\varepsilon)$.
 Let us now set $a_i=y_i - \frac{\varepsilon}{2\sqrt{n}}$ and
 $b_i=y_i + \frac{\varepsilon}{2\sqrt{n}}$
 for all $i=1,\ldots, n$.
 Then $D=[a_1,b_1]\times \cdots \times [a_n, b_n]$ can be
 parametrized as
 $$D=\{y+(\lambda_1,\ldots, \lambda_n) \frac{\varepsilon}{2\sqrt{n}} \mid \lambda_i\in[-1,1]\,\mbox{for all}\, i=1,\ldots, n\}.$$
 For an arbitrary point in $D$, we have
 \begin{eqnarray*}
 |y+(\lambda_1,\ldots, \lambda_n) \frac{\varepsilon}{2\sqrt{n}}-y| &=& |(\lambda_1,\ldots, \lambda_n) \frac{\varepsilon}{2\sqrt{n}}| \\
 &=& \frac{\varepsilon}{2\sqrt{n}} \sqrt{\lambda_1^2 + \cdots + \lambda_n^2} \\
 &\le& \frac{\varepsilon}{2} < \varepsilon,
 \end{eqnarray*}
 so $D\subset B(y,\epsilon)\subset U$, and the claim follows. $\Box$
%%%%%
%%%%%
\end{document}
