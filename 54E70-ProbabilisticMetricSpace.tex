\documentclass[12pt]{article}
\usepackage{pmmeta}
\pmcanonicalname{ProbabilisticMetricSpace}
\pmcreated{2013-03-22 16:49:38}
\pmmodified{2013-03-22 16:49:38}
\pmowner{CWoo}{3771}
\pmmodifier{CWoo}{3771}
\pmtitle{probabilistic metric space}
\pmrecord{12}{39066}
\pmprivacy{1}
\pmauthor{CWoo}{3771}
\pmtype{Definition}
\pmcomment{trigger rebuild}
\pmclassification{msc}{54E70}
\pmdefines{distance distribution function}
\pmdefines{triangle function}

\endmetadata

\usepackage{amssymb,amscd}
\usepackage{amsmath}
\usepackage{amsfonts}

% used for TeXing text within eps files
%\usepackage{psfrag}
% need this for including graphics (\includegraphics)
%\usepackage{graphicx}
% for neatly defining theorems and propositions
\usepackage{amsthm}
% making logically defined graphics
%%\usepackage{xypic}
\usepackage{pst-plot}
\usepackage{psfrag}

% define commands here
\newtheorem{prop}{Proposition}
\newtheorem{thm}{Theorem}
\newtheorem{ex}{Example}
\newcommand{\real}{\mathbb{R}}
\begin{document}
Recall that a metric space is a set $X$ equipped with a \emph{distance function} $d:X\times X\to [0,\infty)$, such that 
\begin{enumerate}
\item $d(a,b)=0$ iff $a=b$,
\item $d(a,b)=d(b,a)$, and
\item $d(a,c)\le d(a,b)+d(b,c)$.
\end{enumerate}
In some real life situations, distance between two points may not be definite.  When this happens, the distance function $d$ may be replaced by a more general function $F$ which takes any pair of points $(a,b)$ to a distribution function $F_{(a,b)}$.  Before precisely describing how this works, we first look at the properties of these $F_{(a,b)}$ should have, and how one translates the triangle inequality in this more general setting.

\textbf{distance distribution functions}.  Since we are dealing with the distance between $a$ and $b$, the distribution function $F_{(a,b)}$ must have the property that $F_{(a,b)}(0)=0$.  Any distribution function $F$ such that $F(0)=0$ is called a \emph{distance distribution function}.  The set of all distance distribution functions is denoted by $\Delta^+$.  For example, for any $r\ge 0$, the step functions defined by 
\begin{eqnarray*}
e_r(x) &=& \left\{ \begin {array}{ll}
0 & \mbox{when}\,\, x\le r, \\
1 & \mbox{otherwise} \\ \end{array} \right.
\end{eqnarray*}
are distance distribution functions.

In addition to $F_{(a,b)}$ being a distance distribution function, we need that $F_{(a,b)}=e_0$ iff $a=b$ and $F_{(a,b)}=F_{(b,a)}$.  These two conditions correspond to the first two conditions on $d$.

\textbf{triangle functions}.  Finally, we need to generalize the binary operation $+$ so it works on the set of distance distribution functions.  Clearly, ordinary addition won't work as the sum of two distribution functions is no longer a distribution function.  \v{S}erstnev developed what is called a \emph{triangle function} that will do the trick.  

First, partial order $\Delta^+$ by $F\le G$ iff $F(x)\le G(x)$ for all $x\in \mathbb{R}$.  It is not hard to see that $e_x\le e_y$ iff $y\le x$ and that $e_0$ is the top element of $\Delta^+$.  From the poset $\Delta^+$, call a binary operator $\tau$ on $\Delta^+$ a \emph{triangle function} if $\tau$ turns $\Delta^+$ into a \PMlinkname{partially ordered}{PartiallyOrderedGroup} commutative monoid with $e_0$ serving as the identity element.  Spelling this out, for any $F,G,H\in \Delta^+$, we have
\begin{itemize}
\item $F\tau G = G\tau F$,
\item $(F\tau G)\tau H = F \tau (G\tau H)$,
\item $F\tau e_0 = e_0 \tau F = F$, and
\item if $G\le H$, then $F\tau G\le F\tau H$,
\end{itemize}
where $F\tau G$ means $\tau(F,G)$.  For example, $F\tau G=F\cdot G$, $F\tau G=\min(F,G)$ are two triangle functions.  In fact, since $F\tau G\le F\tau e_0=F$ and $F\tau G\le G$ similarly, we have $F\tau G\le \min(F,G)$ for any triangle function $\tau$.

With this, we are ready for our main definition:

\textbf{Definition}.  A \emph{probabilistic metric space} is a (non-empty) set $X$, equipped with a function $F:X\times X\to \Delta^+$, where $\Delta^+$ is the set of distance distribution functions on which a triangle function $\tau$ is defined, such that
\begin{enumerate}
\item $F_{(a,b)}=e_0$ iff $a=b$, where $F_{(a,b)}:=F(a,b)$,
\item $F_{(a,b)}=F_{(b,a)}$, and 
\item $F_{(a,c)}\ge F_{(a,b)}\tau F_{(b,c)}$.
\end{enumerate}

Given a metric space $(X,d)$, if we can find a triangle function $\tau$ such that $e_x\tau e_y= e_{x+y}$, then $(X,F)$ with $F_{(a,b)}:=e_{d(a,b)}$ is a probabilistic metric space.

\begin{thebibliography}{9}
\bibitem{ss} B. Schweizer, A. Sklar, \emph{Probabilistic Metric Spaces}, Elsevier Science Publishing Company, (1983).
\bibitem{s} A. N. \v{S}erstnev, \emph{Random normed spaces: problems of completeness}, Kazan. Gos. Univ. U\v{c}en. Zap. 122, 3-20, (1962).
\end{thebibliography}
%%%%%
%%%%%
\end{document}
