\documentclass[12pt]{article}
\usepackage{pmmeta}
\pmcanonicalname{Tangle}
\pmcreated{2013-03-22 18:16:41}
\pmmodified{2013-03-22 18:16:41}
\pmowner{apollonius}{16438}
\pmmodifier{apollonius}{16438}
\pmtitle{tangle}
\pmrecord{5}{40884}
\pmprivacy{1}
\pmauthor{apollonius}{16438}
\pmtype{Definition}
\pmcomment{trigger rebuild}
\pmclassification{msc}{54C25}
\pmrelated{Knot}
\pmrelated{Link}
\pmrelated{Braid}

% this is the default PlanetMath preamble.  as your knowledge
% of TeX increases, you will probably want to edit this, but
% it should be fine as is for beginners.

% almost certainly you want these
\usepackage{amssymb}
\usepackage{amsmath}
\usepackage{amsfonts}

% used for TeXing text within eps files
%\usepackage{psfrag}
% need this for including graphics (\includegraphics)
%\usepackage{graphicx}
% for neatly defining theorems and propositions
%\usepackage{amsthm}
% making logically defined graphics
%%%\usepackage{xypic}

% there are many more packages, add them here as you need them

% define commands here

\begin{document}
A tangle is a $1$-manifold, i.e. a disjoint union of arcs and circles, embedded in $(0,1)^{2}\times[0,1]$. The boundary of a tangle is contained in $(0,1)^{2}\times\{0,1\}$. Two tangles are considered equivalent if and only if they are ambient isotopic relative to their boundaries. Combinatorially, tangles can be understood as tangle diagrams. Any two tangle diagrams which represent the same tangle can be connected by Reidemeister moves. This is the content of a slight generalization of Reidemeister's theorem. Algebraically, tangles form the morphisms of a tortile monoidal category. This is a corollary of Shum's theorem. Specifically, they form the tortile monoidal category generated by a self-dual,unframed object.
%%%%%
%%%%%
\end{document}
