\documentclass[12pt]{article}
\usepackage{pmmeta}
\pmcanonicalname{TopologyViaConvergingNets}
\pmcreated{2013-03-22 17:14:27}
\pmmodified{2013-03-22 17:14:27}
\pmowner{CWoo}{3771}
\pmmodifier{CWoo}{3771}
\pmtitle{topology via converging nets}
\pmrecord{10}{39570}
\pmprivacy{1}
\pmauthor{CWoo}{3771}
\pmtype{Definition}
\pmcomment{trigger rebuild}
\pmclassification{msc}{54A20}

\endmetadata

\usepackage{amssymb,amscd}
\usepackage{amsmath}
\usepackage{amsfonts}
\usepackage{mathrsfs}

% used for TeXing text within eps files
%\usepackage{psfrag}
% need this for including graphics (\includegraphics)
%\usepackage{graphicx}
% for neatly defining theorems and propositions
\usepackage{amsthm}
% making logically defined graphics
%%\usepackage{xypic}
\usepackage{pst-plot}
\usepackage{psfrag}

% define commands here
\newtheorem{prop}{Proposition}
\newtheorem{thm}{Theorem}
\newtheorem{ex}{Example}
\newcommand{\real}{\mathbb{R}}
\newcommand{\pdiff}[2]{\frac{\partial #1}{\partial #2}}
\newcommand{\mpdiff}[3]{\frac{\partial^#1 #2}{\partial #3^#1}}
\begin{document}
Given a topological space $X$, one can define the concept of convergence of a sequence, and more generally, the convergence of a net.  Conversely, given a set $X$, a class of nets, and a suitable definition of ``convergence'' of a net, we can topologize $X$.  The procedure is done as follows: 
\begin{quote}
Let $C$ be the class of all pairs of the form $(x,y)$ where $x$ is a net in $X$ and $y$ is an element of $X$.  For any subset $U$ of $X$ with $y\in U$, we say that a net $x$ \emph{converges} to $y$ \emph{with respect to} $U$ if $x$ is eventually in $U$.  We denote this by $x\to_U y$.  Let $$\mathcal{T}:=\lbrace U\subseteq X\mid (x,y)\in C\mbox{ and }y\in U\mbox{ imply }x\to_U y\rbrace.$$
Then $\mathcal{T}$ is a topology on $X$.
\begin{proof}
Clearly $x\to_X y$ for any pair $(x,y)\in C$.  In addition, $x\to_{\varnothing} y$ is vacuously true.  For any $U,V\in \mathcal{T}$, we want to show that $W:=U\cap V\in \mathcal{T}$.  Since $x$ is eventually in $U$ and $V$, there are $i,j\in D$ (where $D$ is the domain of $x$), such that $x_r\in U$ and $x_s\in V$ for all $r\ge i$ and $s\ge j$.  Since $D$ is directed, there is a $k\in D$ such that $k\ge i$ and $k\ge j$.  It is clear that $x_k\in W$ and that any $t\ge k$ we have that $x_t\in W$ as well.  Next, if $U_{\alpha}$ are sets in $\mathcal{T}$, we want to show their union $U:=\bigcup \lbrace U_{\alpha}\rbrace$ is also in $\mathcal{T}$.  If $y$ is a point in $U$ then $y$ is a point in some $U_{\alpha}$.  Since $(x,y)\in C$ with $x$ is eventually in $U_{\alpha}$, we have that $x$ is eventually in $U$ as well.
\end{proof}
\end{quote}

\textbf{Remark}.  The above can be generalized.  In fact, if the class of pairs $(x,y)$ satisfies some ``axioms'' that are commonly found as properties of convergence, then $X$ can be topologized.  Specifically, let $X$ be a set and $C$ again be the class of all pairs $(x,y)$ as described above.  A subclass $\mathcal{C}$ of $C$ is called a \emph{convergence class} if the following conditions are satisfied
\begin{enumerate}
\item $x$ is a constant net with value $y\in X$, then $(x,y)\in \mathcal{C}$
\item $(x,y)\in \mathcal{C}$ implies $(z,y)\in \mathcal{C}$ for any subnet $z$ of $x$
\item if every subnet $z$ of a net $x$ has a subnet $t$ with $(t,y)\in \mathcal{C}$, then $(x,y)\in \mathcal{C}$
\item suppose $(x,y)\in \mathcal{C}$ with $D=\operatorname{dom}(x)$, and for each $i\in D$, we have that $(z_i,x_i)\in \mathcal{C}$, with $D_i=\operatorname{dom}(z_i)$.  Then $(z,x)\in \mathcal{C}$, where $z$ is the net whose domain is $D\times F$ with $F:=\prod \lbrace D_i \mid i\in D\rbrace$, given by $z(i,f)=(i,f(i))$.
\end{enumerate}
If $(x,y)\in \mathcal{C}$, we write $x\to y$ or $\lim_D x=y$.  The last condition can then be visualized as
$$
\begin{array}{cccccccccccccccccccc}
       & \vdots     & \vdots  & \vdots     & \vdots  & \vdots     &         &          &   & &               
\ddots &            &         &            &         &            &         &          &   \\
\cdots & z_{ia}     & \vdots  & z_{jf}     & \vdots  & z_{kp}     & \cdots  &          &   & &               
       & z_{if(i)}  &         &            &         &            &         &          &   \\
\cdots & \vdots     & \vdots  & \vdots     & \vdots  & \vdots     & \cdots  &          &   & &               
       &            & \ddots  &            &         &            &         &          &   \\
\cdots & z_{ib}     & \vdots  & z_{jg}     & \vdots  & z_{kq}     & \cdots  &          &   & &               
       &            &         & z_{jf(j)}  &         &            &         &          &   \\
\cdots & \vdots     & \vdots  & \vdots     & \vdots  & \vdots     & \cdots  &          &   & \Rightarrow &
       &            &         &            & \ddots  &            &         &          &   \\
\cdots & z_{ic}     & \vdots  & z_{jh}     & \vdots  & z_{kr}     & \cdots  &          &   & &               
       &            &         &            &         & z_{kf(k)}  &         &          &   \\
\cdots & \vdots     & \vdots  & \vdots     & \vdots  & \vdots     & \cdots  &          &   & &               
       &            &         &            &         &            & \ddots  &          &   \\
\cdots & \downarrow & \vdots  & \downarrow & \vdots  & \downarrow & \cdots  &          &   & &               
       &            &         &            &         &            &         & \searrow &   \\
\cdots & x_i        & \cdots  & x_j        & \cdots  & x_k        & \cdots  & \to      & y & &               
       &            &         &            &         &            &         &          & y, 
\end{array}
$$
which is reminiscent of Cantor's diagonal argument.

Now, for any subset $A$ of $X$, we define $A^c$ to be the subset of $X$ consisting of all points $y\in X$ such that there is a net $x$ in $A$ with $x\to y$.  It can be shown that $^c$ is a closure operator, which induces a topology $\mathcal{T}_{\mathcal{C}}$ on $X$.  Furthermore, under this induced topology, the notion of converging nets (as defined by the topology) is exactly the same as the notion of convergence described by the convergence class $\mathcal{C}$.

In addition, it may be shown that there is a one-to-one correspondence between the topologies and the convergence classes on the set $X$.  The correspondence is order reversing in the sense that if $\mathcal{C}_1\subseteq \mathcal{C}_2$ as convergent classes, then $\mathcal{T}_{\mathcal{C}_2}\subseteq \mathcal{T}_{\mathcal{C}_1}$ as topologies.
%%%%%
%%%%%
\end{document}
