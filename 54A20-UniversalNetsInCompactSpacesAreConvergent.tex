\documentclass[12pt]{article}
\usepackage{pmmeta}
\pmcanonicalname{UniversalNetsInCompactSpacesAreConvergent}
\pmcreated{2013-03-22 17:31:29}
\pmmodified{2013-03-22 17:31:29}
\pmowner{asteroid}{17536}
\pmmodifier{asteroid}{17536}
\pmtitle{universal nets in compact spaces are convergent}
\pmrecord{4}{39918}
\pmprivacy{1}
\pmauthor{asteroid}{17536}
\pmtype{Theorem}
\pmcomment{trigger rebuild}
\pmclassification{msc}{54A20}

% this is the default PlanetMath preamble.  as your knowledge
% of TeX increases, you will probably want to edit this, but
% it should be fine as is for beginners.

% almost certainly you want these
\usepackage{amssymb}
\usepackage{amsmath}
\usepackage{amsfonts}

% used for TeXing text within eps files
%\usepackage{psfrag}
% need this for including graphics (\includegraphics)
%\usepackage{graphicx}
% for neatly defining theorems and propositions
%\usepackage{amsthm}
% making logically defined graphics
%%%\usepackage{xypic}

% there are many more packages, add them here as you need them

% define commands here

\begin{document}
{\bf Theorem -} A universal net $ (x_{\alpha})_{\alpha \in \mathcal{A}}$ in a compact space $ X$ is convergent.

{\bf Proof :} Suppose by contradiction that $(x_{\alpha})_{\alpha \in \mathcal{A}}$ was not convergent. Then for every $x \in X$ we would find neighborhoods $U_x$ such that
\[
\forall_{\alpha \in \mathcal{A}}\;\;\; \exists_{\alpha \leq \alpha_0} \;\;\; x_{\alpha_0} \notin U_x
\]

The collection of all this neighborhoods cover $X$, and as $X$ is compact, a finite number
$U_{x_1}, U_{x_2}, \dots, U_{x_n}$ also cover $X$.

The net $(x_{\alpha})_{\alpha \in \mathcal{A}}$ is not eventually in $U_{x_k}$ so it must be eventually in $X-U_{x_k}$ (because it is a \PMlinkescapetext{universal} net). Therefore we can find $\alpha_k \in \mathcal{A}$ such that
\[ \forall_{\alpha_k \leq \alpha} \;\;\; x_{\alpha} \in X-U_{x_k} \]

Because we have a finite number $\alpha_1, \alpha_2 \dots, \alpha_n \in \mathcal{A}$ we can find $\gamma \in \mathcal{A}$ such that $\alpha_k \leq \gamma$ for each $1 \leq k \leq n$. 

Then $x_{\gamma} \in X-U_{x_k}$ for all $k$, i.e. $x_{\gamma} \notin U_{x_k}$ for all $k$. But $U_{x_1}, U_{x_2}, \dots, U_{x_n}$ cover $X$ and thus we have a contradiction. $\square$
%%%%%
%%%%%
\end{document}
