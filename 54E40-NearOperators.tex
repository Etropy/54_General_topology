\documentclass[12pt]{article}
\usepackage{pmmeta}
\pmcanonicalname{NearOperators}
\pmcreated{2013-03-22 14:03:21}
\pmmodified{2013-03-22 14:03:21}
\pmowner{mathcam}{2727}
\pmmodifier{mathcam}{2727}
\pmtitle{near operators}
\pmrecord{13}{35411}
\pmprivacy{1}
\pmauthor{mathcam}{2727}
\pmtype{Topic}
\pmcomment{trigger rebuild}
\pmclassification{msc}{54E40}
\pmsynonym{Campanato theory of near operators}{NearOperators}
%\pmkeywords{nearness}
%\pmkeywords{Lax-Milgram}
%\pmkeywords{Banach}
%\pmkeywords{fixed point}
%\pmkeywords{Riesz}
\pmdefines{perturbation}
\pmdefines{small perturbation}
\pmdefines{near operator}

% this is the default PlanetMath preamble.  as your knowledge
% of TeX increases, you will probably want to edit this, but
% it should be fine as is for beginners.

% almost certainly you want these
\usepackage{amssymb}
\usepackage{amsmath}
\usepackage{amsthm}
\usepackage{amsfonts}

% used for TeXing text within eps files
%\usepackage{psfrag}
% need this for including graphics (\includegraphics)
%\usepackage{graphicx}
% for neatly defining theorems and propositions
%\usepackage{amsthm}
% making logically defined graphics
%%%\usepackage{xypic}

% there are many more packages, add them here as you need them

% define commands here
\newcommand{\headstyle}{\bfseries}
\newlength\thindent
%\settowidth\thindent{{\headstyle No}}
\thindent=\parindent

\newlength\thtopskip
\setlength\thtopskip{4pt plus 2pt minus 2pt}
\newlength\thbotskip
\setlength\thbotskip{4pt plus 2pt minus 2pt}
\newtheoremstyle{normale}{\thtopskip}{\thbotskip}{\slshape}{-\thindent}{\headstyle}{}{.5em}{}
\newtheoremstyle{liscio}{\thtopskip}{\thbotskip}{\upshape}{-\thindent}{\headstyle}{}{.5em}{}

\theoremstyle{normale}
\newtheorem{theorem}{Theorem}[section]
\newtheorem{corollary}[theorem]{Corollary}
\newtheorem{lemma}[theorem]{Lemma}

\theoremstyle{liscio}
\newtheorem{remark}[theorem]{Remark}
\newtheorem{definition}[theorem]{Definition}

\providecommand{\norm}[1]{\lVert#1\rVert}
\begin{document}
\tableofcontents

\section{Perturbations and small perturbations: definitions and
some results}

We start our discussion on the Campanato theory of near operators
with some preliminary tools.

Let $X, Y$ be two sets and let a metric $d$ be defined on $Y$. If
$F\colon X\to Y$ is an injective map, we can define a metric on $X$ by
putting
\[
d_F(x', x'') = d(F(x'),F(x'')).
\]

Indeed, $d_F$ is zero if and only if $x'=x''$ (since $F$ is
injective); $d_F$ is obviously symmetric and the triangle
inequality follows from the triangle inequality of $d$.

Moreover, if $F(X)$ is a complete subspace of $Y$, then $X$ is
complete with respect to the metric $d_F$.

Indeed, let $(u_n)$ be a Cauchy sequence in $X$. By definition of
$d$, then $(F(u_n))$ is a Cauchy sequence in $Y$, and in
particular in $F(X)$, which is complete. Thus, there exists $y_0
= F(x_0) \in F(X)$ which is limit of the sequence $(F(u_n))$.
$x_0$ is the limit of $(x_n)$ in $(X, d_F)$, which completes the
proof.

A particular case of the previous statement is when $F$ is onto
(and thus a bijection) and $(Y,d)$ is complete.

Similarly, if $F(X)$ is compact in $Y$, then $X$ is compact with
the metric $d_F$.

\begin{definition}
Let $X$ be a set and $Y$ be a metric space. Let $F, G$ be two maps
from $X$ to $Y$. We say that $G$ is a perturbation of $F$ if there
exist a constant $k>0$ such that for each $x', x'' \in X$ one has:
\[
d(G(x'), G(x'')) \le k d(F(x'), F(x''))
\]
\end{definition}

\begin{remark}\label{gcont}
In particular, if $F$ is injective then $G$ is a perturbation of
$F$ if $G$ is uniformly continuous with respect to the metric induced on
$X$ by $F$.
\end{remark}

\begin{definition}
In the same hypothesis as in the previous definition, we say that
$G$ is a small perturbation of $F$ if it is a perturbation of
constant $k<1$.
\end{definition}

We can now prove this generalization of the Banach-Caccioppoli
fixed point theorem:

\begin{theorem}\label{fixptgen}
Let $X$ be a set and $(Y, d)$ be a complete metric space. Let $F,
G$ be two mappings from $X$ to $Y$ such that:
\begin{enumerate}
\item \label{bij} $F$ is bijective;
\item \label{smp} $G$ is a small perturbation of $F$.
\end{enumerate}

Then, there exists a unique $u \in X$ such that $G(u)=F(u)$
\begin{proof}
The hypothesis (\ref{bij}) ensures that the metric space $(X,
d_F)$ is complete. If we now consider the function $T:X \to X$
defined by
\[
T(x) = F^{-1}(G(x))
\]
we note that, by (\ref{smp}), we have
\[
d(G(x'), G(x'')) \le k d(F(x'), F(x''))
\]
where $k \in (0,1)$ is the constant of the small perturbation;
note that, by the definition of $d_F$ and applying $F \circ
F^{-1}$ to the first side, the last equation can be rewritten as
\[
d_F(T(x'), T(x'')) \le k d_F(x', x'');
\]
in other words, since $k < 1$, $T$ is a contraction in the
complete metric space $(X, d_F)$; therefore (by the classical
Banach-Caccioppoli fixed point theorem) $T$ has a unique fixed
point: there exist $u \in X$ such that $T(u) = u$; by definition
of $T$ this is equivalent to $G(u) = F(u)$, and the proof is hence
complete.
\end{proof}
\end{theorem}

\begin{remark}
The hypothesis of the theorem can be generalized as such: let $X$
be a set and $Y$ a metric space (not necessarily complete); let
$F, G$ be two mappings from $X$ to $Y$ such that $F$ is injective,
$F(X)$ is complete and $G(X) \subseteq F(X)$; then there exists $u
\in X$ such that $G(u) = F(u)$.

(Apply the theorem using $F(X)$ instead of $Y$ as target space.)
\end{remark}

\begin{remark}
The Banach-Caccioppoli fixed point theorem is obtained when
$X=Y$ and $F$ is the identity.
\end{remark}

We can use theorem~\ref{fixptgen} to prove a result that applies
to perturbations which are not necessarily small (i.e. for which
the constant $k$ can be greater than one). To prove it, we must
assume some supplemental structure on the metric of $Y$: in
particular, we have to assume that the metric $d$ is invariant by
dilations, that is that $d(\alpha y', \alpha y'') = \alpha d (y',
y'')$ for each $y', y'' \in Y$. The most common case of such
a metric is when the metric is deduced from a norm (i.e. when $Y$
is a normed space, and in particular a Banach space). The result
follows immediately:

\begin{corollary}\label{fixm}
Let $X$ be a set and $(Y, d)$ be a complete metric space with
a metric $d$ invariant by dilations. Let $F, G$ be two mappings
from $X$ to $Y$ such that $F$ is bijective and $G$ is
a perturbation of $F$, with constant $K > 0$.

Then, for each $M > K$ there exists a unique $u_M \in X$ such that
$G(u)=M F(u)$
\begin{proof}
The proof is an immediate consequence of theorem~\ref{fixptgen}
given that the map $\tilde G(u) = G(u)/M$ is a small perturbation
of $F$ (a property which is ensured by the dilation invariance of
the metric $d$).
\end{proof}
\end{corollary}

We also have the following
\begin{corollary}\label{fixk}
Let $X$ be a set and $(Y, d)$ be a complete, compact metric space
with a metric $d$ invariant by dilations. Let $F, G$ be two
mappings from $X$ to $Y$ such that $F$ is bijective and $G$ is
a perturbation of $F$, with constant $K > 0$.

Then there exists at least one $u_K \in X$ such that
$G(u_\infty)=K F(u_\infty)$
\begin{proof}
Let $(a_n)$ be a decreasing sequence of real numbers greater than
one, converging to one ($a_n \downarrow 1$) and let $M_n = a_n K$
for each $n \in \mathbb{N}$. We can apply corollary~\ref{fixm} to
each $M_n$, obtaining a sequence $u_n$ of elements of $X$ for
which one has
\begin{equation}\label{gmf}
G(u_n) = M_n F(u_n).
\end{equation}

Since $(X, d_F)$ is compact, there exist a subsequence of $u_n$
which converges to some $u_\infty$; by continuity of $G$ and $F$
we can pass to the limit in \eqref{gmf}, obtaining
\[
G(u_\infty) = K F(u_\infty)
\]
which completes the proof.
\end{proof}
\end{corollary}

\begin{remark}
For theorem~\ref{fixk} we cannot ensure uniqueness of $u_\infty$,
since in general the sequence $u_n$ may change with the choice of
$a_n$, and the limit might be different. So the corollary can only
be applied as an existence theorem.
\end{remark}

\section{Near operators}

We can now introduce the concept of near operators and discuss
some of their properties.

A historical remark: Campanato initially introduced the concept in
Hilbert spaces; subsequently, it was remarked that most of the
theory could more generally be applied to Banach spaces; indeed,
it was also proven that the basic definition can be generalized to
make part of the theory available in the more general environment
of metric vector spaces.

We will here discuss the theory in the case of Banach spaces,
with only a couple of exceptions: to see some of the extra
properties that are available in Hilbert spaces and to discuss
a generalization of the Lax-Milgram theorem to metric vector spaces.

\subsection{Basic definitions and properties}

\begin{definition}
Let $X$ be a set and $Y$ a Banach space. Let $A, B$ be two
operators from $X$ to $Y$. We say that $A$ is near $B$ if and only
if there exist two constants $\alpha > 0$ and $k \in (0, 1)$ such
that, for each $x', x'' \in X$ one has
\[
\norm{B(x') - B(x'') - \alpha ( A(x') - A(x'') ) } \leq
k \norm{B(x') - B(x'')}
\]
\end{definition}

In other words, $A$ is near $B$ if $B - \alpha A$ is a small
perturbation of $B$ for an appropriate value of $\alpha$.

Observe that in general the property is not symmetric: if $A$ is
near $B$, it is not necessarily true that $B$ is near $A$; as we
will briefly see, this can only be proven if $\alpha < 1/2$, or in
the case that $Y$ is a Hilbert space, by using an equivalent
condition that will be discussed later on. Yet it is possible to
define a topology with some interesting properties on the space of
operators, by using the concept of nearness to form a base.

The core point of the nearness between operators is that it allows
us to ``transfer'' many important properties from $B$ to $A$; in
other words, if $B$ satisfies certain properties, and $A$ is near
$B$, then $A$ satisfies the same properties. To prove this, and to
enumerate some of these ``nearness-invariant'' properties, we will
emerge a few important facts.

In what follows, unless differently specified, we will always
assume that $X$ is a set, $Y$ is a Banach space and $A, B$ are two
operators from $X$ to $Y$.

\begin{lemma}\label{control}
If $A$ is near $B$ then there exist two positive constants $M_1,
M_2$ such that
\begin{gather*}
\norm{B(x') - B(x'')} \leq M_1 \norm{A(x') - A(x'')} \\
\norm{A(x') - A(x'')} \leq M_2 \norm{B(x') - B(x'')} \\
\end{gather*}
\begin{proof}
We have:
\begin{multline*}
\norm{B(x')- B(x'')} \leq \\
\leq \norm{B(x') - B(x'') - \alpha ( A(x') - A(x'') ) } +
\alpha \norm{A(x') - A(x'')} \leq \\
\leq k \norm{B(x')- B(x'')} + \alpha \norm{A(x') - A(x'')}
\end{multline*}
and hence
\[
\norm{B(x') - B(x'')} \leq \frac{\alpha}{1-k} \norm{A(x') - A(x'')}
\]
which is the first inequality with $M_1 = \alpha/(1-k)$ (which is
positive since $k < 1$).

But also
\begin{multline*}
\norm{A(x')- A(x'')} \leq \\
\leq \frac{1}{\alpha}\norm{B(x') - B(x'') - \alpha ( A(x') - A(x'') ) } +
\frac{1}{\alpha} \norm{B(x') - B(x'')} \leq \\
\leq \frac{k}{\alpha} \norm{B(x')- B(x'')} + 
\frac{1}{\alpha} \norm{B(x') - B(x'')}
\end{multline*}
and hence
\[
\norm{A(x') - A(x'')} \leq \frac{1+k}{\alpha} \norm{B(x') - B(x'')}
\]
which is the second inequality with $M_2 = (1+k)/\alpha$.
\end{proof}
\end{lemma}

The most important corollary of the previous lemma is the
following
\begin{corollary}
If $A$ is near $B$ then two points of $X$ have the same image
under $A$ if and only if the have the same image under $B$.
\end{corollary}

We can express the previous concept in the following formal way:
for each $y$ in $B(X)$ there exist $z$ in $Y$ such that
$A(B^{-1}(y)) = \{z\}$ and conversely. In yet other words: each
fiber of $A$ is a fiber (for a different point) of $B$, and
conversely.

It is therefore possible to define a map $T_A:B(X) \to Y$ by
putting $T_A(y) = z$; the range of $T_A$ is $A(X)$. Conversely, it
is possible to define $T_B:A(X) \to Y$, by putting $T_B(z) = y$;
the range of $T_B$ is $B(X)$. Both maps are injective and, if
restricted to their respective ranges, one is the inverse of the
other.

Also observe that $T_B$ and $T_A$ are continuous. This follows
from the fact that for each $x \in X$ one has
\[
T_A(B(x)) = A(x), \qquad T_B(A(x)) = B(x)
\]
and that the lemma ensures that given a sequence $(x_n)$ in $X$,
the sequence $(B(x_n))$ converges to $B(x_0)$ if and only if
$(A(x_n))$ converges to $A(x_0)$.

We can now list some invariant properties of operators with
respect to nearness. The properties are given in the form ``if
and only if'' because each operator is near itself (therefore
ensuring the ``only if'' part).

\begin{enumerate}
\item\label{inj} a map is injective if and only if it is near an
injective operator;
\item\label{surj} a map is surjective if and only if it is near a surjective
operator;
\item\label{open} a map is open if and only if it is near an open map;
\item\label{dense} a map has dense range if and only if it is near a map with
dense range.
\end{enumerate}

To prove (\ref{surj}) it is necessary to use
theorem~\ref{fixptgen}.

Another important property that follows from the lemma is that if
there exist $y \in Y$ such that $A^{-1}(y) \cap B^{-1}(y) \ne
\varnothing$, then it is $A^{-1}(y) = B^{-1}(y)$: intersecting
fibers are equal. (Campanato only stated this property for the
case $y = 0$ and called it ```the kernel property''; I prefer to
call it the ``fiber persistence'' property.)

\subsubsection{A topology based on nearness}

In this section we will show that the concept of nearness between
operator can indeed be connected to a topological understanding of
the set of maps from $X$ to $Y$.

\def\M{\mathcal{M}}%
\def\U{\mathcal{U}}%
\def\H{\mathcal{H}}%

Let $\M$ be the set of maps between $X$ and $Y$. For each
$F \in \M$ and for each $k \in (0,1)$ we let $U_k(F)$ the set of
all maps $G \in \M$ such that $F-G$ is a small perturbation of $F$
with constant $k$. In other words, $G \in U_k(F)$ if and only if $G$ is near
$F$ with constants $1, k$.

The set $\U(F) = \{ U_k(F) \mid 0<k<1 \}$ satisfies the axioms of
the set of fundamental neighbourhoods. Indeed:
\begin{enumerate}
\item $F$ belongs to each $U_k(F)$;
\item $U_k(F) \subset U_h(F)$ if and only if $k < h$, and thus the
intersection property of neighbourhoods is trivial;
\item for each $U_k(F)$ there exist $U_h(F)$ such that for each $G
\in U_h(F)$ there exist $U_j(G) \subseteq U_k(F)$.
\end{enumerate}

This last property (permanence of neighbourhoods) is somewhat less
trivial, so we shall now prove it.
\begin{proof}
Let $U_k(F)$ be given.

 Let $U_h(F)$ be another arbitrary neighbourhood of $F$ and let
$G$ be an arbitrary element in it. We then have:
\begin{equation}\label{g}
\norm{F(x') - F(x'') - (G(x')-G(x''))} \le h \norm{F(x') - F(x'')}.
\end{equation}
but also (lemma~\ref{control})
\begin{equation}\label{g}
\norm{(G(x')-G(x''))} \le (1+h) \norm{F(x') - F(x'')}.
\end{equation}

Let also $U_j(G)$ be an arbitrary neighbourhood of $G$ and $H$ an
arbitrary element in it. We then have:
\begin{equation}\label{h}
\norm{G(x') - G(x'') - (H(x')-H(x''))} \le j \norm{G(x') - G(x'')}.
\end{equation}

The nearness between $F$ and $H$ is calculated as such:
\begin{multline}\label{h}
\norm{F(x') - F(x'') - (H(x')-H(x''))} \le \\
\norm{F(x') - F(x'') - (G(x')-G(x''))} +
\norm{G(x') - G(x'') - (H(x')-H(x''))} \le \\
h \norm{F(x') - F(x'')} + j \norm{G(x') - G(x'')} \le
(h + j (1+h)) \norm{F(x') - F(x'')}.
\end{multline}
We then want $h + j (1 + h) \le k$, that is $j \le (k-h)/(1+h)$;
the condition $0<j<1$ is always satisfied on the right side, and
the left side gives us $h<k$.
\end{proof}

It is important to observe that the topology generated this way is
not a Hausdorff topology: indeed, it is not possible to separate
$F$ and $F+y$ (where $F\in\M$ and $y$ is a constant element of~$Y$).
On the other hand, the subset of all maps with with a fixed valued
at a fixed point ($F(x_0) = y_0$) is a Hausdorff subspace.

Another important characteristic of the topology is that the set
$\H$ of invertible operators from $X$ to $Y$ is open in $\M$
(because a map is invertible if and only if it is near an invertible map).
This is not true in the topology of uniform convergence, as is
easily seen by choosing $X = Y = \mathbb{R}$ and the sequence with
generic element $F_n(x) = x^3 - x/n$: the sequence converges (in
the uniform convergence topology) to $F(x) = x^3$, which is
invertible, but none of the $F_n$ is invertible. Hence $F$ is an
element of $\H$ which is not inside $\H$, and $\H$ is not open.

\subsection{Some applications}

As we mentioned in the introduction, the Campanato theory of near
operators allows us to generalize some important theorems; we will
now present some generalizations of the Lax-Milgram theorem, and
a generalization of the Riesz representation theorem.

[TODO]
%%%%%
%%%%%
\end{document}
