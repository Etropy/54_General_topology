\documentclass[12pt]{article}
\usepackage{pmmeta}
\pmcanonicalname{BoundaryOfAClosedSetIsNowhereDense}
\pmcreated{2013-03-22 18:34:01}
\pmmodified{2013-03-22 18:34:01}
\pmowner{neapol1s}{9480}
\pmmodifier{neapol1s}{9480}
\pmtitle{boundary of a closed set is nowhere dense}
\pmrecord{4}{41290}
\pmprivacy{1}
\pmauthor{neapol1s}{9480}
\pmtype{Derivation}
\pmcomment{trigger rebuild}
\pmclassification{msc}{54A99}
%\pmkeywords{nowhere dense}
%\pmkeywords{boundary}

% this is the default PlanetMath preamble.  as your knowledge
% of TeX increases, you will probably want to edit this, but
% it should be fine as is for beginners.

% almost certainly you want these
\usepackage{amssymb}
\usepackage{amsmath}
\usepackage{amsfonts}

% used for TeXing text within eps files
%\usepackage{psfrag}
% need this for including graphics (\includegraphics)
%\usepackage{graphicx}
% for neatly defining theorems and propositions
%\usepackage{amsthm}
% making logically defined graphics
%%%\usepackage{xypic}

% there are many more packages, add them here as you need them

% define commands here

\begin{document}
Let $A$ be closed. In general, the boundary of a set is closed. So it suffices to show that $\partial A$ has empty interior. 

Let $U\subset\partial A$ be open. Since $\partial A\subset \overline{A}=A$, this implies that $U\subset A$. Since $\operatorname{int}(A)$ is the largest open subset of $A$, we must have $U\subset\operatorname{int}(A)$. Therefore $U\subset \partial A \cap \operatorname{int}(A)$. But $\partial A \cap \operatorname{int}(A)=(\overline{A}-\operatorname{int}(A))\cap\operatorname{int}(A)=\emptyset$, so $U=\emptyset$.
%%%%%
%%%%%
\end{document}
