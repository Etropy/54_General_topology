\documentclass[12pt]{article}
\usepackage{pmmeta}
\pmcanonicalname{ThePropertyThatCompactSetsInASpaceAreClosedLiesStrictlyBetweenT1AndT2}
\pmcreated{2013-03-22 17:38:59}
\pmmodified{2013-03-22 17:38:59}
\pmowner{dfeuer}{18434}
\pmmodifier{dfeuer}{18434}
\pmtitle{The property that compact sets in a space are closed lies strictly between T1 and T2}
\pmrecord{5}{40082}
\pmprivacy{1}
\pmauthor{dfeuer}{18434}
\pmtype{Result}
\pmcomment{trigger rebuild}
\pmclassification{msc}{54D30}
\pmclassification{msc}{54D10}
%\pmkeywords{separation axiom}
%\pmkeywords{T1}
%\pmkeywords{T2}
%\pmkeywords{Hausdorff}
%\pmkeywords{compactness}
%\pmkeywords{compact}
\pmrelated{T1Space}
\pmrelated{T2Space}
\pmrelated{Compact}

% this is the default PlanetMath preamble.  as your knowledge
% of TeX increases, you will probably want to edit this, but
% it should be fine as is for beginners.

% almost certainly you want these
\usepackage{amssymb}
\usepackage{amsmath}
\usepackage{amsfonts}

% used for TeXing text within eps files
%\usepackage{psfrag}
% need this for including graphics (\includegraphics)
%\usepackage{graphicx}
% for neatly defining theorems and propositions
%\usepackage{amsthm}
% making logically defined graphics
%%%\usepackage{xypic}

% there are many more packages, add them here as you need them

% define commands here

\begin{document}
If a topological space is Hausdorff ($T_2$), then every compact subset of that space is closed.  If every compact subset of a space is closed, then (since singletons are always compact) then the space is accessible ($T_1$).  There are spaces that are $T_1$ and have compact sets that are not closed, and there are spaces in which compact sets are always closed but that are not $T_2$.

Let $X$ be an infinite set with the finite complement topology.  Singletons are finite, and therefore closed, so $X$ is $T_1$.  Let $S \subset X$, and let $\mathbb{F}$ be an open cover of $S$.  Let $F \in \mathbb{F}$.  Then $X\setminus F$ is finite.  Choosing a member of $\mathbb{F}$ for each remaining element of $S$ shows that $\mathbb{F}$ has a finite subcover.  Thus, every subset of $X$ is compact.  An infinite subset of $X$ will then be compact, but not closed.

Let $Y$ be an uncountable set with the countable complement topology.  No two open sets are disjoint, so $Y$ is not Hausdorff.  Let $C$ be a compact subset of $Y$.  I shall show that $C$ is finite.  Suppose $C$ is infinite, and let $S$ be an infinite sequence in $C$ without any repetitions.  For any natural number $n$, let $U_n$ be all the elements of $C$ except for all the $S_k$, where $k>n$.  Then $U_n$ is open for each $n$, and $\{U_n \mid n \in \mathbb{N}\}$ covers $C$, but has no finite subset that covers $C$, contradicting the fact that $C$ is compact.  This contradiction arose by assuming a compact subset of $Y$ was infinite, all compact subsets of $Y$ are finite.  $Y$ is $T_1$ (singleton sets are countable), so all compact subsets of $Y$ are closed.

These examples were suggested by the person known as Polytope on EFNet's math channel.
%%%%%
%%%%%
\end{document}
