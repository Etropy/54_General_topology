\documentclass[12pt]{article}
\usepackage{pmmeta}
\pmcanonicalname{CompactnessIsPreservedUnderAContinuousMap}
\pmcreated{2013-03-22 13:55:50}
\pmmodified{2013-03-22 13:55:50}
\pmowner{yark}{2760}
\pmmodifier{yark}{2760}
\pmtitle{compactness is preserved under a continuous map}
\pmrecord{13}{34689}
\pmprivacy{1}
\pmauthor{yark}{2760}
\pmtype{Theorem}
\pmcomment{trigger rebuild}
\pmclassification{msc}{54D30}
\pmrelated{ContinuousImageOfACompactSpaceIsCompact}
\pmrelated{ContinuousImageOfACompactSetIsCompact}
\pmrelated{ConnectednessIsPreservedUnderAContinuousMap}

\endmetadata

\usepackage{amssymb}
\usepackage{amsmath}
\usepackage{amsfonts}

\newcommand*{\norm}[1]{\lVert #1 \rVert}
\newcommand*{\abs}[1]{| #1 |}

\def\emptyset{\varnothing}

% The below lines should work as the command
% \renewcommand{\bibname}{References}
% without creating havoc when rendering an entry in 
% the page-image mode.
\makeatletter
\@ifundefined{bibname}{}{\renewcommand{\bibname}{References}}
\makeatother
\begin{document}
\PMlinkescapeword{between}
\PMlinkescapeword{theorem}

{\bf Theorem} \cite{singer, kelley} 
Suppose $f\colon X\to Y$ is a continuous map 
between topological spaces $X$ and $Y$. 
If $X$ is compact and $f$ is surjective, then $Y$ is compact.

The inclusion map $[0,1]\hookrightarrow[0,2)$ shows that the requirement for $f$ to be surjective cannot be omitted.
If $X$ is compact and $f$ is continuous we can always conclude, however, that $f(X)$ is compact, since \PMlinkname{$f\colon X\to f(X)$ is continuous}{IfFcolonXtoYIsContinuousThenFcolonXtoFXIsContinuous}.

\emph{Proof of theorem.} (Following \cite{singer}.)
Suppose $\{V_\alpha \mid \alpha \in I\}$ is an arbitrary
open cover for $f(X)$. Since $f$ is continuous, it follows
that 
$$\{f^{-1} (V_\alpha) \mid \alpha \in I\}$$
is a collection of open sets in $X$. 
Since $A\subseteq f^{-1} f (A)$ for any $A\subseteq X$,
and since the inverse commutes with unions 
(see \PMlinkname{this page}{InverseImage}), 
we have 
\begin{eqnarray*}
X &\subseteq & f^{-1} f(X) \\
  &= & f^{-1} \big( \bigcup_{\alpha \in I} (V_\alpha)\big) \\
  &=& \bigcup_{\alpha \in I} f^{-1}(V_\alpha).
\end{eqnarray*}
Thus $\{ f^{-1}(V_\alpha) \mid \alpha \in I\}$ is an open cover for $X$. 
Since $X$ is compact, there exists a finite subset $J\subseteq I$ 
such that $\{ f^{-1}(V_\alpha) \mid \alpha \in J\}$ is a 
finite open cover for $X$. 
Since $f$ is a surjection, we have $ff^{-1}(A)=A$ for any $A\subseteq Y$
(see \PMlinkname{this page}{InverseImage}). Thus 
\begin{eqnarray*}
f(X) &= & f \big(\bigcup_{i\in J} f^{-1}(V_\alpha)\big)\\
     &= & ff^{-1} \bigcup_{i\in J} f^{-1}(V_\alpha)\\
     &=&  \bigcup_{i\in J} V_\alpha.
\end{eqnarray*}
Thus $\{ V_\alpha \mid \alpha \in J\}$ is an open cover for $f(X)$,
and $f(X)$ is compact.
$\Box$

A shorter proof can be given using the
\PMlinkname{characterization of compactness by the finite intersection
property}{ASpaceIsCompactIfAndOnlyIfTheSpaceHasTheFiniteIntersectionProperty}:

\emph{Shorter proof.}
Suppose $\{A_i\mid i\in I\}$ is a collection of closed
subsets of $Y$ with the finite intersection property.
Then $\{f^{-1}(A_i)\mid i\in I\}$ is a collection of closed subsets of $X$
with the finite intersection property,
because if $F\subseteq I$ is finite then
\[
 \bigcap_{i\in F} f^{-1}(A_i) = f^{-1}\!\left(\,\bigcap_{i\in F} A_i\!\right),
\]
which is nonempty as $f$ is a surjection.
As $X$ is compact, we have
\[
 f^{-1}\left(\,\bigcap_{i\in I} A_i\!\right) = \bigcap_{i\in I} f^{-1}(A_i) \neq \emptyset
\]
and so $\bigcap_{i\in I} A_i\neq \emptyset$.
Therefore $Y$ is compact.
$\Box$

\begin{thebibliography}{9}
 \bibitem{singer}
 I.M. Singer, J.A.Thorpe,
 \emph{Lecture Notes on Elementary Topology and Geometry},
 Springer-Verlag, 1967.
 \bibitem{kelley}
 J.L. Kelley, \emph{General Topology}, D. van Nostrand Company, Inc., 1955.
 \bibitem{jameson} G.J. Jameson, \emph{Topology and Normed Spaces},
 Chapman and Hall, 1974.
\end{thebibliography}
%%%%%
%%%%%
\end{document}
