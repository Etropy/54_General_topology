\documentclass[12pt]{article}
\usepackage{pmmeta}
\pmcanonicalname{PrecompactSet}
\pmcreated{2013-03-22 14:39:59}
\pmmodified{2013-03-22 14:39:59}
\pmowner{matte}{1858}
\pmmodifier{matte}{1858}
\pmtitle{precompact set}
\pmrecord{12}{36264}
\pmprivacy{1}
\pmauthor{matte}{1858}
\pmtype{Definition}
\pmcomment{trigger rebuild}
\pmclassification{msc}{54D45}
\pmsynonym{precompact}{PrecompactSet}
\pmsynonym{relatively compact}{PrecompactSet}

% this is the default PlanetMath preamble.  as your knowledge
% of TeX increases, you will probably want to edit this, but
% it should be fine as is for beginners.

% almost certainly you want these
\usepackage{amssymb}
\usepackage{amsmath}
\usepackage{amsfonts}
\usepackage{amsthm}

% used for TeXing text within eps files
%\usepackage{psfrag}
% need this for including graphics (\includegraphics)
%\usepackage{graphicx}
% for neatly defining theorems and propositions
%
% making logically defined graphics
%%%\usepackage{xypic}

% there are many more packages, add them here as you need them

% define commands here

\newcommand{\sR}[0]{\mathbb{R}}
\newcommand{\sC}[0]{\mathbb{C}}
\newcommand{\sN}[0]{\mathbb{N}}
\newcommand{\sZ}[0]{\mathbb{Z}}

 \usepackage{bbm}
 \newcommand{\Z}{\mathbbmss{Z}}
 \newcommand{\C}{\mathbbmss{C}}
 \newcommand{\R}{\mathbbmss{R}}
 \newcommand{\Q}{\mathbbmss{Q}}



\newcommand*{\norm}[1]{\lVert #1 \rVert}
\newcommand*{\abs}[1]{| #1 |}



\newtheorem{thm}{Theorem}
\newtheorem{defn}{Definition}
\newtheorem{prop}{Proposition}
\newtheorem{lemma}{Lemma}
\newtheorem{cor}{Corollary}
\begin{document}
\begin{defn} A subset in a topological space is 
\emph{precompact} if its closure is compact \cite{lee}.
\end{defn}

For metric spaces, we have the following theorem due to Hausdorff
\cite{cristescu}.

{\bf Theorem} Suppose $K$ is a set in a complete metric space $X$.
Then $K$ relatively compact if and only if for any $\varepsilon>0$
there is a finite \PMlinkname{$\varepsilon$-net}{VarepsilonNet} for $K$.

\subsubsection*{Examples}
\begin{enumerate}
\item In $\sR^n$ every point has a precompact neighborhood.
\item On a manifold, every point has a precompact neighborhood. 
This follows from the previous example, since a homeomorphism 
commutes with the closure operator, and since the continuous image
of a compact set is compact. 
\end{enumerate}

\subsubsection*{Notes}
A synonym is \emph{relatively compact} \cite{cristescu, kreyszig}.

Some authors (notably Bourbaki see \cite{bour}) use precompact differently -  as a synonym for \PMlinkname{totally bounded}{TotallyBounded} (in the generality of topological groups). ``Relatively compact'' is then used to mean ``precompact ''as it is defined here
\begin{thebibliography}{9}
\bibitem{lee} J.M. Lee, \emph{Introduction to Smooth Manifolds},
Graduate Texts in Mathematics series, 218,
Springer-Verlag, 2002.
\bibitem{cristescu} R. Cristescu, \emph{Topological vector spaces},
Noordhoff International Publishing, 1977.
\bibitem{kreyszig} E. Kreyszig,
\emph{Introductory Functional Analysis With Applications},
John Wiley \& Sons, 1978.
\bibitem{bour} N. Bourbaki, \emph{Topological Vector Spaces} Springer-Verlag, 1981
\end{thebibliography}
%%%%%
%%%%%
\end{document}
