\documentclass[12pt]{article}
\usepackage{pmmeta}
\pmcanonicalname{ContinuityAndConvergentNets}
\pmcreated{2013-03-22 18:37:53}
\pmmodified{2013-03-22 18:37:53}
\pmowner{azdbacks4234}{14155}
\pmmodifier{azdbacks4234}{14155}
\pmtitle{continuity and convergent nets}
\pmrecord{5}{41369}
\pmprivacy{1}
\pmauthor{azdbacks4234}{14155}
\pmtype{Theorem}
\pmcomment{trigger rebuild}
\pmclassification{msc}{54A20}
%\pmkeywords{net}
%\pmkeywords{directed set}
%\pmkeywords{continuous function}
%\pmkeywords{convergence}
\pmrelated{Net}
\pmrelated{Continuous}

\endmetadata

%packages
\usepackage{amsmath,mathrsfs,amsfonts,amsthm}
%theorem environments
\theoremstyle{plain}
\newtheorem*{thm*}{Theorem}
\newtheorem*{lem*}{Lemma}
\newtheorem*{cor*}{Corollary}
\newtheorem*{prop*}{Proposition}
%delimiters
\newcommand{\set}[1]{\{#1\}}
\newcommand{\medset}[1]{\big\{#1\big\}}
\newcommand{\bigset}[1]{\bigg\{#1\bigg\}}
\newcommand{\Bigset}[1]{\Bigg\{#1\Bigg\}}
\newcommand{\abs}[1]{\vert#1\vert}
\newcommand{\medabs}[1]{\big\vert#1\big\vert}
\newcommand{\bigabs}[1]{\bigg\vert#1\bigg\vert}
\newcommand{\Bigabs}[1]{\Bigg\vert#1\Bigg\vert}
\newcommand{\norm}[1]{\Vert#1\Vert}
\newcommand{\mednorm}[1]{\big\Vert#1\big\Vert}
\newcommand{\bignorm}[1]{\bigg\Vert#1\bigg\Vert}
\newcommand{\Bignorm}[1]{\Bigg\Vert#1\Bigg\Vert}
\newcommand{\vbrack}[1]{\langle#1\rangle}
\newcommand{\medvbrack}[1]{\big\langle#1\big\rangle}
\newcommand{\bigvbrack}[1]{\bigg\langle#1\bigg\rangle}
\newcommand{\Bigvbrack}[1]{\Bigg\langle#1\Bigg\rangle}
\newcommand{\sbrack}[1]{[#1]}
\newcommand{\medsbrack}[1]{\big[#1\big]}
\newcommand{\bigsbrack}[1]{\bigg[#1\bigg]}
\newcommand{\Bigsbrack}[1]{\Bigg[#1\Bigg]}
%operators
\DeclareMathOperator{\Hom}{Hom}
\DeclareMathOperator{\Tor}{Tor}
\DeclareMathOperator{\Ext}{Ext}
\DeclareMathOperator{\Aut}{Aut}
\DeclareMathOperator{\End}{End}
\DeclareMathOperator{\Inn}{Inn}
\DeclareMathOperator{\lcm}{lcm}
\DeclareMathOperator{\ord}{ord}
\DeclareMathOperator{\rank}{rank}
\DeclareMathOperator{\tr}{tr}
\DeclareMathOperator{\Mat}{Mat}
\DeclareMathOperator{\Gal}{Gal}
\DeclareMathOperator{\GL}{GL}
\DeclareMathOperator{\SL}{SL}
\DeclareMathOperator{\SO}{SO}
\DeclareMathOperator{\ann}{ann}
\DeclareMathOperator{\im}{im}
\DeclareMathOperator{\Char}{char}
\DeclareMathOperator{\Spec}{Spec}
\DeclareMathOperator{\supp}{supp}
\DeclareMathOperator{\diam}{diam}
\DeclareMathOperator{\Ind}{Ind}
\DeclareMathOperator{\vol}{vol}

\begin{document}
\begin{thm*}
Let $X$ and $Y$ be topological spaces. A function $f:X\rightarrow Y$ is continuous at a point $x\in X$ if and only if for each net $(x_\alpha)_{\alpha\in A}$ in $X$ converging to $x$, the net $(f(x_\alpha))_{\alpha\in A}$ converges to $f(x)$.
\end{thm*}
\begin{proof}
If $f$ is continuous, $(x_\alpha)_{\alpha\in A}$ converges to $x$, and $V$ is an open neighborhood of $f(x)$ in $Y$, then $f^{-1}(V)$ is an open neighborhood of $x$ in $X$, so there exists $\alpha_0\in A$ such that $x_\alpha\in f^{-1}(V)$ for $\alpha\geq\alpha_0$. It follows that $f(x_\alpha)\in V$ for $\alpha\geq\alpha_0$, hence that $f(x_\alpha)\rightarrow f(x)$. Conversely, suppose there exists a net $(x_\alpha)_{\alpha\in A}$ in $X$ converging to $x$ such that $(f(x_\alpha))_{\alpha\in A}$ does not converge to $f(x)$, so that, for some open subset $V$ of $Y$ containing $f(x)$ and every $\alpha_0\in A$, there exists $\alpha\geq\alpha_0\in A$ such that $f(x_{\alpha})\notin V$, hence such that $x_\alpha\notin f^{-1}(V)$; as $x_\alpha\rightarrow x$ by hypothesis, this implies that $f^{-1}(V)$ cannot be a neighborhood of $x$, and thus that $f$ fails to be continuous at $x$.
\end{proof}
%%%%%
%%%%%
\end{document}
