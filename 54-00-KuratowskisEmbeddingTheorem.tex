\documentclass[12pt]{article}
\usepackage{pmmeta}
\pmcanonicalname{KuratowskisEmbeddingTheorem}
\pmcreated{2013-03-22 18:24:48}
\pmmodified{2013-03-22 18:24:48}
\pmowner{puuhikki}{9774}
\pmmodifier{puuhikki}{9774}
\pmtitle{Kuratowski's embedding theorem}
\pmrecord{10}{41062}
\pmprivacy{1}
\pmauthor{puuhikki}{9774}
\pmtype{Theorem}
\pmcomment{trigger rebuild}
\pmclassification{msc}{54-00}

% this is the default PlanetMath preamble.  as your knowledge
% of TeX increases, you will probably want to edit this, but
% it should be fine as is for beginners.

% almost certainly you want these
\usepackage{amssymb}
\usepackage{amsmath}
\usepackage{amsfonts}
\usepackage[T1]{fontenc}
\usepackage[latin1]{inputenc}
% used for TeXing text within eps files
%\usepackage{psfrag}
% need this for including graphics (\includegraphics)
%\usepackage{graphicx}
% for neatly defining theorems and propositions
%\usepackage{amsthm}
% making logically defined graphics
%%%\usepackage{xypic}

% there are many more packages, add them here as you need them

% define commands here

\begin{document}
Let $X$ be a set and $\operatorname{Bou}(X,\mathbb{R})$ be the set of bounded functions $f:X\to\mathbb{R}$ with norm \,$||f||=\operatorname{sup}\{|f(x)|:\;x\in X\}$.  Kuratowski's embedding theorem states that every metric space $(X,d)$ can be embedded isometrically into the Banach space \,$E=\operatorname{Bou}(X,\,\mathbb{R})$.

{\em Proof.}\,
One can assume that $X\ne \emptyset$.  Fix a point $a_0\in X$ and for every $a\in X$ define a function $f_a:X\to\mathbb{R}$ by
\[f_a(x)=d(x,a)-d(x,a_0).\]
Then $|f_a(x)|\leq d(a,a_0)$ for every $x\in X$ so $f_a$ is bounded. By setting \,$\varphi :X\to E$,\, $\varphi (a)=f_a$, we have the mapping $\varphi : X\to E$.  It requires to prove that $\varphi$ is an isometry.

Let $a,\,b\in X$.  As $x\in X$ we have that
\[|f_a(x)-f_b(x)|=|d(x,a)-d(x,b)|\leq d(a,b).\]
Therefore $||f_a-f_b||\leq d(a,b)$.  On the other hand
\[|f_a(a)-f_b(a)|=|d(a,a)-d(a,a_0)-d(a,b)+d(a,a_0)|=d(a,b).\]
Therefore $||\varphi (a)-\varphi (b)||=||f_a-f_b||=d(a,b)$.
\begin{thebibliography}{9}
\bibitem{J.V.}{\sc J. Väisälä:} {\em Topologia II}.\, 2nd corrected issue, Limes ry., Helsinki, Finland (2005), ISBN 951-745-209-8
\end{thebibliography}
%%%%%
%%%%%
\end{document}
