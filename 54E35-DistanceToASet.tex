\documentclass[12pt]{article}
\usepackage{pmmeta}
\pmcanonicalname{DistanceToASet}
\pmcreated{2013-03-22 13:38:37}
\pmmodified{2013-03-22 13:38:37}
\pmowner{bbukh}{348}
\pmmodifier{bbukh}{348}
\pmtitle{distance to a set}
\pmrecord{4}{34294}
\pmprivacy{1}
\pmauthor{bbukh}{348}
\pmtype{Definition}
\pmcomment{trigger rebuild}
\pmclassification{msc}{54E35}

\endmetadata

% this is the default PlanetMath preamble.  as your knowledge
% of TeX increases, you will probably want to edit this, but
% it should be fine as is for beginners.

% almost certainly you want these
\usepackage{amssymb}
\usepackage{amsmath}
\usepackage{amsfonts}

% used for TeXing text within eps files
%\usepackage{psfrag}
% need this for including graphics (\includegraphics)
%\usepackage{graphicx}
% for neatly defining theorems and propositions
%\usepackage{amsthm}
% making logically defined graphics
%%%\usepackage{xypic}

% there are many more packages, add them here as you need them

% define commands here
\begin{document}
Let $X$ be a metric space with a metric $d$. If $A$ is a non-empty
subset of $X$ and $x\in X$, then the \emph{distance from $x$ to $A$}
\cite{kelley} is defined as
$$ d(x,A) := \inf_{a\in A} d(x,a).$$
We also write $d(x,A)=d(A,x)$.

Suppose that $x,y$ are points in $X$, and $A\subset X$ is non-empty.
Then we have the following triangle inequality
\begin{eqnarray*}
d(x,A) &=& \inf_{a\in A} d(x,a) \\
        &\le& d(x,y) + \inf_{a\in A} d(y,a) \\
        &=& d(x,y) + d(y,A).
\end{eqnarray*}

If $X$ is only a pseudo-metric space, then the above definition
and triangle-inequality also hold.

\begin{thebibliography}{9}
 \bibitem{kelley}
 J.L. Kelley,
 \emph{General Topology},
 D. van Nostrand Company, Inc., 1955.
 \end{thebibliography}
%%%%%
%%%%%
\end{document}
