\documentclass[12pt]{article}
\usepackage{pmmeta}
\pmcanonicalname{ClosedSet}
\pmcreated{2013-03-22 12:30:23}
\pmmodified{2013-03-22 12:30:23}
\pmowner{yark}{2760}
\pmmodifier{yark}{2760}
\pmtitle{closed set}
\pmrecord{10}{32739}
\pmprivacy{1}
\pmauthor{yark}{2760}
\pmtype{Definition}
\pmcomment{trigger rebuild}
\pmclassification{msc}{54-00}
\pmsynonym{closed subset}{ClosedSet}
\pmdefines{closed}

\usepackage{amsfonts}
\usepackage{amssymb}

\def\R{\mathbb{R}}
\def\emptyset{\varnothing}
\begin{document}
\PMlinkescapephrase{closed under}
\PMlinkescapeword{contains}
\PMlinkescapeword{contained}

Let $(X,\tau)$ be a topological space. Then a subset $C\subseteq X$ is \emph{closed} if its complement $X\setminus C$ is open under the topology $\tau$.

Examples:
\begin{itemize}
\item In any topological space $(X,\tau)$, the sets $X$ and $\emptyset$ are always closed.

\item Consider $\R$ with the standard topology. Then $[0,1]$ is closed since its complement $(-\infty,0) \cup (1,\infty)$ is open (being the union of two open sets).

\item Consider $\R$ with the lower limit topology. Then $[0,1)$ is closed since its complement $(-\infty,0)\cup[1,\infty)$ is open.
\end{itemize}

Closed subsets can also be characterized as follows:

A subset $C\subseteq X$ is closed if and only if $C$ contains all of its cluster points, that is, $C'\subseteq C$.

So the set $\{1,1/2,1/3,1/4,\ldots\}$ is not closed under the standard topology on $\R$ since $0$ is a cluster point not contained in the set.
%%%%%
%%%%%
\end{document}
