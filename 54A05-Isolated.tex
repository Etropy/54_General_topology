\documentclass[12pt]{article}
\usepackage{pmmeta}
\pmcanonicalname{Isolated}
\pmcreated{2013-03-22 12:05:59}
\pmmodified{2013-03-22 12:05:59}
\pmowner{djao}{24}
\pmmodifier{djao}{24}
\pmtitle{isolated}
\pmrecord{8}{31201}
\pmprivacy{1}
\pmauthor{djao}{24}
\pmtype{Definition}
\pmcomment{trigger rebuild}
\pmclassification{msc}{54A05}
\pmsynonym{discrete set}{Isolated}
\pmdefines{isolated set}
\pmdefines{isolated point}

\endmetadata

\usepackage{amssymb}
\usepackage{amsmath}
\usepackage{amsfonts}
\usepackage{graphicx}
%%%\usepackage{xypic}
\begin{document}
Let $X$ be a topological space, let $S \subset X$, and let $x \in S$. The point $x$ is said to be an \emph{isolated} point of $S$ if there exists an open set $U \subset X$ such that $U \cap S = \{x\}$.

The set $S$ is \emph{isolated} or \emph{discrete} if every point in $S$ is an isolated point.
%%%%%
%%%%%
%%%%%
\end{document}
