\documentclass[12pt]{article}
\usepackage{pmmeta}
\pmcanonicalname{Suspension}
\pmcreated{2013-03-22 13:25:37}
\pmmodified{2013-03-22 13:25:37}
\pmowner{antonio}{1116}
\pmmodifier{antonio}{1116}
\pmtitle{suspension}
\pmrecord{10}{33984}
\pmprivacy{1}
\pmauthor{antonio}{1116}
\pmtype{Definition}
\pmcomment{trigger rebuild}
\pmclassification{msc}{54B99}
\pmrelated{Cone}
\pmrelated{LoopSpace}
\pmrelated{Join3}
\pmrelated{SuspensionIsomorphism}
\pmdefines{suspension}
\pmdefines{reduced suspension}
\pmdefines{based suspension}
\pmdefines{unreduced suspension}
\pmdefines{unbased suspension}

\endmetadata

% used for TeXing text within eps files
%\usepackage{psfrag}
% need this for including graphics (\includegraphics)
%\usepackage{graphicx}
% for neatly defining theorems and propositions
%\usepackage{amsthm}
% making logically defined graphics
%%%\usepackage{xypic}

\usepackage{amsthm}
\usepackage{amsmath}
\usepackage{amsfonts}
\usepackage{amssymb}

\newcommand{\limv}[2]{\lim\limits_{#1\rightarrow #2}}
\newcommand{\eb}{\mathbf{e}}    % Standard basis
\newcommand{\comp}{\circ}       % Function composition
\newcommand{\reals}{{\mathbb R}}        % The reals
\newcommand{\integs}{{\mathbb Z}}        % The integers
\newcommand{\setc}[2]{\left\{#1:\: #2\right\}}
\newcommand{\set}[1]{\left\{#1\right\}}
\newcommand{\cycle}[1]{\left(#1\right)}
\newcommand{\tuple}[1]{\left(#1\right)}
\newcommand{\Partial}[2]{\frac{\partial #1}{\partial #2}}
\newcommand{\PartialSl}[2]{\partial #1/\partial #2}
\newcommand{\funcsig}[2]{#1\rightarrow #2}
\newcommand{\funcdef}[3]{#1:\funcsig{#2}{#3}}
\newcommand{\supp}{\mathop{\mathrm{Supp}}} % Support of a function
\newcommand{\sgn}{\mathop{\mathrm{sgn}}} % Sign function
\newcommand{\tr}[1]{#1^\mathrm{tr}} % Transpose of a matrix
\newcommand{\inprod}[2]{\left<#1,#2\right>} % Inner product
\newenvironment{smallbmatrix}{\left[\begin{smallmatrix}}{\end{smallmatrix}\right]}
\newcommand{\maps}[2]{\mathop{\mathrm{Maps}}\left(#1,#2\right)}
\newcommand{\bmaps}[2]{\mathop{\mathrm{Maps}_*}\left(#1,#2\right)}
\newcommand{\intoc}[2]{\left(#1,#2\right]}
\newcommand{\intco}[2]{\left[#1,#2\right)}
\newcommand{\intoo}[2]{\left(#1,#2\right)}
\newcommand{\intcc}[2]{\left[#1,#2\right]}
\newcommand{\transv}{\pitchfork}
\newcommand{\pair}[2]{\left\langle#1,#2\right\rangle}
\newcommand{\norm}[1]{\left\|#1\right\|}
\newcommand{\sqnorm}[1]{\left\|#1\right\|^2}
\newcommand{\bdry}{\partial}
\newcommand{\inv}[1]{#1^{-1}}
\newcommand{\tensor}{\otimes}
\newcommand{\bigtensor}{\bigotimes}
\newcommand{\im}{\operatorname{im}}
\newcommand{\coker}{\operatorname{im}}
\newcommand{\map}{\operatorname{Map}}
\newcommand{\crit}{\operatorname{Crit}}
\newtheorem{thm}{Theorem}[section]
\newtheorem{dthm}{Desired Theorem}[section]
\newtheorem{cor}[thm]{Corollary}
\newtheorem{dcor}[thm]{Desired Corollary}
\newtheorem{lem}[thm]{Lemma}
\newtheorem{defn}{Definition}
\newcommand{\cross}{\times}
\newcommand{\del}{\nabla}
\newcommand{\homeo}{\cong}
\newcommand{\isom}{\cong}
\newcommand{\codim}{\operatorname{codim}}
\newcommand{\susp}{\Sigma}
\begin{document}
\section{The unreduced suspension}

Given a topological space $X,$ the {\em suspension} of $X,$ often denoted by $SX,$ is defined to be the quotient space $X\cross[0,1]/\sim,$ where $(x,0)\sim(y,0)$ and $(x,1)\sim(y,1)$ for any $x, y\in X.$ 

Given a continuous map $\funcdef{f}{X}{Y},$ there is a map 
$\funcdef{Sf}{SX}{SY}$ defined by $Sf([x,t]):=[f(x),t].$ This makes $S$ into a functor from the category of topological spaces into itself. 

Note that $SX$ is homeomorphic to the join $X\star S^0,$ where $S^0$ is a discrete space with two points.

The space $SX$ is sometimes called the {\em unreduced}, {\em unbased} or {\em free} suspension of $X,$ to distinguish it from the reduced suspension described below.


\section{The reduced suspension}
If $(X,x_0)$ is a based topological space, the {\em reduced suspension} of $X,$ often denoted $\susp X$ (or $\susp_{x_0} X$ when the basepoint needs to be explicit), is defined to be the quotient space $X\times[0,1]/(X\cross\set{0}\cup X\cross\set{1}\cup\set{x_0}\cross[0,1].$ Setting the basepoint of $\susp X$ to be the equivalence class of $(x_0,0),$ the reduced suspension is a functor from the category of based topological spaces into itself.

An important property of this functor is that it is a left adjoint to the functor $\Omega$ taking a (based) space $X$ to its loop space $\Omega X$. In other words, $\bmaps{\susp X}{Y}\isom\bmaps{X}{\Omega Y}$ naturally, where $\bmaps{X}{Y}$ stands for continuous maps which preserve basepoints.

The reduced suspension is also known as the {\em based\/} suspension.
%%%%%
%%%%%
\end{document}
