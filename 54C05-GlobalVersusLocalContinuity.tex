\documentclass[12pt]{article}
\usepackage{pmmeta}
\pmcanonicalname{GlobalVersusLocalContinuity}
\pmcreated{2013-03-22 19:09:07}
\pmmodified{2013-03-22 19:09:07}
\pmowner{CWoo}{3771}
\pmmodifier{CWoo}{3771}
\pmtitle{global versus local continuity}
\pmrecord{4}{42056}
\pmprivacy{1}
\pmauthor{CWoo}{3771}
\pmtype{Result}
\pmcomment{trigger rebuild}
\pmclassification{msc}{54C05}
\pmclassification{msc}{26A15}

\usepackage{amssymb,amscd}
\usepackage{amsmath}
\usepackage{amsfonts}
\usepackage{mathrsfs}

% used for TeXing text within eps files
%\usepackage{psfrag}
% need this for including graphics (\includegraphics)
%\usepackage{graphicx}
% for neatly defining theorems and propositions
\usepackage{amsthm}
% making logically defined graphics
%%\usepackage{xypic}
\usepackage{pst-plot}

% define commands here
\newcommand*{\abs}[1]{\left\lvert #1\right\rvert}
\newtheorem{prop}{Proposition}
\newtheorem{thm}{Theorem}
\newtheorem{ex}{Example}
\newcommand{\real}{\mathbb{R}}
\newcommand{\pdiff}[2]{\frac{\partial #1}{\partial #2}}
\newcommand{\mpdiff}[3]{\frac{\partial^#1 #2}{\partial #3^#1}}
\begin{document}
In this entry, we establish a very basic fact about continuity:

\begin{prop} A function $f:X\to Y$ between two topological spaces is continuous iff it is continuous at every point $x\in X$. \end{prop}
\begin{proof}
Suppose first that $f$ is continuous, and $x\in X$.  Let $f(x)\in V$ be an open set in $Y$.  We want to find an open set $x\in U$ in $X$ such that $f(U)\subseteq V$.  Well, let $U=f^{-1}(V)$.  So $U$ is open since $f$ is continuous, and $x\in U$.  Furthermore, $f(U) = f(f^{-1}(V))=V$.

On the other hand, if $f$ is not continuous at $x \in X$.  Then there is an open set $f(x)\in V$ in $Y$ such that no open sets $x\in U$ in $X$ have the property 
\begin{equation}
f(U)\subseteq V.
\end{equation}  
Let $W=f^{-1}(V)$.  If $W$ is open, then $W$ has the property $(1)$ above, a contradiction.  Since $W$ is not open, $f$ is not continuous.
\end{proof}
%%%%%
%%%%%
\end{document}
