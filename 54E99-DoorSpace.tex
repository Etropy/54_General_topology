\documentclass[12pt]{article}
\usepackage{pmmeta}
\pmcanonicalname{DoorSpace}
\pmcreated{2013-03-22 18:46:11}
\pmmodified{2013-03-22 18:46:11}
\pmowner{CWoo}{3771}
\pmmodifier{CWoo}{3771}
\pmtitle{door space}
\pmrecord{6}{41551}
\pmprivacy{1}
\pmauthor{CWoo}{3771}
\pmtype{Definition}
\pmcomment{trigger rebuild}
\pmclassification{msc}{54E99}

\endmetadata

\usepackage{amssymb,amscd}
\usepackage{amsmath}
\usepackage{amsfonts}
\usepackage{mathrsfs}

% used for TeXing text within eps files
%\usepackage{psfrag}
% need this for including graphics (\includegraphics)
%\usepackage{graphicx}
% for neatly defining theorems and propositions
\usepackage{amsthm}
% making logically defined graphics
%%\usepackage{xypic}
\usepackage{pst-plot}

% define commands here
\newcommand*{\abs}[1]{\left\lvert #1\right\rvert}
\newtheorem{prop}{Proposition}
\newtheorem{thm}{Theorem}
\newtheorem{ex}{Example}
\newcommand{\real}{\mathbb{R}}
\newcommand{\pdiff}[2]{\frac{\partial #1}{\partial #2}}
\newcommand{\mpdiff}[3]{\frac{\partial^#1 #2}{\partial #3^#1}}
\begin{document}
A topological space $X$ is called a \emph{door space} if every subset of $X$ is either open or closed.

From the definition, it is immediately clear that any discrete space is door.  

To find more examples, let us look at the singletons of a door space $X$.  For each $x\in X$, either $\lbrace x\rbrace$ is open or closed.  Call a point $x$ in $X$ open or closed according to whether $\lbrace x\rbrace$ is open or closed.  Let $A$ be the collection of open points in $X$.  If $A=X$, then $X$ is discrete.  So suppose now that $A\ne X$.  We look at the special case when $X-A=\lbrace x\rbrace$.  It is now easy to see that the topology $\tau$ generated by all the open singletons makes $X$ a door space:
\begin{proof}
If $B\subseteq X$ does not contain $x$, it is the union of elements in $A$, and therefore open.  If $x\in B$, then its complement $B^c$ does not, so is open, and therefore $B$ is closed.
\end{proof}
Since $\tau=P(A)\cup \lbrace X\rbrace$, the space $X$ not discrete.  In addition, $X$ and $\varnothing$ are the only clopen sets in $X$.

When $X-A$ has more than one element, the situation is a little more complicated.  We know that if $X$ is door, then its topology $\mathcal{T}$ is strictly finer then the topology $\tau$ generated by all the open singletons.  McCartan has shown that $\mathcal{T}=\tau \cup \mathcal{U}$ for some ultrafilter in $X$.  In fact, McCartan showed $\mathcal{T}$, as well as the previous two examples, are the only types of possible topologies on a set making it a door space.

\begin{thebibliography}{9}
\bibitem{jlk} J.L. Kelley, \emph{General Topology}, D. van Nostrand Company, Inc., 1955.
\bibitem{sdm} S.D. McCartan, \emph{Door Spaces are identifiable}, Proc. Roy. Irish Acad. Sect. A, 87 (1) 1987, pp. 13-16.
\end{thebibliography}
%%%%%
%%%%%
\end{document}
