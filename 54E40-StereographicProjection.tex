\documentclass[12pt]{article}
\usepackage{pmmeta}
\pmcanonicalname{StereographicProjection}
\pmcreated{2013-03-22 15:18:35}
\pmmodified{2013-03-22 15:18:35}
\pmowner{GrafZahl}{9234}
\pmmodifier{GrafZahl}{9234}
\pmtitle{stereographic projection}
\pmrecord{5}{37110}
\pmprivacy{1}
\pmauthor{GrafZahl}{9234}
\pmtype{Definition}
\pmcomment{trigger rebuild}
\pmclassification{msc}{54E40}
\pmclassification{msc}{54C25}
\pmclassification{msc}{54C05}
\pmclassification{msc}{51M15}
%\pmkeywords{projection}
%\pmkeywords{map}
\pmrelated{CoordinateSystems}
\pmrelated{ClosedComplexPlane}
\pmrelated{RiemannSphere}
\pmdefines{north pole}
\pmdefines{south pole}

\endmetadata

% this is the default PlanetMath preamble.  as your knowledge
% of TeX increases, you will probably want to edit this, but
% it should be fine as is for beginners.

% almost certainly you want these
\usepackage{amssymb}
\usepackage{amsmath}
\usepackage{amsfonts}
\usepackage[latin1]{inputenc}

% used for TeXing text within eps files
%\usepackage{psfrag}
% need this for including graphics (\includegraphics)
\usepackage{graphicx}
% for neatly defining theorems and propositions
\usepackage{amsthm}
% making logically defined graphics
%%%\usepackage{xypic}

% there are many more packages, add them here as you need them

% define commands here
\newcommand{\Bigcup}{\bigcup\limits}
\newcommand{\Prod}{\prod\limits}
\newcommand{\Sum}{\sum\limits}
\newcommand{\mbb}{\mathbb}
\newcommand{\mbf}{\mathbf}
\newcommand{\mc}{\mathcal}
\newcommand{\mmm}[9]{\left(\begin{array}{rrr}#1&#2&#3\\#4&#5&#6\\#7&#8&#9\end{array}\right)}
\newcommand{\ol}{\overline}

% Math Operators/functions
\DeclareMathOperator{\Frob}{Frob}
\DeclareMathOperator{\cwe}{cwe}
\DeclareMathOperator{\id}{id}
\DeclareMathOperator{\we}{we}
\DeclareMathOperator{\wt}{wt}
\begin{document}
\PMlinkescapeword{restricted}
The $n$-dimensional Euclidean \PMlinkid{unit sphere}{186} $S^n$ is
defined as a subset
of $\mbb{R}^{n+1}$:
\begin{equation*}
S^n=\biggl\{(x_1,\ldots,x_{n+1})\in\mbb{R}^{n+1}\mid\Sum_{k=1}^{n+1}x_k^2=1\biggr\}.
\end{equation*}
The \emph{stereographic projection} maps all points of $S^n$ to
the $n$-dimensional Euclidean space $\mbb{R}^n$ except one. Let
$N:=(0,\ldots,0,1)\in S^n$ be this point (it is usually called the
\emph{north pole}). Then the stereographic projection is defined by
\begin{equation*}
\sigma\colon S^n\setminus
N\to\mbb{R}^n,\quad(x_1,\ldots,x_{n+1})\mapsto\frac{c-1}{x_{n+1}-1}(x_1,\ldots,x_n).
\end{equation*}
Here, $c$ is an arbitrary real number. If $c=1$, the projection
degenerates; in all other cases, however, $\sigma$ is a smooth
bijective mapping.

The image $P'$ of a point $P$ under $\sigma$ can be geometrically
constructed as follows. Embed $\mbb{R}^n$ into
$\mbb{R}^{n+1}$ as a hyperplane at $x_{n+1}=c$. Unless $c=1$, the
straight line defined by $N$ and $P$ intersects with $\mbb{R}^n$ in
precisely one point, $P'$. The most common values for $c$ are $c=-1$
and $c=0$, see figures~\ref{fig1} and~\ref{fig2}.

\begin{figure}
\label{fig1}
\begin{center}
\includegraphics{StereographicProjection.1.eps}
\end{center}
\sf\caption{Stereographic projection of the one dimensional unit
sphere for $c=-1$}
\end{figure}

\begin{figure}
\label{fig2}
\begin{center}
\includegraphics{StereographicProjection.2.eps}
\end{center}
\sf\caption{Stereographic projection of the one dimensional unit
sphere for $c=0$}
\end{figure}

Let $-\id\colon\mbb{R}^{n+1}\to\mbb{R}^{n+1}$ be the map $x\mapsto
-x$, then $\tilde{\sigma}:=\sigma\circ(-\id)$ (a suitably restricted composition) maps all points of $S^n$
except the \emph{south pole} $S:=(0,\ldots,0,-1)$ smoothly and
bijectively to $\mbb{R}^n$. Together, $\sigma$ and $\tilde{\sigma}$
form an atlas of $S^n$, so $S^n$ is an $n$-dimensional smooth
manifold.
%%%%%
%%%%%
\end{document}
