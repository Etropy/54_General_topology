\documentclass[12pt]{article}
\usepackage{pmmeta}
\pmcanonicalname{HewittMarczewskiPondiczeryTheorem}
\pmcreated{2013-03-22 17:16:56}
\pmmodified{2013-03-22 17:16:56}
\pmowner{yark}{2760}
\pmmodifier{yark}{2760}
\pmtitle{Hewitt-Marczewski-Pondiczery theorem}
\pmrecord{6}{39623}
\pmprivacy{1}
\pmauthor{yark}{2760}
\pmtype{Theorem}
\pmcomment{trigger rebuild}
\pmclassification{msc}{54D65}
\pmclassification{msc}{54A25}
\pmrelated{Dense}
\pmrelated{Separable}

\usepackage{amsthm}

\newtheorem*{thm*}{Theorem}

% The following lines should work as the command
% \renewcommand{\bibname}{References}
% without creating havoc when rendering an entry in
% the page-image mode.
\makeatletter
\@ifundefined{bibname}{}{\renewcommand{\bibname}{References}}
\makeatother

\begin{document}
\PMlinkescapeword{theorem}

The \emph{Hewitt--Marczewski--Pondiczery Theorem}
is a result on the density of products of topological spaces.
This theorem was arrived at independently by 
Hewitt\cite{hewitt}, Marczewski\cite{marcz} and Pondiczery\cite{pond}
in the 1940s.

\begin{thm*}
Let $\kappa$ be an infinite cardinal number and $S$ an index set
of cardinality at most $2^\kappa$.
If $X_s$ $(s\in S)$ are topological spaces with $d(X_s)\le\kappa$ then
\[
  d\left(\prod_{s\in S}X_s\right)\le\kappa.
\]
\end{thm*}

The special case $\kappa=\aleph_0$
says that the product of at most continuum many separable spaces is separable.

\begin{thebibliography}{3}
\bibitem{hewitt}
 Edwin Hewitt,
 {\it A remark on density characters},
 Bull.\ Amer.\ Math.\ Soc.\ 52 (1946), 641--643.
 (This paper is available as a PDF file from the AMS website:
  \PMlinkexternal{Bull. Amer. Math. Soc., Volume 52, Number 8}{http://www.ams.org/journals/bull/1946-52-08/home.html}.)
\bibitem{marcz}
 Edward Marczewski,
 {\it S\'eparabilit\'e et multiplication cart\'esienne des espaces topologiques},
 Fund.\ Math.\ 34 (1947), 127--143.
 (This paper is available as a PDF file from the Polish Virtual Library of Science:
  \PMlinkexternal{Fundamenta Mathematicae, Volume 34}{http://matwbn.icm.edu.pl/tresc.php?wyd=1&tom=34&jez=en}.)
\bibitem{pond}
 E.\ S.\ Pondiczery,
 {\it Power problems in abstract spaces},
 Duke Math.\ J.\ 11 (1944), 835--837.
\end{thebibliography}
%%%%%
%%%%%
\end{document}
