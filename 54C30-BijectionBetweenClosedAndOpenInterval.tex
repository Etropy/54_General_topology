\documentclass[12pt]{article}
\usepackage{pmmeta}
\pmcanonicalname{BijectionBetweenClosedAndOpenInterval}
\pmcreated{2013-03-22 19:36:06}
\pmmodified{2013-03-22 19:36:06}
\pmowner{pahio}{2872}
\pmmodifier{pahio}{2872}
\pmtitle{bijection between closed and open interval}
\pmrecord{10}{42593}
\pmprivacy{1}
\pmauthor{pahio}{2872}
\pmtype{Example}
\pmcomment{trigger rebuild}
\pmclassification{msc}{54C30}
\pmclassification{msc}{26A30}
\pmrelated{BijectionBetweenUnitIntervalAndUnitSquare}

% this is the default PlanetMath preamble.  as your knowledge
% of TeX increases, you will probably want to edit this, but
% it should be fine as is for beginners.

% almost certainly you want these
\usepackage{amssymb}
\usepackage{amsmath}
\usepackage{amsfonts}

% used for TeXing text within eps files
%\usepackage{psfrag}
% need this for including graphics (\includegraphics)
%\usepackage{graphicx}
% for neatly defining theorems and propositions
 \usepackage{amsthm}
% making logically defined graphics
%%%\usepackage{xypic}

% there are many more packages, add them here as you need them

% define commands here

\theoremstyle{definition}
\newtheorem*{thmplain}{Theorem}

\begin{document}
\PMlinkescapeword{means}
For mapping the end points of the closed unit interval\, $[0,\,1]$\, and its inner points bijectively onto the corresponding open unit interval\, $(0,\,1)$,\, one has to discern suitable denumerable subsets in both sets:
\begin{align*}
& [0,\,1]\, \;=\; \{0,\,1,\,1/2,\,1/3,\,1/4,\,\ldots\}\cup S,\\
&(0,\,1) \;=\; \{1/2,\,1/3,\,1/4,\,\ldots\}\cup S,
\end{align*}
where
$$S \;:=\; [0,\,1]\smallsetminus\{0,\,1,\,1/2,\,1/3,\,1/4,\,\ldots\}.$$
Then the mapping $f$ from\, $[0,\,1]$\, to\, $(0,\,1)$\, defined by
$$f(x) \;:=\;
\begin{cases}
 1/2 \quad \mbox{for}\quad x = 0,\\
 1/(n\!+\!2) \quad \mbox{for} \quad x = 1/n \quad (n = 1,\,2,\,3,\,\ldots),\\
 x \qquad \mbox{for} \quad x \in S
\end{cases}$$
is apparently a bijection.\, This means the equicardinality of both intervals.\\

Note that the bijection is neither monotonic (e.g. $0 \mapsto \frac{1}{2}$,\; $\frac{1}{2}\mapsto \frac{1}{4}$,\; 
$1 \mapsto \frac{1}{3}$) nor continuous.\, Generally, there does not exist any continuous surjective mapping\, 
$[0,\,1] \to (0,\,1)$,\, since by the intermediate value theorem a continuous function maps a closed interval to a closed interval.

\begin{thebibliography}{8}
\bibitem{L}{\sc S. Lipschutz}: {\em Set theory}.\, Schaum Publishing Co., New York (1964).
\end{thebibliography}

%%%%%
%%%%%
\end{document}
