\documentclass[12pt]{article}
\usepackage{pmmeta}
\pmcanonicalname{IndiscreteTopology}
\pmcreated{2013-03-22 12:48:11}
\pmmodified{2013-03-22 12:48:11}
\pmowner{mathwizard}{128}
\pmmodifier{mathwizard}{128}
\pmtitle{indiscrete topology}
\pmrecord{20}{33120}
\pmprivacy{1}
\pmauthor{mathwizard}{128}
\pmtype{Definition}
\pmcomment{trigger rebuild}
\pmclassification{msc}{54-00}
\pmsynonym{trivial topology}{IndiscreteTopology}
\pmsynonym{coarse topology}{IndiscreteTopology}

\endmetadata

% this is the default PlanetMath preamble.  as your knowledge
% of TeX increases, you will probably want to edit this, but
% it should be fine as is for beginners.

% almost certainly you want these
\usepackage{amssymb}
\usepackage{amsmath}
\usepackage{amsfonts}

% used for TeXing text within eps files
%\usepackage{psfrag}
% need this for including graphics (\includegraphics)
%\usepackage{graphicx}
% for neatly defining theorems and propositions
%\usepackage{amsthm}
% making logically defined graphics
%%%\usepackage{xypic}

% there are many more packages, add them here as you need them

% define commands here
\begin{document}
If $X$ is a set and it is endowed with a topology defined by


$$\tau=\{X,\emptyset\} \label{eq12}$$
then $X$ is said to have the \emph{indiscrete topology}. 

 Furthermore $\tau$ is the coarsest topology a set can possess, since $\tau$
would be a subset of any other possible topology. This topology
gives $X$ many properties:
\begin{itemize}
\item Every subset of $X$ is sequentially compact.  
\item Every function to a space with the indiscrete topology is continuous.
\item $X$ is path connected and hence connected but is arc connected only if $X$ is uncountable or if $X$ has at most a single point. However, $X$ is both hyperconnected and ultraconnected. 
\item If $X$ has more than one point, it is not metrizable because it is not Hausdorff. However it is pseudometrizable with the metric $d(x,y)=0$.
\end{itemize}
%%%%%
%%%%%
\end{document}
