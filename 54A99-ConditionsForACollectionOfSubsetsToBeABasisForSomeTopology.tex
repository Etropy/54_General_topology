\documentclass[12pt]{article}
\usepackage{pmmeta}
\pmcanonicalname{ConditionsForACollectionOfSubsetsToBeABasisForSomeTopology}
\pmcreated{2013-03-22 14:21:49}
\pmmodified{2013-03-22 14:21:49}
\pmowner{waj}{4416}
\pmmodifier{waj}{4416}
\pmtitle{conditions for a collection of subsets to be a basis for some topology}
\pmrecord{4}{35845}
\pmprivacy{1}
\pmauthor{waj}{4416}
\pmtype{Proof}
\pmcomment{trigger rebuild}
\pmclassification{msc}{54A99}
\pmclassification{msc}{54D70}
%\pmkeywords{teaching proofs}
%\pmkeywords{characterization of a basis}
%\pmkeywords{what a basis looks like}

% this is the default PlanetMath preamble.  as your knowledge
% of TeX increases, you will probably want to edit this, but
% it should be fine as is for beginners.

% almost certainly you want these
\usepackage{amssymb}
\usepackage{amsmath}
\usepackage{amsfonts}

% used for TeXing text within eps files
%\usepackage{psfrag}
% need this for including graphics (\includegraphics)
%\usepackage{graphicx}
% for neatly defining theorems and propositions
\usepackage{amsthm}
% making logically defined graphics
%%%\usepackage{xypic}

% there are many more packages, add them here as you need them

% define commands here
\def\co{\colon\thinspace}
\theoremstyle{definition}
\newtheorem*{thm}{Theorem}
\begin{document}
Not just any collection of subsets of $X$ can be a basis for a topology on $X$.  For instance, if we took $\mathcal{C}$ to be all open intervals of length $1$ in $\mathbb{R}$, $\mathcal{C}$ isn't the basis for any topology on $\mathbb{R}$: $(0,1)$ and $(.5, 1.5)$ are unions of elements of $\mathcal{C}$, but their intersection $(.5,1)$ is not.  The collection formed by arbitrary unions of members of $\mathcal{C}$ isn't closed under finite intersections and isn't a topology.

We'd like to know which collections $\mathcal{B}$ of subsets of $X$ could be the basis for some topology on $X$.  Here's the result:

\begin{thm}
A collection $\mathcal{B}$ of subsets of $X$ is a basis for some topology on $X$ if and only if:
\begin{enumerate}
\item
Every $x\in X$ is contained in some $B_x\in \mathcal{B}$, and
\item
If $B_1$ and $B_2$ are two elements of $\mathcal{B}$ containing $x\in X$, then there's a third element $B_3$ of $\mathcal{B}$ such that $x\in B_3\subset B_1\cap B_2$. 
\end{enumerate}

\end{thm}
\begin{proof}
First, we'll show that if $\mathcal{B}$ is the basis for some topology $\mathcal{T}$ on $X$, then it satisfies the two conditions listed.

$\mathcal{T}$ is a topology on $X$, so $X\in \mathcal{T}$.  Since $\mathcal{B}$ is a basis for $\mathcal{T}$, that means $X$ can be written as a union of members of $\mathcal{B}$: since every $x\in X$ is in this union, every $x\in X$ is contained in some member of $\mathcal{B}$.  That takes care of the first condition.

For the second condition: if $B_1$ and $B_2$ are elements of $\mathcal{B}$, they're also in $\mathcal{T}$.  $\mathcal{T}$ is closed under intersection, so $B_1\cap B_2$ is open in $\mathcal{T}$.  Then $B_1\cap B_2$ can be written as a union of members of $\mathcal{B}$, and any $x\in B_1\cap B_2$ is contained by some basis element in this union.

Second, we'll show that if a collection $\mathcal{B}$ of subsets of $X$ satisfies the two conditions, then the collection $\mathcal{T}$ of unions of members of $\mathcal{B}$ is a topology on $X$.

\begin{itemize}
\item
$\emptyset \in \mathcal{T}$: $\emptyset$ is the null union of zero elements of $\mathcal{B}$.

\item
$X\in \mathcal{T}$: by the first condition, every $X$ is contained in some member of $\mathcal{B}$.  The union of all the members of $\mathcal{B}$ is then all of $X$.

\item
$\mathcal{T}$ is closed under arbitrary unions: Say we have a union of sets $T_{\alpha}\in \mathcal{T}$...

\begin{align*}
\bigcup_{\alpha \in I} T_{\alpha} &= \bigcup_{\alpha \in I} \bigcup_{\beta \in J_{\alpha}} B_{\beta} \\
\intertext{(since each $T_{\alpha}$ is a union of sets in $\mathcal{B}$)}
&= \bigcup_{\beta \in \bigcup_{\alpha \in I} J_{\alpha}} B_{\beta}
\end{align*}

Since that's a union of elements of $\mathcal{B}$, it's also a member of $\mathcal{T}$.

\item
$\mathcal{T}$ is closed under finite intersections: since a collection of sets is closed under finite intersections if and only if it is closed under pairwise intersections, we need only check that the intersection of two members $T_1, T_2$ of $\mathcal{T}$ is in $\mathcal{T}$.

Any $x\in T_1\cap T_2$ is contained in some $B_x^1\subset T_1$ and $B_x^2\subset T_2$.  By the second condition, $x\in B_x^1\cap B_x^2$ gets us a $B_x^3$ with $x\in B_x^3 \subset B_x^1\cap B_x^2 \subset T_1\cap T_2$.  Then

\[ T_1\cap T_2 = \bigcup_{x\in T_1\cap T_2} B_x^3 \]

which is in $\mathcal{T}$.
\end{itemize}

\end{proof}
%%%%%
%%%%%
\end{document}
