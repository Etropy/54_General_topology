\documentclass[12pt]{article}
\usepackage{pmmeta}
\pmcanonicalname{ProximityGeneratedByUniformity}
\pmcreated{2013-03-22 16:56:20}
\pmmodified{2013-03-22 16:56:20}
\pmowner{porton}{9363}
\pmmodifier{porton}{9363}
\pmtitle{proximity generated by uniformity}
\pmrecord{9}{39206}
\pmprivacy{1}
\pmauthor{porton}{9363}
\pmtype{Definition}
\pmcomment{trigger rebuild}
\pmclassification{msc}{54E17}
\pmclassification{msc}{54E15}
\pmclassification{msc}{54E05}
%\pmkeywords{proximity space}
%\pmkeywords{nearness space}
%\pmkeywords{uniformity}
%\pmkeywords{uniform space}
\pmdefines{uniform proximity}

% this is the default PlanetMath preamble.  as your knowledge
% of TeX increases, you will probably want to edit this, but
% it should be fine as is for beginners.

% almost certainly you want these
\usepackage{amssymb}
\usepackage{amsmath}
\usepackage{amsfonts}

% used for TeXing text within eps files
%\usepackage{psfrag}
% need this for including graphics (\includegraphics)
%\usepackage{graphicx}
% for neatly defining theorems and propositions
%\usepackage{amsthm}
% making logically defined graphics
%%%\usepackage{xypic}

% there are many more packages, add them here as you need them

% define commands here

\begin{document}
{\bf Definition}. Let $X$ be a uniform space with uniformity $\mathcal{U}$.  The \emph{uniform proximity}, or \emph{proximity generated by $\mathcal{U}$}, is a binary relation $\mathrel{\delta}$ on the subsets of $X$, given by the formula
$$A\mathrel{\delta}B \qquad \iff \qquad \forall U\in\mathcal{U}:U\cap(A\times B)\neq\emptyset.$$

Correctness of the above proximity is given by the below theorem:

{\bf Theorem}. Proximity generated by a uniformity is a proximity.  In other words, $X$ is a proximity space with proximity $\mathrel{\delta}$.
%%%%%
%%%%%
\end{document}
