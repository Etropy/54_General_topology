\documentclass[12pt]{article}
\usepackage{pmmeta}
\pmcanonicalname{ProofThatACompactSetInAHausdorffSpaceIsClosed}
\pmcreated{2013-03-22 13:34:54}
\pmmodified{2013-03-22 13:34:54}
\pmowner{yark}{2760}
\pmmodifier{yark}{2760}
\pmtitle{proof that a compact set in a Hausdorff space is closed}
\pmrecord{7}{34203}
\pmprivacy{1}
\pmauthor{yark}{2760}
\pmtype{Proof}
\pmcomment{trigger rebuild}
\pmclassification{msc}{54D10}
\pmclassification{msc}{54D30}

\endmetadata

\usepackage{amssymb}
\usepackage{amsmath}
\usepackage{amsfonts}

\newcommand{\bbC}{\mathbb{C}}
\newcommand{\bbF}{\mathbb{F}}
\newcommand{\bbN}{\mathbb{N}}
\newcommand{\bbR}{\mathbb{R}}
\newcommand{\bbZ}{\mathbb{Z}}
\newcommand{\calO}{\mathcal{O}}
\newcommand{\fkm}{\mathfrak{m}}
\newcommand{\fkp}{\mathfrak{p}}
\newcommand{\Gal}{\mathrm{Gal}}
\newcommand{\Hom}{\mathrm{Hom}}
\newcommand{\ov}[1]{\overline{#1}}
\newcommand{\ur}{\mathrm{ur}}
\newcommand{\vp}{\varphi}
\newcommand{\wh}[1]{\widehat{#1}}
\newcommand{\wt}[1]{\widetilde{#1}}
\newcommand{\smashedleftarrow}{\setbox0=\hbox{$\longleftarrow$}\ht0=1pt\box0}
\newcommand{\smashedrightarrow}{\setbox0=\hbox{$\longrightarrow$}\ht0=1pt\box0}
\newcommand{\lisom}{\buildrel{\hskip+0.04cm\sim}\over{\smashedleftarrow}}
\newcommand{\risom}{\buildrel{\hskip-0.04cm\sim}\over{\smashedrightarrow}}
\begin{document}
Let $X$ be a Hausdorff space, and $C \subseteq X$ a compact subset.
We are to show that $C$ is closed.
We will do so, by showing that the complement $U = X \setminus C$ is open.
To prove that $U$ is open, it suffices to demonstrate that,
for each $x \in U$,
there exists an open set $V$ with $x \in V$ and $V \subseteq U$.

Fix $x \in U$.
For each $y \in C$, using the Hausdorff assumption,
choose disjoint open sets $A_y$ and $B_y$ with $x \in A_y$ and $y \in B_y$.

Since every $y \in C$ is an element of $B_y$,
the collection $\{B_y \mid y \in C\}$ is an open covering of $C$.
Since $C$ is compact, this open cover admits a finite subcover.
So choose $y_1, \ldots, y_n \in C$ such that
$C \subseteq B_{y_1} \cup \cdots \cup B_{y_n}$.

Notice that $A_{y_1} \cap \cdots \cap A_{y_n}$,
being a finite intersection of open sets, is open, and contains $x$.
Call this neighborhood of $x$ by the name $V$.
All we need to do is show that $V \subseteq U$.

For any point $z \in C$, we have $z \in B_{y_1} \cup \cdots \cup B_{y_n}$,
and therefore $z \in B_{y_k}$ for some $k$.
Since $A_{y_k}$ and $B_{y_k}$ are disjoint, $z \notin A_{y_k}$,
and therefore $z \notin A_{y_1} \cap \cdots \cap A_{y_n} = V$.
Thus $C$ is disjoint from $V$, and $V$ is contained in $U$.
%%%%%
%%%%%
\end{document}
