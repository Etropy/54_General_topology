\documentclass[12pt]{article}
\usepackage{pmmeta}
\pmcanonicalname{ExampleOfContinuousBijectionsWhichAreNotHomeomorphisms}
\pmcreated{2013-03-22 18:54:31}
\pmmodified{2013-03-22 18:54:31}
\pmowner{joking}{16130}
\pmmodifier{joking}{16130}
\pmtitle{example of continuous bijections which are not homeomorphisms}
\pmrecord{4}{41757}
\pmprivacy{1}
\pmauthor{joking}{16130}
\pmtype{Example}
\pmcomment{trigger rebuild}
\pmclassification{msc}{54C05}

% this is the default PlanetMath preamble.  as your knowledge
% of TeX increases, you will probably want to edit this, but
% it should be fine as is for beginners.

% almost certainly you want these
\usepackage{amssymb}
\usepackage{amsmath}
\usepackage{amsfonts}

% used for TeXing text within eps files
%\usepackage{psfrag}
% need this for including graphics (\includegraphics)
%\usepackage{graphicx}
% for neatly defining theorems and propositions
%\usepackage{amsthm}
% making logically defined graphics
%%%\usepackage{xypic}

% there are many more packages, add them here as you need them

% define commands here

\begin{document}
\textbf{Example 1.} Assume that $X$ is a topological space, which neither discrete nor antidiscrete. We will show that there are topological spaces $Y$ and $Z$ such that there are continuous bijections $X\to Y$ and $Z\to X$ which are not homeomorphisms.

Let $Y=Z=X$ as a sets but topology on $Y$ is antidiscrete and on $Z$ is discrete. Then obviously identity mappings $\mathrm{id}:X\to Y$ and $\mathrm{id}:Z\to X$ are continuous, but since $X$ is neither discrete nor antidiscrete, these mappings are not homeomorphisms.

\textbf{Example 2.} Consider the function $f:[0,1)\to S^1$ (here $S^1$ denotes the unit circle in a complex plane) defined by the formula $f(t)=e^{2\pi i t}$. It is easy to see that $f$ is a continuous bijection, but $f$ is not a homeomorphism (because $[0,1)$ is not compact).
%%%%%
%%%%%
\end{document}
