\documentclass[12pt]{article}
\usepackage{pmmeta}
\pmcanonicalname{TotallyBoundedUniformSpace}
\pmcreated{2013-03-22 16:44:09}
\pmmodified{2013-03-22 16:44:09}
\pmowner{CWoo}{3771}
\pmmodifier{CWoo}{3771}
\pmtitle{totally bounded uniform space}
\pmrecord{5}{38958}
\pmprivacy{1}
\pmauthor{CWoo}{3771}
\pmtype{Definition}
\pmcomment{trigger rebuild}
\pmclassification{msc}{54E35}
\pmdefines{totally bounded}
\pmdefines{totally bounded uniformity}

\usepackage{amssymb,amscd}
\usepackage{amsmath}
\usepackage{amsfonts}

% used for TeXing text within eps files
%\usepackage{psfrag}
% need this for including graphics (\includegraphics)
%\usepackage{graphicx}
% for neatly defining theorems and propositions
\usepackage{amsthm}
% making logically defined graphics
%%\usepackage{xypic}
\usepackage{pst-plot}
\usepackage{psfrag}

% define commands here
\newtheorem{prop}{Proposition}
\newtheorem{thm}{Theorem}
\newtheorem{ex}{Example}
\newcommand{\real}{\mathbb{R}}
\begin{document}
A uniform space $X$ with uniformity $\mathcal{U}$ is called \emph{totally bounded} if for every entourage $U\in \mathcal{U}$, there is a finite cover $C_1,\ldots,C_n$ of $X$, such that $C_i\times C_i\in U$ for every $i=1,\ldots,n$.  $\mathcal{U}$ is called a totally bounded uniformity.

\textbf{Remark}.  A uniform space is compact (under the uniform topology) iff it is complete and totally bounded.

\begin{thebibliography}{9}
\bibitem{willard} S. Willard, \emph{General Topology},
Addison-Wesley, Publishing Company, 1970.
\end{thebibliography}
%%%%%
%%%%%
\end{document}
