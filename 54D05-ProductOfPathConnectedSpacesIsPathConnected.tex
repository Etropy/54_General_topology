\documentclass[12pt]{article}
\usepackage{pmmeta}
\pmcanonicalname{ProductOfPathConnectedSpacesIsPathConnected}
\pmcreated{2013-03-22 18:31:02}
\pmmodified{2013-03-22 18:31:02}
\pmowner{joking}{16130}
\pmmodifier{joking}{16130}
\pmtitle{product of path connected spaces is path connected}
\pmrecord{4}{41204}
\pmprivacy{1}
\pmauthor{joking}{16130}
\pmtype{Theorem}
\pmcomment{trigger rebuild}
\pmclassification{msc}{54D05}

% this is the default PlanetMath preamble.  as your knowledge
% of TeX increases, you will probably want to edit this, but
% it should be fine as is for beginners.

% almost certainly you want these
\usepackage{amssymb}
\usepackage{amsmath}
\usepackage{amsfonts}

% used for TeXing text within eps files
%\usepackage{psfrag}
% need this for including graphics (\includegraphics)
%\usepackage{graphicx}
% for neatly defining theorems and propositions
%\usepackage{amsthm}
% making logically defined graphics
%%%\usepackage{xypic}

% there are many more packages, add them here as you need them

% define commands here

\begin{document}
\textbf{Proposition}. Let $X$ and $Y$ be topological spaces. Then $X\times Y$ is path connected if and only if both $X$ and $Y$ are path connected.

\textit{Proof}. ''$\Leftarrow$'' Assume that $X$ and $Y$ are path connected and let $(x_1,y_1), (x_2,y_2)\in X\times Y$ be arbitrary points. Since $X$ is path connected, then there exists a continous map $$\sigma:\mathrm{I}\to X$$ such that $$\sigma(0)=x_1\ \ \mathrm{and}\ \ \sigma(1)=x_{2}.$$ Analogously there exists a continous map $$\tau:\mathrm{I}\to Y$$ such that $$\tau(0)=y_1\ \ \mathrm{and}\ \ \tau(1)=y_{2}.$$ Then we have an induced map $$\sigma\times\tau:\mathrm{I}\to X\times Y$$ defined by the formula: $$( \sigma\times\tau )(t)=(\sigma(t),\tau(t)),$$
which is continous path from $(x_1,y_1)$ to $(x_2,y_2)$.

''$\Rightarrow$'' On the other hand assume that $X\times Y$ is path connected. Let $x_1,x_2\in X$ and $y_0\in Y$. Then there exists a path $$\sigma:\mathrm{I}\to X\times Y$$
such that $$\sigma(0)=(x_1,y_0)\ \ \mathrm{and}\ \ \sigma(1)=(x_1,y_0).$$ We also have the projection map $\pi: X\times Y\to X$ such that $\pi(x,y)=x$. Thus we have a map $$\tau:\mathrm{I}\to X$$ defined by the formula $$\tau(t)=\pi(\sigma(t)).$$
This is a continous path from $x_1$ to $x_2$, therefore $X$ is path connected. Analogously $Y$ is path connected. $\square$

%%%%%
%%%%%
\end{document}
