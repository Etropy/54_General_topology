\documentclass[12pt]{article}
\usepackage{pmmeta}
\pmcanonicalname{AlexandroffSpace}
\pmcreated{2013-03-22 18:45:41}
\pmmodified{2013-03-22 18:45:41}
\pmowner{joking}{16130}
\pmmodifier{joking}{16130}
\pmtitle{Alexandroff space}
\pmrecord{5}{41541}
\pmprivacy{1}
\pmauthor{joking}{16130}
\pmtype{Definition}
\pmcomment{trigger rebuild}
\pmclassification{msc}{54A05}

% this is the default PlanetMath preamble.  as your knowledge
% of TeX increases, you will probably want to edit this, but
% it should be fine as is for beginners.

% almost certainly you want these
\usepackage{amssymb}
\usepackage{amsmath}
\usepackage{amsfonts}

% used for TeXing text within eps files
%\usepackage{psfrag}
% need this for including graphics (\includegraphics)
%\usepackage{graphicx}
% for neatly defining theorems and propositions
%\usepackage{amsthm}
% making logically defined graphics
%%%\usepackage{xypic}

% there are many more packages, add them here as you need them

% define commands here

\begin{document}
Topological space $X$ is called \textit{Alexandroff} if the intersection of every family of open sets is open.

Of course every finite topological space is Alexandroff, but there are also bigger Alexandroff spaces. For example let $\mathbb{R}$ denote the set of real numbers and let $\tau=\{[a,\infty)\ |\ a\in\mathbb{R}\} \cup \{(b,\infty)\ |\ b\in\mathbb{R}\}$. Then $\tau$ is a topology on $\mathbb{R}$ and $(\mathbb{R},\tau)$ is an Alexandroff space.

If $X$ is an Alexandroff space and $A\subseteq X$, then we may talk about smallest open neighbourhood of $A$. Indeed, let $$A^{o}=\bigcap \{U\subseteq X\ |\ U\mbox{ is open and }A\mbox{ is contained in }U\}.$$
Then $A^{o}$ is open.
%%%%%
%%%%%
\end{document}
