\documentclass[12pt]{article}
\usepackage{pmmeta}
\pmcanonicalname{PseudocompactSpace}
\pmcreated{2013-03-22 14:20:36}
\pmmodified{2013-03-22 14:20:36}
\pmowner{yark}{2760}
\pmmodifier{yark}{2760}
\pmtitle{pseudocompact space}
\pmrecord{7}{35815}
\pmprivacy{1}
\pmauthor{yark}{2760}
\pmtype{Definition}
\pmcomment{trigger rebuild}
\pmclassification{msc}{54D30}
\pmsynonym{pseudo compact space}{PseudocompactSpace}
\pmsynonym{pseudo-compact space}{PseudocompactSpace}
\pmrelated{LimitPointCompact}
\pmdefines{pseudocompact}
\pmdefines{pseudocompactness}
\pmdefines{pseudo-compact}
\pmdefines{pseudo-compactness}
\pmdefines{pseudo compact}
\pmdefines{pseudo compactness}

\endmetadata

\usepackage{amssymb}
\usepackage{amsmath}
\usepackage{amsfonts}

\renewcommand{\le}{\leqslant}
\renewcommand{\ge}{\geqslant}
\renewcommand{\leq}{\leqslant}
\renewcommand{\geq}{\geqslant}

\def\R{\mathbb{R}}
\begin{document}
A topological space $X$ is said to be \emph{pseudocompact} if every continuous function $f\colon X\to\R$ has bounded image.

All countably compact spaces (which includes all compact spaces and all sequentially compact spaces) are pseudocompact.
A metric space is pseudocompact if and only if it is compact.
A Hausdorff normal space is pseudocompact if and only if it is countably compact.

%%%%%
%%%%%
\end{document}
