\documentclass[12pt]{article}
\usepackage{pmmeta}
\pmcanonicalname{Closure}
\pmcreated{2013-03-22 12:05:40}
\pmmodified{2013-03-22 12:05:40}
\pmowner{mathwizard}{128}
\pmmodifier{mathwizard}{128}
\pmtitle{closure}
\pmrecord{9}{31191}
\pmprivacy{1}
\pmauthor{mathwizard}{128}
\pmtype{Definition}
\pmcomment{trigger rebuild}
\pmclassification{msc}{54A99}
%\pmkeywords{topology}
\pmrelated{ClosureAxioms}
\pmrelated{Interior}

\usepackage{amssymb}
\usepackage{amsmath}
\usepackage{amsfonts}
\usepackage{graphicx}
%%%\usepackage{xypic}
\begin{document}
The \emph{closure} $\overline{A}$ of a subset $A$ of a topological space $X$ is the intersection of all closed sets containing $A$.

Equivalently, $\overline{A}$ consists of $A$ together with all limit points of $A$ in $X$ or equivalently $x\in\overline{A}$ if and only if every neighborhood of $x$ intersects $A$. Sometimes the notation $\operatorname{cl}(A)$ is used.

If it is not clear, which topological space is used, one writes $\overline{A}^X$. Note that if $Y$ is a subspace of $X$, then $\overline{A}^X$ may not be the same as $\overline{A}^Y$.  For example, if $X=\mathbb{R}$, $Y=(0,1)$ and $A=(0,1)$, then $\overline{A}^X=[0,1]$ while $\overline{A}^Y=(0,1)$.
%%%%%
%%%%%
%%%%%
\end{document}
