\documentclass[12pt]{article}
\usepackage{pmmeta}
\pmcanonicalname{CompactSubspaceOfAHausdorffSpaceIsClosed}
\pmcreated{2013-03-22 16:31:23}
\pmmodified{2013-03-22 16:31:23}
\pmowner{ehremo}{15714}
\pmmodifier{ehremo}{15714}
\pmtitle{compact subspace of a Hausdorff space is closed}
\pmrecord{7}{38702}
\pmprivacy{1}
\pmauthor{ehremo}{15714}
\pmtype{Proof}
\pmcomment{trigger rebuild}
\pmclassification{msc}{54D30}
%\pmkeywords{compact subspace subset Hausdorff space closed}

\endmetadata

% this is the default PlanetMath preamble.  as your knowledge
% of TeX increases, you will probably want to edit this, but
% it should be fine as is for beginners.

% almost certainly you want these
\usepackage{amssymb}
\usepackage{amsmath}
\usepackage{amsfonts}

% used for TeXing text within eps files
%\usepackage{psfrag}
% need this for including graphics (\includegraphics)
%\usepackage{graphicx}
% for neatly defining theorems and propositions
%\usepackage{amsthm}
% making logically defined graphics
%%%\usepackage{xypic}

% there are many more packages, add them here as you need them

% define commands here

\begin{document}
Let $X$ be a Hausdorff space, and $Y$ be a compact subspace of $X$. We prove that $X\setminus Y$ is open, by finding for every point $x \in X \setminus Y$ a neighborhood $U_x$ disjoint from $Y$.

Let $y \in Y$. $x \neq y$, so by the definition of a Hausdorff space, there exist open neighborhoods $U_x^{(y)}$ of $x$ and $V_x^{(y)}$ of $y$ such that $U_x^{(y)} \cap V_x^{(y)} = \emptyset$. Clearly
$$Y \subseteq \bigcup_{y \in Y} V_x^{(y)}$$
but since $Y$ is compact, we can select from these a finite subcover of $Y$
$$Y \subseteq V_x^{(y_1)} \cup \cdots \cup V_x^{(y_n)}$$
Now for every $y \in Y$ there exists $k \in 1...n$ such that $y \in V_x^{(y_k)}$. Since $U_x^{(y_k)}$ and $V_x^{(y_k)}$ are disjoint, $y \not\in U_x^{(y_k)}$, therefore neither is it in the intersection
$$U_x = \bigcap_{j=1}^n U_x^{(y_j)}$$
A finite intersection of open sets is open, hence $U_x$ is a neighborhood of $x$ disjoint from $Y$.
%%%%%
%%%%%
\end{document}
