\documentclass[12pt]{article}
\usepackage{pmmeta}
\pmcanonicalname{ClosureMap}
\pmcreated{2013-03-22 18:53:55}
\pmmodified{2013-03-22 18:53:55}
\pmowner{CWoo}{3771}
\pmmodifier{CWoo}{3771}
\pmtitle{closure map}
\pmrecord{6}{41746}
\pmprivacy{1}
\pmauthor{CWoo}{3771}
\pmtype{Definition}
\pmcomment{trigger rebuild}
\pmclassification{msc}{54A05}
\pmclassification{msc}{06A15}
\pmsynonym{closure}{ClosureMap}
\pmsynonym{closure function}{ClosureMap}
\pmsynonym{closure operator}{ClosureMap}
\pmdefines{dual closure}
\pmdefines{fixed point}

\endmetadata

\usepackage{amssymb,amscd}
\usepackage{amsmath}
\usepackage{amsfonts}
\usepackage{mathrsfs}

% used for TeXing text within eps files
%\usepackage{psfrag}
% need this for including graphics (\includegraphics)
%\usepackage{graphicx}
% for neatly defining theorems and propositions
\usepackage{amsthm}
% making logically defined graphics
%%\usepackage{xypic}
\usepackage{pst-plot}

% define commands here
\newcommand*{\abs}[1]{\left\lvert #1\right\rvert}
\newtheorem{prop}{Proposition}
\newtheorem{thm}{Theorem}
\newtheorem{ex}{Example}
\newcommand{\real}{\mathbb{R}}
\newcommand{\pdiff}[2]{\frac{\partial #1}{\partial #2}}
\newcommand{\mpdiff}[3]{\frac{\partial^#1 #2}{\partial #3^#1}}
\begin{document}
Let $P$ be a poset.  A function $c:P \to P$ is called a \emph{closure map} if
\begin{itemize}
\item $c$ is order preserving,
\item $1_P \le c$,
\item $c$ is idempotent: $c\circ c = c$.
\end{itemize}

If the second condition is changed to $c\le 1_P$, then $c$ is called a \emph{dual closure map} on $P$.

For example, the real function $f$ such that $f(r)$ is the least integer greater than or equal to $r$ is a closure map (see Archimedean property).  The rounding function $[\cdot]$ is an example of a dual closure map.

A \emph{fixed point} of a closure map $c$ on $P$ is an element $x\in P$ such that $c(x)=x$.  It is evident that every image point of $c$ is a fixed point: for if $x=c(a)$ for some $a\in P$, then $c(x)=c(c(a))=c(a)=x$.

In the example above, any integer is a fixed point of $f$.

Every closure map can be characterized by an interesting decomposition property: $c: P\to P$ is a closure map iff there is a set $Q$ and a residuated function $f: P\to Q$ such that $c=f^+\circ f$, where $f^+$ denotes the residual of $f$.

Again, in the example above, $f=g^+\circ g$, where $g: \mathbb{R}\to\mathbb{Z}$ is the function taking any real number $r$ to the largest integer smaller than $r$.  $g$ is residuated, and its residual is $g^+(x)=x+1$.

\textbf{Remark}.  Closure maps are generalizations to closure operator on a set (see the parent entry).  Indeed, any closure operator on a set $X$ takes a subset $A$ of $X$ to a subset $A^c$ of $X$ satisfying the closure axioms, where Axiom 2 corresponds to condition 2 above, and Axiom 3 says the operator is idempotent.  To see that the operator is order preserving, suppose $A\subseteq B$.  Then $B^c = (A\cup B)^c =A^c \cup B^c$ by Axiom 4, and hence $A^c\subseteq B^c$.  Axiom 1 says that the empty set $\varnothing$ is a fixed point of the operator.  However, in general, this is not the case, for $P$ may not even have a minimal element, as indicated by the above example.

\begin{thebibliography}{6}
\bibitem{tsb} T.S. Blyth, {\em Lattices and Ordered Algebraic Structures}, Springer, New York (2005).
\end{thebibliography}
%%%%%
%%%%%
\end{document}
