\documentclass[12pt]{article}
\usepackage{pmmeta}
\pmcanonicalname{T0Space}
\pmcreated{2013-03-22 12:18:12}
\pmmodified{2013-03-22 12:18:12}
\pmowner{yark}{2760}
\pmmodifier{yark}{2760}
\pmtitle{T0 space}
\pmrecord{13}{31850}
\pmprivacy{1}
\pmauthor{yark}{2760}
\pmtype{Definition}
\pmcomment{trigger rebuild}
\pmclassification{msc}{54D10}
\pmsynonym{Kolmogorov space}{T0Space}
\pmsynonym{Kolmogoroff space}{T0Space}
%\pmkeywords{Topology}
\pmrelated{Ball}
\pmrelated{T1Space}
\pmrelated{T2Space}
\pmrelated{RegularSpace}
\pmrelated{T3Space}
\pmdefines{T0}

%\usepackage{graphicx}
%%%\usepackage{xypic} 
\usepackage{bbm}
\newcommand{\Z}{\mathbbmss{Z}}
\newcommand{\C}{\mathbbmss{C}}
\newcommand{\R}{\mathbbmss{R}}
\newcommand{\Q}{\mathbbmss{Q}}
\newcommand{\mathbb}[1]{\mathbbmss{#1}}
\begin{document}
A topological space $(X,\tau)$ is said to be $T_0$
(or to satisfy the $T_0$ axiom )
if for all distinct $x,y\in X$
there exists an open set $U\in\tau$ such that
either $x\in U$ and $y\notin U$ or $x\notin U$ and $y\in U$.

All \PMlinkname{$T_1$ spaces}{T1Space} are $T_0$.
An example of $T_0$ space that is not $T_1$
is the $2$-point Sierpinski space.
%%%%%
%%%%%
\end{document}
