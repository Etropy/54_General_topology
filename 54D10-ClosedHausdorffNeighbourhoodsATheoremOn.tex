\documentclass[12pt]{article}
\usepackage{pmmeta}
\pmcanonicalname{ClosedHausdorffNeighbourhoodsATheoremOn}
\pmcreated{2013-03-22 18:30:46}
\pmmodified{2013-03-22 18:30:46}
\pmowner{yark}{2760}
\pmmodifier{yark}{2760}
\pmtitle{closed Hausdorff neighbourhoods, a theorem on}
\pmrecord{7}{41198}
\pmprivacy{1}
\pmauthor{yark}{2760}
\pmtype{Theorem}
\pmcomment{trigger rebuild}
\pmclassification{msc}{54D10}

\endmetadata

\usepackage{amssymb}

\def\emptyset{\varnothing}

\begin{document}
\PMlinkescapeword{proof}
\PMlinkescapeword{QED}

{\bf Theorem.}
If $X$ is a topological space in which
every point has a closed Hausdorff neighbourhood,
then $X$ is Hausdorff.

{\bf Note.}
In this theorem (and the proof that follows)
neighbourhoods are not assumed to be open.
That is, a neighbourhood of a point $x$
is a set $A$ such that $x$ lies in the interior of $A$.

{\bf Proof of theorem.}
Let $X$ be a topological space in which
every point has a closed Hausdorff neighbourhood.
Suppose $a,b\in X$ are distinct.
It suffices to show that $a$ and $b$ have disjoint neighbourhoods.
By assumption, there is a closed Hausdorff neighbourhood $N$ of $b$.
If $a\notin N$, then $X\setminus N$ and $N$
are disjoint neighbourhoods of $a$ and $b$ (as $N$ is closed).

So suppose $a\in N$.
As $N$ is Hausdorff,
there are disjoint sets $U_0,V_0\subseteq N$
that are open in $N$, such that $a\in U_0$ and $b\in V_0$.
There are open sets $U$ and $V$ of $X$
such that $U_0=U\cap N$ and $V_0=V\cap N$.
Note that $U$ is a neighbourhood of $a$, and $V$ is a neighbourhood of $b$.
As $N$ is a neighbourhood of $b$,
it follows that $V\cap N$ (that is, $V_0$) is a neighbourhood of $b$.
We have $U\cap V_0=U_0\cap V_0=\emptyset$.
So $U$ and $V_0$ are disjoint neighbourhoods of $a$ and $b$.
QED.
%%%%%
%%%%%
\end{document}
