\documentclass[12pt]{article}
\usepackage{pmmeta}
\pmcanonicalname{SmirnovMetrizationTheorem}
\pmcreated{2013-03-22 18:01:00}
\pmmodified{2013-03-22 18:01:00}
\pmowner{rm50}{10146}
\pmmodifier{rm50}{10146}
\pmtitle{Smirnov metrization theorem}
\pmrecord{7}{40532}
\pmprivacy{1}
\pmauthor{rm50}{10146}
\pmtype{Theorem}
\pmcomment{trigger rebuild}
\pmclassification{msc}{54E35}
\pmdefines{locally metrizable}

% this is the default PlanetMath preamble.  as your knowledge
% of TeX increases, you will probably want to edit this, but
% it should be fine as is for beginners.

% almost certainly you want these
\usepackage{amssymb}
\usepackage{amsmath}
\usepackage{amsfonts}

% used for TeXing text within eps files
%\usepackage{psfrag}
% need this for including graphics (\includegraphics)
%\usepackage{graphicx}
% for neatly defining theorems and propositions
%\usepackage{amsthm}
% making logically defined graphics
%%%\usepackage{xypic}

% there are many more packages, add them here as you need them

% define commands here

\begin{document}
The Smirnov metrization theorem establishes necessary and sufficient conditions for a topological space to be metrizable. The theorem reduces questions of metrizability to paracompactness and a local metrizability condition.

\textbf{Definition:}\ A space $X$ is \emph{locally metrizable} if every point $x\in X$ has a neighborhood that is metrizable in the subspace topology.

\textbf{Theorem (Smirnov metrization theorem):}\ A space is metrizable if and only if it is paracompact and locally metrizable.
%%%%%
%%%%%
\end{document}
