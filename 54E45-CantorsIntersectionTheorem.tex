\documentclass[12pt]{article}
\usepackage{pmmeta}
\pmcanonicalname{CantorsIntersectionTheorem}
\pmcreated{2013-03-22 15:12:35}
\pmmodified{2013-03-22 15:12:35}
\pmowner{paolini}{1187}
\pmmodifier{paolini}{1187}
\pmtitle{Cantor's Intersection Theorem}
\pmrecord{4}{36970}
\pmprivacy{1}
\pmauthor{paolini}{1187}
\pmtype{Theorem}
\pmcomment{trigger rebuild}
\pmclassification{msc}{54E45}

\endmetadata

% this is the default PlanetMath preamble.  as your knowledge
% of TeX increases, you will probably want to edit this, but
% it should be fine as is for beginners.

% almost certainly you want these
\usepackage{amssymb}
\usepackage{amsmath}
\usepackage{amsfonts}

% used for TeXing text within eps files
%\usepackage{psfrag}
% need this for including graphics (\includegraphics)
%\usepackage{graphicx}
% for neatly defining theorems and propositions
\usepackage{amsthm}
% making logically defined graphics
%%%\usepackage{xypic}

% there are many more packages, add them here as you need them

% define commands here
\newcommand{\R}{\mathbb R}
\newtheorem{theorem}{Theorem}
\newtheorem{definition}{Definition}
\theoremstyle{remark}
\newtheorem{example}{Example}
\begin{document}
\begin{theorem}
Let $K_1\supset K_2\supset K_3 \supset \ldots \supset K_n\supset \ldots$ be a sequence of non-empty, compact subsets of a metric space $X$. Then the intersection $\bigcap_i K_i$ is not empty.
\end{theorem}
\begin{proof}
Choose a point $x_i\in K_i$ for every $i=1,2,\ldots$
Since $x_i \in K_i \subset K_1$ is a sequence in a compact set, by Bolzano-Weierstrass Theorem, there exists a subsequence $x_{i_j}$ which converges to a point $x\in K_1$. Notice, however, that for a fixed index $n$, the sequence $x_{i_j}$ lies in $K_n$ for all $j$ sufficiently large (namely for all $j$ such that $i_j>n$). So one has $x\in K_n$. 
Since this is true for every $n$, the result follows.
\end{proof}
%%%%%
%%%%%
\end{document}
