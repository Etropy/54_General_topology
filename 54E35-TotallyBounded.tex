\documentclass[12pt]{article}
\usepackage{pmmeta}
\pmcanonicalname{TotallyBounded}
\pmcreated{2013-03-22 13:09:54}
\pmmodified{2013-03-22 13:09:54}
\pmowner{Mathprof}{13753}
\pmmodifier{Mathprof}{13753}
\pmtitle{totally bounded}
\pmrecord{12}{33608}
\pmprivacy{1}
\pmauthor{Mathprof}{13753}
\pmtype{Definition}
\pmcomment{trigger rebuild}
\pmclassification{msc}{54E35}
%\pmkeywords{bounded}
%\pmkeywords{totally}
%\pmkeywords{totally bounded}
%\pmkeywords{total}
%\pmkeywords{bound}
%\pmkeywords{finite bound}
\pmrelated{MetricSpace}
\pmrelated{Bounded}
\pmrelated{Subset}

% this is the default PlanetMath preamble.  as your knowledge
% of TeX increases, you will probably want to edit this, but
% it should be fine as is for beginners.

% almost certainly you want these
\usepackage{amssymb}
\usepackage{amsmath}
\usepackage{amsfonts}

% used for TeXing text within eps files
%\usepackage{psfrag}
% need this for including graphics (\includegraphics)
%\usepackage{graphicx}
% for neatly defining theorems and propositions
%\usepackage{amsthm}
% making logically defined graphics
%%%\usepackage{xypic}

% there are many more packages, add them here as you need them

% define commands here

\begin{document}
Let $A$ be a subset of a topological vector space $X$.

$A$ is called \emph{totally bounded} if , for each neighborhood $G$ of 0,
there exists a finite subset $S$ of $A$ with $A$ contained in the sumset $S + G$.

The definition can be restated in the following form when $X$ is a metric space:

A set $A \subseteq X$ is said to be {\em totally bounded} if  for every $\epsilon>0$, there exists a finite subset $\{s_1,s_2,\ldots ,s_n\}$ of $A$ such that $A\subseteq \bigcup _{k=1} ^n B(s_k,\epsilon )$, where $B(s_k,\epsilon)$ denotes the open ball around $s_k$ with radius $\epsilon$.


 

\begin{thebibliography}{9}
\bibitem{bachman} G. Bachman, L. Narici, \emph{Functional analysis}, Academic Press, 1966.
\bibitem{wil} A. Wilansky, \emph{Functional Analysis}, Blaisdell Publishing Co., 1964
\bibitem{rudin} W. Rudin, \emph{Functional Analysis}, 2nd ed. McGraw-Hill , 1973
\end{thebibliography}

%%%%%
%%%%%
\end{document}
