\documentclass[12pt]{article}
\usepackage{pmmeta}
\pmcanonicalname{UniformNeighborhood}
\pmcreated{2013-03-22 16:42:31}
\pmmodified{2013-03-22 16:42:31}
\pmowner{CWoo}{3771}
\pmmodifier{CWoo}{3771}
\pmtitle{uniform neighborhood}
\pmrecord{6}{38924}
\pmprivacy{1}
\pmauthor{CWoo}{3771}
\pmtype{Definition}
\pmcomment{trigger rebuild}
\pmclassification{msc}{54E15}
\pmrelated{TopologyInducedByAUniformStructure}

\usepackage{amssymb,amscd}
\usepackage{amsmath}
\usepackage{amsfonts}

% used for TeXing text within eps files
%\usepackage{psfrag}
% need this for including graphics (\includegraphics)
%\usepackage{graphicx}
% for neatly defining theorems and propositions
\usepackage{amsthm}
% making logically defined graphics
%%\usepackage{xypic}
\usepackage{pst-plot}
\usepackage{psfrag}

% define commands here

\begin{document}
Let $X$ be a uniform space with uniformity $\mathcal{U}$.  For each $x\in X$ and $U\in \mathcal{U}$, define the following items
\begin{itemize}
\item $U[x]:=\lbrace y\mid (x,y)\in U\rbrace$, and
\item $\mathfrak{N}_x:=\lbrace (x,U[x])\mid U\in \mathcal{U}\rbrace$
\item $\mathfrak{N}=\bigcup_{x\in X} \mathfrak{N}_x$.
\end{itemize}
\textbf{Proposition}.  $\mathfrak{N}_x$ is the abstract neighborhood system around $x$, hence $\mathfrak{N}$ is the abstract neighborhood system of $X$.
\begin{proof}  We show that all five defining conditions of a neighborhood system on a set are met:
\begin{enumerate}
\item
For each $(x,U[x])\in \mathfrak{N}$, $x\in U[x]$, since every entourage contains the diagonal relation.
\item
Every $x\in X$ and every entourage $U\in \mathcal{U}$, $U[x]\subseteq X$ with $(x,U[x])\in \mathfrak{N}$
\item
Suppose $(x,U[x])\in \mathfrak{N}$ and $U[x]\subseteq Y\subseteq X$.  Showing that $(x,Y)\in\mathfrak{N}$ amounts to showing $Y=V[x]$ for some $V\in \mathcal{U}$.  First, note that each entourage $U$ can be decomposed into disjoint union of sets ``slices'' of the form $\lbrace a\rbrace \times U[a]$.  We replace the ``slice'' $\lbrace x\rbrace \times U[x]$ by $\lbrace x\rbrace \times Y$.  The resulting disjoint union is a set $V$, which is a superset of $U$.  Since $\mathcal{U}$ is a filter, $V\in \mathcal{U}$.  Furthermore, $V[x]=Y$.
\item $a\in U[x]\cap V[x]$ iff $(x,a)\in U\cap V$ iff $a\in (U\cap V)[x]$.  This implies that if $(x,U[x]),(x,V[x])\in \mathfrak{N}$, then $(x,U[x]\cap V[x]) = (x,(U\cap V)[x])\in \mathfrak{N}$.
\item Suppose $(x,U[x])\in \mathfrak{N}$.  There is $V\in\mathcal{U}$ such that $(V\circ V)[x]\subseteq U[x]$.  We show that $V[x]\subseteq X$ is what we want.  Clearly, $x\in V[x]$.  For any $y\in V[x]$, and any $a\in V[y]$, we have $(x,a)=(x,y)\circ (y,a)\in V\circ V$, or $a\in (V\circ V)[x]\subseteq U[x]$.  So $V[y]\subseteq U[x]$ for any $y\in V[x]$.  In order to show that $(y,U[x])\in \mathfrak{N}$, we must find $W \in \mathcal{U}$ such that $U[x]=W[y]$.  By the third step above, since $V[y]\subseteq U[x]$, there is $W\in \mathcal{U}$ with $W[y]=U[x]$.  Thus $(y,U[x])=(y,W[y])\in \mathfrak{N}$.
\end{enumerate}
\end{proof}

\textbf{Definition}.  For each $x$ in a uniform space $X$ with uniformity $\mathcal{U}$, a \emph{uniform neighborhood} of $x$ is a set $U[x]$ for some entourage $U\in\mathcal{U}$.  In general, for any $A\subseteq X$, the set $$U[A]:=\lbrace y \in X \mid (x,y)\in U\mbox{ for some }x\in A\rbrace $$ is called a \emph{uniform neighborhood} of $A$.

Two immediate properties that we have already seen in the proof above are: (1). for each $U\in\mathcal{U}$, $x\in U[x]$; and (2). $U[x]\cap V[x]=(U\cap V)[x]$.  More generally, $\bigcap U_i[x]=(\bigcap U_i)[x]$.

\textbf{Remark}.  If we define $T_{\mathcal{U}}:=\lbrace A\subseteq X\mid \forall x\in A, \exists U\in \mathcal{U}\mbox{ such that }U[x]\subseteq A\rbrace$, then $T_{\mathcal{U}}$ is a \PMlinkname{topology induced by the uniform structure}{TopologyInducedByAUniformStructure} $\mathcal{U}$.  Under this topology, uniform neighborhoods are synonymous with neighborhoods.

%%%%%
%%%%%
\end{document}
