\documentclass[12pt]{article}
\usepackage{pmmeta}
\pmcanonicalname{LocallyHomeomorphic}
\pmcreated{2013-03-22 15:14:34}
\pmmodified{2013-03-22 15:14:34}
\pmowner{GrafZahl}{9234}
\pmmodifier{GrafZahl}{9234}
\pmtitle{locally homeomorphic}
\pmrecord{4}{37020}
\pmprivacy{1}
\pmauthor{GrafZahl}{9234}
\pmtype{Definition}
\pmcomment{trigger rebuild}
\pmclassification{msc}{54-00}
\pmsynonym{local homeomorphy}{LocallyHomeomorphic}
\pmrelated{LocallyEuclidean}

\endmetadata

% this is the default PlanetMath preamble.  as your knowledge
% of TeX increases, you will probably want to edit this, but
% it should be fine as is for beginners.

% almost certainly you want these
\usepackage{amssymb}
\usepackage{amsmath}
\usepackage{amsfonts}

% used for TeXing text within eps files
%\usepackage{psfrag}
% need this for including graphics (\includegraphics)
%\usepackage{graphicx}
% for neatly defining theorems and propositions
%\usepackage{amsthm}
% making logically defined graphics
%%%\usepackage{xypic}

% there are many more packages, add them here as you need them

% define commands here
\newcommand{\Prod}{\prod\limits}
\newcommand{\Sum}{\sum\limits}
\newcommand{\mbb}{\mathbb}
\newcommand{\mbf}{\mathbf}
\newcommand{\mc}{\mathcal}
\newcommand{\ol}{\overline}

% Math Operators/functions
\DeclareMathOperator{\Frob}{Frob}
\DeclareMathOperator{\cwe}{cwe}
\DeclareMathOperator{\we}{we}
\DeclareMathOperator{\wt}{wt}
\begin{document}
Let $X$ and $Y$ be topological spaces. Then $X$ is \emph{locally
  homeomorphic} to $Y$, if for every $x\in X$ there is a neighbourhood
  $U\subseteq X$ of $x$ and an \PMlinkid{open set}{380} $V\subseteq Y$, such that $U$
  and $V$ with their respective subspace topology are homeomorphic.

\subsubsection*{Examples}
\begin{itemize}
\item Let $X=\{1\}$ and $Y=\{2,3\}$ be discrete spaces with one resp.\
  two elements. Since $X$ and $Y$ have different cardinalities,
  they cannot be homeomorphic. They are, however, locally homeomorphic
  to each other.
\item Again, let $X=\{1\}$ be a discrete space with one element, but
  now let $Y=\{2,3\}$ the space with topology
  $\{\emptyset,\{2\},Y\}$. Then $X$ is still locally homeomorphic to
  $Y$, but $Y$ is not locally homeomorphic to $X$, since the smallest
  neighbourhood of $3$ already has more elements than $X$.
\item Now, let $X$ be as in the previous examples, and $Y=\{2,3\}$
  be \PMlinkid{indiscrete}{3120}. Then neither $X$ is locally homeomorphic to $Y$ nor
  the other way round.
\item Non-trivial examples arise with locally Euclidean spaces,
  especially manifolds.
\end{itemize}
%%%%%
%%%%%
\end{document}
