\documentclass[12pt]{article}
\usepackage{pmmeta}
\pmcanonicalname{LastNonzeroDigitOfFactorial}
\pmcreated{2013-03-24 0:23:36}
\pmmodified{2013-03-24 0:23:36}
\pmowner{rspuzio}{6075}
\pmmodifier{rspuzio}{6075}
\pmtitle{last non-zero digit of factorial}
\pmrecord{4}{87304}
\pmprivacy{1}
\pmauthor{rspuzio}{6075}
\pmtype{Definition}
\pmcomment{tgf}

% this is the default PlanetMath preamble.  as your knowledge
% of TeX increases, you will probably want to edit this, but
% it should be fine as is for beginners.

% almost certainly you want these
\usepackage{amssymb}
\usepackage{amsmath}
\usepackage{amsfonts}

% need this for including graphics (\includegraphics)
\usepackage{graphicx}
% for neatly defining theorems and propositions
\usepackage{amsthm}

% making logically defined graphics
%\usepackage{xypic}
% used for TeXing text within eps files
%\usepackage{psfrag}

% there are many more packages, add them here as you need them

% define commands here

\begin{document}
We will show how to compute the last non-zero
digit of the factorial of a number from its
digits without having to compute the factorial itself.

Let $L(n)$ denote the last non-zero digit of $n$ in base 10.
We note some basic properties of $L$ which can easily
be checked:
\begin{itemize}
\item For all $n$, we have $L(10n) = L(n)$.
\item If $L(m) \neq 5$ and $L(n) \neq 5$, then $L(mn) = L (L(m) L(n))$.
\end{itemize}

We also tabulate the values of $L(n!)$ for small values of $n$:
\begin{eqnarray*}
L(0!) &=& 1 \\
L(1!) &=& 1 \\
L(2!) &=& 2 \\
L(3!) &=& 6 \\
L(4!) &=& 4 \\
L(5!) &=& 2 \\
L(6!) &=& 2 \\
L(7!) &=& 4 \\
L(8!) &=& 2 \\
L(9!) &=& 8 \\
L(10!) &=& 8 
\end{eqnarray*}

Next, we make two less trivial observations:

\begin{theorem}
For all positive integers $n$, we have $L(n!) \neq 5$.
\end{theorem}
\end{document}
