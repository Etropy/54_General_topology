\documentclass[12pt]{article}
\usepackage{pmmeta}
\pmcanonicalname{GluingTogetherContinuousFunctions}
\pmcreated{2013-03-22 15:17:20}
\pmmodified{2013-03-22 15:17:20}
\pmowner{yark}{2760}
\pmmodifier{yark}{2760}
\pmtitle{gluing together continuous functions}
\pmrecord{16}{37081}
\pmprivacy{1}
\pmauthor{yark}{2760}
\pmtype{Theorem}
\pmcomment{trigger rebuild}
\pmclassification{msc}{54C05}

\usepackage{amsfonts}
\usepackage{amsthm}

\def\R{\mathbb{R}}
\newtheorem{thm}{Theorem}

\begin{document}
\PMlinkescapeword{closed}
\PMlinkescapephrase{closed sets}
\PMlinkescapeword{domains}
\PMlinkescapeword{intersect}
\PMlinkescapeword{necessary}
\PMlinkescapeword{obvious}
\PMlinkescapeword{open}
\PMlinkescapephrase{open set}
\PMlinkescapephrase{open sets}
\PMlinkescapephrase{open subset}
\PMlinkescapeword{restriction}
\PMlinkescapeword{sufficient}
\PMlinkescapeword{theorem}
\PMlinkescapeword{theorems}
\PMlinkescapeword{union}

\section*{Introduction}

Suppose we have a collection $\cal S$ of subsets of a topological space $X$,
and for each $A\in\cal S$ we have a continuous function $f_A\colon A\to Y$,
where $Y$ is another topological space.
If the functions $f_A$ agree wherever their domains intersect,
then we can glue them together in the obvious way to form a new function
$f\colon\cup{\cal S}\to Y$.
The theorems in this entry
give some sufficient conditions for $f$ to be continuous.

\section*{Theorems}

\begin{thm}
Let $X$ and $Y$ be topological spaces,
let $\cal S$ be a locally finite collection of \PMlinkname{closed subsets}{ClosedSet} of $X$,
and let $f\colon\cup{\cal S}\to Y$ be a function
such that the \PMlinkname{restriction}{RestrictionOfAFunction} $f|_A$
is continuous for all $A\in\cal S$.
Then $f$ is continuous.
\end{thm}

\begin{thm}
Let $X$ and $Y$ be topological spaces,
let $\cal S$ be a collection of \PMlinkname{open subsets}{OpenSet} of $X$,
and let $f\colon\cup{\cal S}\to Y$ be a function
such that the restriction $f|_A$ is continuous for all $A\in\cal S$.
Then $f$ is continuous.
\end{thm}

\section*{Notes}

Note that the theorem for closed subsets
requires the collection to be locally finite.
To see that this is condition cannot be omitted,
notice that any function $f\colon\R\to\R$
restricts to a continuous function on each singleton,
yet need not be continuous itself.

\section*{Proofs}

The two theorems are proved in essentially in the same way,
but for the first theorem we need to make use of the fact that
the union of a locally finite collection of closed sets is closed.

{\bf Proof of Theorem 1.}
Let $C$ be a closed subset of $Y$.
Then $f^{-1}(C)=\bigcup_{A\in\cal S}(A\cap f^{-1}(C))
=\bigcup_{A\in\cal S}(f|_A)^{-1}(C)$.
By continuity, each $(f|_A)^{-1}(C)$ is closed in $A$.
But by assumption each $A$ is closed in $X$,
so it follows that each $(f|_A)^{-1}(C)$ is closed in $X$.
Thus $f^{-1}(C)$ is the union of a locally finite collection of closed sets,
and is therefore closed in $X$, and so closed in $\cup\cal S$.
So $f$ is continuous.
\qed

{\bf Proof of Theorem 2.}
Let $U$ be an open subset of $Y$.
Then $f^{-1}(U)=\bigcup_{A\in\cal S}(A\cap f^{-1}(U))
=\bigcup_{A\in\cal S}(f|_A)^{-1}(U)$.
By continuity, each $(f|_A)^{-1}(U)$ is open in $A$.
But by assumption each $A$ is open in $X$,
so it follows that each $(f|_A)^{-1}(U)$ is open in $X$.
Thus $f^{-1}(U)$ is the union of a collection of open sets,
and is therefore open in $X$, and so open in $\cup\cal S$.
So $f$ is continuous.
\qed
%%%%%
%%%%%
\end{document}
