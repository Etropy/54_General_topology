\documentclass[12pt]{article}
\usepackage{pmmeta}
\pmcanonicalname{ListOfCommonTopologies}
\pmcreated{2013-03-22 14:38:27}
\pmmodified{2013-03-22 14:38:27}
\pmowner{matte}{1858}
\pmmodifier{matte}{1858}
\pmtitle{list of common topologies}
\pmrecord{5}{36227}
\pmprivacy{1}
\pmauthor{matte}{1858}
\pmtype{Topic}
\pmcomment{trigger rebuild}
\pmclassification{msc}{54-00}
\pmclassification{msc}{55-00}
\pmclassification{msc}{22-00}

% this is the default PlanetMath preamble.  as your knowledge
% of TeX increases, you will probably want to edit this, but
% it should be fine as is for beginners.

% almost certainly you want these
\usepackage{amssymb}
\usepackage{amsmath}
\usepackage{amsfonts}
\usepackage{amsthm}

% used for TeXing text within eps files
%\usepackage{psfrag}
% need this for including graphics (\includegraphics)
%\usepackage{graphicx}
% for neatly defining theorems and propositions
%
% making logically defined graphics
%%%\usepackage{xypic}

% there are many more packages, add them here as you need them

% define commands here

\newcommand{\sR}[0]{\mathbb{R}}
\newcommand{\sC}[0]{\mathbb{C}}
\newcommand{\sN}[0]{\mathbb{N}}
\newcommand{\sZ}[0]{\mathbb{Z}}

 \usepackage{bbm}
 \newcommand{\Z}{\mathbbmss{Z}}
 \newcommand{\C}{\mathbbmss{C}}
 \newcommand{\R}{\mathbbmss{R}}
 \newcommand{\Q}{\mathbbmss{Q}}



\newcommand*{\norm}[1]{\lVert #1 \rVert}
\newcommand*{\abs}[1]{| #1 |}



\newtheorem{thm}{Theorem}
\newtheorem{defn}{Definition}
\newtheorem{prop}{Proposition}
\newtheorem{lemma}{Lemma}
\newtheorem{cor}{Corollary}
\begin{document}
The aim of this entry is to give an overview of topologies
commonly used in mathematics with links to the
entries at Planetmath where these are discussed in greater detail. 

\subsubsection*{Topologies on arbitrary sets}
The below topologies show that any set can be endowed with a topology. 
\begin{enumerate}
\item discrete topology
\item indiscrete topology
\item cofinite topology
\item cocountable topology
\end{enumerate}

\subsubsection*{New topologies from old ones}
\begin{enumerate}
\item subspace topology
\item quotient topology
\item box topology
\item product topology
\item inductive limit topology
\end{enumerate}

\subsubsection*{Topologies on sets with more structure}
\begin{enumerate}
\item order topology (standard topologies on $\sZ, \sN, \sR$)
\item metric topology
\item seminorm topology (topological vector spaces)
\item graph topology (graph theory)
\item Zariski topology (algebraic geometry)
\end{enumerate}

\subsubsection*{Topologies for functions spaces and mappings}
An extensive list of \PMlinkescapetext{function spaces} can be 
found \PMlinkname{here}{FunctionSpaces}.
\begin{enumerate}
\item Whitney topology (differential geometry)
\item Compact-open topology
\end{enumerate}
%%%%%
%%%%%
\end{document}
