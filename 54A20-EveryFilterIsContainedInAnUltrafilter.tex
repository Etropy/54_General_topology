\documentclass[12pt]{article}
\usepackage{pmmeta}
\pmcanonicalname{EveryFilterIsContainedInAnUltrafilter}
\pmcreated{2013-03-22 14:41:41}
\pmmodified{2013-03-22 14:41:41}
\pmowner{rspuzio}{6075}
\pmmodifier{rspuzio}{6075}
\pmtitle{every filter is contained in an ultrafilter}
\pmrecord{6}{36304}
\pmprivacy{1}
\pmauthor{rspuzio}{6075}
\pmtype{Theorem}
\pmcomment{trigger rebuild}
\pmclassification{msc}{54A20}
\pmrelated{LindenbaumsLemma}

% this is the default PlanetMath preamble.  as your knowledge
% of TeX increases, you will probably want to edit this, but
% it should be fine as is for beginners.

% almost certainly you want these
\usepackage{amssymb}
\usepackage{amsmath}
\usepackage{amsfonts}

% used for TeXing text within eps files
%\usepackage{psfrag}
% need this for including graphics (\includegraphics)
%\usepackage{graphicx}
% for neatly defining theorems and propositions
%\usepackage{amsthm}
% making logically defined graphics
%%%\usepackage{xypic}

% there are many more packages, add them here as you need them

% define commands here
\begin{document}
Let $X$ be a set and $\mathcal{F}$ be a filter on $X$.  Then there exists an ultrafilter $\mathcal{U}$ on $X$ which is finer than $\mathcal{F}$.

An importance consequnce of this theorem is the existence of free ultrafilters on infinite sets.  According to the theorem, there must exist an ultrafilter which is finer than the cofinite filter.  Since the cofinite filter is free, every filter finer than it must also be free, and hence there exists a free ultafilter.

Also note that this theorem requires the axiom of choice.
%%%%%
%%%%%
\end{document}
