\documentclass[12pt]{article}
\usepackage{pmmeta}
\pmcanonicalname{Cembedding}
\pmcreated{2013-03-22 16:57:37}
\pmmodified{2013-03-22 16:57:37}
\pmowner{CWoo}{3771}
\pmmodifier{CWoo}{3771}
\pmtitle{$C$-embedding}
\pmrecord{6}{39231}
\pmprivacy{1}
\pmauthor{CWoo}{3771}
\pmtype{Definition}
\pmcomment{trigger rebuild}
\pmclassification{msc}{54C45}
\pmsynonym{C-embedded}{Cembedding}
\pmsynonym{C embedded}{Cembedding}
\pmsynonym{C*-embedded}{Cembedding}
\pmsynonym{C* embedded}{Cembedding}
\pmdefines{$C$-embedded}
\pmdefines{$C^*$-embedded}

\usepackage{amssymb,amscd}
\usepackage{amsmath}
\usepackage{amsfonts}

% used for TeXing text within eps files
%\usepackage{psfrag}
% need this for including graphics (\includegraphics)
%\usepackage{graphicx}
% for neatly defining theorems and propositions
\usepackage{amsthm}
% making logically defined graphics
%%\usepackage{xypic}
\usepackage{pst-plot}
\usepackage{psfrag}

% define commands here
\newtheorem{prop}{Proposition}
\newtheorem{thm}{Theorem}
\newtheorem{ex}{Example}
\newcommand{\real}{\mathbb{R}}
\begin{document}
Let $X$ be a topological space, and $C(X)$ the ring of continuous functions on $X$.  A subspace $A\subseteq X$ is said to be \emph{$C$-embedded} (in $X$) if every function in $C(A)$ can be extended to a function in $C(X)$.  More precisely, for every real-valued continuous function $f:A\to \mathbb{R}$, there is a real-valued continuous function $g:X\to \mathbb{R}$ such that $g(x)=f(x)$ for all $x\in A$.

If $A\subseteq X$ is $C$-embedded, $f\mapsto g$ (defined above) is an embedding of $C(A)$ into $C(X)$ by axiom of choice, and hence the nomenclature.

Similarly, one may define \emph{$C^*$-embedding} on subspaces of a topological space.  Recall that for a topological space $X$, $C^*(X)$ is the ring of bounded continuous functions on $X$.  A subspace $A\subseteq X$ is said to be \emph{$C^*$-embedded} (in $X$) if every $f\in C^*(A)$ can be extended to some $g\in C^*(X)$.

\textbf{Remarks}.  Let $A$ be a subspace of $X$.
\begin{enumerate}
\item 
If $A$ is $C$-embedded in $X$, and $A\subseteq Y\subseteq X$, then $A$ is $C$-embedded in $Y$.  This is also true for $C^*$-embeddedness.
\item
If $A$ is $C$-embedded, then $A$ is $C^*$-embedded: for if $f$ is a bounded continuous function on $A$, say $-n\le f\le n$, and $g$ is its continuous extension on $X$, then $-n\vee (g\wedge n)$ is a bounded continuous extension of $f$ on $X$.
\item
The converse, however, is not true in general.  A necessary and sufficient condition that a $C^*$-embedded set $A$ is $C$-embedded is: 
\begin{center}
if a zero set is disjoint from $A$, the it is completely separated from $A$.  
\end{center}
Since any pair of disjoint zero sets are completely separated, we have that if $A$ is a $C^*$-embedded zero set, then $A$ is $C$-embedded.
\end{enumerate}

\begin{thebibliography}{7}
\bibitem{gj} L. Gillman, M. Jerison: {\em Rings of Continuous Functions}, Van Nostrand, (1960).
\end{thebibliography}
%%%%%
%%%%%
\end{document}
