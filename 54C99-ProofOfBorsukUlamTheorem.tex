\documentclass[12pt]{article}
\usepackage{pmmeta}
\pmcanonicalname{ProofOfBorsukUlamTheorem}
\pmcreated{2013-03-22 13:10:33}
\pmmodified{2013-03-22 13:10:33}
\pmowner{bwebste}{988}
\pmmodifier{bwebste}{988}
\pmtitle{proof of Borsuk-Ulam theorem}
\pmrecord{5}{33621}
\pmprivacy{1}
\pmauthor{bwebste}{988}
\pmtype{Proof}
\pmcomment{trigger rebuild}
\pmclassification{msc}{54C99}

\endmetadata

% this is the default PlanetMath preamble.  as your knowledge
% of TeX increases, you will probably want to edit this, but
% it should be fine as is for beginners.

% almost certainly you want these
\usepackage{amssymb}
\usepackage{amscd}
\usepackage{amsmath}
\usepackage{amsfonts}

% used for TeXing text within eps files
%\usepackage{psfrag}
% need this for including graphics (\includegraphics)
%\usepackage{graphicx}
% for neatly defining theorems and propositions
%\usepackage{amsthm}
% making logically defined graphics
%%%\usepackage{xypic}

% there are many more packages, add them here as you need them

% define commands here
\begin{document}
\newcommand{\Z}{\mathbb{Z}}
\newcommand{\Zt}{\Z_2}
Proof of the Borsuk-Ulam theorem:  I'm going to prove a stronger statement than the one given in 
the statement of the Borsak-Ulam theorem here, which is:

{\em Every odd (that is, antipode-preserving) map $f:S^n\to S^n$ has odd degree.}

Proof: We go by induction on $n$.  Consider the pair $(S^n,A)$ where $A$ is the equatorial sphere.  
$f$ defines a map $$\tilde{f}:\mathbb{R}P^n\to\mathbb{R}P^n$$.  By cellular approximation, this may be 
assumed to take the hyperplane at infinity (the $n-1$-cell of the standard cell structure on 
$\mathbb{R}P^n$) to itself.  Since whether a map lifts to a covering depends only on its homotopy 
class, $f$ is homotopic to an odd map taking $A$ to itself.  We may assume that $f$ is such a map.

The map $f$ gives us a morphism of the long exact sequences:
$$\begin{CD}
H_n(A;\Zt)@>i>> H_n(S^n;\Zt)@>j>> H_n(S^n,A;\Zt)@>\partial>> H_{n-1}(A;\Zt)
@>i>> H_{n-1}(S^n,A;\Zt)\\
@Vf^*VV @Vf^*VV @Vf^*VV @Vf^*VV @Vf^*VV \\
H_n(A;\Zt)@>i>> H_n(S^n;\Zt)@>j>> H_n(S^n,A;\Zt)@>\partial>> H_{n-1}(A;\Zt)
@>i>> H_{n-1}(S^n,A;\Zt)\\
\end{CD}$$

Clearly, the map $f|_A$ is odd, so by the induction hypothesis, $f|_A$ has odd degree.
Note that a map has odd degree if and only if $f^*:H_n(S^n;\Zt)\to H_n(S^n,\Zt)$ is an 
isomorphism.    Thus $$f^*:H_{n-1}(A;\Zt)\to H_{n-1}(A;\Zt)$$ is an isomorphism.
By the commutativity of the diagram, the map $$f^*:H_n(S^n,A;\Zt)\to H_n(S^n,A;\Zt)$$ is
not trivial.  I claim it is an isomorphism.  $H_n(S^n,A;\Zt)$ is generated by cycles $[R^+]$ and 
$[R^-]$ which are the fundamental classes of the upper and lower hemispheres, and the antipodal 
map exchanges these.  Both of these map to the fundamental class of $A$, 
$[A]\in H_{n-1}(A;\Zt)$.  By the commutativity of the diagram, 
$\partial(f^*([R^\pm]))=f^*(\partial([R^\pm]))=f^*([A])=[A]$.  Thus $f^*([R^+])=[R^\pm]$ and $f^*([R^-])
=[R^\mp]$ since $f$ commutes with the antipodal map.  Thus $f^*$ is an isomorphism on
$H_n(S^n,A;\Zt)$.  Since $H_n(A,\Zt)=0$, by the exactness of the sequence $i:H_n(S^n;\Zt)
\to H_n(S^n,A;\Zt)$ is injective, and so by the commutativity of the diagram (or equivalently
by the $5$-lemma) $f^*:H_n(S^n;\Zt)\to H_n(S^n;\Zt)$ is an isomorphism.  Thus
$f$ has odd degree.

The other statement of the Borsuk-Ulam theorem is:

{\em There is no odd map $S^n\to S^{n-1}$.}

Proof: If $f$ where such a map, consider $f$ restricted to the equator $A$ of $S^n$.  This is an odd
map from $S^{n-1}$ to $S^{n-1}$ and thus has odd degree.  But the map 
$$f^*H_{n-1}(A)\to H_{n-1}(S^{n-1})$$ factors through $H_{n-1}(S^n)=0$, and so must be zero. Thus $f|_A$ has degree 0, a
contradiction.
%%%%%
%%%%%
\end{document}
