\documentclass[12pt]{article}
\usepackage{pmmeta}
\pmcanonicalname{Neighborhood}
\pmcreated{2013-03-22 12:04:49}
\pmmodified{2013-03-22 12:04:49}
\pmowner{djao}{24}
\pmmodifier{djao}{24}
\pmtitle{neighborhood}
\pmrecord{13}{31151}
\pmprivacy{1}
\pmauthor{djao}{24}
\pmtype{Definition}
\pmcomment{trigger rebuild}
\pmclassification{msc}{54A05}
\pmsynonym{neighbourhood}{Neighborhood}
\pmrelated{TopologicalSpace}
\pmrelated{NeighborhoodSystem}
\pmrelated{Ball}
\pmrelated{MetricSpace}
\pmrelated{LocalBase}
\pmdefines{deleted neighborhood}

\usepackage{amssymb}
\usepackage{amsmath}
\usepackage{amsfonts}
\usepackage{graphicx}
%%%\usepackage{xypic}
\begin{document}
The meaning of the word neighborhood in topology is not well standardized. For most authors, a \emph{neighborhood} of a point $x$ in a topological space $X$ is an open subset $U$ of $X$ which contains $x$. If $X$ is a metric space, then an open ball around $x$ is one example of a neighborhood. Unless otherwise specified, this definition of neighborhood predominantes on this site.

More generally, a neighborhood of any subset $S$ of $X$ is defined to be an open set of $X$ containing $S$.

Some authors use the word neighborhood to denote any subset $U$ that contains an open subset containing $x$. This alternative usage has the advantage that it is easier to develop the theory of filters for topological spaces. At the other extreme, some analysis texts which deal only in metric spaces define a neighborhood to be an open ball around a point $x$.

Topologists tolerate this ambiguity because the most common usage, ``$S$ contains a neighborhood of $x$'' is unaffected by the choice.  In fact, almost any argument involving neighborhoods would be unaffected by shrinking a neighborhood to a smaller open set or to an open ball (in the context of metric spaces). 

A \emph{deleted neighborhood} of $x$ is an open set of the form $U \setminus \{x\}$, where $U$ is an open subset of $X$ which contains $x$.
%%%%%
%%%%%
%%%%%
\end{document}
