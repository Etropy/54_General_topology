\documentclass[12pt]{article}
\usepackage{pmmeta}
\pmcanonicalname{SubspaceOfASubspace}
\pmcreated{2013-03-22 15:17:53}
\pmmodified{2013-03-22 15:17:53}
\pmowner{matte}{1858}
\pmmodifier{matte}{1858}
\pmtitle{subspace of a subspace}
\pmrecord{5}{37096}
\pmprivacy{1}
\pmauthor{matte}{1858}
\pmtype{Theorem}
\pmcomment{trigger rebuild}
\pmclassification{msc}{54B05}

\endmetadata

% this is the default PlanetMath preamble.  as your knowledge
% of TeX increases, you will probably want to edit this, but
% it should be fine as is for beginners.

% almost certainly you want these
\usepackage{amssymb}
\usepackage{amsmath}
\usepackage{amsfonts}
\usepackage{amsthm}

\usepackage{mathrsfs}

% used for TeXing text within eps files
%\usepackage{psfrag}
% need this for including graphics (\includegraphics)
%\usepackage{graphicx}
% for neatly defining theorems and propositions
%
% making logically defined graphics
%%%\usepackage{xypic}

% there are many more packages, add them here as you need them

% define commands here

\newcommand{\sR}[0]{\mathbb{R}}
\newcommand{\sC}[0]{\mathbb{C}}
\newcommand{\sN}[0]{\mathbb{N}}
\newcommand{\sZ}[0]{\mathbb{Z}}

 \usepackage{bbm}
 \newcommand{\Z}{\mathbbmss{Z}}
 \newcommand{\C}{\mathbbmss{C}}
 \newcommand{\F}{\mathbbmss{F}}
 \newcommand{\R}{\mathbbmss{R}}
 \newcommand{\Q}{\mathbbmss{Q}}



\newcommand*{\norm}[1]{\lVert #1 \rVert}
\newcommand*{\abs}[1]{| #1 |}



\newtheorem{thm}{Theorem}
\newtheorem{defn}{Definition}
\newtheorem{prop}{Proposition}
\newtheorem{lemma}{Lemma}
\newtheorem{cor}{Corollary}
\begin{document}
\begin{thm} Suppose $X\subseteq Y \subseteq Z$ are sets and $Z$ is a
topological space with topology $\tau_Z$.
Let $\tau_{Y,Z}$ be the subspace topology in $Y$ given by $\tau_Z$,
and let $\tau_{X,Y,Z}$ be the subspace topology in $X$ given by
$\tau_{Y,Z}$, and let $\tau_{X,Z}$ be the subspace topology in $X$
given by $\tau_Z$. Then $\tau_{X,Z}=\tau_{X,Y,Z}$.
\end{thm}

\begin{proof}
Let $U_X\in\tau_{X,Z}$, then there is by the definition of the subspace
topology an open set $U_Z\in\tau_Z$ such that $U_X=U_Z\cap X$. Now
$U_Z\cap Y\in\tau_{Y,Z}$ and therefore $U_Z\cap Y\cap
X\in\tau_{X,Y,Z}$. But since $X\subseteq Y$, we have $U_Z\cap Y\cap
X=U_Z\cap X=U_X$, so $U_X\in\tau_{X,Y,Z}$ and thus
$\tau_{X,Z}\subseteq\tau_{X,Y,Z}$.

To show the reverse inclusion, take an open set
$U_X\in\tau_{X,Y,Z}$. Then there is an open set $U_Y\in\tau_{Y,Z}$
such that $U_X=U_Y\cap X$. Furthermore, there is an open set
$U_Z\in\tau_Z$ such that $U_Y=U_Z\cap Y$. Since $X\subseteq Y$, we
have
\begin{equation*}
U_Z\cap X=U_Z\cap Y\cap X=U_Y\cap X=U_X,
\end{equation*}
so $U_X\in\tau_{X,Z}$ and thus $\tau_{X,Y,Z}\subseteq\tau_{X,Z}$.

Together, both inclusions yield the equality $\tau_{X,Z}=\tau_{X,Y,Z}$.
\end{proof}
%%%%%
%%%%%
\end{document}
