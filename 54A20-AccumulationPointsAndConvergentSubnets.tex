\documentclass[12pt]{article}
\usepackage{pmmeta}
\pmcanonicalname{AccumulationPointsAndConvergentSubnets}
\pmcreated{2013-03-22 18:37:40}
\pmmodified{2013-03-22 18:37:40}
\pmowner{azdbacks4234}{14155}
\pmmodifier{azdbacks4234}{14155}
\pmtitle{accumulation points and convergent subnets}
\pmrecord{9}{41365}
\pmprivacy{1}
\pmauthor{azdbacks4234}{14155}
\pmtype{Theorem}
\pmcomment{trigger rebuild}
\pmclassification{msc}{54A20}
%\pmkeywords{net}
%\pmkeywords{accumulation point}
%\pmkeywords{cluster point}
%\pmkeywords{subnet}
%\pmkeywords{convergence}
\pmrelated{Net}
\pmrelated{Neighborhood}
\pmrelated{DirectedSet}
\pmrelated{CompactnessAndConvergentSubnets}

%packages
\usepackage{amsmath,mathrsfs,amsfonts,amsthm}
%theorem environments
\theoremstyle{plain}
\newtheorem*{thm*}{Theorem}
\newtheorem*{lem*}{Lemma}
\newtheorem*{cor*}{Corollary}
\newtheorem*{prop*}{Proposition}
%delimiters
\newcommand{\set}[1]{\{#1\}}
\newcommand{\medset}[1]{\big\{#1\big\}}
\newcommand{\bigset}[1]{\bigg\{#1\bigg\}}
\newcommand{\Bigset}[1]{\Bigg\{#1\Bigg\}}
\newcommand{\abs}[1]{\vert#1\vert}
\newcommand{\medabs}[1]{\big\vert#1\big\vert}
\newcommand{\bigabs}[1]{\bigg\vert#1\bigg\vert}
\newcommand{\Bigabs}[1]{\Bigg\vert#1\Bigg\vert}
\newcommand{\norm}[1]{\Vert#1\Vert}
\newcommand{\mednorm}[1]{\big\Vert#1\big\Vert}
\newcommand{\bignorm}[1]{\bigg\Vert#1\bigg\Vert}
\newcommand{\Bignorm}[1]{\Bigg\Vert#1\Bigg\Vert}
\newcommand{\vbrack}[1]{\langle#1\rangle}
\newcommand{\medvbrack}[1]{\big\langle#1\big\rangle}
\newcommand{\bigvbrack}[1]{\bigg\langle#1\bigg\rangle}
\newcommand{\Bigvbrack}[1]{\Bigg\langle#1\Bigg\rangle}
\newcommand{\sbrack}[1]{[#1]}
\newcommand{\medsbrack}[1]{\big[#1\big]}
\newcommand{\bigsbrack}[1]{\bigg[#1\bigg]}
\newcommand{\Bigsbrack}[1]{\Bigg[#1\Bigg]}
%operators
\DeclareMathOperator{\Hom}{Hom}
\DeclareMathOperator{\Tor}{Tor}
\DeclareMathOperator{\Ext}{Ext}
\DeclareMathOperator{\Aut}{Aut}
\DeclareMathOperator{\End}{End}
\DeclareMathOperator{\Inn}{Inn}
\DeclareMathOperator{\lcm}{lcm}
\DeclareMathOperator{\ord}{ord}
\DeclareMathOperator{\rank}{rank}
\DeclareMathOperator{\tr}{tr}
\DeclareMathOperator{\Mat}{Mat}
\DeclareMathOperator{\Gal}{Gal}
\DeclareMathOperator{\GL}{GL}
\DeclareMathOperator{\SL}{SL}
\DeclareMathOperator{\SO}{SO}
\DeclareMathOperator{\ann}{ann}
\DeclareMathOperator{\im}{im}
\DeclareMathOperator{\Char}{char}
\DeclareMathOperator{\Spec}{Spec}
\DeclareMathOperator{\supp}{supp}
\DeclareMathOperator{\diam}{diam}
\DeclareMathOperator{\Ind}{Ind}
\DeclareMathOperator{\vol}{vol}

\begin{document}
\begin{prop*}
Let $X$ be a topological space and $(x_\alpha)_{\alpha\in A}$ a net in $X$. A point $x\in X$ 
is an accumulation point of $(x_\alpha)$ if and only if some subnet of $(x_\alpha)$ converges to $x$.
\end{prop*}
\begin{proof}
Suppose first that $(x_{\alpha_\beta})_{\beta\in B}$ is a subnet of $(x_\alpha)$ converging to $x$. Given an open subset $U$ of $X$ containing $x$ and $\alpha\in A$, we may select $\beta_1\in B$ such that $x_{\alpha_\beta}\in U$ for $\beta\geq\beta_1$, as well as $\beta_2\in B$ such that $\alpha_{\beta}\geq\alpha$ for $\beta\geq\beta_2$. Finally, because $B$ is directed, there exists $\beta\in B$ such that $\beta\geq\beta_1$ and $\beta\geq\beta_2$; we then have $\alpha_\beta\geq\alpha$ and $x_{\alpha_\beta}\in U$, so that $(x_\alpha)$ is frequently in $U$, whence $x$ is an accumulation point of $(x_\alpha)$. Conversely, suppose that $x$ is an accumulation point of $(x_\alpha)$, let $N$ be the set of open neighborhoods of $x$ in $X$, directed by reverse inclusion, and let $B=A\times N$, directed in the natural way. For each pair $(\gamma,U)\in B$, select $\alpha_{(\gamma,U)}\in B$ such that $\alpha\geq\gamma$ and $x_{\alpha_{(\gamma,U)}}\in U$; $(x_{\alpha_{(\gamma,U)}})_{(\gamma,U)\in B}$ is then a subnet of $(x_\alpha)$ that converges to $x$, for given $U\in N$ and $\gamma\in A$, if $(\gamma^\prime,U^\prime)\geq(\gamma,U)$, then $\alpha_{(\gamma^\prime,U)}\geq\gamma^\prime\geq\gamma$ and $x_{\alpha_{(\gamma^\prime,U^\prime)}}\in U^\prime\subseteq U$.
\end{proof}
%%%%%
%%%%%
\end{document}
