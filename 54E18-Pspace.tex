\documentclass[12pt]{article}
\usepackage{pmmeta}
\pmcanonicalname{Pspace}
\pmcreated{2013-03-22 18:53:13}
\pmmodified{2013-03-22 18:53:13}
\pmowner{CWoo}{3771}
\pmmodifier{CWoo}{3771}
\pmtitle{P-space}
\pmrecord{6}{41735}
\pmprivacy{1}
\pmauthor{CWoo}{3771}
\pmtype{Definition}
\pmcomment{trigger rebuild}
\pmclassification{msc}{54E18}
\pmclassification{msc}{16S60}

\endmetadata

\usepackage{amssymb,amscd}
\usepackage{amsmath}
\usepackage{amsfonts}
\usepackage{mathrsfs}

% used for TeXing text within eps files
%\usepackage{psfrag}
% need this for including graphics (\includegraphics)
%\usepackage{graphicx}
% for neatly defining theorems and propositions
\usepackage{amsthm}
% making logically defined graphics
%%\usepackage{xypic}
\usepackage{pst-plot}

% define commands here
\newcommand*{\abs}[1]{\left\lvert #1\right\rvert}
\newtheorem{prop}{Proposition}
\newtheorem{thm}{Theorem}
\newtheorem{ex}{Example}
\newcommand{\real}{\mathbb{R}}
\newcommand{\pdiff}[2]{\frac{\partial #1}{\partial #2}}
\newcommand{\mpdiff}[3]{\frac{\partial^#1 #2}{\partial #3^#1}}
\begin{document}
Suppose $X$ is a completely regular topological space.  Then $X$ is said to be a \emph{P-space} if every prime ideal in $C(X)$, the ring of continuous functions on $X$, is maximal.

For example, every space with the discrete topology is a P-space.

Algebraically, a commutative reduced ring $R$ with $1$ such that every prime ideal is maximal is equivalent to any of the following statements:
\begin{itemize}
\item $R$ is von-Neumann regular,
\item every ideal in $R$ is the intersection of prime ideals,
\item every ideal in $R$ is the intersection of maximal ideals,
\item every principal ideal is generated by an idempotent.
\end{itemize}

When $R=C(X)$, then $R$ is commutative reduced with $1$.  In addition to the algebraic characterizations of $R$ above, $X$ being a P-space is equivalent to any of the following statements:
\begin{itemize}
\item every zero set is open
\item if $f,g\in C(X)$, then $(f,g)=(f^2+g^2)$.
\end{itemize}

Some properties of P-spaces:
\begin{enumerate}
\item Every subspace of a P-space is a P-space,
\item Every quotient space of a P-space is a P-space,
\item Every finite product of P-spaces is a P-space,
\item Every P-space has a base of clopen sets.
\end{enumerate}

For more properties of P-spaces, please see the reference below.  For proofs of the above properties and equivalent characterizations, see \PMlinkexternal{here}{http://planetmath.org/?op=getobj&from=books&id=46}.

\begin{thebibliography}{7}
\bibitem{gj} L. Gillman, M. Jerison: {\em Rings of Continuous Functions}, Van Nostrand, (1960).
\end{thebibliography}
%%%%%
%%%%%
\end{document}
