\documentclass[12pt]{article}
\usepackage{pmmeta}
\pmcanonicalname{KleinBottle}
\pmcreated{2013-03-22 13:37:00}
\pmmodified{2013-03-22 13:37:00}
\pmowner{vernondalhart}{2191}
\pmmodifier{vernondalhart}{2191}
\pmtitle{Klein bottle}
\pmrecord{12}{34249}
\pmprivacy{1}
\pmauthor{vernondalhart}{2191}
\pmtype{Definition}
\pmcomment{trigger rebuild}
\pmclassification{msc}{54B15}
\pmrelated{MobiusStrip}

\endmetadata

% this is the default PlanetMath preamble.  as your knowledge
% of TeX increases, you will probably want to edit this, but
% it should be fine as is for beginners.

% almost certainly you want these
\usepackage{amssymb}
\usepackage{amsmath}
\usepackage{amsfonts}

% used for TeXing text within eps files
%\usepackage{psfrag}
% need this for including graphics (\includegraphics)
\usepackage{graphicx}
% for neatly defining theorems and propositions
%\usepackage{amsthm}
% making logically defined graphics
%%%\usepackage{xypic}

% there are many more packages, add them here as you need them

% define commands here
\begin{document}
Where a M\"obius strip is a two dimensional object with only one surface and one edge, a Klein bottle is a two dimensional object with a single surface, and no edges. Consider for comparison, that a sphere is a two dimensional surface with no edges, but that has two surfaces.

A Klein bottle can be constructed by taking a rectangular subset of $\mathbb{R}^2$ and identifying opposite edges with each other, in the following fashion:

Consider the rectangular subset $[-1,1] \times [-1,1]$. Identify the points $(x, 1)$ with $(x, -1)$, and the points $(1,y)$ with the points $(-1,-y)$. Doing these two operations simultaneously will give you the Klein bottle.

Visually, the above is accomplished by the following. Take a rectangle, and match up the arrows on the edges so that their orientation matches:

\begin{center}
\includegraphics{klein.eps}
\end{center}

This of course is completely impossible to do physically in 3-dimensional space; to be able to properly create a Klein bottle, one would need to be able to build it in 4-dimensional space.

To construct a pseudo-Klein bottle in 3-dimensional space, you would first take a cylinder and cut a hole at one point on the side. Next, bend one end of the cylinder through that hole, and attach it to the other end of the clyinder.

A Klein bottle may be parametrized by the following equations:
\begin{align*}
x &= \begin{cases}
a\cos(u)\bigl(1+\sin(u)\bigr) + r\cos(u)\cos(v) & 0 \le u < \pi\\
a\cos(u)\bigl(1+\sin(u)\bigr) + r\cos(v + \pi) & \pi < u \le 2\pi
\end{cases}\\
y &=\begin{cases}
b\sin(u) + r\sin(u)\cos(v) & 0 \le u < \pi\\
b\sin(u) & \pi < u \le 2\pi
\end{cases}\\
z &= r\sin(v)
\end{align*}

where $v\in [0,2\pi], u \in [0, 2\pi], r = c(1-\frac{\cos(u)}{2})$ and $a, b, c$ are chosen arbitrarily.
%%%%%
%%%%%
\end{document}
