\documentclass[12pt]{article}
\usepackage{pmmeta}
\pmcanonicalname{OrdinalSpace}
\pmcreated{2013-03-22 17:10:56}
\pmmodified{2013-03-22 17:10:56}
\pmowner{CWoo}{3771}
\pmmodifier{CWoo}{3771}
\pmtitle{ordinal space}
\pmrecord{8}{39498}
\pmprivacy{1}
\pmauthor{CWoo}{3771}
\pmtype{Definition}
\pmcomment{trigger rebuild}
\pmclassification{msc}{54F05}

\endmetadata

\usepackage{amssymb,amscd}
\usepackage{amsmath}
\usepackage{amsfonts}
\usepackage{mathrsfs}

% used for TeXing text within eps files
%\usepackage{psfrag}
% need this for including graphics (\includegraphics)
%\usepackage{graphicx}
% for neatly defining theorems and propositions
\usepackage{amsthm}
% making logically defined graphics
%%\usepackage{xypic}
\usepackage{pst-plot}
\usepackage{psfrag}

% define commands here
\newtheorem{prop}{Proposition}
\newtheorem{thm}{Theorem}
\newtheorem{ex}{Example}
\newcommand{\real}{\mathbb{R}}
\newcommand{\pdiff}[2]{\frac{\partial #1}{\partial #2}}
\newcommand{\mpdiff}[3]{\frac{\partial^#1 #2}{\partial #3^#1}}
\newcommand{\up}{\uparrow\!\!}
\newcommand{\down}{\downarrow\!\!}
\begin{document}
Let $\alpha$ be an ordinal.  The set $W(\alpha):=\lbrace \beta \mid \beta < \alpha\rbrace$ ordered by $\le$ is a well-ordered set.  $W(\alpha)$ becomes a topological space if we equip $W(\alpha)$ with the interval topology.  An \emph{ordinal space} $X$ is a topological space such that $X=W(\alpha)$ (with the interval topology) for some ordinal $\alpha$.  In this entry, we will always assume that $W(\alpha)\ne \varnothing$, or $0<\alpha$.

Before examining some basic topological structures of $W(\alpha)$, let us look at some of its order structures.  
\begin{enumerate}
\item
First, it is easy to see that $W(\alpha)=\up y \cup W(y)$, for any $y\in W(\alpha)$.  Here, $\up y$ is the upper set of $y$.
\item
Another way of saying that $W(\alpha)$ is well-ordered is that for any non-empyt subset $S$ of $W(\alpha)$, $\bigwedge S$ exists.  Clearly, $0\in W(\alpha)$ is its least element.  If in addition $1<\alpha$, $W(\alpha)$ is also atomic, with $1$ as the sole atom.
\item
Next, $W(\alpha)$ is bounded complete.  If $S\subseteq W(\alpha)$ is bounded from above by $a\in W(\alpha)$, then $b=\bigvee S$ is an ordinal such that $b\le a<\alpha$, therefore $b\in W(\alpha)$ as well.  
\item
Finally, we note that $W(\alpha)$ is a complete lattice iff $\alpha$ is not a limit ordinal.  If $W(\alpha)$ is complete, then $z=\bigvee W(\alpha)\in W(\alpha)$.  So $z<\alpha$.  This means that $z+1\le \alpha$.  If $z+1<\alpha$, then $z+1\in W(\alpha)$ so that $z+1\le \bigvee W(\alpha)=z$, a contradiction.  As a result, $z+1=\alpha$.  On the other hand, if $\alpha=z+1$, then $z=\bigvee W(\alpha)\in W(\alpha)$, so that $W(\alpha)$ is complete.
\end{enumerate}

In any ordinal space $W(\alpha)$ where $0<\alpha$, a typical open interval may be written $(x,y)$, where $0\le x\le y<\alpha$.  If $y$ is not a limit ordinal, we can also write $(x,y)=[x+1,z]$ where $z+1=y$.  This means that $(x,y)$ is a clopen set if $y$ is not a limit ordinal.  In particular, if $y$ is not a limit ordinal, then $\lbrace y\rbrace = (z,y+1)$ is clopen, where $z+1=y$, so that $y$ is an isolated point.  For example, any finite ordinal is an isolated point in $W(\alpha)$.  

Conversely, an isolated point can not be a limit ordinal.  If $y$ is isolated, then $\lbrace y\rbrace$ is open.  Write $\lbrace y\rbrace$ as the union of open intervals $(a_i,b_i)$.  So $a_i<y<b_i$.  Since $y+1$ covers $y$, each $b_i$ must be $y+1$ or $(a_i,b_i)$ would contain more than a point.  If $y$ is a limit ordinal, then $a_i<a_i+1<y$ so that, again, $(a_i,b_i)$ would contain more than just $y$.  Therefore, $y$ can not be a limit ordinal and all $a_i$ must be the same.  Therefore $(a_i,b_i)=(z,y+1)$, where $z$ is the predecessor of $y$: $z+1=y$.

Several basic properties of an ordinal space are: 
\begin{enumerate}
\item
Isolated points in $W(\alpha)$ are exactly those points that are limit ordinals (just a summary of the last two paragraphs).
\item
$W(y)$ is open in $W(\alpha)$ for any $y\in W(\alpha)$.  $W(y)$ is closed iff $y$ is not a limit ordinal.
\item
For any $y\in W(\alpha)$, the collection of intervals of the form $(a,y]$ (where $a<y$) forms a neighborhood base of $y$.
\item
$W(\alpha)$ is a normal space for any $\alpha$; 
\item
$W(\alpha)$ is compact iff $\alpha$ is not a limit ordinal.
\end{enumerate}

Some interesting ordinal spaces are 
\begin{itemize}
\item
$W(\omega)$, which is homeomorphic to the set of natural numbers $\mathbb{N}$.
\item
$W(\omega_1)$, where $\omega_1$ is the first uncountable ordinal.  $W(\omega_1)$ is often written $\Omega_0$.  $\Omega_0$ is not a compact space.
\item
$W(\omega_1+1)$, or $\Omega$.  $\Omega$ is compact, and, in fact, a one-point compactification of $\Omega_0$.
\end{itemize}

\begin{thebibliography}{9}
\bibitem{willard} S. Willard, \emph{General Topology}, Addison-Wesley, Publishing Company, 1970.
\end{thebibliography}
%%%%%
%%%%%
\end{document}
