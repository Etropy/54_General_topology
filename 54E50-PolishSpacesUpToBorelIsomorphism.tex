\documentclass[12pt]{article}
\usepackage{pmmeta}
\pmcanonicalname{PolishSpacesUpToBorelIsomorphism}
\pmcreated{2013-03-22 18:47:00}
\pmmodified{2013-03-22 18:47:00}
\pmowner{gel}{22282}
\pmmodifier{gel}{22282}
\pmtitle{Polish spaces up to Borel isomorphism}
\pmrecord{6}{41578}
\pmprivacy{1}
\pmauthor{gel}{22282}
\pmtype{Theorem}
\pmcomment{trigger rebuild}
\pmclassification{msc}{54E50}
%\pmkeywords{Polish space}
%\pmkeywords{Borel isomorphism}
\pmrelated{CategoryOfPolishGroups}

\endmetadata

% almost certainly you want these
\usepackage{amssymb}
\usepackage{amsmath}
\usepackage{amsfonts}

% used for TeXing text within eps files
%\usepackage{psfrag}
% need this for including graphics (\includegraphics)
%\usepackage{graphicx}
% for neatly defining theorems and propositions
\usepackage{amsthm}
% making logically defined graphics
%%%\usepackage{xypic}

% there are many more packages, add them here as you need them

% define commands here
\newtheorem*{theorem*}{Theorem}
\newtheorem*{lemma*}{Lemma}
\newtheorem*{corollary*}{Corollary}
\newtheorem*{definition*}{Definition}
\newtheorem{theorem}{Theorem}
\newtheorem{lemma}{Lemma}
\newtheorem{corollary}{Corollary}
\newtheorem{definition}{Definition}

\begin{document}
\PMlinkescapeword{borel isomorphism}
\PMlinkescapeword{inverse}
\PMlinkescapeword{function}
\PMlinkescapeword{isomorphic}
\PMlinkescapeword{countable}

Two topological spaces $X$ and $Y$ are \PMlinkname{Borel isomorphic}{BorelIsomorphism} if there is a Borel measurable function $f\colon X\rightarrow Y$ with Borel inverse. Such a function is said to be a Borel isomorphism. The following result classifies all Polish spaces up to Borel isomorphism.

\begin{theorem*}
Every uncountable Polish space is Borel isomorphic to $\mathbb{R}$ with the standard topology.
\end{theorem*}

As the Borel $\sigma$-algebra on any countable metric space is just its power set, this shows that every Polish space is Borel isomorphic to one and only one of the following.

\begin{enumerate}
\item $\left\{1,2,\ldots,n\right\}$ for some $n\ge 0$, with the discrete topology.
\item $\mathbb{N}=\left\{1,2,\ldots\right\}$ with the discrete topology.
\item $\mathbb{R}$ with the standard topology.
\end{enumerate}

In particular, two Polish spaces are Borel isomorphic if and only if they have the same cardinality, and any uncountable Polish space has the cardinality of the continuum.

%%%%%
%%%%%
\end{document}
