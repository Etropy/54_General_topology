\documentclass[12pt]{article}
\usepackage{pmmeta}
\pmcanonicalname{RealNumber}
\pmcreated{2013-03-22 11:52:22}
\pmmodified{2013-03-22 11:52:22}
\pmowner{djao}{24}
\pmmodifier{djao}{24}
\pmtitle{real number}
\pmrecord{23}{30454}
\pmprivacy{1}
\pmauthor{djao}{24}
\pmtype{Definition}
\pmcomment{trigger rebuild}
\pmclassification{msc}{54C30}
\pmclassification{msc}{26-00}
\pmclassification{msc}{12D99}
\pmsynonym{real}{RealNumber}
\pmsynonym{$\mathbb{R}$}{RealNumber}
\pmrelated{DedekindCuts}

\endmetadata

% this is the default PlanetMath preamble.  as your knowledge
% of TeX increases, you will probably want to edit this, but
% it should be fine as is for beginners.

% almost certainly you want these
\usepackage{amssymb}
\usepackage{amsmath}
\usepackage{amsfonts}

% used for TeXing text within eps files
%\usepackage{psfrag}
% need this for including graphics (\includegraphics)
%\usepackage{graphicx}
% for neatly defining theorems and propositions
%\usepackage{amsthm}
% making logically defined graphics
%%%%%\usepackage{xypic} 

% there are many more packages, add them here as you need them
% define commands here
\begin{document}
\section{Definition}
There are several equivalent definitions of real number, all in common use. We give one definition in detail and mention the other ones.

A {\em Cauchy sequence of rational numbers} is a sequence $\{x_i\},\ i=0,1,2,\dots$ of rational numbers with the property that, for every rational number $\epsilon > 0$, there exists a natural number $N$ such that, for all natural numbers $n,m > N$, the absolute value $|x_n - x_m|$ satisfies $|x_n - x_m| < \epsilon$.

The set $\mathbb{R}$ of {\em real numbers} is the set of equivalence classes of Cauchy sequences of rational numbers, under the equivalence relation $\{x_i\} \sim \{y_i\}$ if the interleave sequence of the two sequences is itself a Cauchy sequence.
The real numbers form a ring, with addition and multiplication defined by
\begin{itemize}
\item $\{x_i\} + \{y_i\} = \{(x_i+y_i)\}$
\item $\{x_i\} \cdot \{y_i\} = \{(x_i \cdot y_i)\}$
\end{itemize}
There is an ordering relation on $\mathbb{R}$, defined by $\{x_i\} \leq \{y_i\}$ if either $\{x_i\} \sim \{y_i\}$ or there exists a natural number $N$ such that $x_n < y_n$ for all $n > N$. This definition is well-defined and does not depend on the choice of Cauchy sequences used to represent the equivalence classes.

One can prove that the real numbers form an ordered field and that they satisfy the \emph{Dedekind completeness property} (also known as the \emph{least upper bound property}): For every nonempty subset $S \subset \mathbb{R}$, if $S$ has an upper bound then $S$ has a lowest upper bound. It is also true that every ordered field with the least upper bound property is isomorphic to the real numbers.

Alternative definitions of the set of real numbers include:
\begin{enumerate}
\item Equivalence classes of decimal sequences (sequences consisting of natural numbers between 0 and 9, and a single decimal point), where two decimal sequences are equivalent if they are identical, or if one has an infinite tail of 9's, the other has an infinite tail of 0's, and the leading portion of the first sequence is one lower than the leading portion of the second.
\item Dedekind cuts of rational numbers (that is, subsets $S$ of $\mathbb{Q}$ with the property that, if $a \in S$ and $b < a$, then $b \in S$).
\item The real numbers can also be defined as the unique (up to isomorphism) ordered field satisfying the least upper bound property, after one has proved that such a field exists and is unique up to isomorphism.
\end{enumerate}

\section{Completeness}

The main reason for introducing the reals is that the reals contain all limits.
More technically, the reals are complete (in the sense of metric spaces or
uniform spaces, which is a different sense than the Dedekind completeness of
the order in the previous section). This means the following:
                                                                                
A sequence $(x_n)$ of real numbers is called a Cauchy sequence if for any
$\varepsilon > 0$
there exists an integer $N$ (possibly depending on $\varepsilon$) such that the
distance $|x_n - x_m|$ is less than $\varepsilon$ provided that $n$ and $m$ are
both greater than $N$. In other
words, a sequence is a Cauchy sequence if its elements $x_n$ eventually come
and remain arbitrarily close to each other.
                                                                                
A sequence $(x_n)$ converges to the limit $x$ if for any $\varepsilon > 0$
there exists an integer $N$ (possibly depending on $\varepsilon$) such that the
distance $|x_n - x|$ is less than $\varepsilon$ provided that $n$ is greater
than $N$. In other words, a sequence has limit $x$ if its elements eventually
come and remain arbitrarily close to $x$.
                                                                                
It is easy to see that every convergent sequence is a Cauchy sequence. Now the
important fact about the real numbers is that the converse is true:
\begin{quotation}
    Every Cauchy sequence of real numbers is convergent.
\end{quotation}
That is, the reals are complete.
                                                                                
Note that the rationals are not complete. For example, the sequence $1$, $1.4$,
$1.41$, $1.414$, $1.4142$, $1.41421$, $\ldots$ is Cauchy but it does not
converge to a rational number. (In the real numbers, in contrast, it converges
to the square root of $2$.)
                                                                                
The existence of limits of Cauchy sequences is what makes calculus work and is
of great practical use. The standard numerical test to determine if a sequence
has a limit is to test if it is a Cauchy sequence, as the limit is typically
not known in advance.
                                                                                
For example the standard series of the exponential function
                                                                                
$$
e^x = \sum_{n=0}^\infty \frac{x^n}{n!}
$$
                                                                                
converges to a real number because for every $x$ the sums
                                                                                
                                                                                
$$
\sum_{n=N}^M \frac{x^n}{n!}
$$
                                                                                
can be made arbitrarily small by choosing $N$ sufficiently large. This proves
that the sequence is Cauchy, so we know that the sequence converges even if we
don't know ahead of time what the limit is.
                                                                               
\section{``The complete ordered field''}

The real numbers are often described as ``the complete ordered field,'' a phrase
that can be interpreted in several ways.

First, an order can be lattice complete. It's easy to see that no ordered field
can be lattice complete, because it can have no largest element (given any
element $z$, $z + 1$ is larger), so this is not the sense that is meant.

Additionally, an order can be Dedekind-complete, as defined in the Definitions section. The uniqueness result at the end of that section justifies using the
word ``the'' in the phrase ``complete ordered field'' when this is the sense of
``complete'' that is meant. This sense of completeness is most closely related to the construction of the reals from Dedekind cuts, since that construction
starts from an ordered field (the rationals) and then forms the Dedekind-completion of it in a standard way. 

These two notions of completeness ignore the field structure. However, an
ordered group (and a field is a group under the operations of addition and
subtraction) defines a uniform structure, and uniform structures have a notion
of completeness (topology); the description in the Completeness section above
is a special case. (We refer to the notion of completeness in uniform spaces
rather than the related and better known notion for metric spaces, since the
definition of metric space relies on already having a characterisation of the
real numbers.) It is not true that $\mathbb{R}$ is the only uniformly complete
ordered
field, but it is the only uniformly complete Archimedean field, and indeed one
often hears the phrase ``complete Archimedean field'' instead of ``complete
ordered field.'' Since it can be proved that any uniformly complete Archimedean
field must also be Dedekind complete (and vice versa, of course), this
justifies using ``the'' in the phrase ``the complete Archimedean field.'' This
sense of completeness is most closely related to the construction of the reals
from Cauchy sequences (the construction carried out in full in this article),
since it starts with an Archimedean field (the rationals) and forms the uniform
completion of it in a standard way.
                                                                                
But the original use of the phrase ``complete Archimedean field'' was by David
Hilbert, who meant still something else by it. He meant that the real numbers
form the largest Archimedean field in the sense that every other Archimedean
field is a subfield of $\mathbb{R}$. Thus $\mathbb{R}$ is ``complete'' in the
sense that nothing
further can be added to it without making it no longer an Archimedean field.
This sense of completeness is most closely related to the construction of the
reals from surreal numbers, since that construction starts with a proper class
that contains every ordered field (the surreals) and then selects from it the
largest Archimedean subfield.

\emph{This article contains material from the \PMlinkexternal{Wikipedia article on Real numbers}{http://en.wikipedia.org/wiki/Real_numbers} which is incorporated herein under the terms of the \PMlinkexternal{GNU Free Documentation License}{http://en.wikipedia.org/wiki/Wikipedia:Text_of_the_GNU_Free_Documentation_License}.}
%%%%%
%%%%%
%%%%%
%%%%%
\end{document}
