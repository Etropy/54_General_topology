\documentclass[12pt]{article}
\usepackage{pmmeta}
\pmcanonicalname{LoopSpace}
\pmcreated{2013-03-22 12:15:26}
\pmmodified{2013-03-22 12:15:26}
\pmowner{mathcam}{2727}
\pmmodifier{mathcam}{2727}
\pmtitle{loop space}
\pmrecord{8}{31640}
\pmprivacy{1}
\pmauthor{mathcam}{2727}
\pmtype{Definition}
\pmcomment{trigger rebuild}
\pmclassification{msc}{54-00}
\pmrelated{Suspension}
\pmrelated{EilenbergMacLaneSpace}

% this is the default PlanetMath preamble.  as your knowledge
% of TeX increases, you will probably want to edit this, but
% it should be fine as is for beginners.

% almost certainly you want these
\usepackage{amssymb}
\usepackage{amsmath}
\usepackage{amsfonts}

% used for TeXing text within eps files
%\usepackage{psfrag}
% need this for including graphics (\includegraphics)
%\usepackage{graphicx}
% for neatly defining theorems and propositions
%\usepackage{amsthm}
% making logically defined graphics
%%%%\usepackage{xypic} 

% there are many more packages, add them here as you need them

% define commands here
\begin{document}
Let $X$ be a topological space, and give the space of continuous maps $[0,1]\to X$, the compact-open topology, that is a subbasis for the topology is the collection of sets $\{\sigma : \sigma(K)\subset U \}$ for $K\subset [0,1]$ compact and $U\subset X$ open.

Then for $x\in X$, let $\Omega_{x}X$ be the subset of loops based at $x$ (that is $\sigma$ such that $\sigma(0) = \sigma(1) = x$), with the relative topology.

$\Omega_{x}X$ is called the loop space of $X$ at $x$.
%%%%%
%%%%%
%%%%%
\end{document}
