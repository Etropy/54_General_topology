\documentclass[12pt]{article}
\usepackage{pmmeta}
\pmcanonicalname{CantorBendixsonDerivative}
\pmcreated{2013-03-22 15:01:37}
\pmmodified{2013-03-22 15:01:37}
\pmowner{CWoo}{3771}
\pmmodifier{CWoo}{3771}
\pmtitle{Cantor-Bendixson derivative}
\pmrecord{9}{36736}
\pmprivacy{1}
\pmauthor{CWoo}{3771}
\pmtype{Definition}
\pmcomment{trigger rebuild}
\pmclassification{msc}{54H05}
\pmclassification{msc}{03E15}
\pmsynonym{set derivative}{CantorBendixsonDerivative}
\pmrelated{DerivedSet}
\pmdefines{Cantor-Bendixson rank}

% this is the default PlanetMath preamble.  as your knowledge
% of TeX increases, you will probably want to edit this, but
% it should be fine as is for beginners.

% almost certainly you want these
\usepackage{amssymb}
\usepackage{amsmath}
\usepackage{amsfonts}

% used for TeXing text within eps files
%\usepackage{psfrag}
% need this for including graphics (\includegraphics)
%\usepackage{graphicx}
% for neatly defining theorems and propositions
%\usepackage{amsthm}
% making logically defined graphics
%%%\usepackage{xypic}

% there are many more packages, add them here as you need them

% define commands here
\def\sse{\subseteq}
\def\spse{\supseteq}
\def\bigtimes{\mathop{\mbox{\Huge $\times$}}}
\def\impl{\Rightarrow}
\begin{document}
\PMlinkescapeword{order}
%
Let $A$ be a subset of a topological space $X$. Its \emph{Cantor-Bendixson
derivative} $A'$ is defined as the set of accumulation points of $A$. In
other words
\[
  A' = \{ x\in X \mid x\in \overline{A\setminus \{x\}} \}.
\]
Through transfinite induction, the Cantor-Bendixson derivative can be
defined to any order $\alpha$, where $\alpha$ is an arbitrary ordinal.
Let $A^{(0)} = A$.  If $\alpha$ is a successor ordinal, then
$A^{(\alpha)} = \left(A^{(\alpha-1)}\right)'$. If $\lambda$ is a limit
ordinal, then $A^{(\lambda)} = \bigcap_{\alpha<\lambda} A^{(\alpha)}$.
The \emph{Cantor-Bendixson rank} of the set $A$ is the least ordinal
$\alpha$ such that $A^{(\alpha)} = A^{(\alpha+1)}$. Note that $A' = A$
implies that $A$ is a perfect set.

Some basic properties of the Cantor-Bendixson derivative include
\begin{enumerate}
\item $(A\cup B)' = A'\cup B'$,
\item $(\bigcup_{i\in I} A_i)' \spse \bigcup_{i\in I} A_i'$,
\item $(\bigcap_{i\in I} A_i)' \sse \bigcap_{i\in I} A_i'$,
\item $(A\setminus B)' \spse A' \setminus B'$,
\item $A\sse B \impl A' \sse B'$,
\item $\overline{A} = A \cup A'$,
\item $\overline{A'} = A'$.
\end{enumerate}
The last property requires some justification. Obviously, $A'\sse
\overline{A'}$. Suppose $a\in \overline{A'}$, then every neighborhood of
$a$ contains some points of $A'$ distinct from $a$. But by definition of
$A'$, each such neighborhood must also contain some points of $A$. This
implies that $a$ is an accumulation point of $A$, that is $a\in A'$.
Therefore $\overline{A'}\sse A'$ and we have $\overline{A'}=A'$.

Finally, from the definition of the Cantor-Bendixson rank and the above
properties, if $A$ has Cantor-Bendixson rank $\alpha$, the sets
\[
  A^{(1)} \supset A^{(2)} \supset \cdots \supset A^{(\alpha)}
\]
form a strictly decreasing chain of closed sets.
%%%%%
%%%%%
\end{document}
