\documentclass[12pt]{article}
\usepackage{pmmeta}
\pmcanonicalname{HausdorffMetric}
\pmcreated{2013-03-22 13:28:34}
\pmmodified{2013-03-22 13:28:34}
\pmowner{Koro}{127}
\pmmodifier{Koro}{127}
\pmtitle{Hausdorff metric}
\pmrecord{11}{34046}
\pmprivacy{1}
\pmauthor{Koro}{127}
\pmtype{Definition}
\pmcomment{trigger rebuild}
\pmclassification{msc}{54E35}
\pmsynonym{Hausdorff distance}{HausdorffMetric}
\pmdefines{Hausdorff hemimetric}

% this is the default PlanetMath preamble.  as your knowledge
% of TeX increases, you will probably want to edit this, but
% it should be fine as is for beginners.

% almost certainly you want these
\usepackage{amssymb}
\usepackage{amsmath}
\usepackage{amsfonts}
\usepackage{mathrsfs}

% used for TeXing text within eps files
%\usepackage{psfrag}
% need this for including graphics (\includegraphics)
%\usepackage{graphicx}
% for neatly defining theorems and propositions
%\usepackage{amsthm}
% making logically defined graphics
%%%\usepackage{xypic}

% there are many more packages, add them here as you need them

% define commands here
\newcommand{\C}{\mathbb{C}}
\newcommand{\R}{\mathbb{R}}
\newcommand{\N}{\mathbb{N}}
\newcommand{\Z}{\mathbb{Z}}
\begin{document}
Let $(X,d)$ be a metric space, and let $\mathcal{F}_X$ be the family of all
closed and bounded subsets of $X$. Given $A\in \mathcal{F}_X$, we will denote by
$N_r(A)$ the neighborhood of $A$ of radius $r$, i.e. the set
$\cup_{x\in A} B(x,r)$.

The \emph{upper Hausdorff hemimetric} is defined by
 $$\delta^*(A,B)=\inf\{r>0 : B\subset N_r(A)\}.$$
Analogously, the \emph{lower Hausdorff hemimetric} is $$\delta_*(A,B) =
\inf\{r>0 : A\subset N_r(B)\}.$$ 
Finally, the \emph{Hausdorff metric} is given by
$$\delta(A,B) = \max\{\delta^*(A,B),\delta_*(A,B)\}.$$
for $A,B\in \mathcal{F}_X$.

The following properties follow straight from the definitions:
\begin{enumerate}
\item $\delta^*(A,B) = \delta_*(B,A)$;
\item $\delta^*(A,B)=0$ if and only if $B\subset A$;
\item $\delta_*(A,B)=0$ if and only if $A\subset B$;
\item $\delta^*(A,C) \leq \delta^*(A,B) + \delta^*(B,C)$, and similarly for
$\delta_*$.
\end{enumerate}

From this it is clear that $\delta$ is a metric: the triangle inequality follows from that of $\delta_*$ and $\delta^*$; symmetry follows from
$\delta^*(A,B)=\delta_*(A,B)$; and $\delta(A,B) = 0$ iff both $\delta_*(A,B)$ and $\delta^*(A,B)$ are zero iff $A\subset B$ and $B\subset A$ iff $A=B$.

Hausdorff metric inherits completeness; i.e. if $(X,d)$ is complete, then so is $(\mathcal{F}_X,\delta)$. Also, if $(X,d)$ is totally bounded, then so is
$(\mathcal{F}_X,\delta)$. 

Intuitively, the Hausdorff hemimetric $\delta^*$ (resp. $\delta_*$) measure how much bigger (resp. smaller) is a set compared to another. This allows us to define hemicontinuity of correspondences.
%%%%%
%%%%%
\end{document}
