\documentclass[12pt]{article}
\usepackage{pmmeta}
\pmcanonicalname{HomeomorphismsPreserveConnectedComponents}
\pmcreated{2013-03-22 18:45:32}
\pmmodified{2013-03-22 18:45:32}
\pmowner{joking}{16130}
\pmmodifier{joking}{16130}
\pmtitle{homeomorphisms preserve connected components}
\pmrecord{5}{41538}
\pmprivacy{1}
\pmauthor{joking}{16130}
\pmtype{Derivation}
\pmcomment{trigger rebuild}
\pmclassification{msc}{54D05}

% this is the default PlanetMath preamble.  as your knowledge
% of TeX increases, you will probably want to edit this, but
% it should be fine as is for beginners.

% almost certainly you want these
\usepackage{amssymb}
\usepackage{amsmath}
\usepackage{amsfonts}

% used for TeXing text within eps files
%\usepackage{psfrag}
% need this for including graphics (\includegraphics)
%\usepackage{graphicx}
% for neatly defining theorems and propositions
%\usepackage{amsthm}
% making logically defined graphics
%%%\usepackage{xypic}

% there are many more packages, add them here as you need them

% define commands here

\begin{document}
Let $X,Y$ be topological spaces and $X=\bigcup\, X_i$, $Y=\bigcup\, Y_j$ be decompositions into connected components.

\textbf{Proposition.} Assume that $f:X\to Y$ is a homeomorphism. Then for any $i$ there exists $j$ such that $f(X_i)=Y_j$.

\textit{Proof.} Take any $i$. Because $f$ is continuous $f(X_i)$ is connected, then there exists $j$ such that $f(X_i)\subseteq Y_j$ (because $Y_j$ is a connected component). Now $f$ is a homeomorphism, $f^{-1}(Y_j)\cap X_i\neq\emptyset$, $Y_j$ is connected and $X_i$ is a connected component, so $f^{-1}(Y_j)\subseteq X_i$. Thus $Y_j\subseteq f(X_i)$, which completes the proof. $\square$
%%%%%
%%%%%
\end{document}
