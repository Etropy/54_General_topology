\documentclass[12pt]{article}
\usepackage{pmmeta}
\pmcanonicalname{CompactlySupportedContinuousFunctionsAreDenseInLp}
\pmcreated{2013-03-22 18:38:53}
\pmmodified{2013-03-22 18:38:53}
\pmowner{asteroid}{17536}
\pmmodifier{asteroid}{17536}
\pmtitle{compactly supported continuous functions are dense in $L^p$}
\pmrecord{6}{41390}
\pmprivacy{1}
\pmauthor{asteroid}{17536}
\pmtype{Theorem}
\pmcomment{trigger rebuild}
\pmclassification{msc}{54C35}
\pmclassification{msc}{46E30}
\pmclassification{msc}{28C15}
\pmsynonym{$C_c(X)$ is dense in $L^p(X)$}{CompactlySupportedContinuousFunctionsAreDenseInLp}

\endmetadata

% this is the default PlanetMath preamble.  as your knowledge
% of TeX increases, you will probably want to edit this, but
% it should be fine as is for beginners.

% almost certainly you want these
\usepackage{amssymb}
\usepackage{amsmath}
\usepackage{amsfonts}

% used for TeXing text within eps files
%\usepackage{psfrag}
% need this for including graphics (\includegraphics)
%\usepackage{graphicx}
% for neatly defining theorems and propositions
%\usepackage{amsthm}
% making logically defined graphics
%%%\usepackage{xypic}

% there are many more packages, add them here as you need them

% define commands here

\begin{document}
Let $(X, \mathcal{B}, \mu)$ be a measure space, where $X$ is a locally compact Hausdorff space, $\mathcal{B}$ a \PMlinkname{$\sigma$-algebra}{SigmaAlgebra} that contains all compact subsets of $X$ and $\mu$ a measure such that:

\begin{itemize}
\item $\mu(K) < \infty$ for all compact sets $K \subset X$.
\item $\mu$ is inner regular, meaning $\mu(A) = \sup\{ \mu(K) : K \subset A, \; K\,\text{is compact}\}$
\item $\mu$ is outer regular, meaning $\mu(A) = \inf\{ \mu(U) : A \subset U,\; U \in \mathcal{B} \text{and}\; U\,\text{is open}\}$
\end{itemize}

We denote by $C_c(X)$ the space of continuous functions $X \to \mathbb{C}$ with compact support.

{\bf Theroem -} For every $1 \leq p < \infty$, $C_c(X)$ is dense in \PMlinkname{$L^p(X)$}{LpSpace}.

{\bf \emph{\PMlinkescapetext{Proof}}:} It is clear that $C_c(X)$ is indeed contained in $L^p(X)$, where we identify each function in $C_c(X)$ with its class in $L^p(X)$.

We begin by proving that for each $A \in \mathcal{B}$ with finite measure, the characteristic function $\chi_A$ can be approximated, in the $L^p$ norm, by functions in $C_c(X)$. Let $\epsilon > 0$. By \PMlinkescapetext{inner and outer regularity} of $\mu$, we know there exist an open set $U$ and a compact set $K$ such that $K \subset A \subset U$ and
\begin{align*}
\mu(U \setminus K) = \mu(U) - \mu(K) < \epsilon
\end{align*}

By the \PMlinkname{Urysohn's lemma for locally compact Hausdorff spaces}{ApplicationsOfUrysohnsLemmaToLocallyCompactHausdorffSpaces}, we know there is a function $f \in C_c(X)$ such that $0 \leq f \leq 1$, $f|_K = 1$ and $\mathrm{supp}\,f \subset U$. Hence,

\begin{align*}
\int_X |\chi_A - f|^p \;d\mu = \int_{U \setminus K} |\chi_A - f|^p \;d\mu < \epsilon
\end{align*}

Thus, $\chi_A$ can be approximated in $L^p$ by functions in $C_c(X)$.

Now, it follows easily that any simple function $\sum_{i=1}^n c_i \chi_{A_i}$, where each $A_i$ has finite measure, can also be approximated by a compactly supported continuous function. Since this kind of simple functions are dense in $L^p(X)$ we see that $C_c(X)$ is also dense in $L^p(X)$. $\square$
%%%%%
%%%%%
\end{document}
