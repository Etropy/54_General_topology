\documentclass[12pt]{article}
\usepackage{pmmeta}
\pmcanonicalname{Diameter}
\pmcreated{2013-03-22 12:20:36}
\pmmodified{2013-03-22 12:20:36}
\pmowner{drini}{3}
\pmmodifier{drini}{3}
\pmtitle{diameter}
\pmrecord{4}{31989}
\pmprivacy{1}
\pmauthor{drini}{3}
\pmtype{Definition}
\pmcomment{trigger rebuild}
\pmclassification{msc}{54-00}
\pmrelated{Pi}

\endmetadata

%\usepackage{graphicx}
%%%\usepackage{xypic} 
\usepackage{bbm}
\newcommand{\Z}{\mathbbmss{Z}}
\newcommand{\C}{\mathbbmss{C}}
\newcommand{\R}{\mathbbmss{R}}
\newcommand{\Q}{\mathbbmss{Q}}
\newcommand{\mathbb}[1]{\mathbbmss{#1}}
\begin{document}
Let $A$ a subset of a pseudometric space $(X,d)$. The \emph{diameter} of $A$ is defined to be
$$\sup\{d(x,y) : x\in A, y\in A\}$$
whenever the supremum exists. If the supremum doesn't exist, diameter of $A$ is defined to be infinite.

Having finite diameter is not a topological invariant.
%%%%%
%%%%%
\end{document}
