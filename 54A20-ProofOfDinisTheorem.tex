\documentclass[12pt]{article}
\usepackage{pmmeta}
\pmcanonicalname{ProofOfDinisTheorem}
\pmcreated{2013-03-22 12:44:13}
\pmmodified{2013-03-22 12:44:13}
\pmowner{mathcam}{2727}
\pmmodifier{mathcam}{2727}
\pmtitle{proof of Dini's theorem}
\pmrecord{5}{33038}
\pmprivacy{1}
\pmauthor{mathcam}{2727}
\pmtype{Proof}
\pmcomment{trigger rebuild}
\pmclassification{msc}{54A20}

\endmetadata

% this is the default PlanetMath preamble.  as your knowledge
% of TeX increases, you will probably want to edit this, but
% it should be fine as is for beginners.

% almost certainly you want these
\usepackage{amssymb}
\usepackage{amsmath}
\usepackage{amsfonts}

% used for TeXing text within eps files
%\usepackage{psfrag}
% need this for including graphics (\includegraphics)
%\usepackage{graphicx}
% for neatly defining theorems and propositions
%\usepackage{amsthm}
% making logically defined graphics
%%%\usepackage{xypic}

% there are many more packages, add them here as you need them

% define commands here
\begin{document}
Without loss of generality we will assume that $X$ is compact and, by replacing
$f_n$ with $f-f_n$, that the net converges monotonically to 0. 

Let $\epsilon > 0$.
For each $x\in X$, we can choose an $n_x$, such that $f_{n_x}(x) <
\epsilon/2$. Since $f_{n_x}$ is continuous, 
there is an open
neighbourhood $U_x$ of $x$, such that for each $y\in U_x$, we have $f_{n_x}(y)
< \epsilon/2$. The open sets $U_x$ cover $X$, which is compact, so we can choose
finitely many $x_1, \ldots, x_k$ such that the $U_{x_i}$ also cover $X$. Then,
if $N\geq n_{x_1}, \ldots, n_{x_k}$, we have $f_n(x) < \epsilon$ for each
$n\geq N$ and $x\in X$, since the sequence $f_n$ is monotonically decreasing.
Thus, $\{f_n\}$ converges to 0 uniformly on $X$, which was to be proven.
%%%%%
%%%%%
\end{document}
