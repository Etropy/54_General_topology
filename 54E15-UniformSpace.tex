\documentclass[12pt]{article}
\usepackage{pmmeta}
\pmcanonicalname{UniformSpace}
\pmcreated{2013-03-22 12:46:26}
\pmmodified{2013-03-22 12:46:26}
\pmowner{mps}{409}
\pmmodifier{mps}{409}
\pmtitle{uniform space}
\pmrecord{12}{33085}
\pmprivacy{1}
\pmauthor{mps}{409}
\pmtype{Definition}
\pmcomment{trigger rebuild}
\pmclassification{msc}{54E15}
\pmdefines{uniform structure}
\pmdefines{uniformity}
\pmdefines{entourage}
\pmdefines{$V$-close}
\pmdefines{vicinity}

\endmetadata

% this is the default PlanetMath preamble.  as your knowledge
% of TeX increases, you will probably want to edit this, but
% it should be fine as is for beginners.

% almost certainly you want these
\usepackage{amssymb}
\usepackage{amsmath}
\usepackage{amsfonts}

% used for TeXing text within eps files
%\usepackage{psfrag}
% need this for including graphics (\includegraphics)
%\usepackage{graphicx}
% for neatly defining theorems and propositions
%\usepackage{amsthm}
% making logically defined graphics
%%%\usepackage{xypic}

% there are many more packages, add them here as you need them

% define commands here
\begin{document}
A \emph{uniform structure} (or \emph{uniformity}) on a set $X$ is a non empty set $\mathcal{U}$ of subsets of $X \times X$ which satisfies the following axioms:
\begin{enumerate}
\item Every subset of $X\times X$ which contains a set of $\mathcal{U}$ belongs to $\mathcal{U}$.
\item Every finite intersection of sets of $\mathcal{U}$ belongs to $\mathcal{U}$.
\item Every set of $\mathcal{U}$ is a reflexive relation on $X$ (i.e. contains the diagonal).
\item If $V$ belongs to $\mathcal{U}$, then $V' = \{(y,x): (x,y) \in V\}$ belongs to $\mathcal{U}$.
\item If $V$ belongs to $\mathcal{U}$, then exists $V'$ in $\mathcal{U}$ such that, whenever $(x,y),(y,z) \in V'$, then $(x,z) \in V$ (i.e. $V'\circ V'\subseteq V$).
\end{enumerate}

The sets of $\mathcal{U}$ are called \emph{entourages} or \emph{vicinities}. The set $X$ together with the uniform structure $\mathcal{U}$ is called a \emph{uniform space}.

If $V$ is an entourage, then for any $(x,y)\in V$ we say that $x$ and $y$ are \emph{$V$-close}.

Every uniform space can be considered a topological space with a natural topology induced by uniform structure. The uniformity, however, provides in general a richer structure, which formalize the concept of relative closeness: in a uniform space we can say that $x$ is close to $y$ as $z$ is to $w$, which makes no sense in a topological space. It follows that uniform spaces are the most natural environment for uniformly continuous functions and Cauchy sequences, in which these concepts are naturally involved.

Examples of uniform spaces are metric spaces, topological groups, and topological vector spaces.
%%%%%
%%%%%
\end{document}
