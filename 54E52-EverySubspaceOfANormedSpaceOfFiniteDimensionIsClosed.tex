\documentclass[12pt]{article}
\usepackage{pmmeta}
\pmcanonicalname{EverySubspaceOfANormedSpaceOfFiniteDimensionIsClosed}
\pmcreated{2013-03-22 14:56:28}
\pmmodified{2013-03-22 14:56:28}
\pmowner{Mathprof}{13753}
\pmmodifier{Mathprof}{13753}
\pmtitle{every subspace of a normed space of finite dimension is closed}
\pmrecord{12}{36632}
\pmprivacy{1}
\pmauthor{Mathprof}{13753}
\pmtype{Theorem}
\pmcomment{trigger rebuild}
\pmclassification{msc}{54E52}
\pmclassification{msc}{15A03}
\pmclassification{msc}{46B99}

\endmetadata

% this is the default PlanetMath preamble.  as your knowledge
% of TeX increases, you will probably want to edit this, but
% it should be fine as is for beginners.

% almost certainly you want these
\usepackage{amssymb}
\usepackage{amsmath}
\usepackage{amsfonts}

% used for TeXing text within eps files
%\usepackage{psfrag}
% need this for including graphics (\includegraphics)
%\usepackage{graphicx}
% for neatly defining theorems and propositions
%\usepackage{amsthm}
% making logically defined graphics
%%%\usepackage{xypic}

% there are many more packages, add them here as you need them

% define commands here
\begin{document}
Let $(V, \| \cdot \|)$ be a  normed vector space, and $S \subset V$ a finite dimensional  subspace. Then $S$ is closed.


\textbf{Proof}

Let $a \in \overline{S} $ and choose a sequence $\{a_n\}$ with $a_n \in S$ such that
$a_n$ converges to $a$. Then $\{a_n\}$ is a Cauchy sequence in $V$ and 
is also a Cauchy sequence in $S$. Since a finite dimensional normed
space is a Banach space, $S$ is complete, so $\{a_n\}$ converges to an 
element of $S$. Since limits in a normed  space are unique, that limit
must be $a$, so $a \in S$.

\textbf{Example}

The result depends on the field being the real or complex numbers.
Suppose the $V = Q \times R$, viewed as a vector space over $Q$ and
$S = Q \times Q$ is the finite dimensional subspace. Then clearly $(1, \sqrt{2})$ is in
$V$ and is a limit point of $S$ which is not in $S$. So $S$ is not closed.

\textbf{Example}

On the other hand, there is an example where $Q$ is the underlying 
field and we can still show a finite dimensional subspace is closed.  Suppose
that $V = Q^n$, the set of $n$-tuples of rational numbers, viewed
as vector space over $Q$. Then if $S$ is a finite dimensional subspace 
it must be that $S = \{x | Ax = 0\}$ for some matrix $A$. 
That is,  $S$ is the inverse image of the closed set $\{0\}$.
Since the map $x \to Ax$ is continuous, it follows that $S$ is a closed set.


%%%%%
%%%%%
\end{document}
