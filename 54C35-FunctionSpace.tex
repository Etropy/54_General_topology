\documentclass[12pt]{article}
\usepackage{pmmeta}
\pmcanonicalname{FunctionSpace}
\pmcreated{2013-03-22 14:08:31}
\pmmodified{2013-03-22 14:08:31}
\pmowner{matte}{1858}
\pmmodifier{matte}{1858}
\pmtitle{function space}
\pmrecord{38}{35556}
\pmprivacy{1}
\pmauthor{matte}{1858}
\pmtype{Topic}
\pmcomment{trigger rebuild}
\pmclassification{msc}{54C35}
\pmclassification{msc}{26-00}
\pmclassification{msc}{46-00}
\pmclassification{msc}{30H05}
\pmsynonym{space of functions}{FunctionSpace}

% this is the default PlanetMath preamble.  as your knowledge
% of TeX increases, you will probably want to edit this, but
% it should be fine as is for beginners.

% almost certainly you want these
\usepackage{amssymb,euscript}
\usepackage{amsmath}
\usepackage{amsfonts}
\usepackage{amsthm}
\usepackage{mathrsfs}

% used for TeXing text within eps files
%\usepackage{psfrag}
% need this for including graphics (\includegraphics)
%\usepackage{graphicx}
% for neatly defining theorems and propositions
%
% making logically defined graphics
%%%\usepackage{xypic}

% there are many more packages, add them here as you need them

% define commands here

\newcommand{\sR}[0]{\mathbb{R}}
\newcommand{\sC}[0]{\mathbb{C}}
\newcommand{\sN}[0]{\mathbb{N}}
\newcommand{\sZ}[0]{\mathbb{Z}}

 \usepackage{bbm}
 \newcommand{\Z}{\mathbbmss{Z}}
 \newcommand{\C}{\mathbbmss{C}}
 \newcommand{\R}{\mathbbmss{R}}
 \newcommand{\Q}{\mathbbmss{Q}}



\newcommand*{\norm}[1]{\lVert #1 \rVert}
\newcommand*{\abs}[1]{| #1 |}



\newtheorem{thm}{Theorem}
\newtheorem{defn}{Definition}
\newtheorem{prop}{Proposition}
\newtheorem{lemma}{Lemma}
\newtheorem{cor}{Corollary}
\begin{document}
Generally speaking, a \textit{function space} is a collection of functions satisfying certain properties.  Typically, these properties are topological in nature, and hence the word ``space''.  Usually, functions in a function space have a common \PMlinkname{domain}{Function} and codomain.  Thus, a function space $\EuScript{F}$, which contains functions acting from set $X$ to set $Y$, is denoted by $\EuScript{F}(X,Y)$.  Evidently, $\EuScript{F}(X,Y)\subseteq Y^X$.  In the case when $Y=\mathbb{R}$ one usually writes only $\EuScript{F}(X)$.

If the codomain $Y$ is a vector space over field $K$, then it is easy to define
operations of the vector space on functions acting to $Y$ in the following way:
\begin{equation}
    \begin{array}{rcl}
        (\alpha\cdot f)\, (x) & = & \alpha\cdot f(x) \\
        (f+g)\,(x) & = & f(x)+g(x)
    \end{array}
\label{op}
\end{equation}
where $\alpha$ is an element of the field $K$, and $x$ is an element of the \PMlinkname{domain}{Function} of functions.
One usually consider function spaces which are closed under operations (\ref{op}) and thus
are vector spaces. Function spaces are also often equipped with some topology.

Below is a list of function spaces, \PMlinkescapetext{links} to entries where they are defined, and notation for these.

The main purpose of this entry is to give a list
of function spaces that already have been 
defined on PlanetMath (or should be), a gallery of function
spaces if you like. 

\subsubsection*{Restrictions on smoothness}
\begin{itemize}
\item $C$; continuous functions
\item $C^k$; $k$ times continuously differentiable functions
\item $C^{k,\alpha}$; H\"older continuous functions
\item $\mathrm{Lip}$; Lipschitz continuous functions
\item $C^\infty$; smooth functions
\item $C^\omega$; analytic functions
\item $\mathcal{O}(G)$; holomorphic functions
\item $C_c^\infty$ or $\mathcal{D}$; smooth functions with compact support 
\end{itemize}

\subsubsection*{Restrictions on integrability}
\begin{itemize}
\item $L^0$; measurable functions
\item $L^1$; integrable functions
\item $L^2$; square integrable functions
\item $L^p$ \PMlinkname{functions}{LpSpace}
\item $L^\infty$; essentially bounded functions
\item $L^1_{\scriptsize{\mbox{loc}}}(U)$; locally integrable function 
\end{itemize}
\subsubsection*{Integrability of derivatives}
\begin{itemize}
\item $BV$; functions of bounded variation, i.e.\ functions whose derivative is a measure
\item $W^{m,p}(\Omega)$; Sobolev space of $p$-integrable functions which have $p$-integrable derivatives of $m$-th order. Space $W^{m,2}(\Omega)$ is a Hilbert space and is usually denoted by $W^{m}(\Omega)$ or $H^{m}(\Omega)$.
\item $BMO$; functions with bounded mean oscillation. $VMO$ functions with vanishing mean oscillation
\end{itemize}

\subsubsection*{Restriction on growth}
\begin{itemize}
\item $B$; bounded functions
\item Functions with polynomial growth
\item $\mathscr{S}$; rapidly decreasing functions (Schwartz space)
\end{itemize}

\subsubsection*{Test function spaces}
\begin{itemize}
\item $\mathscr{S}$; rapidly decreasing functions (Schwartz space)
\item $\mathscr{D}$; smooth functions with compact support
\end{itemize}

\subsubsection*{Distribution spaces}
\begin{itemize}
\item $\mathscr{S}'$; tempered distributions
\item $\mathscr{D}'$; distributions
\item $\mathscr{E}'$; distributions with compact support
\item $\mathscr{M}$; Radon measures
\end{itemize}

\subsubsection*{Piecewise properties}
\begin{itemize}
\item $PC$; piecewise continuous functions
\item $PC^k$; piecewise k times continuous differentiable functions
\item $PC^{\infty}$; \PMlinkname{piecewise smooth functions}{PiecewiseSmooth}
\item piecewise linear functions
\item simple functions
\end{itemize}

It is possible to attach a number which we call
\emph{regularity index}, to many of these spaces. 
If a space $X$ has a regularity index which is strictly less than the regularity index of $Y$, then (under some hypothesis on the domain of the functions) $X$ contains $Y$. 

Here is a list of regularity indices ($n$ is the dimension of the domain):
\begin{center}
\begin{tabular}{|l|c|}
\hline
$C$ & $0$\\
\hline
$C^k$ & $k$\\
\hline
$C^\infty$ & $\infty$\\
\hline
$C^\omega$ & $\infty$\\
\hline
$C^{k,\alpha}$ & $k+\alpha$\\
\hline
$\mathrm{Lip}$ & $1$\\
\hline
$L^p$ & $-n/p$\\
\hline
$L^\infty$ & $0$\\
\hline
$W^{k,p}$ & $k-n/p$\\
\hline
$W^{k,\infty}$ & $k$\\
\hline
$BV$ & $0$\\
\hline
$\mathscr D'$ & $-\infty$\\
\hline
$\mathscr M$ & $-n$\\
\hline
\end{tabular}
\end{center}

\subsubsection*{Selected links}

\begin{itemize}
\item The entry \htmladdnormallink{Function space}{http://en.wikipedia.org/wiki/Function_space}
at the \htmladdnormallink{Wikipedia}{http://en.wikipedia.org/}.

\item Chapter \htmladdnormallink{Function spaces}{http://www.math.uiowa.edu/~dstewart/classes/22m176/dfs-notes/node2.html}
from the
\htmladdnormallink{Notes on distributions and function spaces}{http://www.math.uiowa.edu/~dstewart/classes/22m176/dfs-notes/}
by
\htmladdnormallink{D.~Stewart}{http://www.math.uiowa.edu/~dstewart/}.

\end{itemize}
%%%%%
%%%%%
\end{document}
