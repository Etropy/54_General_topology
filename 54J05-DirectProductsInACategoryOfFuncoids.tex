\documentclass[12pt]{article}
\usepackage{pmmeta}
\pmcanonicalname{DirectProductsInACategoryOfFuncoids}
\pmcreated{2013-09-12 19:40:35}
\pmmodified{2013-09-12 19:40:35}
\pmowner{porton}{9363}
\pmmodifier{porton}{9363}
\pmtitle{Direct products in a category of funcoids}
\pmrecord{11}{87329}
\pmprivacy{1}
\pmauthor{porton}{9363}
\pmtype{Definition}
\pmcomment{most of the problem is now sovled}
\pmclassification{msc}{54J05}
\pmclassification{msc}{54A05}
\pmclassification{msc}{54D99}
\pmclassification{msc}{54E05}
\pmclassification{msc}{54E17}
\pmclassification{msc}{54E99}

\endmetadata

% this is the default PlanetMath preamble.  as your knowledge
% of TeX increases, you will probably want to edit this, but
% it should be fine as is for beginners.

% almost certainly you want these
\usepackage{amssymb}
\usepackage{amsmath}
\usepackage{amsfonts}

% need this for including graphics (\includegraphics)
\usepackage{graphicx}
% for neatly defining theorems and propositions
\usepackage{amsthm}

% making logically defined graphics
%\usepackage{xypic}
% used for TeXing text within eps files
%\usepackage{psfrag}

% there are many more packages, add them here as you need them

% define commands here

\begin{document}
ADDED: I've proved that subatomic product is the categorical product.

There are defined (\href{http://www.mathematics21.org/algebraic-general-topology.html}{see my book}) several kinds of product of (any possibly infinite number) funcoids:

\begin{enumerate}
\item cross-composition product
\item subatomic product
\item displaced product
\end{enumerate}

There is one more kind of product, for which it is not proved that the product of funcoids are (pointfree) funcoids:

$$\left\langle f_1 \times f_2 \right\rangle x = \bigsqcup \left\{ \left\langle
f_1 \right\rangle X \times^{\mathsf{\operatorname{FCD}}} \left\langle f_2
\right\rangle X \hspace{1em} | \hspace{1em} X \in \operatorname{atoms} x \right\}.$$

It is considered natural by analogy with the category {\bf Top} of topological spaces to consider this category:

\begin{itemize}
\item Objects are endofuncoids on small sets.
\item Morphisms between a endofuncoids $\mu$ and $\nu$ are continuous (that is corresponding to a continuous funcoid) functions from the object of $\mu$ to the object of $\nu$.
\item Composition is induced by composition of functions.
\end{itemize}

It is trivial to show that the above is really a category.

The product of functions is the same as in {\bf Set}.
\end{document}
