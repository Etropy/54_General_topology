\documentclass[12pt]{article}
\usepackage{pmmeta}
\pmcanonicalname{Coarser}
\pmcreated{2013-03-22 12:56:03}
\pmmodified{2013-03-22 12:56:03}
\pmowner{rspuzio}{6075}
\pmmodifier{rspuzio}{6075}
\pmtitle{coarser}
\pmrecord{10}{33290}
\pmprivacy{1}
\pmauthor{rspuzio}{6075}
\pmtype{Definition}
\pmcomment{trigger rebuild}
\pmclassification{msc}{54-00}
\pmsynonym{stronger}{Coarser}
\pmrelated{InitialTopology}
\pmrelated{LatticeOfTopologies}
\pmdefines{weaker}
\pmdefines{finer}
\pmdefines{refinement}
\pmdefines{expansion}

\endmetadata

% this is the default PlanetMath preamble.  as your knowledge
% of TeX increases, you will probably want to edit this, but
% it should be fine as is for beginners.

% almost certainly you want these
\usepackage{amssymb}
\usepackage{amsmath}
\usepackage{amsfonts}

% used for TeXing text within eps files
%\usepackage{psfrag}
% need this for including graphics (\includegraphics)
%\usepackage{graphicx}
% for neatly defining theorems and propositions
%\usepackage{amsthm}
% making logically defined graphics
%%%\usepackage{xypic}

% there are many more packages, add them here as you need them

% define commands here
\begin{document}
The set of topologies which can be defined on a set is partially ordered under inclusion.  Below, we list several synonymous terms which are used to refer to this order.  Let $\mathcal{U}$ and $\mathcal{V}$ be two topologies defined on a set $E$.  All of the following expressions mean that $\mathcal{U} \subset \mathcal{V}$:
\begin{itemize}
\item $\mathcal{U}$ is \textbf{weaker} than $\mathcal{V}$
\item $\mathcal{U}$ is \textbf{coarser} than $\mathcal{V}$
\item $\mathcal{V}$ is \textbf{finer} than $\mathcal{U}$
\item $\mathcal{V}$ is a \textbf{refinement} of $\mathcal{U}$
\item $\mathcal{V}$ is an \textbf{expansion} of $\mathcal{U}$
\end{itemize}

It is worth noting that this condition is equivalent to the requirement that  the identity map from $(E, \mathcal{V})$ to $(E, \mathcal{U})$ is continuous.
%%%%%
%%%%%
\end{document}
