\documentclass[12pt]{article}
\usepackage{pmmeta}
\pmcanonicalname{Generic}
\pmcreated{2013-03-22 13:40:30}
\pmmodified{2013-03-22 13:40:30}
\pmowner{Koro}{127}
\pmmodifier{Koro}{127}
\pmtitle{generic}
\pmrecord{10}{34341}
\pmprivacy{1}
\pmauthor{Koro}{127}
\pmtype{Definition}
\pmcomment{trigger rebuild}
\pmclassification{msc}{54E52}
\pmdefines{generically}

\endmetadata

% this is the default PlanetMath preamble.  as your knowledge
% of TeX increases, you will probably want to edit this, but
% it should be fine as is for beginners.

% almost certainly you want these
\usepackage{amssymb}
\usepackage{amsmath}
\usepackage{amsfonts}
\usepackage{mathrsfs}

% used for TeXing text within eps files
%\usepackage{psfrag}
% need this for including graphics (\includegraphics)
%\usepackage{graphicx}
% for neatly defining theorems and propositions
%\usepackage{amsthm}
% making logically defined graphics
%%%\usepackage{xypic}

% there are many more packages, add them here as you need them

% define commands here
\newcommand{\C}{\mathbb{C}}
\newcommand{\R}{\mathbb{R}}
\newcommand{\N}{\mathbb{N}}
\newcommand{\Z}{\mathbb{Z}}
\newcommand{\Per}{\operatorname{Per}}
\begin{document}
\PMlinkescapeword{satisfies}
A property that holds for all $x$ in some residual subset of a Baire space $X$ is said to be \emph{generic} in $X$, or to \emph{hold generically} in $X$. In the study of generic properties, it is common to state ``generically, $P(x)$'', where $P(x)$ is some proposition about $x\in X$. The useful fact about generic properties is that, given countably many generic properties $P_n$, all of them hold simultaneously in a residual set, i.e. we have that, generically, $P_n(x)$ holds for each $n$.
%%%%%
%%%%%
\end{document}
