\documentclass[12pt]{article}
\usepackage{pmmeta}
\pmcanonicalname{SeparatedUniformSpace}
\pmcreated{2013-03-22 16:42:34}
\pmmodified{2013-03-22 16:42:34}
\pmowner{CWoo}{3771}
\pmmodifier{CWoo}{3771}
\pmtitle{separated uniform space}
\pmrecord{5}{38925}
\pmprivacy{1}
\pmauthor{CWoo}{3771}
\pmtype{Definition}
\pmcomment{trigger rebuild}
\pmclassification{msc}{54E15}
\pmsynonym{separating}{SeparatedUniformSpace}
\pmsynonym{Hausdorff uniform space}{SeparatedUniformSpace}

\usepackage{amssymb,amscd}
\usepackage{amsmath}
\usepackage{amsfonts}

% used for TeXing text within eps files
%\usepackage{psfrag}
% need this for including graphics (\includegraphics)
%\usepackage{graphicx}
% for neatly defining theorems and propositions
\usepackage{amsthm}
% making logically defined graphics
%%\usepackage{xypic}
\usepackage{pst-plot}
\usepackage{psfrag}

% define commands here

\begin{document}
\PMlinkescapeword{separated}
\PMlinkescapeword{separation axiom}

Let $X$ be a uniform space with uniformity $\mathcal{U}$.  $X$ is said to be \emph{separated} or \emph{Hausdorff} if it satisfies the following \emph{separation axiom}:
$$\bigcap \mathcal{U}=\Delta,$$
where $\Delta$ is the diagonal relation on $X$ and $\bigcap \mathcal{U}$ is the intersection of all elements (entourages) in $\mathcal{U}$.  Since $\Delta\subseteq \bigcap \mathcal{U}$, the separation axiom says that the only elements that belong to every entourage of $\mathcal{U}$ are precisely the diagonal elements $(x,x)$.  Equivalently, if $x\ne y$, then there is an entourage $U$ such that $(x,y)\notin U$.

The reason for calling $X$ separated has to do with the following assertion:
\begin{quote}
$X$ is separated iff $X$ is a Hausdorff space under the topology $T_{\mathcal{U}}$ \PMlinkname{induced by}{TopologyInducedByAUniformStructure} $\mathcal{U}$.
\end{quote}

Recall that $T_{\mathcal{U}}=\lbrace A\subseteq X\mid \mbox{for each }x\in A\mbox{, there is }U\in \mathcal{U}\mbox{, such that }U[x]\subseteq A\rbrace$, where $U[x]$ is some uniform neighborhood of $x$ where, under $T_{\mathcal{U}}$, $U[x]$ is also a neighborhood of $x$.  To say that $X$ is Hausdorff under $T_{\mathcal{U}}$ is the same as saying every pair of distinct points in $X$ have disjoint uniform neighborhoods.

\begin{proof}
$(\Rightarrow)$.  Suppose $X$ is separated and $x,y\in X$ are distinct.  Then $(x,y)\notin U$ for some $U\in \mathcal{U}$.  Pick $V\in \mathcal{U}$ with $V\circ V\subseteq U$.  Set $W=V\cap V^{-1}$, then $W$ is symmetric and $W\subseteq V$.  Furthermore, $W\circ W\subseteq V\circ V\subseteq U$.  If $z\in W[x]\cap W[y]$, then $(x,z),(y,z)\in W$.  Since $W$ is symmetric, $(z,y)\in W$, so $(x,y)=(x,z)\circ (z,y)\in W\circ W\subseteq U$, which is a contradiction.

$(\Leftarrow)$.  Suppose $X$ is Hausdorff under $T_{\mathcal{U}}$ and $(x,y)\in U$ for every $U\in \mathcal{U}$ for some $x,y\in X$.  If $x\ne y$, then there are $V[x]\cap W[y]=\varnothing$ for some $V,W\in \mathcal{U}$.  Since $(x,y)\in V$ by assumption, $y\in V[x]$.  But $y\in W[y]$, contradicting the disjointness of $V[x]$ and $W[y]$.  Therefore $x=y$.
\end{proof}
%%%%%
%%%%%
\end{document}
